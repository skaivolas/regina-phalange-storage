% This file was converted to LaTeX by Writer2LaTeX ver. 1.6
% see http://writer2latex.sourceforge.net for more info
\documentclass[a5paper]{article}
\usepackage{amsmath,amssymb,amsfonts}
\usepackage{fontspec}
\usepackage{xunicode}
\usepackage{xltxtra}
\usepackage{color}
\usepackage{array}
\usepackage{supertabular}
\usepackage{hhline}
\usepackage{hyperref}
\hypersetup{colorlinks=true, linkcolor=blue, citecolor=blue, filecolor=blue, urlcolor=blue}
\usepackage{graphicx}
\usepackage{polyglossia}
\setdefaultlanguage{arabic}
\setmainfont[Script=Arabic]{FreeSerif}
% Text styles
\newcommand\textstylePolicepardfaut[1]{#1}
\newcommand\textstyleDropCaps[1]{#1}
\newcommand\textstyleCaptioncharacters[1]{#1}
\newcommand\textstylePageNumber[1]{#1}
% Outline numbering
\setcounter{secnumdepth}{0}
\makeatletter
\newcommand\arraybslash{\let\\\@arraycr}
\makeatother
% Page layout (geometry)
\setlength\voffset{-1in}
\setlength\hoffset{-1in}
\setlength\topmargin{0.5in}
\setlength\oddsidemargin{0.5in}
\setlength\textheight{7.0499997in}
\setlength\textwidth{4.8272in}
\setlength\footskip{0.4181in}
\setlength\headheight{0cm}
\setlength\headsep{0cm}
% Footnote rule
\setlength{\skip\footins}{0.0469in}
\renewcommand\footnoterule{\vspace*{-0.0071in}\setlength\leftskip{0pt}\setlength\rightskip{0pt plus 1fil}\noindent\textcolor{black}{\rule{0.25\columnwidth}{0.0071in}}\vspace*{0.0398in}}
% Pages styles
\makeatletter
\newcommand\ps@Standard{
  \renewcommand\@oddhead{}
  \renewcommand\@evenhead{}
  \renewcommand\@oddfoot{\textstylePageNumber{\thepage{} / ?}}
  \renewcommand\@evenfoot{\@oddfoot}
  \renewcommand\thepage{\arabic{page}}
}
\makeatother
\pagestyle{Standard}
\setlength\tabcolsep{1mm}
\renewcommand\arraystretch{1.3}
\title{بسم الله الرحمن الرحيم}
\author{Žāle Mottahedin}
\date{2018-12-12}
\begin{document}
بِسْـــمِ اللهِ الرَّحْمَنِ الرَّحِيمِ

الدُّرُوسُ الاَوَّلِيَّةُ

تأليفُ بَهَاء الدّين محمّد الدّاغستانِىِّ

Бисмиллахи ррахмани ррахим!

Вашему вниманию предлагается электронная версия легендарного учебника арабского языка «Первые уроки» (Ад-дурусу ль-авваия) Багауддина Мухаммада. Это самый эффективный и самый известный учебник арабского языка на всем постсоветском пространстве. Забудьте нудные уроки иностранного в школе, которые большинству учеников прививают отвращение к изучению языков на всю оставшуюся жизнь. Смело открывайте со словами «бисмиллах» первую страницу книги и, одновременно, новую страницу в своей жизни – успешное изучение арабского языка. 

Открыв и начав листать учебник, вы заметите, что в нем нет грамматических правил, многосложный заданий, надоедливых тестов и т.п. Но это нисколько не убавляет ценности данного учебника. А скорее наоборот. Чтобы понять секрет педагогического успеха Багауддина Мухаммада, обратимся к трудам современных языковедов.

Е.А. Умрюхин в своей книге "Иностранный легко и с удовольствием" выделяет три основы изучения иностранного языка: 

1) желание изучать язык 

2) ощущение успешности при изучении 

3) регулярность занятий

Желание изучать арабский язык свойственно мусульманам всех времен, а остальные два принципа успешно реализованы в «Первых уроках» Багауддина Мухаммада. Сделано это следующим образом.

Весь учебник состоит из небольших уроков, благодаря чему вы можете регулярно проходить по одному уроку без особых трудностей. Каждый урок – это набор слов и словосочетаний, который вам предстоит выучить наизусть, а также последующий текст для упражнения и закрепления, содержащий исключительно те слова, которые вы уже выучили в предыдущих уроках. Это позволяет вам сосредоточить все свои усилия на глубоком усвоение материала учебника, так как вам не придется отвлекаться на обращение к словарю и другой дополнительной литературе. Моральный настрой учащегося при такой схеме уроков выглядят примерно следующим образом.

Представьте, ученик дошел до 20-го урока и у него уже есть кое-какой словарный запас. Он переходит к упражнению по закреплению пройденного и встречает там только что изученные слова, а также слова начиная с самого первого урока. Читая каждое новое предложение, он радуется в душе: «Как здорово! Я понимаю все это предложение целиком! Я знаю каждое отдельное слово, которое сюда входит, а также понимаю смысл всей фразы, состоящей из этих отдельных слов! Мне не надо копаться в словаре! Я и так все знаю» и т.д. Перейдя к следующему уроку он снова не сталкивается ни с какими проблемами, но зато испытывает радость обладания все большим и большим словарным запасом. Поверьте, это очень воодушевляет как маленького, так и взрослого ученика и создает ощущение успеха, который побуждает к дальнейшим регулярным занятиям с максимальным вниманием и интересом.

Путем постепенного освоения новых слов и работы только со знакомыми словами достигается еще одна важная цель – развитие навыка чтения.

В.В. Лебедев в своем учебном пособии «Читаем Хадисы по-арабски» пишет:

«Навык чтения может эффективно развиваться только на основе текста, все элементы которого знакомы читающему». 

Помимо развития чтения, при работе с текстом из знакомых элементов вы самопроизвольно, даже не делая усилия над собой, запоминаете порядок слов в предложении и то, каким образом слова связываются друг с другом, образуя смысловые конструкции (как стыкуются предлог с существительным, существительное с прилагательным, глагол с существительным и т.д.). Если вы до этого не изучали иностранные языки, наверное, вам будет полезно знать следующие вещи.

Любой язык условно можно разделить на две составляющие. Первая – сами слова, которые обозначают предметы, явления, действия и т.п. Вторая – законы, по которым эти слова изменяются в зависимости от ситуации их использования. 

Несмотря на то, что первые уроки маленькие, они даются наиболее трудно. Это связано с двумя сложностями. 

Первая сложность заключается в том, что вы еще не осознали закономерностей словоизменения арабского языка и вам приходится делать особые усилия для изменения падежей, склонения глаголов и т.д. Этих закономерностей на самом деле не много и через некоторое время вы заметите, что будете правильно изменять слова, даже не задумываясь над этим.

Вторая сложность связана с тем, что вы еще не поняли, как лучше запоминать новые слова. Многие люди, которые когда-то совсем не знали арабского языка, а теперь хорошо разговаривают на нем, вспоминают, как повторяли одно и то же слово по 100-200 раз, чтобы запомнить его. Так продолжалось некоторое время. А потом слова стали запоминаться буквально после одного прочтения. 

Поэтому не стоит бояться первых трудностей – стоит проявить немного терпения и на смену трудностям обязательно придет успех.

Важно также отметить, что «Первые уроки» написаны авторитетным исламским ученым. На страницах этого учебника дети слушаются родителей, младшие уважают старших. Персонажи этой книги просыпаются рано утром и, услышав Азан, спешат на молитву в мечеть, мальчики обращают внимание на скромность и прилежность девушек и их веру в Бога, и от этого хотят жениться на них, юноши стремятся вырасти и участвовать в Джихаде и т.д. В общем, работая с данной книгой, вы не просто имеете дело с иностранным языком, но и окунаетесь с головой в удивительный Исламский мир со всеми его многочисленными проявлениями.

И последнее замечание. Когда будете продвигаться от урока к урока не удивляйтесь, когда поймете, что некоторые слова из первых уроков вы уже успели забыть. Нужно много раз повторять слова, чтобы они закрепились в вашей памяти. Нужно придумать регулярную систему прохождения новых уроков и повторения старых и следовать этой системе, стараясь не отклоняться от нее.

Многие деревенские ребята, жители сельских районов Дагестана, Чечни, а также других регионов Кавказа, которые не имели ни малейшего представления о лингвистике и плохо знали даже русский язык занимались по методике Багауддина Мухаммада и довольно быстро начинали свободно общаться на арабском языке, а также в дальнейшем продолжили самостоятельное изучение языка, читать арабские книги и т.д.. Поэтому, если у вас не получается учить арабский язык – это говорит лишь о том, что вы не достаточно сильно хотите этого, либо не можете справиться с ленью. 

Да поможет нам Бог во всех наших благих начинаниях! Аминь. 

\ Абдуллах Хайдар.

\subsection{اَلدَّرْسُ الأَوَّلُ 1}
هُوَ, هُمْ. أَنْتَ, أَنْتُمْ. أَنَا, نَحْنُ.\ \  كَبِيرٌ, كِبَارٌ. صَغِيرٌ, صِغَارٌ

\_\_\_\_\_\_\_\_\_\_\_\_\_\_\_\_\_\_

هُوَ كَبِيرٌ. هُمْ كِبَارٌ. أَنْتَ صَغِيرٌ. اَنْتُمْ صِغَارٌ. اَنَا كَبِيرٌ. نَحْنُ كِبَارٌ. هُوَ صَغِيرٌ. هُمْ صِغَارٌ. اَنْتَ كَبِيرٌ. اَنْتُمْ كِبَارٌ. اَنَا صَغِيرٌ. نَحْنُ صِغَارٌ.

\subsubsection{Урок 1}
Он большой. Они большие. Ты маленький. Вы маленькие. Я большой. Мы большие. Он маленький. Они маленькие. Ты большой. Вы большие. Я маленький. Мы маленькие.

\subsection{اَلدَّرْسُ الثَّانِى 2}
هِىَ, هُنَّ. اَنْتِ, اَنْتُنَّ. اَنَا, نَحْنُ. كَبِيرَةٌ, كَبِيرَاتٌ. صَغِيرَةٌ, صَغِيرَاتٌ.

\_\_\_\_\_\_\_\_\_\_\_\_\_\_\_\_\_\_\_\_\_\_\_\_\_\_\_\_\_\_

هِىَ كَبِيرَةٌ. هُنَّ كَبِيرَاتٌ. اَنْتِ صَغِيرَةٌ. اَنْتُنَّ صَغِيرَاتٌ. اَنَا كَبِيرَةٌ. نَحْنُ كَبِيرَاتٌ. هِىَ صَغِيرَةٌ. هُنَّ صَغِيرَاتٌ. اَنْتِ كَبِيرَةٌ. اَنْتُنَّ كَبِيرَاتٌ. اَنَا صَغِيرَةٌ. نَحْنُ صَغِيرَاتٌ.

\subsubsection{Урок 2}
Она большая. Они (ж.р.) большие. Ты маленькая. Вы (ж.р.) маленькие. Я большая. Мы (ж.р.) большие. Она маленькая. Они (ж.р.) маленькие. Ты большая. Вы (ж.р.) большие. Я маленькая. Мы маленькие.

\subsection{اَلدَّرْسُ الثَّالِثُ 3}
هَذَا. هَذِهِ. هَؤُلاَءِ. رَجُلٌ, رِجَالٌ. اِمْرَأَةٌ, نِسَاءٌ. طَوِيلٌ, طِوَالٌ. قَصِيرٌ, قِصَارٌ.

\_\_\_\_\_\_\_\_\_\_\_\_\_\_\_\_\_\_\_\_\_\_\_\_

هَذَا الرَّجُلُ طَوِيلٌ. هَؤُلاَءِ الرِّجَالُ طِوَالٌ. هَذِهِ الْمَرْأَةُ قَصِيرَةٌ. هَؤُلاَءِ النِّسَاءُ قَصِيرَاتٌ. هَذَا الرَّجُلُ كَبِيرٌ. هَؤُلاَءِ الرِّجَالُ كِبَارٌ. هَذِهِ الْمَرْأَةُ صَغِيرَةٌ. هَؤُلاَءِ النِّسَاءُ صَغِيرَاتٌ. هُوَ طَوِيلٌ. هُمْ طِوَالٌ. اَنْتَ قَصِيرٌ. اَنْتُمْ قِصَارٌ. اَنَا طَوِيلٌ. نَحْنُ طِوَالٌ.

\subsubsection{Урок 3.}
Этот мужчина высокий. Эти мужчины высокие. Эта женщина невысокая. Эти женщины невысокие. Этот мужчина большой. Эти мужчины большие. Эта женщина маленькая. Эти женщины маленькие. Он высокий. Они высокие. Ты не высокий. Вы не высокие. Я высокий. Мы высокие.

\subsection{اَلدَّرْسُ الرَّابِعُ 4}
 \includegraphics[width=4.2398in,height=2.1457in]{images/MuhammadBagauddinprettified-img001.jpg} 

مُعَلِّمٌ, معَلِّمُونَ (ون). تِلْمِيذٌ, تَلاَمِيذُ, تَلاَمِذَةٌ. مُعَلِّمَةٌ, مُعَلِّمَاتٌ (ات). تِلْمِيذَةٌ (ات) مَنْ؟

\_\_\_\_\_\_\_\_\_\_\_\_\_\_\_\_\_\_\_\_\_\_\_\_

مَنْ هَذَا؟ - مُعَلِّمٌ. مَنْ هَؤُلاَءِ؟ - مُعَلِّمُونَ. مَنْ هَذِهِ؟ - تِلْمِيذَةٌ. مَنْ هَؤُلاَءِ؟ - تِلْمِيذَاتٌ. هُوَ مُعَلِّمٌ. هُمْ مُعَلِّمُونَ. اَنْتَ تِلْمِيذٌ. اَنْتُمْ تَلاَمِيذُ. هِىَ تِلْمِيذَةٌ. هُنَّ تِلْمِيذَاتٌ. اَنَا مُعَلِّمَةٌ. نَحْنُ مُعَلِّمَاتٌ. هَذَا الرَّجُلُ مُعَلِّمٌ. هَؤُلاَءِ الرِّجَالُ مُعَلِّمُونَ. هَذِهِ الْمَرْأَةُ مُعَلِّمَةٌ. هَؤُلاَءِ النِّسَاءُ مُعَلِّمَاتٌ. مَنْ هُوَ؟ مَنْ هِىَ؟ مَنْ اَنْتَ؟ مَنْ اَنْتُمْ؟ مَنْ هُمْ؟

\subsubsection{Урок 4. }
Кто это? - Учитель. Кто это? - Учителя. Кто это? - Ученица. Кто это? - Ученицы. Он учитель. Они учителя. Ты ученик. Вы ученики. Она ученица. Они ученицы. Я учительница. Мы учительницы. Этот мужчина учитель. Эти мужчины учителя. Эта женщина учительница. Эти женщины учительницы. Кто он? Кто она? Кто ты? Кто вы? Кто они?u

\subsection{اَلدَّرْسُ الْخَامِسُ 5}
\  \includegraphics[width=0.8752in,height=0.7189in]{images/MuhammadBagauddinprettified-img002.jpg}   \includegraphics[width=1.0937in,height=1.0937in]{images/MuhammadBagauddinprettified-img003.jpg}   \includegraphics[width=1.448in,height=0.9689in]{images/MuhammadBagauddinprettified-img004.jpg}   \includegraphics[width=1.2709in,height=0.6772in]{images/MuhammadBagauddinprettified-img005.jpg} 

كِتَابٌ, كُتُبٌ. دَفْتَرٌ, دَفَاتِرُ. مِحْفَظَةٌ, مَحَافِظُ. قَلَمٌ, اَقْلاَمٌ. مَا؟ مَا هَذَا؟ لِمَنْ؟ لِلْمُعَلِّمِ.

\_\_\_\_\_\_\_\_\_\_\_\_\_\_\_\_\_\_\_\_\_\_\_\_\_

مَا هَذَا؟ - هَذَا كِتَابٌ. مَا هَذِهِ؟ - هَذِهِ مِحْفَظَةٌ. هَذَا دَفْتَرٌ. هَذِهِ دَفَاتِرُ. هَذَا الْكِتَابُ كَبِيرٌ. هَذِهِ الْكُتُبُ كَبِيرَةٌ. هَذَا الدَّفْتَرُ صَغِيرٌ. هَذِهِ الدَّفَاتِرُ صَغِيرَةٌ. هَذَا الْقَلَمُ طَوِيلٌ. هَذِهِ الأَقْلاَمُ طَوِيلَةٌ. لِمَنْ هَذَا الْكِتَابُ؟ - لِلْمُعَلِّمِ. لِمَنْ هَذَا الدَّفْتَرُ؟ - لِلتِّلْمِيذِ. لِمَنْ هَذِهِ الْمِحْفَظَةُ؟ لِلتِّلْمِيذَةِ.

\subsubsection{Урок 5}
Что это? - Это книга. Что это? - Это портфель. Это тетрадь. Это тетради. Эта книга большая. Эти книги книги большие. Эта тетрадь маленькая. Эти тетради маленькие. Этот карандаш длинный. Эти карандаши длинные. Чья эта книга? - Учителя. Чья эта тетрадь? - Ученика. Чей этот портфель? - Ученицы. 

\subsection{اَلدَّرْسُ السَّادِسُ 6}
\  \includegraphics[width=1.1354in,height=1.0937in]{images/MuhammadBagauddinprettified-img006.jpg}   \includegraphics[width=1.8528in,height=1.1965in]{images/MuhammadBagauddinprettified-img007.jpg} \ \  \includegraphics[width=2.0354in,height=1.4374in]{images/MuhammadBagauddinprettified-img008.jpg} 

بَيْتٌ, بُيُوتٌ. حُجْرَةٌ, حُجُرَاتٌ. مَدْرَسَةٌ, مَدَارِسُ. 

\ ذَاكَ. تِلْكَ. اُولَئِكَ. اَيْنَ؟ هُنَا.

هُنَاكَ. فِى. فِى الْمَدْرَسَةِ. فِى الْحُجْرَةِ. فَصْلٌ, فُصُولٌ

\_\_\_\_\_\_\_\_\_\_\_\_\_\_\_\_\_\_\_\_\_\_

هَذَا بَيْتٌ. هِذِهِ بُيُوتٌ. هَذِهِ حُجْرَةٌ. هَذِهِ حُجُرَاتٌ. أَيْنَ الْمُعَلِّمُ؟ - اَلْمُعَلِّمُ فِي الْمَدْرَسَةِ. أَيْنَ التِّلْمِيذُ؟ - اَلتِّلْمِيذُ فِي الْفَصْلِ. مَنْ فِي الْبَيْتِ؟ - فِي الْبَيْتِ رَجُلٌ. أَيْنَ الْمُعَلِّمُونَ؟ - هُمْ هُنَاكَ فِي الْمَدْرَسَةِ. أَيْنَ التَّلاَمِذَةُ؟ - هُمْ هُنَا فِي الْفَصْلِ. لِمَنْ ذَاكَ الْبَيْتُ؟ - لِذَاكَ الرَّجُلِ. لِمَنْ تِلْكَ الْحُجْرَةُ؟ - لِتِلْكَ الْمَرْأَةِ. أُولَئِكَ الرِّجَالُ مُعَلِّمُونَ. أُولَئِكَ النِّسَاءُ مُعَلِّمَاتٌ.

\subsubsection{УРОК 6}
Это дом. Это дома. Это комната. Это комнаты. Где учитель? — Учитель в школе. Где ученик? — Ученик в классе. Кто в доме? — В доме (один) мужчина. Где учителя? — Они там, в школе. Где ученики? — Они здесь, в классе. Чей тот дом? — Того мужчины. Чья та комната? — Той женщины. Те мужчины учителя. Те женщины учительницы.

\subsection{7 اَلدَّرْسُ السَّابِعُ}
 \includegraphics[width=1.8543in,height=1.1457in]{images/MuhammadBagauddinprettified-img009.jpg}   \includegraphics[width=1.6874in,height=1.1252in]{images/MuhammadBagauddinprettified-img010.jpg} 

جَرِيدَةٌ، جَرَائِدُ. مَجَلَّةٌ (ات). عَرَبِيٌّ. رُوسِيٌّ. جَمِيلٌ. وَ. كِتَابٌ وَ دَفْتَرٌ. لَهُ. لَهَا. لَكَ. لَكِ. لِي.

\_\_\_\_\_\_\_\_\_\_\_\_\_\_\_\_\_\_\_\_\_\_\_\_\_

لَهُ كِتَابٌ كَبِيرٌ. اَلْكِتَابُ جَمِيلٌ. لَهَا مِحْفَظَةٌ. اَلْمِحْفَظَةُ جَمِيلَةٌ. لِي كِتَابٌ وَ دَفْتَرٌ. لَكِ قَلَمٌ وَ مِحْفَظَةٌ. هَذِهِ الْجَرِيدَةُ عَرَبِيَّةٌ. تِلْكَ الْمَجَلَّةُ رُوسِيَّةٌ. لِمَنْ تِلْكَ الْجَرِيدَةُ؟ - لِي. لِمَنْ هَذِهِ الْمَجَلَّةُ؟ - لَكَ. أَيْنَ الْمُعَلِّمُ وَ الْمُعَلِّمَةُ؟ - اَلْمُعَلِّمُ وَ الْمُعَلِّمَةُ فِي الْفَصْلِ. أَيْنَ التِّلْمِيذُ وَ التِّلْمِيذَةُ؟ - اَلتِّلْمِيذُ وَ التِّلْمِيذَةُ فِي الْبَيْتِ. لَهُ كِتَابٌ. لَهَا دَفْتَرٌ. لَكَ قَلَمٌ. لِي ِمحْفَظَةٌ.

\subsubsection{УРОК 7}
У него большая книга. Книга красивая. У неё портфель. Портфель красивый. У меня книга и тетрадь. У тебя (ж.р.) карандаш и портфель. Эта газета — арабская. Тот журнал — русский. Чья та газета? — Моя. Чей этот журнал? — Твой. Где учитель и учительница? — Учитель и учительница в классе. Где ученик и ученица? — Ученик и ученица дома. У него книга. У неё тетрадь. У тебя карандаш. У меня портфель.

\subsection{8 اَلدَّرْسُ الثَّامِنُ}
 \includegraphics[width=1.2083in,height=1.2811in]{images/MuhammadBagauddinprettified-img011.jpg}   \includegraphics[width=1.1354in,height=1.2917in]{images/MuhammadBagauddinprettified-img012.jpg}   \includegraphics[width=1.0311in,height=1.302in]{images/MuhammadBagauddinprettified-img013.jpg} 

كَلْبٌ، كِلاَبٌ. دِيكٌ، دِيَكَةٌ. دَجَاجَةٌ (ات). مَكْتَبَةٌ، مَكَاتِبُ. هَلْ؟ نَعَمْ. لاَ. لَيْسَ. لَيْسَتْ.

\_\_\_\_\_\_\_\_\_\_\_\_\_\_\_\_\_\_\_\_\_\_\_\_\_ 

هَلْ هَذَا بَيْتٌ؟ - نَعَمْ، هَذَا بَيْتٌ. هَلْ هُوَ كَبِيرٌ؟ - نَعَمْ، هُوَ كَبِيرٌ. هَلْ هَذِهِ مَدْرَسَةٌ؟ - لاَ، هَذِهِ لَيْسَتْ بِمَدْرَسَةٍ. هَذِهِ مَكْتَبَةٌ. هَلْ هَذَا تِلْمِيذٌ؟ - لاَ، هَذَا لَيْسَ بِتِلْمِيذٍ. هَلْ لَكَ كَلْبٌ؟ - لاَ، لَيْسَ لِي كَلْبٌ. لِمَنْ ذَاكَ الدِّيكُ؟ - لِتِلْكَ الْمَرْأَةِ. أَيْنَ الْكُتُبُ؟ - اَلْكُتُبُ فِي الْمَكْتَبَةِ. هَلِ الْمُعَلِّمُ فِي الْمَدْرَسَةِ؟ - لاَ، هُوَ لَيْسَ فِي الْمَدْرَسَةِ. فِي الْمَكْتَبَةِ كُتُبٌ عَرَبِيَّةٌ وَ جَرَائِدُ رُوسِيَّةٌ.

\subsubsection{УРОК 8}
Это дом? — Да, это дом. Он большой? — Да, он большой. Это школа? — Нет, это не школа, это библиотека. Это ученик? — Нет, это не ученик. У тебя есть собака? — Нет, у меня нет собаки. Чей тот петух? — Той женщины. Где книги? — Книги в библиотеке. Учитель в школе? — Нет, он не в школе. В библиотеке арабские книги и русские газеты.

\subsection{9 اَلدَّرْسُ التَّاسِعُ}
دَرْسٌ، دُرُوسٌ. إِقْرَأْ، إِقْرَئِي. أُدْخُلْ. أُخْرُجْ. أُكْتُبْ. يَا. يَا مُحَمَّدُ. يَا فَاطِمَةُ. يَا مُحَمَّدُ اقْرَأْ. يَا فَاطِمَةُ اقْرَئِي. مِنْ. مِنَ الْبَيْتِ.

\_\_\_\_\_\_\_\_\_\_\_\_\_\_\_\_\_\_\_\_\_\_\_ 

يَا مُحَمَّدُ اقْرَأِ الْجَرِيدَةَ. يَا فَاطِمَةُ اقْرَئِي الْمَجَلَّةَ. يَا أَحْمَدُ ادْخُلِ الْفَصْلَ وَ اقْرَأِ الدَّرْسَ. يَا عَائِشَةُ اخْرُجِي مِنَ الْمَدْرَسَةِ. يَا عَبْدُ اللهِ اكْتُبِ الدَّرْسَ. يَا مَرْيَمُ ادْخُلِي الْحُجْرَةَ وَ اقْرَئِي هُنَاكَ. اِقْرَأْ هُنَا. أُكْتُبْ هُنَاكَ. أُخْرُجِي مِنْ هُنَا وَ ادْخُلِي هُنَاكَ. أَيْنَ التَّلاَمِذَةُ وَ التِّلْمِيذَاتُ؟ - هُمْ فِي الْفَصْلِ.

\subsubsection{УРОК 9}
Мухаммед, читай газету. Фатима, читай журнал. Ахмед, войди в клacc и читай Урок. Аиша, выйди из школы. Абдулла, пиши урок. Марьям заходи в комнату и читай там. Читай здесь. Пиши там. Выйди отсюда и войди туда. Где ученики и ученицы? — Они в классе.

\subsection{10 اَلدَّرْسُ العَاشِرُ}
 \includegraphics[width=1.4374in,height=1.4791in]{images/MuhammadBagauddinprettified-img014.jpg}   \includegraphics[width=1.0937in,height=1.4689in]{images/MuhammadBagauddinprettified-img015.jpg}   \includegraphics[width=1.2398in,height=1.4689in]{images/MuhammadBagauddinprettified-img016.jpg} 

مَقْعَدُ التِّلْمِيذِ. يَقْرَأُ. يَكْتُبُ.

خُذْ، خُذِي. هَاتِ، هَاتِي. وَرقٌ، أَوْرَاقٌ. كُرْسِيٌّ، كَرَاسِيُّ.

\_\_\_\_\_\_\_\_\_\_\_\_\_\_\_\_\_\_\_\_\_\_\_ 

أَيْنَ مَقْعَدُ التِّلْمِيذِ؟ - هُنَاكَ فِي الْفَصْلِ. مَا هَذَا؟ - كُرْسِيٌّ. لِمَنْ هُوَ؟ - لِلْمُعَلِّمِ. خُذِ الْقَلَمَ. هَاتِ الْكِتَابَ. عَبْدُ اللهِ يَقْرَأُ الدَّرْسَ. سَلِيمٌ يَقْرَأُ الْكِتَابَ وَ مَحْمُودٌ يَكْتُبُ الدَّرْسَ. مَا هَذَا؟ - مَقْعَدُ التِّلْمِيذِ. أَيْنَ رَشِيدٌ؟ - هُنَاكَ فِي الْحُجْرَةِ يَقْرَأُ الدَّرْسَ. أَيْنَ عَبْدُ اللهِ؟ - فِي الْفَصْلِ يَكْتُبُ. يَا فَاطِمَةُ، خُذِي هَذِهِ الأَوْرَاقَ. يَا زَيْنَبُ، هَاتِي تِلْكَ الأَقْلاَمَ. لِمَنْ هَذَا الْكُرْسِيُّ؟ - لِلْمُعَلِّمِ.

\subsubsection{УРОК 10}
Где парта? — Там, в классе. Что это? — Стул. Чей он? — Учителя. На (бери) карандаш. Дай книгу. Абдулла читает урок. Салим читает книгу, а Махмуд пишет (делает) урок. Что это? — Парта. Где Рашид? — Там, в комнате, читает урок. Где Абдулла? — В классе, пишет. Фатима, бери эти бумаги. Зайнаб, дай те карандаши. Чей это стул? — Учителя.

\subsection{11 اَلدَّرْسُ الحَادِي عَشَرَ}
 \includegraphics[width=3.1252in,height=1.3752in]{images/MuhammadBagauddinprettified-img017.jpg}   \includegraphics[width=1.0209in,height=1.3646in]{images/MuhammadBagauddinprettified-img018.jpg} 

\ مَدِينَةٌ، مُدُنٌ. مَكْتَبٌ، مَكَاتِبُ.

كَاتِبٌ، كُتَّابٌ. كَاتِبَةٌ (ات). مَكْتَبُ الْمُعَلِّمِِ. عَلَى...

عَلَى مَقْعَدِ التِّلْمِيذِ. عَلَيْهِ. بَلْ.

\_\_\_\_\_\_\_\_\_\_\_\_\_\_\_\_\_\_\_\_\_\_\_ 

أَيْنَ الْكِتَابُ؟ - اَلْكِتَابُ عَلَى مَكْتَبِ الْمُعَلِّمِ. أَيْنَ الْمِحْفَظَةُ؟ - اَلْمِحْفَظَةُ فِي مَقْعَدِ التِّلْمِيذِ. مَا هَذِهِ؟ - هَذِهِ مَدِينَةٌ. هَلْ هَذِهِ الْمَدِينَةُ كَبِيرَةٌ؟ - نَعَمْ، هَذِهِ الْمَدِينَةُ كَبِيرَةٌ وَ جَمِيلَةٌ. مَنْ أَنْتَ؟ - أَنَا كَاتِبٌ. هَلْ أَنْتَ كَاتِبٌ عَرَبِيٌّ؟ - نَعَمْ، أَنَا كَاتِبٌ عَرَبِيٌّ. مَنْ هِيَ؟ - هِيَ كَاتِبَةٌ. هَلْ هِيَ كَاتِبَةٌ عَرَبِيَّةٌ؟ - لاَ، بَلْ هِيَ كَاتِبَةٌ رُوسِيَّةٌ وَ لَيْسَتْ بِعَرَبِيَّةٍ. يَا تِلْمِيذَةُ خُذِي هَذَا الْوَرَقَ وَ اكْتُبِي عَلَيْهِ الدُّرُوسَ.

\subsubsection{УРОК 11}
Где книга? - Книга на (письменном) столе учителя. Где порт­фель? — Портфель в парте. Что это? — Это город. Этот город большой? — Да, этот город большой и красивый. Кто ты? — Я писатель. Ты арабский писатель? — Да, я арабский писатель, Кто она? — Она писательница. Она арабская писательница? — Нет, наоборот, она русская писательница, а не арабская. Девочка (ученица), возьми этот лист бумаги и делай на нём уроки.

\subsection[12 اَلدَّرْسُ الثَّانِي عَشَرَ]{12 اَلدَّرْسُ الثَّانِي عَشَرَ}
\  \includegraphics[width=0.7398in,height=1.0209in]{images/MuhammadBagauddinprettified-img019.jpg}   \includegraphics[width=1.5102in,height=1.0311in]{images/MuhammadBagauddinprettified-img020.jpg}   \includegraphics[width=0.5835in,height=1.0311in]{images/MuhammadBagauddinprettified-img021.jpg} 

وَلَدٌ، أَوْلاَدٌ. خُبْزٌ، أَخْبَازٌ. كُوبٌ، أَكْوَابٌ. لَبَنٌ، أَلْبَانٌ.

\ مَاءٌ، مِيَاهٌ. بَارِدٌ. سَخِينٌ. لَذِيذٌ. كُلْ. اِشْرَبْ. فِيهِ. يَا أَيُّهَا الْوَلَدُ. اِقْرَأْهُ. أُكْتُبْهُ.

\_\_\_\_\_\_\_\_\_\_\_\_\_\_\_\_\_\_\_\_\_\_\_\_ 

مَا هَذَا؟ - هَذَا كُوبٌ. مَا فِيهِ؟ - فِيهِ لَبَنٌ. هَلْ هُوَ بَارِدٌ؟ - لاَ، هُوَ لَيْسَ بِبَارِدٍ. هَلْ هُوَ لَذِيذٌ؟ - نَعَمْ، هُوَ لَذِيذٌ. يَا أَيُّهَا الْوَلَدُ، كُلِ الْخُبْزَ وَ اشْرَبِ اللَّبَنَ. هَذَا الْخُبْزُ سَخِينٌ وَ ذَاكَ اللَّبَنُ بَارِدٌ. مَا فِي هَذَا الْكُوبِ؟ - فِيهِ مَاءٌ. مَنْ فِي ذَاكَ الْبَيْتِ؟ - فِيهِ وَلَدٌ. لِمَنْ هُوَ؟ - هُوَ لِي. أَيْنَ الْمَاءُ؟ - اَلْمَاءُ فِي الْكُوبِ. وَ أَيْنَ الْكُوبُ؟ - اَلْكُوبُ هُنَاكَ فِي الْحُجْرَةِ. هَاتِ الْمَاءَ يَا وَلَدُ.

\subsubsection{УРОК 12}
Что это? — Это стакан. Что в нём? — В нём молоко. Оно холод­ное? — Нет, оно не холодное. Оно вкусное? — Да, оно вкусное. Мальчик, ешь хлеб и пей молоко. Этот хлеб горячий, а то молоко холодное. Что в этом стакане? — Там вода. Кто в том доме? — там (один) мальчик. Чей он? — Он мой. Где вода? — Вода в стакане. А где стакан? — Стакан там, в комнате. Дай воды, мальчик.

\subsection{13 اَلدَّرْسُ الثَّالِثَ عَشَرَ}
 \includegraphics[width=1.4898in,height=1.4583in]{images/MuhammadBagauddinprettified-img022.jpg}   \includegraphics[width=1.3228in,height=1.448in]{images/MuhammadBagauddinprettified-img023.jpg}   \includegraphics[width=1.7083in,height=1.4689in]{images/MuhammadBagauddinprettified-img024.jpg}   \includegraphics[width=1.5874in,height=1.2508in]{images/MuhammadBagauddinprettified-img025.jpg} 

بَابٌ، أَبْوَابٌ. شُبَّاكٌ، شَبَابِيكُ. طَاوِلَةٌ (ات). مِقْلَمَةٌ، مَقَالِمُ.

بَابُ الْبَيْتِ. شُبَّاكُ الْحُجْرَةِ. مَفْتُوحٌ. مُغْلَقٌ. بَابٌ مَفْتُوحٌ.

تَعَالَ. تَعَالَيْ.

\_\_\_\_\_\_\_\_\_\_\_\_\_\_\_\_\_\_\_\_\_\_\_\_

أَيْنَ الأَقْلاَمُ؟ - فِي الْمِقْلَمَةِ. وَ أَيْنَ الْمِقْلَمَةُ؟ - اَلْمِقْلَمَةُ فِي الْمِحْفَظَةِ، وَ الْمِحْفَظَةُ عَلَى الطَّاوِلَةِ. يَا تِلْمِيذُ، تَعَالَ هُنَا خُذِ الْكِتَابَ وَ اقْرَأِ الدَّرْسَ. يَا تِلْمِيذَةُ، تَعَالَيْ هُنَا وَ اقْرَئِي هَذِهِ الْجَرِيدَةَ الْعَرَبِيَّةَ. هَذَا بَابٌ وَ هَذَا شُبَّاكٌ. اَلْبَابُ مَفْتُوحٌ وَ الشُّبَّاكُ مُغْلَقٌ. هَذَا بَابُ الْحُجْرَةِ وَ هَذَا شُبَّاكُ الْحُجْرَةِ. أَيْنَ شُبَّاكُ الْحُجْرَةِ؟ - شُبَّاكُ الْحُجْرَةِ هُنَاكَ. اَلْكُتُبُ وَ الْمَجَلاَّتُ وَ الْجَرَائِدُ عَلَى الطَّاوِلَةِ. أُدْخُلِ الْبَابَ. أُخْرُجْ مِنَ البَابِ.

\subsubsection{УРОК 13}
Где карандаши? — В пенале. А где пенал? — Пенал в портфеле, а портфель на столе. Мальчик (ученик), иди сюда, бери книгу и читай урок. Девочка (ученица), иди сюда и читай эту арабскую газету. Это дверь, а это окно. Дверь открыта, а окно закрыто. Это дверь комнаты, а это окно комнаты. Где окно комнаты? — Окно комнаты там. Книги, журналы и газеты на столе. Войди в дверь. Выйди из двери.

\subsection{14 اَلدَّرْسُ الرَّابِعَ عَشَرَ}
\  \includegraphics[width=1.3437in,height=1.3854in]{images/MuhammadBagauddinprettified-img026.jpg}   \includegraphics[width=1.7811in,height=1.3854in]{images/MuhammadBagauddinprettified-img027.jpg}   \includegraphics[width=0.802in,height=1.3543in]{images/MuhammadBagauddinprettified-img028.jpg} 

فِنْجَانٌ، فَنَاجِينُ. سُكَّرٌ. سُكَّرِيَّةٌ (ات). قَهْوَةٌ. شَايٌ. بِالسُّكَّرِ. مَاذَا؟ مَاذَا تَفْعَلُ؟ لاَ يَفْعَلُ. هَلْ تَفْعَلُ؟

\begin{flushleft}
\tablefirsthead{}
\tablehead{}
\tabletail{}
\tablelasttail{}
\begin{supertabular}{|m{1.1712599in}|m{1.2962599in}|m{1.2962599in}|m{1.4212599in}|m{1.0670599in}|}
\hline
هُوَ يَفْعَلُ &
هِيَ تَفْعَلُ &
أَنْتَ تَفْعَلُ &
أَنْتِ تَفْعَلِينَ &
أَنَا أَفْعَلُ\\\hline
\end{supertabular}
\end{flushleft}
\_\_\_\_\_\_\_\_\_\_\_\_\_\_\_\_\_\_\_\_\_\_

لِي فِنْجَانٌ كَبِيرٌ. لَكَ فِنْجَانٌ صَغِيرٌ. لِمَنْ ذَاكَ الْفِنْجَانُ؟ - لِهَذَا الْوَلَدِ. هَلْ تَشْرَبُ الشَّايَ بِالسُّكَّرِ؟ - نَعَمْ، أَشْرَبُ. هَلْ تَشْرَبُ الْقَهْوَةَ بِاللَّبَنِ؟ - لاَ، لاَ أَشْرَبُ الْقَهْوَةَ بِاللَّبَنِ. يَا فَاطِمَةُ أَيْنَ أَنْتِ؟ - أَنَا هُنَا فِي الْحُجْرَةِ. مَاذَا تَفْعَلِينَ فِيهَا؟ - أَقْرَأُ كِتَابًا عَرَبِيًّا. أَيْنَ السُّكَّرُ؟ - فِي السُّكَّرِيَةِ عَلَى الطَّاوِلَةِ. سَعِيدٌ يَشْرَبُ الشَّايَ. عُثْمَانُ يَأْكُلُ الْخُبْزَ. مَنْ يَقْرَأُ فِي الْبَيْتِ؟ - فَاطِمَةُ تَقْرَأُ.

\subsubsection{УРОК 14}
У меня большая чашка. У тебя маленькая чашка. Чья та чашка? — Этого мальчика. Ты будешь пить чай с сахаром? — Да, буду пить. Ты будешь пить кофе с молоком? — Нет, я не буду пить кофе с молоком. Фатима, где ты? — Я здесь, в комнате. Что ты там дела­ешь? — Читаю арабскую книгу. Где сахар? — В сахарнице, на столе. Саид пьёт чай. Осман ест хлеб. Кто читает дома? — Фатима читает.

\subsection{15 اَلدَّرْسُ الْخَامِسَ عَشَرَ}
أَبٌ، آبَاءٌ. أُمٌّ، أُمَّهَاتٌ. أَخٌ، إِخْوَةٌ. أُخْتٌ، أَخَوَاتٌ. إِبْنٌ، أَبْنَاءٌ. بِنْتٌ، بَنَاتٌ. أَبُوهُ. أَبُوهَا. أَبُوكَ. أَبُوكِ. أَبِي.

\ ...ـهُ ...ـهَا ...ـكَ ...ـكِ ...ـِي

\_\_\_\_\_\_\_\_\_\_\_\_\_\_\_\_\_\_\_\_\_\_\_\_

هُوَ أَبُوكَ. أَنْتَ أَبِي. أَنَا أَخُوهُ. أَنْتِ أُخْتُهُ. أَنَا أُمُّهَا. هَذَا كِتَابُ مُحَمَّدٍ. ذَاكَ قَلَمُ مَحْمُودٍ. خُذْ دَفْتَرَكَ هَذَا. هَاتِ دَفْتَرِي ذَاكَ. أَيْنَ كِتَابِي يَا أُمِّي؟ - هُنَاكَ عَلَى الطَّاوِلَةِ يَا بِنْتِي. أَيْنَ دَفْتَرُكِ يَا بِنْتِي؟ - هُنَا فِي مِحْفَظَتِي يَا أُمِّي. مَنْ أَبُوكَ؟ - أَبِي عَلِيُّ بْنُ مَحْمُودٍ. مَنْ أُمُّكَ؟ - أُمِّي خَدِيجَةُ بِنْتُ سَالِمٍ. أَيْنَ أَخُوكَ؟ - هُوَ فِي الْمَكْتَبَةِ. مَاذَا يَفْعَلُ فِيهَا؟ - يَكْتُبُ دُرُوسَهُ. هَلْ لَّكَ أَوْلاَدٌ؟ - نَعَمْ، لِي أَبْنَاءٌ وَ بَنَاتٌ. أَخُوكَ مُعَلِّمِي. إِبْنِي تِلْمِيذُكَ.

\subsubsection{УРОК 15}
Он твой отец. Ты мой отец. Я его брат. Ты его сестра. Я её мать. Это книга Мухаммеда. Вон (то) карандаш Махмуда. Возьми эту свою (твою) тетрадь. Дай ту мою тетрадь. Где моя книга, мама? — Там не столе, дочка. Где твоя тетрадь, дочка? — Здесь, в моём портфеле, мама. Кто твой отец? — Мой отец Али, сын Махмуда. Кто твоя мать? — Моя мать Хадиджа, дочь Салима. Где твой (ж.р.) брат? — Он в библиотетке. Что он там делает? — Делает (пишет) свои уроки. У тебя есть дети? — Да, у меня есть сыновья и дочери. Твой брат — мой учитель. Мой сын — твой ученик.

\subsection[16 اَلدَّرْسُ السَّادِسَ عَشَرَ]{16 اَلدَّرْسُ السَّادِسَ عَشَرَ}
\  \includegraphics[width=1.948in,height=1.3957in]{images/MuhammadBagauddinprettified-img029.jpg}   \includegraphics[width=0.7291in,height=0.6772in]{images/MuhammadBagauddinprettified-img030.jpg}   \includegraphics[width=0.7602in,height=0.5in]{images/MuhammadBagauddinprettified-img031.jpg} 

دَارٌ، دُورٌ. سَاحَةٌ (ات). مِمْحَاةٌ، مَمَاحٍ. رِيشَةٌ (ات). يَرْكُضُ. يَلْعَبُ. أُرْكُضْ. إِلْعَبْ. فِي سَاحَةِ الدَّارِ. فِي سَاحَةِ الْمَدْرَسَةِ.

\_\_\_\_\_\_\_\_\_\_\_\_\_\_\_\_\_\_\_\_\_\_\_\_\_

\ أَيْنَ الْكَلْبُ؟ - فِي سَاحَةِ الدَّارِ يَرْكُضُ. أَيْنَ التِّلْمِيذُ؟ - هُوَ فِي سَاحَةِ الْمَدْرَسَةِ يَلْعَبُ. لِمَنْ هَذِهِ الْمِمْحَاةُ؟ - هِيَ ِلأَخِي. لِمَنْ تِلْكَ الرِّيشَةُ؟ - هِيَ ِلأُخْتِكَ. اَلتِّلْمِيذُ يَكْتُبُ دُرُوسَهُ بِالرِّيشَةِ. مَا فِي مِحْفَظَتِكَ؟ - فِيهَا كُتُبٌ وَ دَفَاتِرُ وَ أَقْلاَمٌ وَ أَوْرَاقٌ وَ مِمْحَاةٌ. مَنْ يَلْعَبُ فِي الْبَيْتِ؟ - اِبْنُكَ الصَّغِيرُ. مَنْ يَرْكُضُ فِي سَاحَةِ الدَّارِ؟ - بِنْتِي الْكَبِيرَةُ. أُدْخُلِ الْحُجْرَةَ مِنْ هَذَا الْبَابِ. اِشْرَبِ الشَّايَ بِالْكُوبِ وَ أَنَا أَشْرَبُ الْقَهْوَةَ بِالْفِنْجَانِ. يَا وَلَدُ خُذِ الْقَلَمَ وَ اكْتُبْ دُرُوسَكَ.

\subsubsection{УРОК 16}
Где собака? — Бегает во дворе. Где ученик? — Он играет во дворе школы. Чья эта резинка? — Она моего брата. Чьё то перо? — Оно твоей сестры. Ученик пишет свои уроки пером. Что в твоём портфеле? — Там книги, тетради, карандаши, бумаги и резинка. Кто играет в доме? — Твой младший сын. Кто бегает во дворе (дома)? — Моя старшая дочь. Войди в комнату через эту дверь. Пей чай из стакана, а я буду пить кофе из чашки. Мальчик, возьми карандаш и делай (пиши) свои уроки.

\subsection{17 اَلدَّرْسُ السَّابِعَ عَشَرَ}

\bigskip

\begin{center}
\tablefirsthead{}
\tablehead{}
\tabletail{}
\tablelasttail{}
\begin{supertabular}{|m{0.9212598in}|m{1.1712599in}|m{1.2962599in}|m{1.2962599in}|m{1.0670599in}|}
\hline
هُوَ قَرَأَ &
هِيَ قَرَأَتْ &
أَنْتَ قَرَأْتَ &
أَنْتِ قَرَأْتِ &
أَنَا قَرَأْتُ\\\hline
\end{supertabular}
\end{center}
(كَتَبَ، يَكْتُبُ، كِتَابَةٌ). (أَخَذَ، يَأْخُذُ، أَخْذٌ). (أَكَلَ، يَأْكُلُ، أَكْلٌ). (شَرِبَ، يَشْرَبُ، شُرْبٌ). (دَخَلَ، يَدْخُلُ، دُخُولٌ).

(خَرَجَ، يَخْرُجُ، خُرُوجٌ). هُوَ مَا قَرَأَ. ثُمَّ. بَعْدُ. مَا قَرَأْتُ بَعْدُ. هَلْ قَرَأْتَ؟ مَاذَا فَعَلْتَ؟

\_\_\_\_\_\_\_\_\_\_\_\_\_\_\_\_\_\_\_\_\_

هَلْ قَرَأْتَ هَذَا الْكِتَابَ يَا أَحْمَدُ؟ - نَعَمْ، قَرَأْتُهُ. هَلْ كَتَبْتِ دَرْسَكِ يَا هِنْدُ؟ - لاَ، مَا كَتَبْتُهُ بَعْدُ. مَنْ أَخَذَ دَفْتَرِي؟ - أَخَذَتْهُ أُخْتُكَ الصَّغِيرَةُ مِنْ مِحْفَظَتِكَ. أَخَذَ التِّلْمِيذُ كِتَابَهُ وَ قَرَأَ دَرْسَهُ. مَنْ أَكَلَ خُبْزِي؟ - أَنَا أَكَلْتُ خُبْزَكَ. مَنْ شَرِبَ قَهْوَتِي؟ - أَنَا مَا شَرِبْتُ قَهْوَتَكَ، بَلْ شَرِبَهَا أَخُوكَ. هَلْ خَرَجَ التِّلْمِيذُ مِنَ الْفَصْلِ؟ - لاَ، هُوَ مَا خَرَجَ بَعْدُ. دَخَلَ الْمُعَلِّمُ الْفَصْلَ ثُمَّ التَّلاَمِذَةُ. خَرَجَتِ الأُمُّ مِنَ الْبَيْتِ ثُمَّ بَنَاتُهَا. مَاذَا فَعَلْتَ فِي الْمَدْرَسَةِ؟ - قَرَأْتُ وَ كَتَبْتُ.

\subsubsection{УРОК 17}
Ты читал эту книгу, Ахмед? — Да, я её читал. Ты делала (писала) свой урок, Хинд? — Нет, я не делала его ещё. Кто взял мою те­традь? — Её взяла твоя младшая сестра из твоего портфеля. Ученик взял свою книгу и читал свой урок. Кто съел мой хлеб? — Я съел твой хлеб. Кто выпил мой кофе? — Я не пил твой кофе, (наоборот) его выпил твой брат. Ученик вышел из класса? — Нет, он ещё" не вышел. Учитель вошёл в класс, потом ученики. Мать вышла из дома, потом её дочери. Что ты делал в школе? — Читал и писал.

\subsection{18 اَلدَّرْسُ الثَّامِنَ عَشَرَ}
 \includegraphics[width=1.0417in,height=1in]{images/MuhammadBagauddinprettified-img032.jpg}   \includegraphics[width=0.6563in,height=0.6772in]{images/MuhammadBagauddinprettified-img033.jpg} 

\ مَقْعَدٌ، مَقَاعِدُ. كَلِمَةٌ (ات). قَاعَةٌ (ات). أَيْضًا. بَعْضُ. أَمْسِ. اَلْيَوْمَ. غَدًا. أَقَرَأْتَ؟ إِلَى. إِلَى الْبَيْتِ.

إِنْ شَاءَ اللهُ. (فَهِمَ، يَفْهَمُ، فَهْمٌ). (جَلَسَ، يَجْلِسُ، جُلُوسٌ). (ذَهَبَ، يَذْهَبُ، ذَهَابٌ).

\_\_\_\_\_\_\_\_\_\_\_\_\_\_\_\_\_\_\_\_\_\_\_\_

\ هَلْ قَرَأْتَ أَمْسِ دَرْسَكَ؟ - نَعَمْ، قَرَأْتُ دَرْسِي أَمْسِ. أَيْنَ قَرَأْتَهُ؟ - قَرَأْتُهُ فِي سَاحَةِ دَارِي. هَلْ فَهِمْتَ الْكَلِمَاتِ فِي هَذِا الدَّرْسِ؟ - فَهِمْتُ بَعْضَ الْكَلِمَاتِ وَ مَا فَهِمْتُ بَعْضَهَا. أَيْنَ جَلَسَ الْمُعَلِّمُ؟ - جَلَسَ الْمُعَلِّمُ عَلَى الْمَقْعَدِ فِي الْقَاعَةِ. أَذَهَبْتَ إَلَى الْمَكْتَبَةِ الْيَوْمَ؟ - نَعَمْ، ذَهَبْتُ. وَ مَاذَا فَعَلَتَ فِيهَا؟ - قَرَأْتُ فِيهَا جَرَائِدَ وَ مَجَلاَّتٍ. وَ هَلْ أَخَذْتَ مِنْهَا الْكُتُبَ إِلَى الْبَيْتِ؟ - نَعَمْ، أَخَذْتُ بَعْضَ الْكُتُبِ. هَلْ تَذْهَبُ إِلَيْهَا غَدًا؟ - نَعَمْ، أَذْهَبُ إِلَيْهَا غَدًا إِنْ شَاءَ اللهُ.

\subsubsection{УРОК 18}
Ты вчера читал свой урок? — Да, я читал вчера свой урок. Где ты читал его? — Я читал его во дворе своего дома. Ты понял слова в этом уроке? — Я понял некоторые слова, а некоторые не понял. Где сидел учитель? — Учитель сидел на скамейке в зале. Ты сегодня ходил в библиотеку? — Да, ходил. А что ты там делал? — Я там читал газеты и журналы. А ты взял оттуда книги домой? — Да, я взял некоторые книги. Ты пойдёшь туда завтра? — Да, я пойду туда завтра тоже, если Богу будет угодно.

\subsection{19 اَلدَّرْسُ التَّاسِعَ عَشَرَ}
\  \includegraphics[width=0.9583in,height=0.9583in]{images/MuhammadBagauddinprettified-img034.jpg} 

كُرَةٌ (ات). لَحْمٌ، لُحُومٌ. زَميِلٌ، زُمَلاَءُ. مَعَ. قَلِيلٌ مِنْ... قَلِيلٌ مِنَ الْخُبْزِ. لاَ تَلْعَبْ. لاَ تَلْعَبِي.

\_\_\_\_\_\_\_\_\_\_\_\_\_\_\_\_\_\_\_\_\_\_\_\_

هَذَا أَخِي. هَؤُلاَءِ إِخْوَتِي. تِلْكَ أُخْتُكَ. أُولَئِكَ أَخَوَاتُكَ. إِبْنِيَ الْكَبِيرُ يَلْعَبُ مَعَ أَخِيهِ بِالْكُرَةِ. إِبْنُكَ الصَّغِيرُ يَجْلِسُ فِي الْحُجْرَةِ عَلَى الْمَقْعَدِ وَ يَشْرَبُ الشَّايَ بِالسُّكَّرِ. بِنْتُ وَلِيٍّ تِلْمِيذَةٌ. بَنَاتُ وَلِيٍّ تِلْمِيذَاتٌ. دَخَلَ التِّلْمِيذُ الْحُجْرَةَ وَ أَخَذَ كِتَابَهُ ثُمَّ خَرَجَ وَ ذَهَبَ إِلَى زَمِيلِهِ. أَبُوكَ دَخَلَ الْقَاعَةَ. أُمُّكَ خَرَجَتْ مِنْ سَاحَةِ الدَّارِ. أُخْتِي كَتَبَتِ الدَّرْسَ. أَخُوهُ شَرِبَ اللَّبَنَ. لاَ تَلْعَبْ هُنَا بَلِ اخْرُجْ مِنْ هُنَا إِلَى السَّاحَةِ وَ الْعَبْ فِيهَا. أَكَلْتُ قَلِيلاً مِنَ الْخُبْزِ وَ شَرِبْتُ مَعَهُ قَلِيلاً مِنَ اللَّبَنِ. يَا وَلَدُ، إِلْعَبْ بِالْكُرَةِ مَعَ زُمَلاَئِكَ. لاَ تَشْرَبْ مَاءً بَارِدًا وَ لاَ تَأْكُلْ خُبْزًا سَخِينًا.

\subsubsection{УРОК 19}
Это мой брат. Это мои братья. Та (девочка) — твоя сестра. Те (девушки) твои сестры. Мой старший сын играет со своим братом в мяч. Твой младший сын сидит в комнате на скамейке и пьет чай с сахаром. Дочь Вали — ученица. Дочери Вали — ученицы. Ученик вошёл в комнату и взял свою книгу, а потом вышел и пошел к своему товарищу. Твой отец вошёл в зал. Твоя мать вышла со двора. Моя сестра написала урок. Его брат выпил молоко. Не играй здесь, а выходи отсюда во двор и играй там. Я поел немного хлеба и выпил с ним немного молока. Мальчик, играй в мяч со своими товарищами. Не пей холодной воды и не ешь горячего хлеба.

\subsection{20 اَلدَّرْسُ الْعِشْرُونَ}
أَيْنَ كُنْتَ؟ كُنْتُ. مَتَى؟ صَبَاحًا. مَسَاءً. الْآنَ. قَبْلَ. بَعْدَ. (قَامَ، يَقُومُ، قِيَامٌ). قَامَ فَقَرَأَ دَرْسَهُ. قُمْ. لاَ تَقُمْ. قُمْ وَ اقْرَأِ الدَّرْسَ.

\_\_\_\_\_\_\_\_\_\_\_\_\_\_\_\_\_\_\_\_\_

أَيْنَ كُنْتَ؟ - كُنْتُ فِي الْمَدْرَسَةِ. مَتَى ذَهَبْتَ إِلَى الْمَدْرَسَةِ؟ - ذَهَبْتُ إِلَيْهَا الْيَوْمَ صَبَاحًا. هَلْ كُنْتَ فِي الْمَكْتَبَةِ الْيَوْمَ؟ - لاَ، اَلْيَوْمَ مَا كُنْتُ فِي الْمَكْتَبَةِ. وَ مَتَى تَذْهَبُ إِلَى الْمَكْتَبَةِ؟ - أَذْهَبُ إِلَيْهَا مَسَاءً إِنْ شَاءَ اللَّهُ. اَلآنَ مَاذَا تَفْعَلُ؟ - أَقْرَأُ دُرُوسِي. هَلْ فَهِمْتَ هَذِهِ الْكَلِمَةَ مِنْ دَرْسِكَ؟ - نَعَمْ، فَهِمْتُهَا. هَلْ تَشْرَبُ الشَّايَ قَبْلَ الدَّرْسِ؟ - نَعَمْ، أَشْرَبُ قَلِيلاً مِنَ الشَّايِ. هَلْ تَلْعَبُ قَبْلَ الدَّرْسِ؟ - لاَ ، أَنَا لاَ أَلْعَبُ قَبْلَ الدَّرْسِ بَلْ بَعْدَهُ. قَامَ التِّلْمِيذُ فَأَخَذَ كِتَابَهُ وَ قَرَأَ دَرْسَهُ. اَلآنَ قُمْ وَ اذْهَبْ إِلَى بَيْتِكَ.

\subsubsection{УРОК 20}
Где ты был? — Я был в школе. Когда ты пошёл в школу? — Я пошёл туда сегодня утром. Ты был в библиотеке сегодня? — Нет, сегодня я не был в библиотеке. А когда ты пойдёшь в библиоте­ку? \_ я пойду туда вечером, если Богу будет угодно. Сейчас что ты делаешь? — Читаю свои уроки. Ты понял это слово из своего урока? — Да, я понял его. Ты будешь пить чай перед уроком? — Да, я выпью немного чая. Ты будешь играть перед уроком? — Нет, я не буду играть перед уроком, а после него. Ученик встал, потом взял свою книгу и прочитал свой урок. Теперь встань и иди к себе домой.

\subsection{21 اَلدَّرْسُ الْحَادِي وَ الْعِشْرُونَ}
 \includegraphics[width=1.75in,height=0.802in]{images/MuhammadBagauddinprettified-img035.jpg} 

\ سَيَّارَةٌ (ات). مُهَنْدِسٌ (ون). مُدَرِّسٌ (ون). مَاهِرٌ، مَهَرَةٌ. جَدِيدٌ، جُدُدٌ. دَافِئٌ. رَكِبَ السَّيَّارَةَ. قَادَ السَّيَّارَةَ. عِنْدَ. لَسْتَ. لَسْتِ. لَسْتُ. (رَكِبَ، يَرْكَبُ، رُكُوبٌ). 

(قَادَ، يَقُودُ، قِيَادَةٌ). (رَجَعَ، يَرْجِعُ، رُجُوعٌ).

\_\_\_\_\_\_\_\_\_\_\_\_\_\_\_\_\_\_\_\_\_\_\_\_

هَلْ أَنْتَ مُهَنْدِسٌ؟ - لاَ، لَسْتُ بِمُهَنْدِسٍ. هَلْ هِيَ مُهَنْدِسَةٌ؟ - نَعَمْ، هِيَ مُهَنْدِسَةٌ مَاهِرَةٌ. هِيَ اِمْرَأَةٌ جَمِيلَةٌ. مَنْ أَنْتِ؟ - أَنَا مُدَرِّسَةٌ. لِي تِلْمِيذٌ جَدِيدٌ . مَنْ هَذَا؟ - مُهَنْدِسٌ عَرَبِيٌّ. لَهُ سَيَّارَةٌ جَدِيدَةٌ. رَكِبْتُ السَّيَّارَةَ مَعَ أَبِي صَبَاحًا فَذَهَبْتُ إِلَى الْمَدْرَسَةِ وَ بَعْدَ الدُّرُوسِ رَجَعْتُ مِنْهَا إِلَى الْبَيْتِ مَعَ زُمَلاَئِي. هَلْ تَرْكَبُ السَّيَّارَةَ؟ - نَعَمْ، أَرْكَبُ. أَتَقُودُ السَّيَّارَةَ؟ - لاَ، أَنَا لاَ أَقُودُ السَّيَّارَةَ. وَ هَلْ يَقُودُهَا أَخُوكَ؟ - نَعَمْ، هُوَ يَقُودُهَا وَ أَنَا أَرْكَبُ مَعَهُ. هَلْ عِنْدَكَ مَاءٌ؟ - نَعَمْ، عِنْدِي مَاءٌ. هَلْ هُوَ بَارِدٌ؟ - لاَ، بَلْ دَافِئٌ.

\subsubsection{УРОК 21}
Ты инженер? — Нет, я не инженер, а преподаватель. Она инже­нер? — Да, она способный инженер. Она красивая женщина. Кто ты (ж.р.)? — Я преподавательница. У меня новый ученик. Кто это? — Арабский инженер. У него новая красивая машина, Я сел в машину со своим отцом утром и поехал в школу, а после уроков я оттуда вернулся домой со своими товарищами. Ты сядешь в машину? — Да, сяду. Ты водишь машину? — Нет, я не вожу машину. А твой брат водит её? — Да, он водит её, и я сажусь с ним. У тебя есть во­да? Да, у меня есть вода. Она холодная? — Нет, наоборот, тёплая.‏

\subsection{22 اَلدَّرْسُ الثَّانِي وَ الْعِشْرُونَ}
 \includegraphics[width=2.0835in,height=1.8752in]{images/MuhammadBagauddinprettified-img036.jpg}   \includegraphics[width=2.2917in,height=1.7291in]{images/MuhammadBagauddinprettified-img037.jpg}   \includegraphics[width=1.448in,height=1.5626in]{images/MuhammadBagauddinprettified-img038.jpg} 

\ بُسْتَانٌ، بَسَاتِينُ. قَرْيَةٌ، قُرًى. شَجَرٌ، أَشْجَارٌ. 

 \includegraphics[width=1.75in,height=1.1457in]{images/MuhammadBagauddinprettified-img039.jpg}  

دِيوَانٌ، دَوَاوِينُ. 

نَبَاتٌ (ات). نَافِذَةٌ، نَوَافِذُ. مَنْزِلٌ، مَنَازِلُ. نَادِرٌ. مُخْتَلِفٌ. كَثِيرٌ. وَثِيرٌ. قَلِيلاً. جِدًّا. قُرْبَ... ذَلِكَ. لاَ أُرِيدُ. تَعَالَ هُنَا نَقْرَإِ الدَّرْسَ.

\_\_\_\_\_\_\_\_\_\_\_\_\_\_\_\_\_\_\_\_\_\_

لِمَنْ هَذَا الْبُسْتَانُ؟ - هَذَا بُسْتَانِي. فِي هَذَا الْبُسْتَانِ نَبَاتَاتٌ نَاذِرَةٌ وَ أَشْجَارٌ مُخْتَلِفَةٌ كَثِيرَةٌ. مَا هَذَا الْبَيْتُ؟ - هَذَا الْبَيْتُ الصَّغِيرُ مَنْزِلٌ، وَ ذَلِكَ الْبَيْتُ الْكَبِيرُ مَدْرَسَةٌ وَ قُرْبَ الْمَدْرَسَةِ مَكْتَبَةٌ فِيهَا كُتُبٌ نَادِرَةٌ مُخْتَلِفَةٌ. مَاذَا تُرِيدُ؟ - أُرِيدُ مَاءً. هَلْ فِي حُجْرَتِكَ دِيوَانٌ؟ - نَعَمْ، فِي حُجْرَتِي دِيوَانٌ وَ هُوَ وَثِيرٌ جِدًّا. جَلَسْتُ عَلَى الدِّيوَانِ الْوَثِيرِ. هَذِهِ الْقَرْيَةُ قُرْبَ الْمَدِينَةِ. بَيْتِي قُرْبَ الْمَدْرَسَةِ. اَلدِّيوَانُ قُرْبَ الْبَابِ. اَلطَّاوِلَةُ وَ الْكُرْسِيُّ قُرْبَ النَّافِذَةِ. مَنْ يَجْلِسُ عَلَى الدِّيوَانِ؟ - أَخُوكَ. إِجْلِسْ مَعِي قَلِيلاً نَقْرَأُ الدَّرْسَ وَ نَشْرَبُ الْقَهْوَةَ ثُمَّ اذْهَبْ إِلَى الْبَيْتِ. إِقْرَأْ قَلِيلاً ثُمَّ اخْرُجْ وَ الْعَبْ.

\subsubsection{Урок 22}
Чей это сад? — Это мой сад. В этом саду редкие растения и много разных деревьев. Что это за дом? — Этот маленький дом — жилище, а тот большой дом — школа, и около школы — библиотека; в ней разные редкие книги. Что ты хочешь? — Я хочу воды. В твоей комнате есть диван? — Да, в моей комнате есть диван, и он очень мягкий. Я сел на мягкий диван. Это селение близ города. Мой дом около школы. Диван около двери. Стол и стул около окна. Кто сидит на диване? — Твой брат. Посиди со мной немного, почитаем урок и выпьем кофе, потом иди домой. Почитай немного, потом выходи и играй.

\subsection{23 اَلدَّرْسُ الثَّالِثُ وَ الْعِشْرُونَ}
\  \includegraphics[width=1.3646in,height=1.2811in]{images/MuhammadBagauddinprettified-img040.jpg} 

مَسْجِدٌ، مَسَاجِدُ. شِقَّةٌ، شِقَقٌ. أَلَيْسَ؟ عَنْ. قَرِيبٌ. بَعِيدٌ. قَدِيمٌ. ...ـهُمْ. ...ـهُنَّ. ...ـكُمْ. ...ـكُنَّ. ...ـنَا. بَيْتُهُمْ. بَيْتُهُنَّ. بَيْتُكُمْ. بَيْتُكُنَّ. بَيْتُنَا. سَكَنَ (و) سَكَنٌ. نَظَرَ (و) نَظَرٌ. رَحَلَ (و) َرحِيلٌ.

\begin{center}
\tablefirsthead{}
\tablehead{}
\tabletail{}
\tablelasttail{}
\begin{supertabular}{|m{1.1712599in}|m{1.1712599in}|m{1.2962599in}|m{1.4212599in}|m{1.3170599in}|}
\hline
هُمْ سَكَنُوا &
هُنَّ سَكَنَّ &
أَنْتُمْ سَكَنْتُمْ &
أَنْتُنَّ سَكَنْتُنَّ &
نَحْنُ سَكَنَا\\\hline
\end{supertabular}
\end{center}
\_\_\_\_\_\_\_\_\_\_\_\_\_\_\_\_\_\_\_\_\_\_\_

بَيْتُهُمْ قَرِيبٌ مِنَ الْمَسْجِدِ. بَيْتُنَا بَعِيدٌ عَنِ الْمَسْجِدِ. أَيْنَ تَسْكُنُ؟ - أَسْكُنُ فِي بَيْتٍ قَدِيمٍ قُرْبَ الْمَسْجِدِ. أَبِي يَسْكُنُ فِي شِقَّةٍ جَدِيدَةٍ. هَؤُلاَءِ الرِّجَالُ سَكَنُوا فِي الْمَدِينَةِ. آبَاؤُنَا سَكَنُوا فِي الْقَرْيَةِ وَ أُمَّهَاتُنَا سَكَنَّ اَيْضًا فِي الْقَرْيَةِ. أَيْنَ أَبُوكُمْ اَلآنَ؟ - هُوَ الآنَ فِي الْمَسْجِدِ. إِلَى أَيْنَ تَنْظُرُ مِنَ النَّافِذَةِ؟ - أَنْظُرُ إِلَى بُسْتَانِنَا. هَؤُلاَءِ التِّلْمِيذَاتُ خَرَجْنَ مِنَ الْمَدْرَسَةِ فَرَكِبْنَ السَّيَّارَةَ وَ رَجَعْنَ إِلَى بُيُوتِهُنَّ. مَسْجِدُ الْقَرْيَةِ قَدِيمٌ وَ مَسْجِدُ الْمَدِينَةِ جَدِيدٌ. أَيْنَ تَسْكُنُ الآنَ؟ - أَنَا أَسْكُنُ فِي شِقَّةِ أَخِي. أَلَيْسَ الْبَابُ مَفْتُوحًا؟ أَلَيْسَتِ النَّافِذَةُ مُقْفَلَةً؟ أَلَسْتَ مُدَرِّسًا؟ أَلَيْسَ الْمَسْجِدُ قَرِيبًا؟ أَخِي الْكَبِيرُ سَكَنَ فِي الْمَدِينَةِ ثُمَّ رَحَلَ مِنْهَا إِلَى الْقَرْيَةِ وَ هُوَ الآنَ فِي الْقَرْيَةِ وَ بَيْتُهُ قُرْبَ الْمَسْجِدِ.

\subsubsection[К уроку 3]{К уроку 3}
Переведите и запишите:

Этот мужчина невысокий. Эти мужчины невысокие. Эта женщина высокая. Эти женщины высокие. Это мужчина, он высокий. Это мужчины, они высокие. Это женщина, она невысокая. Это женщины, они невысокие.

\subsection{24 اَلدَّرْسُ الرَّابِعُ وَ الْعِشْرُونَ}
 \includegraphics[width=1.0311in,height=0.7811in]{images/MuhammadBagauddinprettified-img041.jpg}   \includegraphics[width=1.5in,height=0.8126in]{images/MuhammadBagauddinprettified-img042.jpg}   \includegraphics[width=0.4791in,height=1.1252in]{images/MuhammadBagauddinprettified-img043.jpg}   \includegraphics[width=0.448in,height=1.0728in]{images/MuhammadBagauddinprettified-img044.jpg} 

هِرٌّ، هِرَرَةٌ. فَأْرٌ، فِئْرَانٌ. شَيْخٌ، شُيُوخٌ. عَجُوزٌ، عَجَائِزُ. فَتًى، فِتْيَانٌ. فَتَاةٌ، فَتَيَاتٌ. شَابٌّ، شُبَّانٌ. أُسْتَاذٌ، أَسَاتِيذُ. فَلاَّحٌ (ون). فلاَّحَةٌ (ات). غَنِيٌّ، أَغْنِيَاءُ. فَقِيرٌ، فُقَرَاءُ. مَشْهُورٌ، مَشَاهِيرُ. مُجْتَهِدٌ (ون). رَفٌّ، رُفُوفٌ. مِصْرُ. مِصْرِيٌّ.

\_\_\_\_\_\_\_\_\_\_\_\_\_\_\_\_\_\_\_\_\_\_\_

هَذَا الْفَتَى تِلْمِيذٌ، هُوَ مُجْتَهِدٌ. تِلْكَ الْفَتَاةُ تِلْمِيذَةٌ، هِيَ مُجْتَهِدَةٌ. أَيْنَ الْقَهْوَةُ؟ - عَلَى الرَّفِّ خُذْهَا مِنْهُ. مَنْ ذَاكَ الْفَتَى؟ - هُوَ زَمِيلِي فِي الْمَدْرَسَةِ، يَجْلِسُ مَعِي عَلَى الْمَقْعَدِ. لِي فِي الْمَدْرَسَةِ زُمَلاَءُ كَثِيرُونَ. مُعَلِّمُنَا رَجُلٌ مَشْهُورٌ وَ هُوَ أُسْتَاذٌ. مُعَلِّمُكُمْ رَجُلٌ غَنِيٌّ. مُعَلِّمُهُمْ شَيْخٌ. مُعَلِّمَتُهُنَّ عَجُوزٌ. أُمُّنَا شَابَّةٌ وَ أُمُّكُمْ عَجُوزٌ. مِنْ أَيْنَ هَذَا الرَّجُلُ؟ - هُوَ مِنْ مِصْرَ، هُوَ فَلاَّحٌ مِصْرِيٌّ. مِنْ أَيْنَ تِلْكَ الْعَجُوزُ؟ - هِيَ مِنْ مِصْرَ أَيْضًا. وَ هَلْ هِيَ فَلاَّحَةٌ أَيْضًا؟ - نَعَمْ هِيَ فَلاَّحَةٌ أَيْضًا. عِنْدَنَا هِرٌّ وَ هُوَ جَمِيلٌ جِدًّا، اَلْهِرُّ يَأْكُلُ الْفَأْرَ. فِي بَيْتِكُمْ فِئْرَانٌ كَثِيرَةٌ وَ لَيْسَ عِنْدَكُمْ هِرٌّ.

\subsubsection{К уроку 4}
Переведите и запишите:

Это учитель. Это учителя. Это учительница. Это учительницы. Кто это? — Ученик. Кто это? — Ученица. Он учитель. Они учителя. Она учительница. Они учительницы. Ты ученик. Вы ученики. Ты ученица. Вы ученицы.

\subsection{25 اَلدَّرْسُ الْخَامِسُ وَ الْعِشْرُونَ}
 \includegraphics[width=1.2398in,height=0.9689in]{images/MuhammadBagauddinprettified-img045.jpg}   \includegraphics[width=1.25in,height=0.8752in]{images/MuhammadBagauddinprettified-img046.jpg}   \includegraphics[width=1.3646in,height=1.0728in]{images/MuhammadBagauddinprettified-img047.jpg}   \includegraphics[width=1.2811in,height=1.3228in]{images/MuhammadBagauddinprettified-img048.jpg} 

ثَوْرٌ، ثِيرَانٌ. بَقَرَةٌ (ات). فَرَسٌ، أَفْرَاسٌ. حِمَارٌ، حُمُرٌ. 

 \includegraphics[width=2.1665in,height=1.25in]{images/MuhammadBagauddinprettified-img049.jpg} 

\ حَقْلٌ، حُقُولٌ. بَلَدٌ، بِلاَدٌ. كُلُّهُمْ. كُلُّكُمْ. كُلُّنَا.

أَنْ يَفْعَلَ. لِيَفْعَلَ.

\begin{center}
\tablefirsthead{}
\tablehead{}
\tabletail{}
\tablelasttail{}
\begin{supertabular}{|m{1.2962599in}|m{1.1712599in}|m{1.4212599in}|m{1.2962599in}|m{1.1920599in}|}
\hline
هُمْ يَقْرَأُونَ &
هُنَّ يَقْرَأْنَ &
أَنْتُمْ تَقْرَأُونَ &
أَنْتُنَّ تَقْرَأْنَ &
نَحْنُ نَقْرَأُ\\\hline
\end{supertabular}
\end{center}
\_\_\_\_\_\_\_\_\_\_\_\_\_\_\_\_\_\_\_\_\_

أَوْلاَدِي كُلُّهُمْ يَقْرَأُونَ الْقُرْآنَ. أَبْنَائِي كُلُّهُمْ يَذْهَبُونَ إِلَى الْمَدْرَسَةِ. اَلْفَتَيَاتُ يَذْهَبْنَ إِلَى الْمَكْتَبَةِ لِيَأْخُذْنَ الْكُتُبَ مِنْهَا. مَاذَا تَفْعَلُونَ هُنَا؟ إِلَى أَيْنَ أَنْتُمْ تَنْظُرُونَ؟ مَاذَا تُرِيدُونَ بَعْدَ الدُّرُوسِ؟ - نُرِيدُ أَنْ نَّرْجِعَ إِلَى بُيُوتِنَا فَنَأْكُلَ الْخُبْزَ وَ نَشْرَبَ الشَّايَ ثُمَّ نَلْعَبَ بِالْكُرَةِ بَعْدَ ذَلِكَ. بَلَدِي كَبِيرٌ. بِلاَدُنَا كَبِيرَةٌ. هَلْ لِّبَلَدِكُمْ حُقُولٌ وَ بَسَاتِينُ؟ - نَعَمْ، لِبَلَدِنَا حُقُولٌ وَ بَسَاتِينُ كَثِيرَةٌ. يَا أَيُّهَا الْفَتَيَاتُ أَتَفْهَمْنَ هَذِهِ الْكَلِمَاتِ الْعَرَبِيَّةَ؟ - نَعَمْ، نَحْنُ نَفْهَمُهَا كُلَّهَا. مَتَى تَرْجِعُونَ إِلَى بِلاَدِكُمْ؟ - بَعْدَ قَلِيلٍ، إِنْ شَاءَ اللهُ. رَكِبْنَا الأَفْرَاسَ فَخَرَجْنَا مِنْ قَرْيَتِنَا الْيَوْمَ صَبَاحًا وَ نُرِيدُ أَنْ نَّرْجِعَ إِلَيْهَا مَسَاءً. هَلْ لِّهَذَا الْفَلاَّحِ بَقَرَةٌ؟ - نَعَمْ، لَهُ بَقَرَةٌ وَ ثَوْرٌ وَ فَرَسٌ وَ حِمَارٌ وَ لَهُ دِيَكَةٌ وَ دَجَاجَاتٌ أَيْضًا وَ هُوَ لَيْسَ بِفَقِيرٍ وَ لَهُ حَقْلٌ كَبِيرٌ وَ بُسْتَانٌ أَيْضًا وَ فِي الْبُسْتَانِ أَشْجَارٌ مُخْتَلِفَةٌ وَ نَبَاتَاتٌ نَادِرَةٌ. هُمْ كُلُّهُمْ فَلاَّحُونَ. أَنْتُمْ كُلُّكُمْ شُيُوخٌ. نَحْنُ كُلُّنَا شُبَّانٌ. إِرْكَبِ الْفَرَسَ وَ ارْكَبِ الْحِمَارَ وَ لاَ تَرْكَبِ الثَّوْرَ وَلاَ تَرْكَبِ الْبَقَرَةَ.

\subsubsection{К уроку 5}
Чья эта книга? Чья эта тетрадь? Чей этот карандаш? Чей этот портфель? Что это? — Книга. Чья она? — Ученика. Что это? — Тетрадь. Чья она? — Ученицы. Что это? — Портфель. Чей он? — Учителя. Что это? — Карандаш. Чей он? — Учительницы. Эта книга маленькая. Эти книги маленькие. Эта тетрадь большая. Эти тетради большие. Этот карандаш длинный. Эти карандаши длинные.

\subsection{26 اَلدَّرْسُ السَّادِسُ وَ الْعِشْرُونَ}
 \includegraphics[width=0.8543in,height=1.3437in]{images/MuhammadBagauddinprettified-img050.jpg}   \includegraphics[width=1.3437in,height=1.4374in]{images/MuhammadBagauddinprettified-img051.jpg}   \includegraphics[width=0.7291in,height=0.7602in]{images/MuhammadBagauddinprettified-img052.jpg}   \includegraphics[width=1.4063in,height=1.2189in]{images/MuhammadBagauddinprettified-img053.jpg} 

\ حَلِيبٌ. مَائِدَةٌ، مَوَائِدُ. لَوْحٌ. أَلْوَاحٌ. طَبَاشِيرُ.

\  \includegraphics[width=0.6354in,height=0.9374in]{images/MuhammadBagauddinprettified-img054.jpg}   \includegraphics[width=0.6146in,height=1.0311in]{images/MuhammadBagauddinprettified-img055.jpg} 

\ مِحْبَرَةٌ، مَحَابِرُ. مِلْحٌ، أَمْلاَحٌ. مِمْلَحَةٌ، مَمَالِحُ. مِمْسَحَةٌ. حِبْرٌ، أَحْبَارٌ. فِلْمٌ، أَفْلاَمٌ. حَارٌّ. جَيِّدٌ. يُوجَدُ. أَجَلْ. فَقَطْ. وَاحِدٌ.

\_\_\_\_\_\_\_\_\_\_\_\_\_\_\_\_\_\_\_\_\_\_

هَذَا الْفِلْمُ جَدِيدٌ. إِجْلِسْ عَلَى الدِّيوَانِ وَ انْظُرْ إِلَى الْفِلْمِ. نَظَرْنَا الْيَوْمَ وَ أَمْسِ إِلَى أَفْلاَمٍ عَرَبِيَّةٍ كَثِيرَةٍ. هَلْ نَظَرْتُمْ إِلَى فِلْمِ "الرِّسَالَةِ"؟ - لاَ، مَا نَظَرْنَا إِلَيْهِ بَعْدُ وَ نَحْنُ نُرِيدُ أَنْ نَنْظُرَ إِلَيْهِ. أَيْنَ اللَّوْحُ؟ - اَللَّوْحُ فِي الْفَصْلِ. وَ الطَّبَاشِيرُ عَلَى اللَّوْحِ. وَ هَلْ تُوجَدُ عَلَيْهِ مِمْسَحَةٌ؟ - نَعَمْ، تُوجَدُ عَلَيْهِ مِمْسَحَةٌ أَيْضًا. اَلْحِبْرُ فِي الْمِحْبَرَةِ. اَلْمِحْبَرَةُ فِي مِحْفَظَةِ التِّلْمِيذِ. هَلْ هَذَا الْحِبْرُ جَيِّدٌ؟ - نَعَمْ، هَذَا الْحِبْرُ جَيِّدٌ وَ جَدِيدٌ. يَا أُمِّي، أُرِيدُ أَنْ أَشْرَبَ قَهْوَةً حَارَّةً أَيْنَ تُوجَدُ الْقَهْوَةُ؟ - تُوجَدُ الْقَهْوَةُ عَلَى الْمَائِدَةِ. هَلْ تَشْرَبُ الْقَهْوَةَ بِالْحَلِيبِ؟ - أَجَلْ، أَشْرَبُهَا بِالْحَلِيبِ وَ السُّكَّرِ. اَلْحَلِيبُ فِي الْكُوبِ. اَلْكُوبُ وُ الْمِمْلَحَةُ وَ السُّكَّرِيَّةُ كُلُّهَا عَلَى الْمَائِدَةِ. لاَ،لاَ، اَلْمِمْلَحَةُ عَلَى الرَّفِّ لاَ عَلَى الْمَائِدَةِ. مَاذَا عَلَى الْمَائِدَةِ؟ عَلَيْهَا خُبْزٌ فَقَطْ. هَلْ لَّكَ ثِيرَانٌ كَثِيرَةٌ؟ - لَيْسَ لِى ثِيرَانٌ كَثِيرَةٌ، بَلْ لِّي ثَوْرٌ وَاحِدٌ فَقَطْ.

\subsubsection{К уроку 6}
Это дом, он большой. Это дома, они большие. Это комната, она маленькая. Это комнаты, они маленькие. Кто дома? Кто в комнате? Кто в школе? Кто в классе? Где учитель? Где учителя? Где учительница? Где учительницы? Кто здесь? Что здесь? Кто там? Чей тот дом? Чьи те дома? Чья та комната? Чьи те комнаты? Тот мужчина учитель. Те мужчины учителя. Та женщина учительница. Те женщины учительницы. Где ты? — Я здесь. Где он? — Он там.

\subsection{27 اَلدَّرْسُ السَّابِعُ وَ الْعِشْرُونَ}
\  \includegraphics[width=0.8957in,height=0.9583in]{images/MuhammadBagauddinprettified-img056.jpg} 

وَرْدَةٌ (ات). زَوْجٌ، أَزْوَاجٌ. زَوْجَةٌ (ات).

مَسْرَحٌ، مَسَارِحُ. مَيْدَانٌ، مَيَادِينُ. حَدِيقَةٌ، حَدَائِقُ. أَجْنَبِيٌّ.

مَنْ هُوَ هَذَا الرَّجُلُ؟ كَثِيرٌ مِنْ... كَثِيرٌ مِنَ الْكُتُبِ. أَمَامَ. وَرَاءَ. يَمِينَ. يَسَارَ. فَوْقَ. تَحْتَ.

\_\_\_\_\_\_\_\_\_\_\_\_\_\_\_\_\_\_\_\_\_\_\_\_

مُعَلِّمُنَا زَوْجُ فَاطِمَةَ. مُعَلِّمَتُنَا زَوْجَةُ أَحْمَدَ. هَلْ يُوجَدُ فِي مَدِينَتِكُمْ مَسْرَحٌ؟ - أَجَلْ، يُوجَدُ فِي مَدِينَتِنَا مَسْرَحٌ كَبِيرٌ. اَلْمَسْرَحُ قُرْبَ الْمَيْدَانِ. وَ تُوجَدُ وَرَاءَ الْمَسْرَحِ حَدِيقَةٌ وَ فِي الْحَدِيقَةِ مَقَاعِدُ. فِي الْحَدِيقَةِ كَثِيرٌ مِنَ الأَشْجَارِ وَ النَّبَاتَاتِ. تُوجَدُ فِي الْحَدِيقَةِ الآنَ وَرْدَاتٌ جَمِيلَةٌ. أَيْنَ تَجْلِسُ؟ - أَجْلِسُ يَسَارَ بَكْرٍ وَ يَمِينَ زَيْدٍ. مَنْ هُوَ هَذَا الرَّجُلُ؟ - هُوَ كَاتِبٌ أَجْنَبِيٌّ. مَنْ هِيَ هَذِهِ الْمَرْأَةُ؟ - هِيَ مُهَنْدِسَةٌ أَجْنَبِيَّةٌ. لِمَدْرَسَتِنَا حَدِيقَةٌ وَ سَاحَةٌ. اَلسَّاحَةُ أَمَامَ الْمَدْرَسَةِ وَ الْحَدِيقَةُ وَرَاءَهَا. بَعْدَ الدُّرُوسِ يَخْرُجُ التَّلاَمِيذُ مِنَ الْفُصُولِ فَيَذْهَبُ بَعْضُهُمْ إِلَى الْحَدِيقَةِ وَ يَجْلِسُونَ فِيهَا عَلَى الْمَقَاعِدِ تَحْتَ الأَشْجَارِ وَ يَذْهَبُ بَعْضُهُمْ إِلَى السَّاحَةِ وَ يَلْعَبُونَ فِيهَا بِالْكُرَةِ. يَا أُخْتِي، أَيْنَ الْخُبْزُ وَ الْحَلِيبُ أُرِيدُ أَنْ آكُلَ؟ - اَلْحَلِيبُ عَلَى الرَّفِّ فَوْقَ الْمَائِدَةِ وَ الْخُبْزُ تَحْتَهَا.

\subsubsection{К уроку 7}
У меня газета, она арабская. У тебя журнал, он русский. Чья эта газета? Чьи эти газеты? Чей этот журнал? Чьи эти журналы? Эта газета русская. Эти газеты русские. Этот журнал арабский. Эти журналы арабские. Где книга и тетрадь? — Книга и тетрадь здесь. Где карандаш и портфель? — Карандаш и портфель там. Где газеты и журна­лы? — Газеты и журналы в портфеле. Мужчины и женщины в доме. Учителя и учительницы в школе. Ученики и ученицы в классе.

\subsection[28 اَلدَّرْسُ الثَّامِنُ وَ الْعِشْرُونَ]{28 اَلدَّرْسُ الثَّامِنُ وَ الْعِشْرُونَ}
 \includegraphics[width=1.2709in,height=1.1457in]{images/MuhammadBagauddinprettified-img057.png}   \includegraphics[width=1.2398in,height=0.9063in]{images/MuhammadBagauddinprettified-img058.png}   \includegraphics[width=0.5417in,height=1.2291in]{images/MuhammadBagauddinprettified-img059.png} 

جَامِعَةٌ (ات). شَارَةٌ (ات). بَدْلَةٌ، بَدَلاَتٌ. عَامِلٌ، عُمَّالٌ. عَامِلَةٌ (ات). مَعْمَلٌ، مَعَامِلُ. حَيَاةٌ. عَمَّاذَا؟ مِمَّنْ؟ كَانَ. تَسَلَّمَ. عِنْدَمَا خَرَجَ. عَنْ... رِسَالَةٌ, رَسَائِلُ.

عَرَفَ (ى) مَعْرِفَةٌ. دَرَسَ (و) دِرَاسَةٌ. عَمِلَ (ا) عَمَلٌ. 

\_\_\_\_\_\_\_\_\_\_\_\_\_\_\_\_\_\_\_\_\_\_\_

مِمَّنْ هَذِهِ الرِّسَالَةُ؟ - هَذِهِ الرِّسَالَةُ مِنْ أَخٍ عَرَبِيٍّ. عَمَّاذَا هِيَ؟ - اَلرِّسَالَةُ عَنْ حَيَاةِ الْفَلاَّحِينَ عِنْدَهُمْ. مَتَى تَسَلَّمْتَ الرِّسَالَةَ؟ - تَسَلَّمْتُهَا أَمْسِ مَسَاءً. مِنْ أَيْنَ تَعْرِفُ هَذَا الأَخَ الْعَرَبِيَّ؟ - أَعْرِفُهُ عِنْدَمَا كَانَ يَدْرُسُ فِي الْجَامِعَةِ فِي مَدِينَتِنَا. كَانَ يَسْكُنُ مَعَنَا فِي مَنْزِلِنَا. أَيَكْتُبُ إِلَيْكُمْ رَسَائِلَ؟ - نَعَمْ، اَلآنَ يَكْتُبُ إِلَيْنَا رَسَائِلَ. لِمَنْ هَذِهِ الْبَدْلَةُ؟ - هِيَ ِلأَبِي. أَيْنَ يَعْمَلُ أَبُوكَ؟ - هُوَ مُهَنْدِسٌ، وَ هُوَ الآنَ فِي الْمَعْمَلِ. يَعْمَلُ فِي هَذَا الْمَعْمَلِ عُمَّالٌ وَ عَامِلاَتٌ. وَ أَيْنَ تَعْمَلُ أَنْتَ؟ - أَنَا فَلاَّحٌ وَ أَعْمَلُ فِي الْحَقْلِ. فِي بِلاَدِنَا مَعَامِلُ كَثِيرَةٌ و يَعْمَلُ فِيهَا كَثِيرٌ مِنَ الْعُمَّالِ وَ الْعَامِلاَتِ وَ الْمُهَنْدِسِينَ. أَيْنَ شَارَتُكَ؟ - شَارَتِي عَلَى بَدْلَتِي. بَدْلَتِي جَدِيدَةٌ. بَدْلَتُكَ قَدِيمَةٌ. أَيْنَ تَدْرُسُ؟ - أَدْرُسُ فِي الْجَامِعَةِ. هُوَ وَ إِخْوَتُهُ يَدْرُسُونَ فِي الْمَدْرَسَةِ.

\subsubsection{К уроку 8}
Эта собака большая. Эти собаки большие. Тот петух маленький. Те петухи маленькие. Та курица красивая. Те курицы красивые. Чья эта собака? — Эта собака моя. Чьи эти собаки? — Эти собаки твои (ж.р.). Чей тот петух? — Тот петух его. Чьи те петухи? — Те петухи их. Где книги, журналы и газеты? — Книги, журналы и газеты в библиотеке. А где библиотека? — Библиотека здесь. Ты учитель? — Да, я учитель. Ты учительница? — Нет, я не учительница. Он уче­ник? \_ Нет, он не ученик. Она ученица? — Да, она ученица.

\subsection{29 اَلدَّرْسُ التَّاسِعُ وَ الْعِشْرُونَ}

\bigskip

\begin{center}
\tablefirsthead{}
\tablehead{}
\tabletail{}
\tablelasttail{}
\begin{supertabular}{|m{0.5455598in}|m{0.7955598in}|m{0.71985984in}|m{0.7441598in}|m{0.9261598in}|m{1.1712599in}|m{1.0462599in}|m{1.0670599in}|}
\hline
إِقْرَأْ &
إِقْرَأُوا &
إِقْرَئِي &
إِقْرَأْنَ &
لاَ تَقْرَأْ &
لاَ تَقْرَأُوا &
لاَ تَقْرَئِي &
لاَ تَقْرَأْنَ\\\hline
\end{supertabular}
\end{center}
(نَزَلَ، يَنْزِلُ، نُزُولٌ). (لَبِسَ، يَلْبَسُ، لُبْسٌ). (فَتَحَ، يَفْتَحُ، فَتْحٌ). (شَرَحَ، يَشْرَحُ، شَرْحٌ). (مَسَحَ، يَمْسَحُ، مَسْحٌ). مَعْنًى، مَعَانٍ. آخَرُ، أُخْرَى. عِنْدَ ذَلِكَ. اَلدُّنْيَا بَرْدٌ.

\_\_\_\_\_\_\_\_\_\_\_\_\_\_\_\_\_\_\_\_\_\_\_\_

إِلَى أَيْنَ تَذْهَبُونَ؟ - نُرِيدُ أَنْ نَذْهَبَ إِلَى الْمَدِينَةِ. هَذِهِ السَّيَارَةُ لاَ تَذْهَبُ إِلَى الْمَدِينَةِ فَلاَ تَرْكَبُوهَا بَلِ انْزِلُوا مِنْهَا وَ ارْكَبُوا سَيَّارَةً أُخْرَى. يَا تِلْمِيذَةُ خُذِي الْمِمْسَحَةَ وَ امْسَحِي اللَّوْحَ بِهَا ثُمَّ اكْتُبِي عَلَيْهِ هَذِهِ الْكَلِمَاتِ الْعَرَبِيَّةَ ثُمَّ اشْرَحِي لَنَا مَعَانِيَهَا. اَلدُّنْيَا بَرْدٌ اليَوْمَ لاَ تَفْتَحُوا الْنَّوَافِذَ وَ لاَ تَفْتَحُوا الأَبْوَابَ. مَنْ فَتَحَ الْبَابَ؟ مَنْ لَبِسَ بَدْلَتِي؟ مَنْ يَشْرَحُ مِنْكُمْ لِي دَرْسَ الْيَوْمِ؟ يَا وَلَدُ لاَ تَلْعَبْ فِي الْحُجْرَةِ. يَا أَوْلاَدُ لاَ تَلْعَبُوا فِي الْحُجْرَةِ. يَا بِنْتُ لاَ تَلْعَبِي فِي الْحُجْرَةِ بَلِ اخْرُجِي إِلَى السَّاحَةِ وَ الْعَبِي فِيهَا. يَا بَنَاتُ لاَ تَلْعَبْنَ فِي الْحُجْرَةِ بَلِ اخْرُجْنَ إِلىَ السَّاحَةِ وَ الْعَبْنَ فِيهَا. لاَ تَأْخُذْ هَذَا الْكِتَابَ بَلْ خُذْ كِتَابًا آخَرَ. أَتُرِيدُ هَذِهِ الْمَجَلَّةَ؟ - لاَ، بَلْ أُرِيدُ مَجَلَّةً أُخْرَى. أَهَذَا مُعَلِّمُكُمْ؟ - لاَ، بَلْ مُعَلِّمُنَا رَجُلٌ آخَرُ. هَلْ كُنْتَ قُرْبَ السَّيَّارَةِ عِنْدَمَا نَزَلَ مِنْهَا الأُسْتَاذُ؟ - نَعَمْ، كُنْتُ قُرْبَهَا عِنْدَ ذَلِكَ وَ نَزَلَ مَعَهُ مِنْهَا رَجُلٌ آخَرُ.

\subsubsection{К уроку 9}
Ахмед, читай урок. Аиша, читай урок. Войди в комнату. Выйди из комнаты. Войди в класс. Выйди из класса. Где книга? — Книга здесь. Чья она? — Она Махмуда. Читай в этой комнате. Пиши в той комнате. Ты войди в дом и читай там. Ты выходи из дома, пиши здесь. Кто в библиотеке? — Там ученики и ученицы. Кто в школе? — Там учителя и учительницы. Кто в доме? — Там мужчины и женщины. Чей этот урок? Чьи эти уроки?

\subsection{30 اَلدَّرْسُ الثَّلاَثوُنَ}
 \includegraphics[width=1.198in,height=1.4583in]{images/MuhammadBagauddinprettified-img060.png}  \includegraphics[width=1.2602in,height=1.4063in]{images/MuhammadBagauddinprettified-img061.png}   \includegraphics[width=1.7083in,height=1.2917in]{images/MuhammadBagauddinprettified-img062.png} 

\ تَمْرٌ. تِينٌ. زَيْتُونٌ. سَمَاءٌ, سَمَاوَاتٌ.

\ مَطَرٌ, أَمْطَارٌ. ثَلْجٌ, ثُلُوجٌ. مَاءُ الْمَطَرِ. وَرْدٌ، وُرُودٌ. زَرْعٌ، زُرُوعٌ. مُفِيدٌ. نَافِعٌ.

(سَأَلَ، يَسْأَلُ، سُؤَالٌ). (رَسَمَ، يَرْسُمُ، رَسْمٌ). (قَالَ، يَقُولُ، قَوْلٌ). (سَقَى، يَسْقِي، سَقْيٌ). لِمَاذَا؟ قِيلَ لَهُ. لاَ أَدْرِي.

\ نَزَلَ الْمَطَرُ.

\_\_\_\_\_\_\_\_\_\_\_\_\_\_\_\_\_\_\_\_\_\_\_\_\_\_

سَأَلَتْ مَيْسُونُ أُخْتَهَا: أَيْنَ سَالِمٌ؟ قَالَتْ لَهَا: سَالِمٌ فِي الْبُسْتَانِ. وَ أَيْنَ فِرَاسٌ؟ - فِرَاسٌ يَرْسُمُ فِي الْبَيْتِ. مَاذَا يَرْسُمُ فِرَاسٌ؟ - هُوَ يَرْسُمُ وَرْدًا. فِي بِلاَدِنَا تَمْرٌ وَ تِينٌ وَ زَيْتُونٌ. اَلْمَطَرُ يَنْزِلُ فَتَقُولُ زَيْنَبُ: هَذَا مَاءُ الْمَطَرِ، مَاءُ الْمَطَرِ نَافِعٌ جِدًّا. اَلْوَرْدُ يُرِيدُ الْمَطَرَ وَ الشَّجَرُ يُرِيدُ الْمَطَرَ وَ الزَّرْعُ يُرِيدُ الْمَطَرَ. اَلْمَطَرُ يَنْزِلُ مِنَ السَّمَاءِ وَ يَسْقِي الزَّرْعَ وَ الثَّلْجُ أَيْضًا يَنْزِلُ مِنَ السَّمَاءِ. اَلْمَطَرُ نَافِعٌ لَنَا وَ لِلزَّرْعِ. يَقُولُ زِيَادٌ: أَيْنَ دَارُ أَسْمَاءَ؟ - قِيلَ لَهُ: دَارُ أَسْمَاءَ لَيْسَ بِبَعِيدٍ، لِمَاذَا تُرِيدُ دَارَهَا؟ - قَالَ: أُرِيدُ أَنْ أَذْهَبَ إِلَيْهَا وَ آخُذَ مِنْهَا كِتَابِي. قَالَ طَلاَّلٌ: بِلاَلٌ يُرِيدُ أَقْلاَمِي يَا أَبِي فَقُلْ لَهُ "لاَ تَأْخُذْ أَقْلاَمَهُ" قَالَ الأَبُ: وَ أَيْنَ أَقْلاَمُ بِلاَلٍ؟ قَالَ طَلاَّلٌ: لاَ أَدْرِي. فَقَالَتْ عَائِشَةُ: أَقْلاَمُ بِلاَلٍ عِنْدِي. هَذَا الْكِتَابُ مُفِيدٌ. مَا قُلْتَ لِي؟ مَا تُرِيدُ أَنْ تَقُولَ لِي؟ قُلْ لِي مَاذَا تُرِيدُ؟ قُلْتُ لَهُ: إِسْقِ بُسْتَانَنَا فَسَقَاهُ. لِمَاذَا أَنْتَ هُنَا؟

\subsubsection{К уроку 10}
Где книга? — В парте. Где парта? — В классе. Где класс? — В школе. Кто читает там? — Салим. Кто пишет там? — Рашид. Где Абдулла? — Абдулла читает книгу в комнате. Где Ахмед? — Ахмед делает (пишет) урок в классе. Ахмед, бери книгу. Фатима, бери тетрадь. Салим, дай ручку. Аиша, дай портфель. Где арабская газе­та? — Дай мне. — На бери (её). Это стул, стул большой. Это парта, парта маленькая. Чья эта бумага? — Моя. Чей этот стул? — Твой (ж.р.).

\subsection[اَلدَّرْسُ الْحَادِى وَ الثَّلاَثُونَ 31]{اَلدَّرْسُ الْحَادِى وَ الثَّلاَثُونَ 31}
\  \includegraphics[width=1.2189in,height=1.1252in]{images/MuhammadBagauddinprettified-img063.png}   \includegraphics[width=1.2189in,height=1.052in]{images/MuhammadBagauddinprettified-img064.png}   \includegraphics[width=1.2709in,height=0.8854in]{images/MuhammadBagauddinprettified-img065.png} 

عَلَمٌ, اَعْلاَمٌ. عَلَمُ الإِسْلاَمِ. جُبْنٌ. طَبْلٌ, طُبُولٌ.

زَرَّاعٌ (ون). بَيَّاعٌ (ون). رَبِيعٌ. مَرْفُوعٌ. حَسَنًا. اَبَدًا. (قَفَزَ, يَقْفِزُ, قَفْزٌ). (زَرَعَ, يَزْرَعُ, زَرْعٌ). (رَفَعَ, يَرْفَعُ, رَفْعٌ). (كَسَرَ, يَكْسِرُ, كَسْرٌ). (نَقَرَ, يَنْقُرُ, نَقْرٌ). (بَاعَ, يَبِيعُ, بَيْعٌ). (رَقَصَ, يَرْقُصُ, رَقْصٌ). نَقَرَ عَلَى الطَّبْلِ. رَفْرَفَ.

\_\_\_\_\_\_\_\_\_\_\_\_\_\_\_\_\_\_\_\_\_\_\_\_\_\_\_

قَالَتْ نَوَالُ: مَنْ قَفَزَ؟ قَالَتْ زَيْنَبُ: اَنَا قَفَزْتُ فَقَالَتْ: لِمَاذَا تَقْفِزِينَ؟ لاَ تَقْفِزِى وَ لاَ تَلْعَبِى بَلِ اذْهَبِى اِلَى حُجْرَتِكِ وَ خُذِى كِتَابَكِ وَ اقْرَئِى دَرْسَكِ. يَقُولُ وَلِيدٌ: اَبِى زَرَّاعٌ يَزْرَعُ الْوَرْدَ فِى الرَّبِيعِ وَ يَسْقِيهِ بِالْمَاءِ وَ اَنَا بَيَّاعٌ اَبِيعُ الْوَرْدَ. عَلَمُ بِلاَدِى عَلَمُ الإِسْلاَمِ, عَلَمُ بِلاَدِى مَرْفُوعٌ. رَفَعَ عَادِلٌ اَلْعَلَمَ وَ قَالَ: رَفْرِفْ يَا عَلَمُ, رَفْرِفْ يَا عَلَمَ الإِسْلاَمِ. اَلأَعْلاَمُ تُرَفْرِفُ فَوْقَ الْمَسَاجِدِ وَ الْبُيُوتِ. سَلْوَى تَنْقُرُ عَلَى الطَّبْلِ وَ لَيْلَى تَنْقُرُ عَلَى الطَّبْلِ فَقَالَتْ نَدَى: اَنَا اَيْضًا اُرِيدُ اَنْ اَنْقُرَ عَلَى الطَّبْلِ مَتَى اَنْقُرُ عَلَى الطَّبْلِ؟ فَقَالَتْ سَلْوَى: اَنْتِ لاَ تَنْقُرِى عَلَى الطَّبْلِ مَتَى أَنْقُرُ عَلَى الطَّبْلِ؟ فَقَالَتْ سَلْوَى: أَنْتِ لا تَنْقُرِي عَلَى الطَّبْلِ بَلِ ارْقُصِى اَمَامَنَا وَ نَحْنُ نَنْقُرُ لَكِ. اَكَلَ الْوَلَدُ الْخُبْزَ مَعَ الْجُبْنِ وَ شَرِبَ الْحَلِيبَ مِنْ فِنْجَانٍ جَمِيلٍ ثُمَّ كَسَرَ الْفِنْجَانَ بَعْدَ ذَلِكَ. مَنْ كَسَرَ قَلَمِى يَا اَحْمَدُ؟ - اَنَا مَا كَسَرْتُ قَلَمَكَ وَ لاَ اَعْرِفُ مَنْ كَسَرَهُ. اَنَا لاَ اَكْسِرُ الْفِنْجَانَ وَ لاَ اَكْسِرُ الْكُوبَ اَبَدًا. وَ لاَ تَكْسِرْ اَنْتَ اَيْضًا. - حَسَنًا.

\subsubsection{К уроку 11}
Он писатель. Они писатели. Она писательница. Они писательницы. Этот город маленький. Эти города маленькие. Этот писатель араб. Эти писатели арабы. Эта писательница русская. Эти писательницы русские. Где книга и тетрадь? — На парте. Где газета и журнал? — На столе учителя. Этот город большой? — Нет, наоборот, маленький. Этот писатель араб? — Нет, наоборот, русский. Кто в школе? — В школе учителя и ученики. Что в классе? — В классе парта, стул и стол учителя. Что в парте? — В парте портфель.

\subsection[اَلدَّرْسُ الثَّانِى وَ الثَّلاَثُونَ 32]{اَلدَّرْسُ الثَّانِى وَ الثَّلاَثُونَ 32}
 \includegraphics[width=1.052in,height=0.8957in]{images/MuhammadBagauddinprettified-img066.png}   \includegraphics[width=1.0835in,height=1.198in]{images/MuhammadBagauddinprettified-img067.png}   \includegraphics[width=1.0728in,height=1.1772in]{images/MuhammadBagauddinprettified-img068.png}   \includegraphics[width=1.2709in,height=1.448in]{images/MuhammadBagauddinprettified-img069.png} 

فَرَاشَةٌ (ات). عُشٌّ, عِشَاشٌ. طَيْرٌ, طُيُورٌ. رِيشٌ, أَرْيَاشٌ. 

 \includegraphics[width=1.7602in,height=1.052in]{images/MuhammadBagauddinprettified-img070.png}   \includegraphics[width=0.6354in,height=1.4063in]{images/MuhammadBagauddinprettified-img071.png}   \includegraphics[width=1.7083in,height=1.0835in]{images/MuhammadBagauddinprettified-img072.png} 

غُصْنٌ, أَغْصَانٌ. شُرْطِىُّ الْمُرُورِ. رَادِيُو. مُسْلِمٌ (ون). نَشِيطٌ, نُشَطَاءُ. اَلْحَمْدُ لِلَّهِ. شُرْطِىٌّ. قَشٌّ. اِنْسَانٌ, نَاسٌ. شَجَرَةٌ (ات). شَارِعٌ, شَوَارِعُ. اِذَاعَةٌ (ات). اِذَاعَةُ رَادِيُو. عُبُورٌ. مُرُورٌ. جَمِيعًا. عَالَمٌ. فِى الْعَالَمِ.

(سَمِعَ, يَسْمَعُ, سَمَاعٌ). (وَقَفَ, يَقِفُ, وُقُوفٌ).

(صَنَعَ, يَصْنَعُ, صُنْعٌ).(جَمَعَ, يَجْمَعُ, جَمْعٌ).(طَارَ,يَطِيرُ,طَيَرَانٌ). سَاعَدَ.

\_\_\_\_\_\_\_\_\_\_\_\_\_\_\_\_\_\_\_\_\_\_\_\_\_

اَلْيَوْمَ صَبَاحًا سَمِعْتُ اِذَاعَةً عَرَبِيَّةً عَنْ حَيَاةِ الْمُسْلِمِينَ فِى الْعَالَمِ. هَلْ سَمِعْتَ هَذِهِ الإِذَاعَةَ؟ - لاَ, اَنَا مَا سَمِعْتُهَا. اَنَا مُسْلِمٌ وَ الْحَمْدُ لِلَّهِ. هَلْ اَنْتَ مُسْلِمٌ؟ - نَعَمْ, اَنَا مُسْلِمٌ وَ اَبِى مُسْلِمٌ وَ اُمِّى مُسْلِمَةٌ وَ اَخِى مُسْلِمٌ وَ اُخْتِى مُسْلِمَةٌ نَحْنُ جَمِيعًا مُسْلِمُونَ وَ الْحَمْدُ لِلَّهِ. شَرِيفٌ شُرْطِىُّ الْمُرُورِ وَ هُوَ شُرْطِىٌّ نَشِيطٌ جِدًّا. شُرْطِىُّ الْمُرُورِ يَقِفُ فِى الشَّارِعِ وَ يُسَاعِدُ النَّاسَ عَلَى عُبُورِ الشَّارِعِ وَ الْمُرُورِ فِيهِ. الْفَرَاشَةُ تَطِيرُ فِى الْبُسْتَانِ وَ الطَّيْرُ يَطِيرُ فِى الْبُسْتَانِ. اَلْفَرَاشَةُ تَطِيرُ مِنْ نَّبَاتٍ اِلَى نَبَاتٍ وَ الطَّيْرُ يَطِيرُ مِنْ شَجَرَةٍ اِلَى شَجَرَةٍ وَ مِنْ غُصْنٍ اِلَى غُصْنٍ. لِلطَّيْرِ رِيشٌ وَ لِلطَّيْرِ عُشٌّ. اَلطَّيْرُ يَصْنَعُ الْعُشَّ عَلَى غُصْنِ الشَّجَرِ مِنَ الْقَشِّ وَ يَجْمَعُ الْقَشَّ فِى الْحَقْلِ. فِى الْعَالَمِ الآنَ مُسْلِمُونَ كَثِيرُونَ. اَلْمُسْلِمُونَ جَمِيعًا اِخْوَةٌ. اِسْمَعِ الإِذَاعَاتِ الْعَرَبِيَّةَ بِالرَّادِيُو كَثِيرًا لِتَعْرِفَ عَنِ الْمُسْلِمِينَ فِى الْعَالَمِ كَثِيرًا.

\subsubsection{К уроку 12}
Что это? — Стакан. Что это? — Молоко. Что это? — Хлеб. Что это? — Вода. Кто это? — Мальчик. Чей он? — Он мой. Где ста­кан? — Стакан в комнате. Что в стакане? — В нём молоко. Молоко горячее? — Молоко не горячее, а холодное. Где вода? — Вода там. Вода холодная? — Да, вода холодная. Эй, ученик, бери книгу и читай её. Эй, мальчик, бери хлеб и ешь его. Это хлеб, он вкусный. Это вода, она горячая. Это молоко, оно холодное. Этот хлеб невкусный. То молоко нехолодное. Та вода негорячая. Эта книга немаленькая.

\subsection{الدَّرْسُ الثَّالِثُ وَ الثَّلاَثُونَ 33}
 \includegraphics[width=2.0209in,height=0.6354in]{images/MuhammadBagauddinprettified-img073.png}   \includegraphics[width=1.1772in,height=1.052in]{images/MuhammadBagauddinprettified-img074.png}   \includegraphics[width=0.7291in,height=0.9898in]{images/MuhammadBagauddinprettified-img075.png} 

\ يَدٌ, اَيْدٍ. جِزَانَةٌ (ات). خِزَانَةُ كُتُبٍ. خَرِيفٌ. مَعْهَدٌ, مَعَاهِدُ. أَىٌّ؟ أَيَّةٌ؟ أَىُّ كِتَابٍ. جَيِّدًا. 

(عَادَ, يَعُودُ, عَوْدَةٌ). (نَامَ, يَنَامُ, نَوْمٌ). اَطَاعَ. تَغَدَّى. تَعَشَّى. لُغَةٌ (ات). غُرْفَةٌ, غُرَفٌ. مَخْزَنٌ, مَخَازِنُ. 

اِنَاءٌ, آنِيَةٌ. دَرَسَ اللُّغَةَ الْعَرَبِيَّةَ. بِاللُّغَةِ الْعَرَبِيَّةِ. نِمْتُ. نِمْنَا. تَغَدَّيْتُ

\_\_\_\_\_\_\_\_\_\_\_\_\_\_\_\_\_\_\_\_\_\_\_\_\_\_

ذَهَبَ التَّلاَمِذَةُ اِلَى الْمَدْرَسَةِ فِى الْخَرِيفِ وَ دَرَسُوا فِيهَا اللُّغَةَ الْعَرَبِيَّةَ. دَخَلَتِ التِّلْمِيذَاتُ الْغُرْفَةَ وَ جَلَسْنَ فِيهَا عَلَى مَقَاعِدَ قُرْبَ النَّافِذَةِ. أَيَّةَ لُغَةٍ تَدْرُسُونَ فِى الْمَعْهَدِ؟ - نَحْنَ نَدْرُسُ فِيهِ اللُّغَةَ الرُّوسِيَّةَ. ذَهَبَتِ الْفَتَيَاتُ اِلَى الْمَكْتَبَةِ وَ اَخَذْنَ مِنْهَا كُتُبًا اَجْنَبِيَّةً مُخْتَلِفَةً مِنَ الْخِزَانَةِ ثُمَّ خَرَجْنَ مِنَ الْمَكْتَبَةِ. هَذِهِ خِزَانَةُ كُتُبٍ فِيهَا كُتُبٌ بِاللُّغَةِ الْعَرَبِيَّةِ. هَلْ تُوجَدُ فِى الْخِزَانَةِ مَجَلاَّةٌ وَ جَرَائِدُ بِاللُّغَةِ الرُّوسِيَّةِ؟ - لاَ, لاَ تُوجَدُ هِىَ فِيهَا. بِأَيَّةِ لُغَةٍ تَقْرَأُ؟ بِأَيَّةِ لُغَةٍ تَكْتُبُ؟ يَا اُمِّى, هَاتِ الْغَدَاءَ, اُرِيدُ اَنْ آكُلَ. مَتَى نَجْلِسُ لِنَقْرَأَ الدُّرُوسَ؟ - نَجْلِسُ لِذَلِكَ بَعْدَ الْعَشَاءِ. هَلْ تَغَدَّيْتَ؟ - نَعَمْ, تَغَدَّيْتُ. هَلْ تَعَشَّيْتِ؟ - لاَ, مَا تَعَشَّيْتُ بَعْدُ. اُرِيدُ اَنْ اَتَغَدَّى. اَتُرِيدُ اَنْ تَتَعَشَّى؟ تَغَدَّيْنَا ثُمَّ اَخَذْنَا كُتُبَنَا وَ ذَهَبْنَا اِلَى الْمَدْرَسَةِ وَ بَعْدَ الرُّجُوعِ مِنْهَا تَعَشَّيْنَا وَ قَرَأْنَا دُرُوسَنَا ثُمَّ نِمْنَا. كَانَتْ اِنْعَامُ فِى الْحَدِيقَةِ. عَادَتْ اِنْعَامُ اِلَى الْبَيْتِ. فِى يَدِ اِنْعَامَ اِنَاءٌ. اِنْعَامُ تَقُولُ: أُطِيعُ أَبِى وَ أُسَاعِدُ اُمِّى. يَا اَحْمَدُ اَطِعْ اَبَاكَ. يَا فَاطِمَةُ سَاعِدِى أُمَّكِ. 

\subsubsection{К уроку 13}
Эй, мальчик, иди сюда и читай урок. Эй, мальчик, иди сюда и пей молоко. Зайнаб, иди сюда и ешь хлеб. Хадиджа, иди сюда и пей воду. Что это? — Это дверь, она большая. Это окно, оно маленькое. Дверь открыта? — Да, дверь открыта. Окно закрыто? — Да, окно закрыто. Что на столе? — На нём газеты и журналы. Что на парте/ — На ней пенал и портфель. Мухаммед, войди в дверь. Хасан, выйди из двери. Асма, войди в комнату. Лайла, выйди из комнаты. Чей этот пенал? — Того ученика. Что в пенале? — В нём карандаши.

\subsection[اَلدَّرْسُ الرَّابِعُ وَ الثَّلاَثُونَ 34]{اَلدَّرْسُ الرَّابِعُ وَ الثَّلاَثُونَ 34}
 \includegraphics[width=1.6665in,height=1.052in]{images/MuhammadBagauddinprettified-img076.png}   \includegraphics[width=1.2602in,height=0.8957in]{images/MuhammadBagauddinprettified-img077.png}   \includegraphics[width=1.2917in,height=1.198in]{images/MuhammadBagauddinprettified-img078.png}   \includegraphics[width=2.1146in,height=1.1665in]{images/MuhammadBagauddinprettified-img079.png} 

\ سَمَكٌ, أَسْمَاكٌ. مِنَشَّةٌ. سَمَّاكٌ (ون). نَهْرٌ, أَنْهُارٌ. 

\  \includegraphics[width=1.9898in,height=1.0835in]{images/MuhammadBagauddinprettified-img080.png}   \includegraphics[width=2.4374in,height=1.6665in]{images/MuhammadBagauddinprettified-img081.png} 

بَحْرٌ, بِحَارٌ. بُحَيْرَةٌ (ات). سُوقٌ, أَسْوَاقٌ. ذُبَابٌ, ذُبَّانٌ. شُكْرًا لَكَ. خَبَرٌ, أَخْبَارٌ. بِحَمْدِ اللَّهِ. قِرَاءَةٌ. كِتَابَةٌ. عَلَّمَ. كُلٌّ. كُلَّ يَوْمٍ. عَادَةً. أَوَّلَ أَمْسِ. بَعْدَ غَدٍ. عَصًا, عِصِىٌّ. عَاشَ (ى) عَيْشٌ. صَادَ (ى) صَيْدٌ.

طَرَدَ (و) طَرْدٌ. سَبَحَ (ا) سِبَاحَةٌ.

\_\_\_\_\_\_\_\_\_\_\_\_\_\_\_\_\_\_\_\_\_\_\_\_\_\_\_\_\_\_

فِى يَدِ مُنْذِرٍ مِنَشَّةٌ. مُنْذِرُ يَطْرُدُ الذُّبَابَ بِالْمِنَشَّةِ. قَالَتْ لَهُ اُمُّهُ مَاذَا تَفْعَلُ يَا مُنْذِرُ؟ قَالَ مُنْذِرٌ: اَنَا اَطْرُدُ الذُّبَابَ بِالْمِنَشَّةِ. مَالِكٌ سَمَّاكٌ. مَالِكٌ يَصِيدُ السَّمَكَ فِى الْبُحَيْرَةِ وَ يَبِيعُهُ فِى السُّوقِ. لَحْمُ السَّمَكِ لَذِيذٌ وَ نَافِعٌ. اَيْنَ يَعِيشُ السَّمَكُ؟ - بَعْضُ الأَسْمَاكِ يَعِيشُ فِي النَّهْرِ وَ بَعْضُهَا يَعِيشُ فِى الْبَحْرِ اَوِ الْبُحَيْرَةِ. اَيْنَ يَسْبَحُ السَّمَكُ؟ - اَلسَّمَكُ يَسْبَحُ فِى الْمَاءِ. مَاذَا تَرْسُمُ عَلَى الْوَرَقِ يَا وَلِيدُ؟ - اَرْسُمُ حِمَارًا وَ عَلَى الْحِمَارِ وَلَدًا فِى يَدِهِ عَصًا. رَسَمَ التِّلْمِيذُ عَلَى اللَّوْحِ عَلَمًا وَ كَتَبَ عَلَيْهِ "لاَ اِلَهَ اِلاَّ اللَّهُ". اَنْتَ عَلَّمْتَنَا الْقِرَاءَةَ وَ الْكِتَابَةَ شُكْرًا لَكَ يَا اُسْتَاذُ, فَاَنْتَ مُعَلِّمُنَا. كُلُّنَا الآنَ بِحَمْدِ اللَّهِ نَقْرَأُ وَ نَكْتُبُ. نَحْنُ نَلْعَبُ بَعْدَ الدُّرُوسِ وَ لاَ نَلْعَبُ قَبْلَهَا عَادَةً. اَيْنَ سَمِعْتَ هَذَا الْخَبَرَ الْجَدِيدَ؟ سَمِعْتُهُ فِى الرَّادِيُو. مَتَى سَمِعْتَهُ؟ - سَمِعْتُهُ اَوَّلَ اَمْسِ. مَتَى تَذْهَبُ اِلَى النَّهْرِ لِصَيْدِ السَّمَكِ؟ - اَذْهَبْ اِلَى النَّهْرِ لِصَيْدِ السَّمَكِ بَعْدَ غَدٍ. بِاَيَّةِ لُغَةٍ هَذَا الْكِتَابُ؟ - هَذَا الْكِتَابُ بِلُغَتِنَا. كُلَّ يَوْمٍ نَذْهَبُ اِلَى الْبُحَيْرَةِ وَ نَصِيدُ فِيهَا السَّمَكَ ثُمَّ نَبِيعُهُ فِى السُّوقِ.

\subsubsection[К уроку 14]{К уроку 14}
Ахмед, где ты? — Я здесь, в комнате. Что ты там делаешь? — Читаю книгу, пишу урок. У меня чашка, она красивая. У тебя (ж.р.) стакан, он некрасивый. В чашке молоко. В стакане чай. Где сахар? — В сахарнице. А где сахарница? — Сахарница там, в комнате, на столе. Ты будешь пить чай? Ты будешь пить кофе? Ты (м.р.) будешь есть хлеб? Ты будешь читать эту книгу? Он читает газету. Она читает газету. Ты читаешь газету. Ты (ж.р.) читаешь газету. Я читаю газету. Ты (ж.р.) будешь есть хлеб? — Да, буду. Ты (ж.р.) будешь пить кофе? — Нет, не буду.

\subsection{اَلدَّرْسُ الْخَامِسُ وَ الثَّلاَثُونَ 35}
 \includegraphics[width=2.3437in,height=1.3126in]{images/MuhammadBagauddinprettified-img082.png}   \includegraphics[width=1.4063in,height=1.2917in]{images/MuhammadBagauddinprettified-img083.png} 

\ مَلْعَبٌ, مَلاَعِبُ. كُرَةُ الْقَدَمِ. دِينٌ, اَدْيَانٌ. يَوْمٌ, اَيَّامٌ. يَوْمُ الْعِيدِ. خَشَبِىٌّ. جِهَادٌ. أَحَبَّ. طَالِبٌ, طُلاَّبٌ. 

طَالِبَةٌ (ات). اِنْتِهَاءٌ. اِخْوَانٌ. صَلاَةٌ, صَلَوَاتٌ. 

عِيدٌ, اَعْيَادٌ. عِيدُكَ مُبَارَكٌ. - وَ عِيدُكَ أَنْتَ. 

دَارَ, يَدُورُ, دَوَرَانٌ. تَهْنِئَةٌ. صَلاَةُ الْعِيدِ. كَثِيرًا.

اَللَّهُ يَرْعَاكَ.

\_\_\_\_\_\_\_\_\_\_\_\_\_\_\_\_\_\_\_\_\_\_\_\_\_\_

أَبُو نَبِيهٍ يَزْرَعُ فِى الْحَقْلِ وَ نَبِيهٌ يَزْرَعُ مَعَهُ وَ يُسَاعِدُهُ. قَالَ لَهُ نَبِيهٌ: اَنْتَ تَزْرَعُ يَا أَبِى وَ اَنَا أُسَاعِدُكَ. فَقَالَ لَهُ اَبُوهُ: نَعَمْ, يَا وَلَدِى, شُكْرًا لَكَ, اَللَّهُ يَرْعَاكَ, اَنْتَ سَاعَدْتَنِى كَثِيرًا. هَذَا الْبَيْتُ خَشَبِىٌّ وَ هَذِهِ الْخِزَانَةُ خَشَبِيَّةٌ. مَنْ اَنْتَ؟ - اَنَا طَالِبٌ. وَ هَلْ اُخْتُكَ طَالِبَةٌ اَيْضًا؟ - نَعَمْ, هِىَ طَالِبَةٌ اَيْضًا. هَلْ اَنْتَ طَالِبٌ فِى الْمَعْهَدِ؟ - لاَ, اَنَا طَالِبٌ فِى الْجَامِعَةِ. اَيْنَ جَامِعَتُكُمْ؟ - جَامِعَتُنَا قُرْبَ الْمَلْعَبِ. طُلاَّبُ جَامِعَتِنَا الآنَ فِى الْمَلْعَبِ يَلْعَبُونَ بِكُرَةِ الْقَدَمِ. طُلاَّبُ جَامِعَتِنَا كُلُّهُمْ نُشَطَاءُ مُجْتَهِدُونَ وَ كُلُّهُمْ مُسْلِمُونَ يُحِبُّونَ دِينَ الإِسْلاَمِ وَ يُحِبُّونَ الْجِهَادَ. عَلَى اَىِّ دِينٍ اَنْتَ؟ - اَلْحَمْدُ لِلَّهِ, اَنَا عَلَى دِينِ الإِسْلاَمِ. اَتُحِبُّ الْجِهَادَ؟ نَعَمْ, اُحِبُّهُ كَثِيرًا. نَحْنُ مُسْلِمُونَ وَ نَحْنُ نُحِبُّ اللَّهَ وَ نُحِبُّ الإِسْلاَمَ وَ نُحِبُّ الْجِهَادَ وَ نُحِبُّ الْمَسْجِدَ وَ نُحِبُّ الصَّلاَةَ. عِيدُكَ مُبَارَكٌ يَا هَاشِمُ! وَ عِيدُكِ اَنْتِ. اَيْنَ نَذْهَبُ فِى يَوْمِ الْعِيدِ؟ - نَذْهَبُ اِلَى الْمَسْجِدِ لِصَلاَةِ الْعِيدِ وَ بَعْدَ انْتِهَاءِ الصَّلاَةِ نَدُورُ عَلَى اِخْوَانِنَا وَ اَخَوَاتِنَا لِتَهْنِئَتِهِمْ بِالْعِيدِ.

\subsubsection{К уроку 15}
Кто это? — Это мой отец. Кто это? — Это твоя (ж.р.) мать. Кто это? — Это его сын. Кто это? — Это её дочь. Его книга здесь. Её тетрадь там. Твой карандаш в пенале. Твоя ручка (ж.р.) в портфеле. Мой портфель в парте. Чей ты? Чей ты (ж.р.)? Чья она? Чей он? Кто твой брат? — Мой брат Хасан. Где он? — Он дома. Кто твоя (ж.р.) сестра? — Моя сестра Наваль. Где она? — Она в школе. Зайд мой ученик. Амр твой учитель. Надия её учительница. Самира твоя ученица. Мой отец писатель. Твоя мать писательница. Она пьёт кофе.

\subsection{اَلدَّرْسُ السَّادِسُ وَ الثَّلاَثُونَ 36}
 \includegraphics[width=1.3126in,height=0.8854in]{images/MuhammadBagauddinprettified-img084.png}   \includegraphics[width=1.3854in,height=1.052in]{images/MuhammadBagauddinprettified-img085.png}   \includegraphics[width=0.802in,height=0.8957in]{images/MuhammadBagauddinprettified-img086.png} 

سَاعَةٌ (ات). طَبِيبٌ, اَطِبَّاءُ. مَرِيضٌ, مَرْضَى. سَطْلٌ, سُطُولٌ. حِكَايَةٌ (ات). عَمٌّ, اَعْمَامٌ. هَدِيَّةٌ, هَدَايَا. سَنَةٌ, سَنَوَاتٌ. وَقْتٌ, اَوْقَاتٌ. ضَعِيفٌ, ضُعَفَاءُ. صَبَاحِىٌّ. مَاضٍ. وَلِذَلِكَ. اَلسَّنَةُ الْمَاضِيَةُ. نَهَارُكَ سَعِيدٌ. اَهْلاً وَ سَهْلاً. دَخَلَ الْجَامِعَةَ. اِنَّ. اِنَّهُ. اَمَا قَرَأَ؟ اَمَا قُلْتُ لَكَ؟ 

زَارَ, يَزُورُ, زِيَارَةٌ.

\_\_\_\_\_\_\_\_\_\_\_\_\_\_\_\_\_\_\_\_\_\_\_\_\_\_\_\_\_

فِى يَدِ زُهْرَةَ سَطْلٌ وَ فِى السَّطْلِ لَبَنٌ. بَعْدَ الْغَدَاءِ نَذْهَبُ اِلَى الْمَلْعَبِ لِنَلْعَبَ بِكُرَةِ الْقَدَمِ. هَلْ مَعَكَ سَاعَتُكَ؟ - لاَ, سَاعَتِى لَيْسَتْ مَعِى بَلْ هِىَ فِى حُجْرَتِى عَلَى مَكْتَبِى. اَىُّ يَوْمٍ هَذَا؟ - هَذَا يَوْمُ الْعِيدِ, هَذَا يَوْمُ عِيدِ الْمُسْلِمِينَ. مِنْ اَيْنَ سَمِعْتَ هَذِهِ الْحِكَايَةَ؟ - قَرَأْتُهَا فِى الْجَرِيدَةِ الصَّبَاحِيَّةِ. بَيْتُ عَمِّى قُرْبَ النَّهْرِ. زُرْتُ عَمِّى وَ قُلْتُ لَهُ: نَهَارُكَ سَعِيدٌ يَا عَمِّى, فَقَالَ لِى: وَ نَهَارُكَ, يَا سُهَيْلُ, اَهْلاً وَ سَهْلاً بِكَ, عِنْدِى لَكَ هَدَايَا. فَاطِمَةُ طَالِبَةٌ فِى الْمَعْهَدِ. فَاطِمَةُ طَالِبَةٌ مُجْتَهِدَةٌ نَشِيطَةٌ تُحِبُّ الْقِرَاءَةَ وَ الْكِتَابَةَ. دَخَلَتِ الْمَعْهَدَ فِى السَّنَةِ الْمَاضِيَةِ وَ قَبْلَ ذَلِكَ دَرَسَتْ فِى الْمَدْرَسَةِ. فَاطِمَةُ الآنَ مَرِيضَةٌ. اَمَا ذَهَبَتْ اِلَى الطَّبِيبِ؟ - لاَ, مَا ذَهَبَتْ اِلَى الطَّبِيبِ اِنَّهَا ضَعِيفَةٌ جِدًّا, وَ لِذَلِكَ مَا ذَهَبَتْ اِلَى الدُّرُوسِ وَ مَا خَرَجَتْ مِنَ الْبَيْتِ. نَاهِدَةُ فَتَاةٌ نَشِيطَةٌ عِنْدَهَا كُرَةٌ تَلْعَبُ بِهَا وَ عِنْدَهَا سَاعَةٌ تَعْرِفُ بِهَا الْوَقْتَ, اَلسَّاعَةُ هَدِيَّةٌ مِنْ أَبِيهَا لَهَا فِى يَوْمِ الْعِيدِ. 

\subsubsection{К уроку 16}
Кто во дворе (дома) бегает? Кто во дворе играет? Кто в комнате читает? Кто в классе пишет? Рашид, иди сюда, бери книгу и читай свой урок. Махмуд, войди в комнату, возьми тетрадь и напиши свой урок. Где твоя резинка? — Здесь, в пенале. Где моё перо? — Там, на столе. Ты читай и пиши в комнате. Бегай и играй во дворе. Мой двор большой. Твой двор маленький. Где ученики? — Они во дворе школы. Где ученицы? — Они во дворе дома. Мой сын бегает. Его сын читает. Её сын пишет. Твой (ж.р.) сын ест хлеб и пьёт кофе с молоком. Мой сын читает книгу и пишет свой урок пером.

\subsection[اَلدَّرْسُ السَّابِعُ وَ الثَّلاَثُونَ 37]{اَلدَّرْسُ السَّابِعُ وَ الثَّلاَثُونَ 37}
 \includegraphics[width=1.5937in,height=2.0417in]{images/MuhammadBagauddinprettified-img087.png}   \includegraphics[width=0.698in,height=1.5in]{images/MuhammadBagauddinprettified-img088.png}   \includegraphics[width=1.2602in,height=2.0102in]{images/MuhammadBagauddinprettified-img089.png} 

الْكُرَةُ الطَّائِرَةُ. مِقَصٌّ. زَهْرِيَّةٌ (ات). رِيَاضَةٌ (ات). رِيَاضِيٌّ (ون). فُرْصَةٌ, فُرَصٌ. طَيَّارَةٌ (ات).

 \includegraphics[width=1.7602in,height=1.0728in]{images/MuhammadBagauddinprettified-img090.png}   \includegraphics[width=1.7811in,height=1.1043in]{images/MuhammadBagauddinprettified-img091.png}   \includegraphics[width=2.2917in,height=1.3543in]{images/MuhammadBagauddinprettified-img092.png} 

\ أَرْضِيَّةٌ. حَائِطٌ, حِيطَانٌ. صَيَّادُ السَّمَكِ. 

 \includegraphics[width=2.1043in,height=1.3854in]{images/MuhammadBagauddinprettified-img093.png}   \includegraphics[width=1.7083in,height=1.1665in]{images/MuhammadBagauddinprettified-img094.png}   \includegraphics[width=1.1665in,height=0.7709in]{images/MuhammadBagauddinprettified-img095.png} 

\ شِصٌّ, شُصُوصٌ. بِسَاطٌ, بُسُطٌ. خَرِيطَةٌ, خَرَائِطُ. 

زَاوِيَةٌ, زَوَايَا. جُغْرَافِىٌّ. خَرِيطَةٌ جُغْرَافِيَّةٌ. عَرِيضٌ. مُدَوَّرٌ. فِى الْغَدِ. قَصَّ (و) قَصٌّ. (رَأَى, يَرَى, رُؤْيَةٌ). (عَمِلَ, يَعْمَلُ, عَمَلٌ)

\_\_\_\_\_\_\_\_\_\_\_\_\_\_\_\_\_\_\_\_\_\_\_\_\_\_\_\_\_\_

هَلْ اَنْتَ رِيَاضِىٌّ؟ - نَعَمْ, اَنَا رِيَاضِىٌّ. اَتُحِبُّ الرِّيَاضَةَ؟ - اَجَلْ, اُحِبُّهَا. ذَهَبَ الرِّيَاضِيُّونَ اِلَى مَلْعَبِ الْمَدِينَةِ لِيَلْعَبُوا بِالْكُرَةِ الطَّائِرَةِ. وَقَّاصٌ تِلْمِيذٌ مُجْتَهِدٌ يَقْرَأُ وَ يَكْتُبُ فِى الدَّرْسِ وَ يَلْعَبُ فِى الْفُرَصِ. عِنْدَ وَقَّاصٍ مِقَصٌّ يَقُصُّ بِهِ الْوَرَقَ وَ يَعْمَلُ مِنْهُ طَيَّارَةً. اَخَذْتُ مِقَصًّا وَ قَصَصْتُ بِهِ الْوَرَقَ. هَلْ كُنْتَ عِنْدَ الأُسْتَاذِ فِى غُرْفَتِهِ؟ - نَعَمْ, كُنْتُ. وَ مَاذَا رَأَيْتَ فِيهَا؟ - رَأَيْتُ فِيهَا طَاوِلَةً مُدَوَّرَةً عَلَيْهَا زَهْرِيَّةٌ, وَ عَلَى الطَّاوِلَةِ رَأَيْتُ كُتُبًا وَ مَجَلاَّتٍ مُخْتَلِفَةً اَيْضًا, وَ رَأَيْتُ عَلَى الْحَائِطِ خَرِيطَةً جُغْرَافِيَّةً وَ عَلَى الأَرْضِيَّةِ بِسَاطًا جَمِيلاً عَرِيضًا, وَ رَأَيْتُ فِى زَاوِيَةِ الْغُرْفَةِ رُفُوفًا عَلَيْهَا كُتُبٌ كَثِيرَةٌ بِاللُّغَةِ الْعَرَبِيَّةِ. اَيْنَ يَجْلِسُ الأُسْتَاذُ؟ - عِنْدَ قِرَاءَةِ الْكُتُبِ وَ الْمَجَلاَّتِ وَ الْجَرَائِدِ يَجْلِسُ عَادَةً عَلَى الدِّيوَانِ وَ عِنْدَ الْكِتَابَةِ يَجْلِسُ عَلَى الْكُرْسِىِّ وَرَاءَ الْمَكْتَبِ. اَنَا صَيَّادُ السَّمَكِ وَ عِنْدِى شِصٌّ اَذْهَبُ بِهِ اِلَى النَّهْرِ وَ اَصِيدُ بِهِ السَّمَكَ وَ فِى الْغَدِ اَذْهَبُ بِالسَّمَكِ اِلَى السُّوقِ وَ اَبِيعُهُ فِى السُّوقِ. تَعَالَ نَجْلِسْ عَلَى هَذَا الْبِسَاطِ الْعَرِيضِ. تَعَالَوْا نَذْهَبْ اِلَى مَلْعَبِ الْمَدْرَسَةِ لِنَلْعَبَ بِالْكُرَةِ الطَّائِرَةِ. 

\subsubsection{К уроку 17}
Ты читал эту книгу? — Да, я её читал. Ты написала свой урок? — Нет, я его ещё не написала. Кто взял мою резинку? — Твой младший брат. Кто съел его хлеб? — Его сын. Кто пил чай? — Твой (ж.р.) отец. Кто вошёл в дом? — В дом вошёл этот мужчина. Кто вышел из комнаты? — Из комнаты вышла та женщина. Кто делал это? — Это делал твой ученик. Учитель вышел из комнаты? — Нет, он ещё не вышел. Ученица вошла в класс? — Да, она вошла туда. Я вошёл в комнату, потом вышел. Она взяла книгу, потом прочитала её.

\subsection{الدَّرْسُ الثَّامِنُ وَ الثَّلاَثُونَ 38}
\  \includegraphics[width=1.198in,height=0.7083in]{images/MuhammadBagauddinprettified-img096.png}   \includegraphics[width=1.2917in,height=1.1457in]{images/MuhammadBagauddinprettified-img097.png}   \includegraphics[width=1.3335in,height=1.0102in]{images/MuhammadBagauddinprettified-img098.png} 

صُنْدُوقٌ, صَنَادِيقُ. قُفْلٌ, اَقْفَالٌ. مِفْتَاحٌ, مَفَاتِيحُ. 

 \includegraphics[width=1.2398in,height=1.3335in]{images/MuhammadBagauddinprettified-img099.png}   \includegraphics[width=1.2709in,height=1.1043in]{images/MuhammadBagauddinprettified-img100.png}   \includegraphics[width=1.3126in,height=1.052in]{images/MuhammadBagauddinprettified-img101.png} 

\ حِنْطَةٌ. طَمَاطِمُ. بَطَاطِسُ. 

رَبٌّ, اَرْبَابٌ. اِلَهٌ, آلِهَةٌ. نَبِىٌّ, اَنْبِيَاءُ. مُؤْمِنٌ (ون). لُعْبَةٌ, لُعَبٌ. نُقُودٌ. خُضْرَوَاتٌ. وَ غَيْرُهَا مِنَ الْخُضْرَوَاتِ. مَكَّةُ. اَلْمَدِينَةُ. اِسْمٌ, اَسْمَاءٌ. مَا اسْمُكَ؟ وُلِدَ. دُفِنَ. فَرِحَ, يَفْرَحُ, فَرَحٌ. مَاتَ, يَمُوتُ, مَوْتٌ.

\_\_\_\_\_\_\_\_\_\_\_\_\_\_\_\_\_\_\_\_\_\_\_

مَنْ رَبُّكَ؟ - رَبِّىَ اللَّهُ. مَنْ اِلَهُكَ؟ - اِلَهِىَاللَّهُ. وَ مَنْ نَبِيُّكَ؟ - نَبِيِّى مُحَمَّدٌ. وَ مَا دِينُكَ؟ - دِينِىَ الإِسْلاَمُ. مَنْ اِخْوَتُكَ؟ - اِخْوَتِى الْمُسْلِمُونَ الْمُؤْمِنُونَ. وَ مَنْ اَخَوَاتُكَ؟ اَخَوَاتِى الْمُسْلِمَاتُ الْمُؤْمِنَاتُ. اَللَّهُ رَبُّ النَّاسِ جَمِيعًا, وَ اِلَهُ النَّاسِ جَمِيعًا وَ مُحَمَّدٌ نَبِيُّ النَّاسِ جَمِيعًا وَ الإِسْلاَمُ دِينُ النَّاسِ جَمِيعًا. اَيْنَ وُلِدَ نَبِيُّنَا مُحَمَّدٌ؟ - وُلِدَ نَبِيُّنَا مُحَمَّدٌ فِى مَكَّةَ. اَيْنَ مَاتَ مُحَمَّدٌ وَ اَيْنَ دُفِنَ؟ - هُوَ مَاتَ وَ دُفِنَ فِى الْمَدِينَةِ. مَا اسْمُ اَبِيهِ؟ - اِسْمُ اَبِيهِ عَبْدُ اللَّهِ. مَا اسْمُ اُمِّهِ؟ - اِسْمُ اُمِّهِ آمِنَةُ. نَبِيُّنَا هُوَ مُحَمَّدُ بْنُ عَبْدِ اللَّهِ.

قَالَ صَبَاحٌ: عِنْدِى صُنْدُوقٌ فِيهِ نُقُودِى, وَ لَهُ قُفْلٌ وَ مِفْتَاحٌ, آخُذُ الْمِفْتَاحَ وَ اَفْتَحُ الصُّنْدُوقَ ثُمَّ اَلْعَبُ مَعَ صَدِيقِى صَالِحٍ وَ نَفْرَحُ. عِنْدَمَا سَمِعْتُ خَبَرَ اَخِى مِنَ الْمَعْهَدِ فَرِحْتُ جِدًّا. مَاذَا زَرَعَ الْفَلاَّحُونَ فِى الْحَقْلِ فِى السَّنَةِ الْمَاضِيَةِ؟ - زَرَعُوا فِيهِ فِى السَّنَةِ الْمَاضِيَةِ الْحِنْطَةَ وَ الْبَطَاطِسَ وَ الطَّمَاطِمَ وَ غَيْرَهَا مِنَ الْخُضْرَوَاتِ. مَا اسْمُكَ؟ - اِسْمِى بَهَاءُ الدِّينِ. مَا اسْمُ مَدِينَتِكَ؟ - اِسْمُ مَدِينَتِى غِيزِلْيُورْتُ.

\subsubsection{К уроку 18}
Что ты делаешь в школе? — Читаю и пишу, бегаю и играю. Ты понял это слово из своего урока? Ты поняла эти слова из арабской книги? — Нет, я не поняла их. Это слово арабское. Эти слова арабские. Что делает ученик в зале? — Читает журнал. Вчера я пошёл в библиотеку и взял оттуда некоторые книги. А сегодня ты ходил в библиотеку? — Да, сегодня тоже я ходил туда. А завтра пойдёшь туда? — Нет, завтра не пойду. Ты (ж.р.) завтра в школу пой­дёшь? — Да, если Богу будет угодно. Где ты сидела? — Я сидела за партой (на парте). Где сидела учительница? — Она сидела на стуле.

\subsection[الدَّرْسُ التَّاسِعُ وَ الثَّلاَثُونَ 39]{الدَّرْسُ التَّاسِعُ وَ الثَّلاَثُونَ 39}
 \includegraphics[width=1.9272in,height=1.0835in]{images/MuhammadBagauddinprettified-img102.png}   \includegraphics[width=0.75in,height=0.802in]{images/MuhammadBagauddinprettified-img103.png}   \includegraphics[width=0.5417in,height=1.0102in]{images/MuhammadBagauddinprettified-img104.png}  

\ عُلْبَةٌ, عُلَبٌ. حَبْلٌ, حِبَالٌ. كِيسٌ, اَكْيَاسٌ. 

 \includegraphics[width=1.6457in,height=1.5209in]{images/MuhammadBagauddinprettified-img105.png}  \includegraphics[width=1.6457in,height=1.6665in]{images/MuhammadBagauddinprettified-img106.png}   \includegraphics[width=1.5209in,height=1.1772in]{images/MuhammadBagauddinprettified-img107.png} 

\ قَمْحٌ. شَعِيرٌ. صُورَةٌ, صُوَرٌ.

\  \includegraphics[width=2.1354in,height=1.6665in]{images/MuhammadBagauddinprettified-img108.png}   \includegraphics[width=2.198in,height=1.552in]{images/MuhammadBagauddinprettified-img109.png} 

\ حَوْضٌ, اَحْوَاضٌ. حَوْضُ السِّبَاحَةِ. حَشِيشٌ. 

 \includegraphics[width=1.8957in,height=1.2602in]{images/MuhammadBagauddinprettified-img110.png}   \includegraphics[width=1.1252in,height=0.6772in]{images/MuhammadBagauddinprettified-img111.png} 

\ صِبْغٌ, اَصْبَاغٌ. عُلْبَةُ أَصْبَاغٍ. حَيَوَانٌ (ات). 

حِصَانٌ, حُصُنٌ. طَعَامٌ, اَطْعِمَةٌ. لَوْنٌ, اَلْوَانٌ. فُطُورٌ. صَمْغٌ. صِغَرٌ. فِى الصِّغَرِ. سِبَاحَةٌ. وَاسِعٌ. عَمِيقٌ. وَقْتَ الْفَرَاغِ. قَدَّمَ. اَلْصَقَ. تَعَلَّمَ. تَعَلَّمْ. 

حَمَلَ (ى) حَمْلٌ. رَبَطَ (ى) رَبْطٌ. صَبَغَ (ا) صَبْغٌ.

\_\_\_\_\_\_\_\_\_\_\_\_\_\_\_\_\_\_\_\_\_\_\_\_\_\_

لِحَمَّادٍ حِصَانٌ, الْحِصَانُ كَبِيرٌ. حَمَّادٌ يَرْكَبُ الْحِصَانَ وَ يَخْرُجُ مِنْ دَارِهِ وَ يَذْهَبُ اِلَى الْحَقْلِ. الْحِصَانُ حَيْوَانٌ مُفِيدٌ فَهُوَ يَحْمِلُ الأَكْيَاسَ. فِى الأَكْيَاسِ قَمْحٌ وَ شَعِيرٌ. يَا رَشِيدُ سَاعِدْنِى عَلَى حَمْلِ هَذَا الْكِيسِ اِلَى السَّيَّارَةِ. فَاطِمَةُ تَرْبِطُ الْحِصَانَ بِالْحَبْلِ وَ تُقَدِّمُ اِلَيْهِ الْحَشِيشَ وَ الْحِصَانُ يَأْكُلُ الْحَشِيشَ. اُمُّنَا تُقَدِّمُ لَنَا الطَّعَامَ و نَحْنُ نَأْكُلُهُ. زَوْجَتِى قَدَّمَتْ لَنَا فُطُورًا وَ نَحْنُ اَكَلْنَاهُ. عِنْدِى عُلْبَةُ اَصْبَاغٍ, اَرْسُمُ وَقْتَ الْفَرَاغِ وَ اَصْبَغُ الصُّورَةَ بِاَلْوَانٍ مُخْتَلِفَةٍ ثُمَّ اُلْصِقُ الصُّورَةَ بِالصَّمْغِ فِى دَفْتَرِى. هَذَا الصَّمْغُ جَدِيدٌ وَ جَيِّدٌ. فِى مَدِينَتِنَا حَوْضُ السِّبَاحَةِ, فِتْيَانُ الْمَدِينَةِ يَذْهَبُونَ اِلَى حَوْضِ السِّبَاحَةِ وَ يَسْبَحَونَ فِيهِ. فِى سَاحَةِ دَارِنَا اَيْضًا حَوْضُ السِّبَاحَةِ, حَوْضُ السِّبَاحَةِ وَاسِعٌ وَ عَمِيقٌ. اَنَا وَ رِيَاضٌ نَسْبَحُ فِى الْحَوْضِ وَ نَلْعَبُ وَ نَرْكُضُ بَعْدَ السِّبَاحَةِ. نَحْنُ نُحِبُّ السِّبَاحَةَ, اَلسِّبَاحَةُ نَافِعَةٌ. رِيَاضٌ تَعَلَّمَ السِّبَاحَةَ فِى صِغَرِهِ وَ اَنَا تَعَلَّمْتُ السِّبَاحَةَ فِى صِغَرِى. تَعَلَّمِ السِّبَاحَةَ فِى صِغَرِكَ.

\subsubsection{К уроку 19}
Чей этот мяч? — Он мой. А где мой мяч? — Там, во дворе. Чей тот мяч? — Твоего брата. Эй, мальчик, не играй здесь, а бери мяч, выйди во двор и играй там. Я поел немного хлеба с мясом, затем выпил немного чая с сахаром, затем взял свой мяч, вышел из комнаты во двор и поиграл в мяч со своими товарищами. С кем ты пошла в школу? — Я пошла с Аишой. С кем ты сидел на парте? — Я сидел со своим товарищем Хасаном. Это мясо вкусное? — Да, оно вкусное. Ты будешь есть его? — Нет, я не буду есть его. Мой мяч большой и красивый, а твой мяч небольшой и некрасивый.

\subsection[اَلدَّرْسُ الأَرْبَعُونَ 40]{اَلدَّرْسُ الأَرْبَعُونَ 40}
 \includegraphics[width=1.8543in,height=1.552in]{images/MuhammadBagauddinprettified-img112.png}   \includegraphics[width=1.6146in,height=1.198in]{images/MuhammadBagauddinprettified-img113.png}   \includegraphics[width=1.552in,height=1.198in]{images/MuhammadBagauddinprettified-img114.png} 

\ أَسَدٌ, أُسُودٌ. نَمِرٌ, نُمُورٌ. قِرْدٌ, قِرَدَةٌ. 

 \includegraphics[width=2.0626in,height=1.0728in]{images/MuhammadBagauddinprettified-img115.png}   \includegraphics[width=1.4583in,height=1.052in]{images/MuhammadBagauddinprettified-img116.png}   \includegraphics[width=1.1457in,height=1.2602in]{images/MuhammadBagauddinprettified-img117.png} 

\ دُبٌّ, دُبَبَةٌ. ذِئْبٌ, ذِئَابٌ. غَزَالٌ, غِزْلاَنٌ. 

\  \includegraphics[width=1.0311in,height=0.9583in]{images/MuhammadBagauddinprettified-img118.png}   \includegraphics[width=1.1043in,height=0.6354in]{images/MuhammadBagauddinprettified-img119.png}   \includegraphics[width=1.052in,height=1.052in]{images/MuhammadBagauddinprettified-img120.png}  

ذُبَابَةٌ (ات). عَنْكَبُوتٌ, عَنَاكِبُ. نَسِيجُ الْعَنْكَبُوتِ. 

شَبَكَةٌ (ات). رَوْضَةٌ, رِيَاضٌ. خَيْطٌ, خُيُوطٌ. نَسَّاجٌ (ون). حَشَرَةٌ (ات). وَحْشِىٌّ. دَقِيقٌ. لَطِيفٌ. حِكَايَةٌ لَطِيفَةٌ. طِفْلٌ, اَطْفَالٌ. رَوْضَةُ الأَطْفَالِ. حَدِيقَةُ الْحَيْوَانَاتِ. 

بَيْتُ الْعَنْكَبُوتِ. نَسَجَ (ى) نَسْجٌ. وَقَعَ (ا) وُقُوعٌ. 

\  \includegraphics[width=0.9165in,height=0.5626in]{images/MuhammadBagauddinprettified-img121.png} 

عَيْنٌ, عُيُونٌ. ضَحِكَ (ا) ضَحِكٌ. وَثَبَ (ى) وُثُوبٌ.

\_\_\_\_\_\_\_\_\_\_\_\_\_\_\_\_\_\_\_\_\_\_\_\_\_\_\_\_\_

زُرْنَا حَدِيقَةَ الْحَيْوَانَاتِ. فِى الْحَدِيقَةِ غَزَالٌ وَ فِيهَا اَسَدٌ وَ نَمِرٌ وَ فِيهَا ذِئْبٌ وَ دُبٌّ وَ فِيهَا غَيْرُهَا مِنَ الْحَيَوَانَاتِ الْوَحْشِيَّةِ. اَلأَسَدُ حَيْوَانٌ وَحْشِيٌّ وَ النَّمِرُ وَ الدُّبُّ وَ الذِّئْبُ وَ الْقِرْدُ كُلُّهَا حَيْوَانَاتٌ وَحْشِيَّةٌ. الْغَزَالُ صَغِيرٌ وَ جَمِيلٌ وَ عَيْنُهُ وَاسِعَةٌ. اَلْغَزَالُ يَقْفِزُ وَ يَلْعَبُ. اَنَا اُحِبُّ الْغَزَالَ. اُطْعِمُ الْغَزَالَ الْخُبْزَ مِنْ يَدِى. ذَهَبَ ضِيَاءٌ اِلَى الرَّوْضَةِ وَ لَعِبَ مَعَ الأَطْفَالِ ثُمَّ جَلَسَ هُوَ وَ الأَطْفَالُ لِلْغَدَاءِ وَ بَعْدَ الْغَدَاءِ سَمِعُوا مِنَ الْمُعَلِّمَةِ حِكَايَةً لَطِيفَةً عَنِ الْقُرُودِ فَفَرِحُوا جَمِيعًا وَ ضَحِكُوا. اَلْعَنْكَبُوتُ نَسَّاجٌ مَاهِرٌ يَنْسِجُ بَيْتَهُ مِنْ خُيُوطٍ دَقِيقَةٍ. بَيْتُ الْعَنْكَبُوتِ شَبَكَةٌ يَصِيدُ بِهَا الْحَشَرَاتِ. وَقَعَتْ ذُبَابَةٌ فِى نَسِيجِ الْعَنْكَبُوتِ فَوَثَبَ عَلَيْهَا الْعَنْكَبُوتُ وَ صَادَهَا. اَيْنَ يَكُونُ اِبْنُكُمُ الصَّغِيرُ عِنْدَمَا تَذْهَبُ اَنْتَ اِلَى الْمَعْمَلِ وَ تَذْهَبُ زَوْجَتُكَ اِلَى الْحَقْلِ؟ - اِبْنُنَا فِى ذَلِكَ الْوَقْتِ يَكُونُ فِى رَوْضَةِ الأَطْفَالِ. وَ فِى الْمَسَاءِ عِنْدَمَا نَرْجِعُ اِلَى الْبَيْتِ نَأْخُذُهُ مِنْهَا. يَا زُهْرَةُ هَلْ اَطْعَمْتِ طِفْلَكِ؟

\subsubsection{К уроку 20}
Где ты был? — Я был во дворе. Что ты там делал? — Играл в мяч с ребятами. Когда ты идёшь в школу? — Я иду в школу утром. Вечером где ты был? — Вечером я был в зале. Что ты там де­лал? — Читал газеты и журналы. Сейчас что ты делаешь? — Сижу на скамейке. Вставай, бери книгу и читай её. Ученик встал и вышел из зала. Ученица встала, взяла книгу и прочла её. Я выпила немного чая перед уроком и поиграла после урока. Кто там сейчас? Кто пошёл сейчас? Кто сейчас встал и вышел? Кто сейчас играл? Кто понял этот урок?

\subsection{الدَّرْسُ الْحَادِى وَ الأَرْبَعُنَ 41}
\  \includegraphics[width=1.3126in,height=0.9165in]{images/MuhammadBagauddinprettified-img122.png}   \includegraphics[width=1.7398in,height=1.1252in]{images/MuhammadBagauddinprettified-img123.png}   \includegraphics[width=1.0728in,height=0.7291in]{images/MuhammadBagauddinprettified-img124.png} 

جِهَازُ التِّلِفُونِ. تِلِفِزْيُونٌ (ات). جِهَازُ الرَّادِيُو. 

مَطْبَخٌ, مَطَابِخُ. مِصْعَدٌ, مَصَاعِدُ. أَثَاثٌ (ات). وَسَطٌ. 

فِى وَسَطِ الْمَدِينَةِ. زَاوِيَةٌ, زَوَايَا. نَظِيفٌ. مُرَبَّعٌ. لِمَ؟ ِلأَنَّهُ. تَقْرِيبًا. لَكِنَّ. لَكِنَّهُ. هَكَذَا. لَمْ يَقْرَأْ. أَتَدْرِى؟ لَوْ كُنْتُ مَكَانَكَ. (بَكَى, يَبْكِى, بُكَاءٌ). (ضَرَبَ, يَضْرِبُ, ضَرْبٌ)

\_\_\_\_\_\_\_\_\_\_\_\_\_\_\_\_\_\_\_\_\_\_\_\_\_\_\_

هَلْ عِنْدَكُمْ تِلِفِزْيُونٌ؟ - نَعَمْ, عِنْدَنَا تِلِفِزْيُونٌ. كَانَ الْيَوْمَ فِى التِّلِفِزْيُونِ فِلْمٌ جَدِيدٌ عَنْ حَيَاةِ نَبِيِّنَا مُحَمَّدٍ, صَلَّى اللَّهُ عَلَيْهِ وَ سَلَّمَ. هَلْ غُرْفَتُكَ كَبِيرَةٌ؟ لاَ, هِىَ صَغِيرَةٌ لَكِنَّهَا نَظِيفَةٌ. مَاذَا يُوجَدُ فِى غُرْفَتِكَ مِنَ الأَثَاثِ؟ - تُوجَدُ فِى غُرْفَتِى خِزَانَةُ كُتُبٍ وَ دِيوَانٌ وَثِيرٌ وَ طَاوِلَةٌ مُرَبَّعَةٌ عَلَيْهَا جِهَازُ رَادِيُو وَ جِهَازُ التِّلِفُونِ وَ طَاوِلَةٌ صَغِيرَةٌ أُخْرَى مُدَوَّرَةٌ فِى زَاوِيَةِ الْغُرْفَةِ عَلَيْهَا جِهَازُ التِّلِفِزْيُونِ. هَلِ الْمَطْبَخُ فِى شِقَّتِكُمْ كَبِيرٌ؟ - لاَ, الْمَطْبَخُ فِى شِقَّتِنَا لَيْسَ كَبِيرًا. هَلْ يُوجَدُ فِى بَيْتِكُمْ مِصْعَدٌ؟ - نَعَمْ, يُوجَدُ فِى بَيْتِنَا مِصْعَدٌ جَيِّدٌ جَمِيلٌ. وَ اَيْنَ بَيْتُكُمْ؟ - بَيْتُنَا فِى وَسَطِ الْمَدِينَةِ قُرْبَ حَدِيقَةِ الْحَيْوَانَاتِ, وَ لِذَلِكَ اَزُورُ اَنَا وَ اَخِى الصَّغِيرُ حَدِيقَةَ الْحَيْوَانَاتِ كُلَّ يَوْمٍ تَقْرِيبًا. مُحَمَّدٌ يَضْحَكُ وَ اَخُوهُ يَبْكِى. لِمَاذَا يَضْحَكُ مُحَمَّدٌ؟ - ِلأَنَّهُ فَرِحَ كَثِيرًا. لِمَاذَا يَبْكِى اَخُوهُ؟ - ِلأَنَّهُ لَمْ يَقْرَاْ دُرُوسَهُ وَ لَمْ يَعْرِفْهَا فَضَرَبَهُ اَبُوهُ. اَتَدْرِى لِمَ تَبْكِى تِلْكَ الْفَتَاةُ؟ - لاَ اَدْرِى لِمَاذَا تَبْكِى. لَوْ كُنْتُ مَكَانَكَ لَفَعَلْتُ هَكَذَا. لَوْ كُنْتُ مَكَانَكَ لَقُلْتُ لَهُ هَكَذَا. لَوْ كُنْتُ مَكَانَكَ مَا فَعَلْتُ هَكَذَا. لَوْ كُنْتُ مَكَانَكَ مَا بَكَيْتُ. 

\subsubsection{К уроку 21}
Я инженер, и мой отец инженер. Ты преподаватель, и твой брат преподаватель. Ты вчера утром сел в машину и поехал, куда ты ехал? — Вчера утром я сел в машину и поехал в Махачкалу. Когда ты оттуда вернулся? — Я оттуда вернулся вчера вечером. Кто с тобой туда ездил? — Со мной туда ездил мой преподаватель. Ты водишь машину? — Да, я вожу машину. А твой младший брат водит машину? — Нет, он ещё не водит. Моя машина новая, твоя машина не новая. Ты не инженер. Ты не преподавательница. Я не писатель. Эта вода тёплая.

\subsection{اَلدَّرْسُ الثَّانِى وَ الأَرْبَعُونَ 42}
 \includegraphics[width=1.6252in,height=1.0311in]{images/MuhammadBagauddinprettified-img125.png}   \includegraphics[width=1.0728in,height=0.8752in]{images/MuhammadBagauddinprettified-img126.png}   \includegraphics[width=1.2189in,height=1.1252in]{images/MuhammadBagauddinprettified-img127.png}   \includegraphics[width=0.948in,height=1.1874in]{images/MuhammadBagauddinprettified-img128.png} 

\ جَبَلٌ, جِبَالٌ. كُوخٌ, اَكْوَاخٌ. مَوْقِدُ الْغَازِ. فَرْخٌ, أَفْرَاخٌ. 

\  \includegraphics[width=1.2398in,height=1.052in]{images/MuhammadBagauddinprettified-img129.png}   \includegraphics[width=1.5311in,height=0.9374in]{images/MuhammadBagauddinprettified-img130.png} 

بَيْضَةٌ, بَيْضٌ (ات). جَوْزٌ. مَصْيَفٌ, مَصَايِفُ. 

صَدِيقٌ, أَصْدِقَاءُ. غَازٌ (ات). سُؤَالٌ, اَسْئِلَةٌ. 

مَسْئَلَةٌ, مَسَائِلُ. مَامَا. اَلأَوَّلُ. اَلثَّانِى. الَثَّالِثُ. عِدَّةٌ. صَفٌّ, صُفُوفٌ. عَالٍ. مَسْرُورٌ. سَهْلٌ. صَعْبٌ. 

مَوْقِدٌ, مَوَاقِدُ. لَمَّا. لَمَّا خَرَجَ. حَقًّا. مِنْهُمْ. اَهْلٌ, اَهَالٍ. صَلَّى. طَبَخَ (ا) طَبْخٌ. وَجَدَ (ى) وُجُودٌ. 

ضَرَبَ - ضُرِبَ. أُخِذَ. كُتِبَ.

\_\_\_\_\_\_\_\_\_\_\_\_\_\_\_\_\_\_\_\_\_\_\_\_\_\_\_\_\_\_

فِى بِلاَدِنَا جِبَالٌ عَالِيَةٌ فِيهَا مَصَايِفُ جَمِيلَةٌ. زُرْتُ مَصْيَفِى اَمْسِ مَعَ اَهْلِى وَ صَلَّيْتُ فِيهِ مَعَ اَوْلاَدِى وَ قَرَأْنَا الْقُرْآنَ تَحْتَ اَشْجَارِ الْجَوْزِ وَ فِى الْمَسَاءِ رَجَعْنَا مِنْهُ اِلَى مَنْزِلِنَا مَسْرُورِينَ. لَيْثٌ لَهُ عِدَّةُ اَصْدِقَاءِ مِنْهُمْ زَيْدٌ وَ هُوَ يَتَعَلَّمُ فِى الصَّفِّ الأَوَّلِ وَ رَائِدٌ وَ هُوَ يَتَعَلَّمُ فِى الصَّفِّ الثَّانِى وَ حَارِثٌ وَ هُوَ يَتَعَلَّمُ فِى الصَّفِّ الثَّالِثِ. رَأَيْتُ كُوخَ دَجَاجٍ فِيهِ بَيْضٌ وَ فِيهِ عِدَّةُ اَفْرَاخٍ. لَمَّا رَجَعْتُ اِلَى الْبَيْتِ وَجَدْتُ مَامَا تَطْبَخُ لَنَا طَعَامًا لَذِيذًا. فِى مَطْبَخِنَا مَوْقِدُ الْغَازِ. نَحْنُ نَطْبَخُ الطَّعَامَ عَلَى مَوْقِدِ الْغَازِ. هَلْ عِنْدَكَ سُؤَالٌ اِلَىَّ؟ - نَعَمْ, عِنْدِى اِلَيْكَ عِدَّةُ اَسْئِلَةٍ. هَلْ اَسْئِلَتُكَ سَهْلَةٌ؟ - نَعَمْ, هِىَ سَهْلَةٌ وَ لَيْسَتْ صَعْبَةً. هَاتِ اَسْئِلَتَكَ؟ - السُّؤَالُ الأَوَّلُ: مَنْ اِلَهُكَ؟ وَ السُّؤَالُ الثَّانِى: مَادِينُكَ؟ وَ السُّؤَالُ الثَّالِثُ: مَنْ نَبِيُّكَ؟ - اِلَهِىَ اللَّهُ وَ دِينِىَ الإِسْلاَمُ وَ نَبِيِّى مُحَمَّدُ بْنُ عَبْدِ اللَّهِ. - اَنْتَ مُسْلِمٌ حَقًّا. دُرُوسُ هَذَا الْكِتَابِ سَهْلَةٌ. دُرُوسُ هَذَا الْكِتَابِ كُتِبَتْ لِلأَطْفَالِ الصِّغَارِ فِيهَا صُوَرُ الْحَيْوَانَاتِ وَ الأَشْجَارِ. هَذِهِ الْمَسْئَلَةُ صَعْبَةٌ وَ كَانَتْ مَسْئَلَةُ الأَمْسِ سَهْلَةً. 

\subsubsection{К уроку 22}
У меня большой сад. В моём саду много деревьев и растений. Мой сад около моего жилища. Утром я выхожу из дома, вхожу в сад и там сижу немного на скамейке, затем встаю, возвращаюсь в свою комнату, ем немного хлеба и пью кофе с молоком, затем сажусь на диван и читаю свои уроки. Мухаммед, что ты хочешь? Зайнаб, что ты хочешь? Что хочет этот мужчина? Что хочет та женщина? Я ученик, и я хочу книги, тетради и карандаши. В моей комнате диван, этот диван мягкий и новый. Я сел на диван и посмотрел из окна. Где твоё село? — Моё село около города. Твоё село большое? — Да, моё село очень большое.

\subsection[اَلدَّرْسُ الثَّالِثُ وَ الأَرْبَعُونَ 43]{اَلدَّرْسُ الثَّالِثُ وَ الأَرْبَعُونَ 43}
 \includegraphics[width=2.0417in,height=0.9791in]{images/MuhammadBagauddinprettified-img131.png}   \includegraphics[width=1.5in,height=1.1563in]{images/MuhammadBagauddinprettified-img132.png}   \includegraphics[width=1in,height=1.2398in]{images/MuhammadBagauddinprettified-img133.png} 

\ قِطَارٌ, قُطُرٌ. حَقِيبَةٌ, حَقَائِبُ. دُمْيَةٌ, دُمًى. 

\  \includegraphics[width=1.5626in,height=1.1772in]{images/MuhammadBagauddinprettified-img134.png}   \includegraphics[width=1.3646in,height=1.2811in]{images/MuhammadBagauddinprettified-img135.png}   \includegraphics[width=1.4374in,height=1.1772in]{images/MuhammadBagauddinprettified-img136.png} 

\ تُفَّاحٌ. كُمَّثْرَى. خَوْخٌ. 

 \includegraphics[width=1.1252in,height=1.7709in]{images/MuhammadBagauddinprettified-img137.png}   \includegraphics[width=0.9791in,height=1.6146in]{images/MuhammadBagauddinprettified-img138.png}   \includegraphics[width=2.1563in,height=1.0417in]{images/MuhammadBagauddinprettified-img139.png} 

\ عِنَبٌ. نَخْلٌ, نَخِيلٌ. خَيْمَةٌ, خِيَامٌ. 

سَفَرٌ, أَسْفَارٌ. ثَوْبٌ, اَثْوَابٌ, ثِيَابٌ. ثِيَابُ الْمَدْرَسَةِ. مِشْمِشٌ. كُرَّاسَةٌ (ات). صَاحِبٌ, اَصْحَابٌ. بَعْدَ سَنَةٍ. تَنَاوَلَ الطَّعَامَ. بَابَا. حَوْلَ... مَايَزَالُ. مَايَزَالُ صَغِيرًا. قَابَلَ. 

سَلَّمَ عَلَى... بَيْنَ... بَيْنَنَا. نَصَبَ (ى) نَصْبٌ. 

نَصَبَ الْخَيْمَةَ. حَانَ (ى) حَيْنُونَةٌ. حَانَ وَقْتُ الصَّلاَةِ. 

كَبُرَ (و) كِبَرٌ.

\_\_\_\_\_\_\_\_\_\_\_\_\_\_\_\_\_\_\_\_\_\_\_\_\_\_\_\_\_\_\_

خَرَجَ الْمُعَلِّمُ مَعَ تَلاَمِيذِهِ اِلَى الْبُسْتَانِ. فِى الْبُسْتَانِ خَوْخٌ وَ تِينٌ وَ تَمْرٌ وَ عِنَبٌ وَ تُفَّاحٌ وَ مِشْمِشٌ وَ كُمَّثْرَى. نَصَبُوا خَيْمَةً بَيْنَ النَّخِيلِ وَ جَلَسُوا فِيهَا وَ لَمَّا حَانَ وَقْتُ الصَّلاَةِ قَامُوا فَصَلَّوْا وَ بَعْدَ الصَّلاَةِ جَلَسُوا لِلْغَدَاءِ فَتَنَاوَلُوا الْغَدَاءَ ثُمَّ قَرَأُوا دُرُوسَهُمْ. عَادَ اَبُونَا مِنْ سَفَرِهِ فَجَلَسْنَا حَوْلَهُ فَفَتَحَ الْحَقِيبَةَ وَ قَدَّمَ اِلَيْنَا هَدَايَا وَ قَالَ: "هَذَا قِطَارٌ لَكَ يَا وَلِيدُ, وَ هَذِهِ دُمْيَةٌ لَكِ يَا هِنْدُ, وَ هَذَا فَرَسٌ خَشَبِىٌّ لَكَ يَا فَرِيدُ". فَقُلْنَا لَهُ: شُكْرًا لَكَ يَا بَابَا, اَللَّهُ يَرْعَاكَ. قَالَ عُمَرُ ِلأُمِّهِ: اَنَا اَلْبَسُ ثِيَابَ الْمَدْرَسَةِ وَ آخُذُ الْكُرَّاسَةَ وَ اَذْهَبُ اِلَى الْمَدْرَسَةِ, فَقَالَتْ اَمَلُ: اَنَا كَبُرْتُ يَا اُمِّى وَ عِنْدِى ثَوْبٌ جَدِيدٌ وَ عِنْدِى كُرَّاسَةٌ جَدِيدَةٌ وَ اُحِبُّ اَنْ اَذْهَبَ مَعَ عُمَرَ اِلَى الْمَدْرَسَةِ وَ اُحِبُّ اَنْ اَدْرُسَ فِى الْمَدْرَسَةِ فَضَحِكَ عُمَرُ وَ قَالَ: اَنْتِ مَا تَزَالِينَ صَغِيرَةً, وَ قَالَتِ اْلأُمُّ: تَذْهَبِينَ بَعْدَ سَنَةِ يَا بِنْتِى. ذَهَبَ عُمَرُ اِلَى الْمَدْرَسَةِ وَ قَابَلَ اَصْحَابَهُ وَ سَلَّمَ عَلَيْهِمْ وَ سَلَّمُوا عَلَيْهِ. 

\subsubsection{К уроку 23}
Кто живёт в этом новом доме? •— Я живу здесь. Кто живёт в том старом доме? — Там жил мой отец, теперь он там не живёт. Где он теперь живёт? — Теперь он живёт в городе, в новой квартире. Твоя мать живёт с ним в новой квартире? — Да, она тоже живёт с ним в новой квартире. Куда ты смотришь? — Я смотрю туда. Наша мечеть новая, она близко отсюда. Наша квартира старая, она далеко отсюда. Где живёт ваш инженер? Где жила ваша (ж.р.) преподавательница? Где жили ваши отцы и матери? — Они жили в деревне.

\subsection{اَلدَّرْسُ الرَّابِعُ وَ اْلأَرْبَعُونَ 44}
\  \includegraphics[width=1.0102in,height=1.3646in]{images/MuhammadBagauddinprettified-img140.png}   \includegraphics[width=0.7602in,height=1.0626in]{images/MuhammadBagauddinprettified-img141.png} 

جُنْدِىٌّ, جُنُودٌ. جَرَسٌ, أَجْرَاسٌ. جُنْدٌ, جُنُودٌ. وَطَنٌ, أَوْطَانٌ. شَعْبٌ, شُعُوبٌ. جَيْشٌ, جُيُوشٌ. بَطَلٌ, أَبْطَالٌ. 

اَلْجَيْشُ الْبَطَلُ. صَفٌّ, صُفُوفٌ. حَوْشٌ, أَحْوَاشٌ. اِسْلاَمِىٌّ. رَعَاكَ اللَّهُ. دَافَعَ عَنْ... دَافَعَ عَنْ دِينِهِ. 

وَقَفَ (ى) وُقُوفٌ. وَقَفَ فِى صَفٍّ. دَقَّ (و) دَقٌّ. 

دَقَّ الْجَرَسُ. يَحْيَى. يَحْيَى اْلإِسْلاَمُ. دَوْلَةٌ, دُوَلٌ. 

تَحْيَى دَوْلَةُ الإِسْلاَمِ. عَلَيْنَا اَنْ نَّذْهَبَ. عَلَيْنَا أَنْ...

\_\_\_\_\_\_\_\_\_\_\_\_\_\_\_\_\_\_\_\_\_\_\_\_\_\_\_\_\_\_\_

اَنَا جُنْدِىٌّ مُسْلِمٌ وَ اَنَا جُنْدِىُّ الإِسْلاَمِ. نَحْنُ جَمِيعًا جُنُودُ الإِسْلاَمِ. وَ عَلَيْنَا أَنْ نُحِبَّ الإِسْلاَمَ وَ نُحِبَّ جُنْدَ الإِسْلاَمِ. الْجُنْدِىُّ الْمُسْلِمُ يُحِبُّ وَطَنَهُ وَ دِينَهُ وَ شَعْبَهُ وَ يُدَافِعُ عَنْهُمْ. جَيْشُنَا جَيْشُ الإِسْلاَمِ وَ نَحْنُ جُنْدُ الإِسْلاَمِ. رَعَاكَ اللَّهُ يَا جَيْشَنَا الإِسْلاَمِىَّ يَا جَيْشَنَا الْبَطَلَ! كَانَتِ الْمُعَلِّمَةُ فِى حَوْشِ الْمَدْرَسَةِ, نَظَرَتْ اِلَيْهَا زَيْنَبُ فَقَالَتْ: اُنْظُرْ يَا عُمَرُ, اَلْمُعَلِّمَةُ. ذَهَبَتْ زَيْنَبُ اِلَى الْمُعَلِّمَةِ وَ سَلَّمَتْ عَلَيْهَا وَ سَلَّمَ عَلَيْهَا عُمَرُ وَ حَازِمٌ أَيْضًا. وَ لَمَّا دَقَّ الْجَرَسُ وَقَفَ عُمَرُ وَ اَصْحَابُهُ فِى صَفٍّ. وَ وَقَفَ التَّلاَمِيذُ مِنْ فُصُولٍ اُخْرَى فِى صُفُوفٍ وَ اَمَامَ كُلِّ صَفٍّ مُعَلِّمٌ اَوْ مُعَلِّمَةٌ. رَفَعَ التَّلاَمِيذُ الْعَلَمَ عَالِيًا وَ قَالُوا: يَحْيَا الإِسْلاَمُ! يَحْيَا الإِسْلاَمُ! تَحْيَا دَوْلَةُ الإِسْلاَمِ! تَحْيَا دَوْلَةُ الإِسْلاَمِ! عَلَى كُلِّ دَوْلَةٍ اَنْ تَكُونَ إِسْلاَمِيَّةً. عَلَى الدُّوَلِ جَمِيعًا اَنْ تَكُونَ دَوْلَةً وَاحِدَةً دَوْلَةً اِسْلاَمِيَّةً. عَلَيْنَا جَمِيعًا اَنْ نَكُونَ جُنُودًا فِى الدَّوْلَةِ الإِسْلاَمِيَّةِ. عَلَى الشُّعُوبِ جَمِيعًا اَنْ تَكُونَ شَعْبًا اِسْلاَمِيًّا وَاحِدًا.

\subsubsection[К уроку 24]{К уроку 24}
Этот юноша — мой товарищ по школе. Эти юноши — мои товарищи по школе. Та девушка — твоя сестра. Те девушки — твои сестры. Мой отец старик. Твоя мать старуха. Наш профессор молодой, и он известный человек. Наши профессора молодые. Мой ученик прилеж­ный, а твой ученик не прилежный. Откуда этот крестьянин? — Он из Дагестана. Откуда та крестьянка? — Она из Чечено-Ингушетии. Те мужчины крестьяне. Те женщины крестьянки. Откуда он? — Он из этого села. Откуда она? — Она из того города. Кто ваш профессор? Кто этот молодой (человек)? Кто этот старик? Кто эта старуха? Кто этот мальчик? — Он мой сын.

\subsection{الدَّرْسُ الْخَامِسُ وَ الأَرْبَعُونَ 45}
 \includegraphics[width=1.9272in,height=0.9791in]{images/MuhammadBagauddinprettified-img142.png}  \includegraphics[width=1.8228in,height=0.7083in]{images/MuhammadBagauddinprettified-img143.png}   \includegraphics[width=1.1043in,height=1.1252in]{images/MuhammadBagauddinprettified-img144.png} 

\ دَبَّابَةٌ (ات). طَائِرَةٌ (ات). سَلَّةٌ, سِلاَلٌ. مُهْمَلاَتٌ. سَلَّةُ الْمُهْمَلاَتِ. مَصْنَعٌ, مَصَانِعُ. وَرَقَةٌ (ات). 

مَصْنَعُ السَّيَّارَاتِ. مَصْنَعُ الدَّبَّابَاتِ. مَصْنَعُ الطَّائِرَاتِ. يَنْبَغِى. يَنْبَغِى اَنْ تَعْلَمَ. اَرْضٌ, اَرَاضٍ. عَلَى الأَرْضِ. رِحْلَةٌ, رِحِلاَتٌ. (مَشَى, يَمْشِى, مَشْىٌ). 

(رَمَى, يَرْمِى, رَمْىٌ). (وَضَعَ, يَضَعُ, وَضْعٌ).

(جَرَى, يَجْرِى, جَرْىٌ). (قَامَ بِ, يَقُومُ, قِيَامٌ). قَامَ بِرِحْلَةٍ. سَلاَمٌ. رَدَّ السَّلاَمَ عَلَى... خَلْفَ... جَرَى خَلْفَهُ.

\_\_\_\_\_\_\_\_\_\_\_\_\_\_\_\_\_\_\_\_\_\_\_\_\_\_\_\_\_\_\_\_\_\_

كَانَ اَحْمَدُ يَمْشِى فِى الْحَوْشِ فَرَأَى وَرَقَةً عَلَى الأَرْضِ فَقَالَ: مَنْ رَمَى الْوَرَقَةَ فِى الْحَوْشِ؟ قَالَ عَلِىٌّ: لاَ اَدْرِى مَنْ رَمَى الْوَرَقَةَ فِى الْحَوْشِ, وَ لَكِنْ هُنَاكَ سَلَّةٌ لِلْمُهْمَلاَتِ نَضَعُ فِيهَا الْوَرَقَ خُذْهَا يَا اَحْمَدُ وَ ضَعْهَا فِى السَّلَّةِ. اَخَذَ اَحْمَدُ الْوَرَقَةَ وَ ذَهَبَ بِهَا اِلَى سَلَّةِ الْمُهْمَلاَتِ وَ وَضَعَ الْوَرَقَةَ فِى السَّلَّةِ وَ قَالَ: مَدْرَسَتُنَا يَنْبَغِى اَنْ تَكُونَ نَظِيفَةً. وَ بَعْدَ قَلِيلٍ دَقَّ الْجَرَسُ فَجَرَى التَّلاَمِيذُ اِلَى الْفَصْلِ وَ دَخَلُوا الْفَصْلَ ثُمَّ دَخَلَ الْمُعَلِّمُ خَلْفَهُمْ وَ سَلَّمَ عَلَى الْجَمِيعِ فَرَدُّوا عَلَيْهِ السَّلاَمَ. أَيُّهَا الْوَلَدُ الْمُسْلِمُ يَنْبَغِى اَنْ تَكُونَ نَظِيفًا, وَ يَنْبَغِى اَنْ تَكُونَ ثِيَابُكَ نَظِيفَةً وَ اَنْ تَكُونَ كُتُبُكَ وَ دَفَاتِرُكَ نَظِيفَةً وَ اَنْ تَكُونَ حُجْرَتُكَ نَظِيفَةً ِلأَنَّكَ مُسْلِمٌ وَ الْمُسْلِمُ دَائِمًا نَظِيفٌ وَ اللَّهُ يُحِبُّ النَّظِيفَ. فِى مَدِينَتِنَا مَصَانِعُ كَثِيرَةٌ: مِنْهَا مَصْنَعُ السَّيَّارَاتِ وَ مَصْنَعُ الطَّائِرَاتِ وَ مَصْنَعُ الدَّبَّابَاتِ وَ غَيْرُهَا مِنَ الْمَصَانِعِ. قُمْنَا أَمْسِ بِرِحْلَةٍ اِلَى مَصْنَعِ الدَّبَّابَاتِ فَرَأَيْنَا فِيهِ دَبَّابَاتٍ جَدِيدَةً.

\subsubsection{К уроку 25}
Кто из вас хочет пить чай? Кто из вас хочет сесть на лошадь и поехать на поле? Кто из вас хочет играть сейчас во дворе? Мы хотим сесть на лошадей и вернуться в деревню. Наша страна большая и богатая. В нашей стране много городов и сёл. В нашей стране мнош садов и полей. В нашей стране много крестьян и крестьянок. Мы все крестьяне. Они все профессора. Вы все преподаватели. Что ты хочешь? — Я хочу сесть и читать свой урок. Что он хочет? — Он хочет пойти в библиотеку, чтобы взять некоторые книги. У тебя ест поле? — Да, у меня есть большое поле. Чья эта корова? — Моей матери. Я сел на лошадь, ты села на осла.

\subsection{الدَّرْسُ السَّادِسُ وَ الأَرْبَعُونَ 46}
 \includegraphics[width=1.0835in,height=1.448in]{images/MuhammadBagauddinprettified-img145.png}   \includegraphics[width=1.1665in,height=0.9272in]{images/MuhammadBagauddinprettified-img146.png}   \includegraphics[width=0.8646in,height=0.6252in]{images/MuhammadBagauddinprettified-img147.png} 

\ سَبُّورَةٌ. قَلَمُ رَصَاصٍ. قَلَمُ حِبْرٍ. فَكَّرَ. فَكَّرَ فِيمَا يَفْعَلُ. آتٍ. عِبَارَةٌ (ات). اُسْبُوعٌ, اَسَابِيعُ. خَرَخَ فِى رِحْلَةٍ. وَهَكَذَا. فَاِذَا. جَنْبَ... قَادِمٌ. 

 \includegraphics[width=1.8335in,height=1.4898in]{images/MuhammadBagauddinprettified-img148.png} 

\ غَابَةٌ (ات). سَأَلَ عَنْ... اِمَّا... اَوْ... اِذًا.

يَكُونُ - سَيَكُونُ. يَقْرَأُ - سَيَقْرَأُ. كَمَا قَالَ. مَا لَوْنُهُ؟ 

اَبْيَضُ, بَيْضَاءُ, بِيضٌ. اَسْوَدُ, سَوْدَاءُ, سُودٌ

اَحْمَرُ, حَمْرَاءُ, حُمْرٌ. شَكَرَ (و) شُكْرٌ

\_\_\_\_\_\_\_\_\_\_\_\_\_\_\_\_\_\_\_\_\_\_

جَرَى خَالِدٌ فِى سَاحَةِ الْمَدْرَسَةِ فَاِذَا بِقَلَمٍ, قَالَ: قَلَمٌ, قَلَمٌ مَنْ؟ فَكَّرَ خَالِدٌ فِيمَا يَفْعَلُ بِهِ فَدَخَلَ الْفَصْلَ وَ ذَهَبَ اِلَى السَّبُّورَةِ وَ أَخَذَ الطَّبَاشِيرَ وَ كَتَبَ عَلَيْهَا الْعِبَارَةَ الآتِيَةَ "اِنِّى وَجَدْتُ قَلَمَ حِبْرٍ فِى الْحَوْشِ, قَلَمُ مَنْ؟". قَرَاَ عَاصِمٌ الْعِبَارَةَ عَلَى السَّبُّورَةِ وَ سَأَلَ خَالِدًا عَنِ الْقَلَمِ. فَقَالَ لَهُ خَالِدٌ: هَلْ هُوَ قَلَمُكَ؟ قَالَ: نَعَمْ, قَالَ: اِذًا فَمَا لَوْنُهُ؟ هُوَ اَبْيَضُ وَ فِيهِ حِبْرٌ اَحْمَرُ. فَنَظَرَ فَاِذَا هُوَ كَمَا قَالَ. فَأَخَذَ عَاصِمٌ الْقَلَمَ وَ شَكَرَ خَالِدًا. قَالَتِ الْمُعَلِّمَةُ لِتَلاَمِيذِهَا: اَلْيَوْمَ نَخْرُجُ فِى رِحْلَةٍ حَوْلَ الْمَدِينَةِ, وَ فِى الاُسْبُوعِ الْقَادِمِ سَنَخْرُجُ فِى رِحْلَةٍ اِلَى الْغَابَةِ اَوْ اِلَى الْحَدِيقَةِ جَنْبَ الْمَسْرَحِ اَوْ اِلَى مَصْنَعٍ مِنْ مَصَانِعِ الْمَدِينَةِ. وَ فِى الاُسْبُوعِ الْقَادِمِ خَرَجَتِ التَّلاَمِيذُ, كَمَا قَالَتِ الْمُعَلِّمَةُ, فِى رِحْلَةٍ اِلَى الْغَابَةِ.

\subsubsection{К уроку 26}
Этот фильм новый? Эта чернильница не твоя? Те чернила не хорошие? Бери мел, иди к доске и пиши на ней арабские слова. — Сейчас пойду. Написал? — Да, написал. А теперь читай их. Соль, молоко, хлеб и мясо на обеденном столе, а сахар, чай и кофе на полке. Кто хочет соль? (кому соль?). Мы едим и пьем за обеденным столом, а читаем и пишем за письменным столом. Мои книги и тетради на письменном столе, а твои книги и тетради на парте в портфеле. Кто посмотрел этот фильм? Кто хочет смотреть этот фильм? Я смотрел этот фильм, а теперь не хочу смотреть его.

\subsection{الدَّرْسُ السَّابِعُ وَ الأَرْبَعُونَ 47}
\  \includegraphics[width=0.6874in,height=0.8957in]{images/MuhammadBagauddinprettified-img149.png}   \includegraphics[width=0.9374in,height=1.052in]{images/MuhammadBagauddinprettified-img150.png}   \includegraphics[width=0.9272in,height=1.2291in]{images/MuhammadBagauddinprettified-img151.png} 

طَابِعٌ, طَوَابِعُ. ظَرْفٌ, ظُرُوفٌ. صُنْدُوقُ الْبَرِيدِ.

 \includegraphics[width=1.8646in,height=1.1252in]{images/MuhammadBagauddinprettified-img152.png}   \includegraphics[width=1.9689in,height=1.0728in]{images/MuhammadBagauddinprettified-img153.png}   \includegraphics[width=1.7811in,height=1.0626in]{images/MuhammadBagauddinprettified-img154.png} 

\ خِطَابٌ (ات). طَرِيقٌ, طُرُقٌ. حَجَرٌ, أَحْجَارٌ. بَرِيدٌ.

 \includegraphics[width=1.552in,height=1.3437in]{images/MuhammadBagauddinprettified-img155.png} 

سَاعِى الْبَرِيدِ. مَكْتَبُ الْبَرِيدِ. اِشْتَرَى. هَذَا هُوَ. هَيَّا.

هَيَّا نَجْلِسْ. لاَبُدَّ مِنْ... ثَقِيلٌ. وَ مَعَ ذَلِكَ. بَعِيدًا. غَيْرَ بَعِيدٍ.

كَيْفَ؟ تَرَكَ (و) تَرْكٌ. دَعَا (و) دَعْوَةٌ. سَدَّ (و) سَدٌّ.

\_\_\_\_\_\_\_\_\_\_\_\_\_\_\_\_\_\_\_\_\_\_\_\_\_\_\_\_\_\_\_\_\_\_

يُوجَدُ فِى شَارِعِنَا غَيْرَ بَعِيدٍ عَنْ دَارِنَا مَكْتَبُ بَرِيدٍ. عَمُّنَا سَاعِى الْبَرِيدِ. هَذَا هُوَ مَكْتَبُ الْبَرِيدِ. فِى مَكْتَبِ الْبَرِيدِ صُنْدُوقٌ لِلْخِطَابَاتِ. فِى مَكْتَبِ الْبَرِيدِ يَشْتَرِى النَّاسُ طَوَابِعَ الْبَرِيدِ وَ الظُّرُوفَ. كَتَبْتُ رِسَالَةً اِلَى صَدِيقِى مِنْ مُوسْكُو فَاشْتَرَيْتُ طَابِعَ بَرِيدٍ مِنْ مَكْتَبِ الْبَرِيدِ وَ ظَرْفًا وَ اَلْصَقْتُ الطَّابِعَ عَلَى الظَّرْفِ وَ وَضَعْتُ الرِّسَالَةَ فِى الظَّرْفِ ثُمَّ وَضَعْتُهُ فِى صُنْدُوقِ الْخِطَابَاتِ. كَانَ حَامِدٌ يَمْشِى فِى الطَّرِيقِ فَرَأَى فِى وَسَطِ الطَّرِيقِ حَجَرًا كَبِيرًا. نَظَرَ حَامِدٌ فَقَالَ: مَنْ وَضَعَ الْحَجَرَ فِى الطَّرِيقِ؟ لاَبُدَّ مِنْ رَفْعِهِ عَنِ الطَّرِيقِ. فَاَرَادَ اَنْ يَرْفَعَهُ وَ لَكِنَّهُ وَجَدَهُ ثَقِيلاً, فَقَالَ: كَيْفَ اَرْفَعَهُ؟ فَكَّرَ قَلِيلاً ثُمَّ تَرَكَ الْحَجَرَ وَ مَشَى وَ دَعَا اَصْحَابَهُ وَ قَالَ لَهُمْ: هُنَاكَ حَجَرٌ يَسُدُّ الطَّرِيقَ, هَيَّا نَذْهَبْ وَ نَرْفَعْهُ عَنْ طَرِيقِ النَّاسِ فَذَهَبُوا فَرَأَوْهُ فِى وَسَطِ الطَّرِيقِ. قَالَ خَلِيلٌ: هَذَا حَجَرٌ ثَقِيلٌ, اَلرَّجُلُ الْوَاحِدُ لاَ يَرْفَعُهُ وَ لَكِنَّنَا جَمِيعًا مَعَ ذَلِكَ نَرْفَعُهُ اِنْ شَاءَ اللَّهُ.

\subsubsection{К уроку 27}
Кто твой (ж.р.) муж? Кто твоя жена? Я со своей женой пошёл в театр. Театр находится в городе. Театр недалеко от площади. Перед площадью имеется большое здание, это школа, а справа от него парк, и в парке много разных деревьев и растений, красивых роз. Мой муж иностранец, он араб. Кто этот мужчина? — Он профессор. Откуда он? — Он из Египта. Египет арабская страна? — Да, Египет арабская страна. Египет большая страна? — Да, Египет большая страна. Позади нашего дома много деревьев. В библиотеке нашего отца много книг. Некоторые их них арабские, а некоторые — русские.

\subsection{اَلدَّرْسُ الثَّامِنُ وَ اْلأَرْبَعُون 48}
\  \includegraphics[width=1.25in,height=0.9689in]{images/MuhammadBagauddinprettified-img156.png}   \includegraphics[width=1.5937in,height=1.0209in]{images/MuhammadBagauddinprettified-img157.png}   \includegraphics[width=0.3752in,height=1.0728in]{images/MuhammadBagauddinprettified-img158.png} 

مِحْرَاثٌ, مَحَارِيثُ. جَرَّارَةٌ (ات). فِرْجَارٌ. 

جَارٌ, جِيرَانٌ. عَمَلٌ, اَعْمَالٌ. نَشَّافَةٌ (ات). نَظَافَةٌ.

بِسُرْعَةٍ. غَيْرُ كَبِيرٍ. غَيْرُ نَظِيفٍ. فِكْرَةٌ. كُلُّ وَاحِدٍ.

 \includegraphics[width=1.75in,height=1.0937in]{images/MuhammadBagauddinprettified-img159.png}   \includegraphics[width=2.1563in,height=1.3752in]{images/MuhammadBagauddinprettified-img160.png} 

مِسْطَرَةٌ, مَسَاطِرُ. غَيْطٌ, غِيطَانٌ. وَالِدٌ. مُهِمٌّ. 

آخِرٌ, أَوَاخِرُ. وَصَلَ (ى) وُصُولٌ. مَرَّ (و) مُرُورٌ. 

جَرَّ (و) جَرٌّ. حَرَثَ (و) حَرْثٌ. حَدَثَ (و) حُدُوثٌ. 

هَلْ حَدَثَ مِنْ شَىْءٍ؟ مَا حَدَثَ شَىْءٌ. دَارَ اِلَى الْيَمِينَ. 

دَارَ اِلَى الْوَرَاءِ. كَعَادَتِهِ. كَنَّاسٌ (ون). نَظَّفَ.

\_\_\_\_\_\_\_\_\_\_\_\_\_\_\_\_\_\_\_\_\_\_\_\_\_\_\_

لاَبُدَّ لِلتِّلْمِيذِ مِنَ النَّشَّافَةِ وَ الْمِسْطَرَةِ وَ الْفِرْجَارِ وَ الْمِمْحَاةِ. نَظَرَ اَحْمَدُ فِى الصَّبَاحِ كَعَادَتِهِ اِلَى الشَّارِعِ فَرَآهُ غَيْرَ نَظِيفٍ وَرَأَى فِيهِ اَوْرَاقًا هُنَا وَ هُنَاكَ, فَقَالَ: شَارِعُنَا كُلَّ يَوْمٍ نَظِيفٌ وَ هُوَ الْيَوْمَ غَيْرُ نَظِيفٍ فَمَاذَا حَدَثَ وَ اَيْنَ الْكَنَّاسُ الْيَومَ؟ لِمَاذَا شَارِعُنَا الْيَوْمَ غَيْرُ نَظِيفٍ؟ قَالَ وَالِدُ اَحْمَدَ: ِلأَنَّ الْكَنَّاسَ الْيَوْمَ مَرِيضٌ. فَقَالَ اَحْمَدُ: وَ مَاذَا نَعْمَلُ؟ قَالَ الْوَالِدُ: عِنْدِى فِكْرَةٌ, قِيلَ لَهُ: وَ مَا هِىَ؟ قَالَ: كُلُّ وَاحِدٍ مِنَّا وَ مِنْ جِيرَانِنَا يُنَظِّفُ اَمَامَ بَيْتِهِ وَ لاَيَرْمِى الْوَرَقَ فِى الشَّارِعِ. قُمْ يَا اَحْمَدُ, مُرَّ عَلَى جِيرَانِنَا وَ قُلْ لَّهُمْ هَذِهِ الْفِكْرَةَ, ثُمَّ قَالَ وَالِدُ اَحْمَدَ: اَلنَّظَافَةُ مُهِمَّةٌ وَ عَمَلُ الْكَنَّاسِ مُهِمٌّ, وَ النَّظَافَةُ مِنَ اْلإِسْلاَمِ. كُلُّ وَاحِدٍ لَهُ عَمَلٌ مُهِمٌّ وَ كُلُّ اْلأَعْمَالِ مُهِمَّةٌ. اَلْجَرَّارَةُ تَجُرُّ الْمِحْرَاثَ وَ تَحْرُثُ اْلأَرْضَ. أَخَذَ اَبِى الْجَرَّارَةَ فِى الصَّبَاحِ وَ ذَهَبَ بِهَا اِلَى الْغَيْطِ لِيَحْرُثَهُ, وَصَلَتِ الْجَرَّارَةُ اِلَى آخِرِ الْغَيْطِ فَدَارَتْ اِلَى الْوَرَاءِ وَ رَجَعَتْ وَ الْمِحْرَاثُ دَارَ مَعَهَا وَ رَجَعَ. الْجَرَّارَةُ تَحْرُثُ اْلأَرْضَ بِسُرْعَةٍ.

\subsubsection{К уроку 28}
От кого это письмо? — Это письмо от моего брата. Где находится твой брат? — Он сейчас находится в Москве. Что он там делает? - Он там учится в университете. У него есть жена? — Нет, у него нет жены. А у тебя есть жена? — Да, у меня есть жена и дети. Они все ходят в школу. Это твой старший брат или младший брат? — Это мой младший брат. Ты написала письмо своей матери? — Нет, не написала ещё. А когда напишешь? — Напишу сегодня вечером. Бери сейчас бумагу и ручку, садись на стул и напиши письмо своей матери о нашей жизни, о жизни крестьян на полях и о жизни рабочих на фабриках. Мой костюм новый. Твой костюм старый. Когда ты вышел из дома, куда ты пошёл? — Я пошёл в мечеть. Ты получил моё письмо? Ты работала на этой фабрике? Ты училась в том университете? Ты узнала того человека?

\subsection[الدَّرْسُ التَّاسِعُ وَ اْلأَرْبَعُونَ 49]{الدَّرْسُ التَّاسِعُ وَ اْلأَرْبَعُونَ 49}
\  \includegraphics[width=1.2602in,height=1.1457in]{images/MuhammadBagauddinprettified-img161.png}   \includegraphics[width=0.8126in,height=1.2602in]{images/MuhammadBagauddinprettified-img162.png} 

عُصْفُورٌ, عَصَافِيرُ. قَفْصٌ, أَقْفَاصٌ. 

قَوْلٌ, أَقْوَالٌ. شَىْءٌ, أَشْيَاءُ. هَوَاءٌ. حُرِّيَّةٌ. بُسْتَانِىٌّ. 

وَقَعَ الطَّيْرُ عَلَى غُصْنٍ. هُوَ لاَ يَفْهَمُ شَيْئًا. أَمْسَكَ. اَطْلَقَ. غَرَّدَ. أَخْبَرَ. أَجَابَ. وَحْشٌ, وُحُوشٌ. سَقَطَ (و) سُقُوطٌ. 

صَاحَ (ى) صِيَاحٌ. حَبَسَ (ى) حَبْسٌ. جَاءَ (ى) مَجِئٌ. أَحَدٌ. هَلْ جَاءَ اَحَدٌ. لَمْ يَجِئْ اَحَدٌ. كُتُبِىٌّ. يَوْمًا.

\_\_\_\_\_\_\_\_\_\_\_\_\_\_\_\_\_\_\_\_\_\_\_\_\_\_\_\_\_\_\_

كَانَ خَالِدٌ يَلْعَبُ يَوْمًا تَحْتَ شَجَرَةٍ فَسَقَطَتْ عُصْفُورَةٌ عَلَى اْلأَرْضِ أَمَامَ خَالِدٍ, صَاحَ خَالِدٌ: عُصْفُورَةٌ! عُصْفُورَةٌ! فَأَمْسَكَ الْعُصْفُورَةَ وَ جَرَى اِلَى أُمِّهِ وَ قَالَ: أُمِّى, أُمِّى, عُصْفُورَةٌ جَمِيلَةٌ, هَلْ عِنْدَكِ قَفَصٌ؟ هَاتِى الْقَفَصَ. قَالَتِ اْلأُمُّ: قَفَصٌ؟ أَتُحِبُّ يَا خَالِدُ اَنْ يَحْبِسَكَ أَحَدٌ؟ فَكَّرَ خَالِدٌ فِى قَوْلِهَا وَ رَجَعَ اِلَى الشَّجَرَةِ وَ اَطْلَقَ الْعُصْفُورَةَ. فَطَارَتِ الْعُصْفُورَةُ وَ وَقَعَتْ عَلَى غُصْنٍ وَ غَرَّدَتْ فَجَاءَتْ أُمُّ الْعُصْفُورَةِ فَطَارَتَا فِى الْهَوَاءِ بَعِيدًا عَنْهُ ثُمَّ رَجَعَ خَالِدٌ اِلَى أُمِّهِ وَ اَخْبَرَهَا بِمَا فَعَلَ. فَرِحَتِ اْلأُمُّ بِمَا فَعَلَ خَالِدٌ وَ قَالَتْ: اِنَّ الْحُرِّيَّةَ يُحِبُّهَا الْجَمِيعُ, يُحِبُّهَا الطَّيْرُ فِى الْهَوَاءِ وَ الْوَحْشُ فِى الْغَابَةِ وَ السَّمَكُ فِى الْمَاءِ وَ اْلإِنْسَانُ فِى الْبِلاَدِ. كَانَ بُسْتَانِيٌّ فِى بُسْتَانِهِ يَسْقِى اْلأَشْجَارَ وَ الْوَرْدَ وَ الزَّرْعَ. مَرَّ التَّلاَمِذَةُ بِهِ وَ وَقَفُوا عِنْدَهُ وَ سَأَلُوهُ عَنِ اْلأَشْجَارِ وَ الْوَرْدِ وَ عَنْ أَشْيَاءَ أُخْرَى فَأَجَابَ الْبُسْتَانِىُّ عَلَى أَسْئِلَتِهِمْ. بَاعَ الْكُتُبِيُّ فِى هَذِهِ السَّنَةِ كُتُبًا كَثِيرَةً.

\subsubsection{К уроку 29}
Руслан, слезай с его машины и садись на мою машину. Ты куда едешь? — Я еду в город. Я тоже еду в город, возьми меня. Открой окно. Не открывай дверь. Откройте окна. Не открывайте двери. На улице холодно, надень костюм. Рашид, иди к доске, бери тряпку и вытри её, потом бери мел и напиши на ней эти русские слова. Кто понимает смысл этого слова? Рамазан, ты понимаешь? — Да, я понимаю. А ты, Самира, понимаешь? — Нет, я не понимаю. Ты учишься в этом университете? — Нет, я учусь не в этом университете (بل) я учусь в другом университете. Твоя сестра работает на этой фабрике? — Нет, (بل) она работает на другой фабрике. Каков ( مَا ) смысл этого слова? Объясни его мне, я его не понимаю. Я не знаю арабского языка, я не был в арабских странах. Я знаю только свой язык, аварский, и русский язык, других языков я не знаю. Какие языки ты знаешь? — Я знаю много языков.

\subsection[اَلدَّرْسُ الْخَمْسُونَ 50 ]{اَلدَّرْسُ الْخَمْسُونَ 50 }
 \includegraphics[width=1.7291in,height=1.3126in]{images/MuhammadBagauddinprettified-img163.png}   \includegraphics[width=1.3543in,height=1.0728in]{images/MuhammadBagauddinprettified-img164.png}   \includegraphics[width=0.698in,height=1.0626in]{images/MuhammadBagauddinprettified-img165.png} 

\ رُمَّانٌ. حَمَامٌ. فُولٌ. 

أُسْرَةٌ, أُسَرٌ. خَالٌ, أَخْوَالٌ. مَرْعًى, مَرَاعٍ. رِيفٌ, أَرْيَافٌ. وِعَاءٌ, أَوْعِيَةٌ. زُبْدٌ. شَحْمٌ. خُضَرٌ. بُرْتُقَالٌ. مَغْلِىٌّ. مَاْكُولاَتٌ. جَزَّارٌ (ون). اَخْضَرُ, خَضْرَاءُ, خُضْرٌ. يُبَاعُ. حَصَلَ عَلَى... (و) حُصُولٌ. حَلَبَ (و) حَلْبٌ. 

رَعَى (ا) رَعْىٌ. فُولٌ اَخْضَرُ. تَفَرَّجَ. فَرْدٌ, اَفْرَادٌ. 

اَفْرَادُ اْلأُسْرَةِ.

\_\_\_\_\_\_\_\_\_\_\_\_\_\_\_\_\_\_\_\_\_\_\_\_\_\_\_\_\_\_\_\_\_\_

فِى الصَّبَاحِ جَلَسَتِ اْلأُسْرَةُ وَرَاءَ الْمَائِدَةِ لِلْفُطُورِ وَ اَمَامَ كُلِّ وَاحِدٍ مِنْ أَفْرَادِ اْلأُسْرَةِ كُوبُ لَبَنٍ مَغْلِىٍّ. شَرِبَتْ اِنْعَامُ اللَّبَنَ وَ سَأَلَتْ عَارِفًا: خَالِى, مِنْ اَيْنَ تَشْتَرِى هَذَا اللَّبَنَ؟ ضَحِكَ عَارِفٌ وَ قَالَ: نَحْنُ هُنَا فِى الرِّيفِ لاَ نَشْتَرِى اللَّبَنَ, نَحْنُ نَحْصُلُ عَلَى اللَّبَنِ مِنَ الْبَقَرَةِ وَ نَعْمَلُ مِنَ اللَّبَنِ الْجُبْنَ وَ الزُّبْدَ. فِى الصَّبَاحِ تَخْرُخُ الْبَقَرَةُ اِلَى الْمَرْعَى فَتَرْعَى فِيهِ اِلَى الْمَسَاءِ, وَ فِى الْمَسَاءِ تَرْجِعُ مِنَ الْمَرْعَى فَتَحْلُبُهَا زَوْجَتِى وَ يَنْزِلُ اللَّبَنُ فِى وِعَاءٍ كَبِيرٍ. الْيَوْمَ سُوقُ الْقَرْيَةِ. اَلسُّوقُ جَنْبَ الْقَرْيَةِ. مَشَى عُمَرُ فِى السُّوقِ يَتَفَرَّجُ. فِى السُّوقِ نَاسٌ كَثِيرٌ, نَاسٌ تَبِيعُ وَ نَاسٌ تَشْتَرِى. هَذِهِ فَلاَّحَةٌ تَبِيعُ الْحَمَامَ, وَ هَذِهِ فَلاَّحَةٌ تَبِيعُ الْجُبْنَ وَ الْبَيْضَ وَ غَيْرُهَا مِنَ الْمَأْكُولاَتِ وَ اَمَامَ كُلِّ وَاحِدَةٍ نَاسٌ تَشْتَرِى. هَذَا فَلاَّحٌ يَبِيعُ الْخُضَرَ وَ هُنَاكَ آخَرُ يَبِيعُ الْبُرْتُقَالَ وَ الرُّمَّانَ وَ هُنَاكَ ثَالِثٌ يَبِيعُ السَّمَكَ وَ هَذَا جَزَّارٌ يَبِيعُ اللَّحْمَ وَ الشَّحْمَ وَ اَمَامَ كُلِّ وَاحِدٍ نَاسٌ تَشْتَرِى. هُنَا يُبَاعُ الْفُولُ اْلأَخْضَرُ.

\subsubsection{К уроку 30}
Вчера шёл дождь. Вчера было тепло, а сегодня холодно. Сегодня идёт снег, а что завтра пойдёт? Кто это знает? Кто знает, что завтра пойдет? В школьном саду финики, инжир и оливы. Я спросил Зейда: „Где ты был?" Он мне сказал: „Я был у учителя". Потом я спросил его: „Зачем ты пошёл к нему?" Он мне сказал: „Я пошёл к нему, чтобы спросить его о новых словах в сегодняшнем уроке". В сегодняшнем уроке много новых слов. Дождь полезный? — Да, дождь полезный. А снег полезный? — Да, снег тоже полезный. Кто нарисовал это? Кто сказал: „мяу"? Почему ты сидишь и не играешь с ребятами? Почему ты стоишь и не читаешь уроки? Почему ты не пишешь мне письма. Что я тебе сказал? Розы, деревья и посевы хогят дождя, почему ты их не поливаешь? Откуда идёт дождь? Откуда идёт снег? Нарисуй мне площадь и на площади мальчика с мячом. Нарисуй мне деревья и розы под ними.

\subsection{الدَّرْسُ الْحَادِى وَ الْخَمْسُونَ 51}
يَوْمُ الْجُمُعَةِ. صَلاَةُ الْجُمُعَةِ. صَلَّى الْجُمُعَةَ. جَمَاعَةٌ. صَلاَةُ الْجَمَاعَةِ. صَلَّى مَعَ الْجَمَاعَةِ. صَلَّى بِالنَّاسِ. رَتَّبَ 

 \includegraphics[width=1.3854in,height=1.6043in]{images/MuhammadBagauddinprettified-img166.png}   \includegraphics[width=0.9898in,height=1.6665in]{images/MuhammadBagauddinprettified-img167.png} 

\ اِمَامٌ, اَئِمَّةٌ. فَرَّاشٌ (ون). مُؤَذِّنٌ (ون). 

\  \includegraphics[width=1.2709in,height=1.1146in]{images/MuhammadBagauddinprettified-img168.png}   \includegraphics[width=1.5in,height=1.1563in]{images/MuhammadBagauddinprettified-img169.png}   \includegraphics[width=1.6043in,height=1.1772in]{images/MuhammadBagauddinprettified-img170.png} 

مِكْنَسَةٌ, مَكَانِسُ. شَفَّاطَةُ الْغُبَارِ. جَامِعٌ, جَوَامِعُ. 

تَمَّ (ى) تَمَامٌ. تَمَّتِ الصَّلاَةُ. كَنَسَ (و) كَنْسٌ. 

حَضَرَ (و) حُضُورٌ. حَضَرَ الصَّلاَةَ. حَضَرَ الدَّرْسَ. 

قَدْ. قَدْ ذَهَبَ. قَدْ جَاءَ.

\_\_\_\_\_\_\_\_\_\_\_\_\_\_\_\_\_\_\_\_\_\_\_\_\_\_\_\_\_\_\_\_\_\_\_\_\_\_

قَالَ عُمَرُ ِلأَبِيهِ: يَا اَبِى خُذْنِى مَعَكَ, خُذْنِى مَعَكَ اِلَى الْجَامِعِ. اَنَا أُحِبُّ أَنْ أُصَلِّىَ فِى الْجَامِعِ, اَنَا قَدْ كَبُرْتُ وَ تَعَلَّمْتُ اَنْ أُصَلِّىَ. قَالَ وَالِدُ عُمَرَ: اَنَا مَسْرُورٌ بِكَ يَا وَلَدِى, اَنْتَ كَبُرْتَ, اَرَى, وَ أُحِبُّ اَنْ تُصَلِّىَ مَعَ الْجَمَاعَةِ فِى الْجَامِعِ, تَعَالَ مَعِى, الْجَامِعُ غَيْرُ بَعِيدٍ نُصَلِّى الْجُمُعَةَ. الْيَوْمَ يَوْمُ الْجُمُعَةِ. الْيَوْمَ يَذْهَبُ الرِّجَالُ جَمِيعًا اِلَى الْجَامِعِ لِصَلاَةِ الْجُمُعَةِ. خَرَجَ عُمَرُ مَعَ وَالِدِهِ وَ وَصَلَ اِلَى الْجَامِعِ. فِى الْجَامِعِ نَاسٌ كَثِيرٌ. فِيهِ رِجَالٌ كِبَارٌ وَ اَوْلاَدٌ صِغَارٌ وَ فِيهِ بَعْضُ النِّسَاءِ اَيْضًا قَدْ جَلَسْنَ فِى زَاوِيَةِ الْمَسْجِدِ. رَأَى عُمَرُ ابْنَ جَارِهِ طَارِقًا فَسَلَّمَ عَلَيْهِ وَ جَلَسَ جَنْبَهُ. صَلَّى طَارِقٌ وَ عُمَرُ الْجُمُعَةَ. فَلَمَّا تَمَّتِ الصَّلاَةُ رَجَعَ عُمَرُ اِلَى بَيْتِهِ مَسْرُورًا لِيُخْبِرَ أُمَّهُ بِمَا فَعَلَ فَقَالَ لَهَا: اَنَا ذَهَبْتُ اِلَى الْجَامِعِ يَا أُمِّى وَ صَلَّيْتُ الْجُمُعَةَ فِيهِ مَعَ الْجَمَاعَةِ, وَ قَدْ حَضَرَ الْجَمَاعَةَ نَاسٌ كَثِيرٌ. لِلْجَامِعِ اِمَامٌ وَ مُؤَذِّنٌ وَ فَرَّاشٌ. اْلإِمَامُ يُصَلِّى بِالنَّاسِ وَ الْمُؤَذِّنُ يَدْعُو النَّاسَ اِلَى الصَّلَوَاتِ فِى اَوْقَاتِهَا, وَ الْفَرَّاشُ يَكْنُسُ الْمَسْجِدَ وَ يُرَتِّبُهُ بِالْمِكْنَسَةِ وَ بِشَفَّاطَةِ الْغُبَارِ. بَعْدَ هَذَا, اِنْ شَاءَ اللَّهُ, اُصَلِّى كُلَّ جُمُعَةٍ فِى الْجَامِعِ كَالرِّجَالِ الْكِبَارِ.

\subsubsection[К уроку 31]{К уроку 31}
Каково знамя ( مَا هُوَ عَلَمُ ) вашей страны? — Знамя нашей страны

— знамя Ислама. Знамя Ислама развевается над мечетью. Знамёна Ислама развеваются и над домами города. Ты продаёшь свою лошадь?

— Да, я продаю свою лошадь, дом и уезжаю в город, чтобы жить там и чтобы учиться в университете. У меня есть барабан, а у моего брата барабана нет, у него только маленький мяч. После уроков мы выходим во двор. Я бью в барабан, а Муса бегает и играет в мяч, после этого мы возвращаемся в свою комнату и пьём немного кофе. Мы не бьём стаканы и не бьём чашки. Что вы посеяли на своём поле? Что они посеяли на своём поле? Что мы будем сеять на своём поле? Вы земледельцы, вы живёте в селе, а мы — рабочие и живём в городе. Я тебе говорю: "Играй". Я ему говорю: "Не играй", и он мне говорит: "Хорошо, не буду".

\subsection{الدَّرْسُ الثَّانِى وَ الخَمْسُونَ 52}
\  \includegraphics[width=1.2398in,height=1in]{images/MuhammadBagauddinprettified-img171.png}   \includegraphics[width=1.7602in,height=0.9374in]{images/MuhammadBagauddinprettified-img172.png}   \includegraphics[width=1.6665in,height=1.052in]{images/MuhammadBagauddinprettified-img173.png} 

خَرُوفٌ, خِرْفَانٌ. مَكِنَةٌ (ات). قَنَاةٌ, قَنَوَاتٌ. 

فَرَاشٌ. شَكْلٌ, أَشْكَالٌ. صُوفٌ, اَصْوَافٌ. 

صَوْتٌ, اَصْوَاتٌ. اَوَّلٌ, اَوَائِلُ. اَلإِثْنَانِ. عَجِيبٌ. 

سَمِينٌ, سِمَانٌ. بِرْسِيمٌ. طُولَ الْيَوْمِ. دَارَتِ الْمَكِنَةُ. 

جَرَى الْمَاءُ. رَوَى (ى) رَىٌّ. مَكِنَةُ الرَّىِّ. سَارَ (ى) سَيْرٌ.

\_\_\_\_\_\_\_\_\_\_\_\_\_\_\_\_\_\_\_\_\_\_\_\_\_\_\_\_\_\_\_\_\_\_\_

ذَهَبَ حَسَّانٌ فِى الصَّبَاحِ اِلَى الْغَيْطِ مَعَ أُخْتِهِ مَرْيَمَ لِيَتَفَرَّجَ عَلَى اْلأَشْجَارِ وَ الزُّرُوعِ وَ الْوَرْدِ. مَشَى حَسَّانٌ بَيْنَ الزُّرُوعِ وَ الشَّجَرِ وَ تَفَرَّجَ عَلَى الْفُولِ وَ الْبِرْسِيمِ وَ رَأَتْ مَرْيَمُ فَرَاشَاتٍ فِى الْحَقْلِ وَ جَرَتْ وَرَاءَهَا وَ قَالَتْ: أُنْظُرْ يَا حَسَّانُ, اَلْفَرَاشُ تَطِيرُ هُنَا وَ هُنَاكَ لَوْنُهَا جَمِيلٌ وَ شَكْلُهَا عَجِيبٌ. كَانَتْ مَرْيَمُ تَسِيرُ فِى الْحَقْلِ مَعَ حَسَّانٍ فَنَظَرَتْ اِلَى خَرُوفٍ يَرْعَى فِيهِ وَ قَالَتْ: خَرُوفٌ سَمِينٌ لَوْنُهُ اَبْيَضُ وَ صُوفُهُ طَوِيلٌ. سَمِعَتْ مَرْيَمُ تِكْ, تِكْ, تِكْ, تِكْ فَقَالَتْ: مَا هَذَا الصَّوْتُ يَا حَسَّانُ؟ قَالَ حَسَّانٌ: هَذِهِ مَكِنَةُ الرَّىِّ, اَلْمَكِنَةُ هُنَاكَ فِى اَوَّلِ الْغَيْطِ. تَعَالَىْ نَتَفَرَّجْ. جَرَتْ مَرْيَمُ مَعَ حَسَّانٍ وَ وَقَفَ اْلإِثْنَانِ اَمَامَ مَكِنَةِ الرَّىِّ. الْمَكِنَةُ تَدُورُ وَ تَدُورُ طُولَ الْيَوْمِ وَ الْمَاءُ يَخْرُجُ مِنْهَا وَ يَجْرِى فِى الْقَنَاةِ وَ يَرْوِى الزَّرْعَ. 

\subsubsection{К уроку 32}
Когда ты слушал арабскую передачу по радио? — Я её слушал вчера вечером. О чём была передача? — Передача была о жизни мусульман в мире. Ты мусульманин? — Слава Богу, я мусульманин. И твой товарищ тоже мусульманин? — Все мои товарищи в университете мусульмане. Мусульмане все — братья. Кто стоит на улице и помогает людям переходить улицу и двигаться по ней? — Регулировщик стоит и помогает людям и машинам двигаться по ней. Из чего птица делает своё гнездо? — Птица делает своё гнездо из соломы. Где птица собирает солому? — Птица собирает солому на поле. Раиса активная ученица. Она учится в школе, а когда возвращается домой, помогает своей матери. В наших садах много птиц и бабочек. Они летают с дерева на дерево и с растения на растение. Гнёзда птиц на деревьях. Кто сломал ветку нашего дерева? — Я знаю, кто её сломал.

\subsection{الدَّرْسُ الثَّالِثُ وَ الْخَمْسُونَ 53}
\  \includegraphics[width=0.9583in,height=0.8957in]{images/MuhammadBagauddinprettified-img174.png}   \includegraphics[width=1.3543in,height=1.052in]{images/MuhammadBagauddinprettified-img175.png} 

عُلْبَةٌ, عُلَبٌ. فَاكِهَةٌ, فَوَاكِهُ. مُرَبًّى, مُرَبَّيَاتٌ. مَحْفُوظَاتٌ. عُلَبُ مَحْفُوظَاتٍ. خُشَافٌ. عَدَدٌ كَبِيرٌ مِنْ... 

عَدَدٌ كَبِيرٌ مِنَ اْلأَشْجَارِ. حَفِظَ (ا) حِفْظٌ. حَفِظَ السَّمَكَ. 

خَدَمَ (ى) خِدْمَةٌ. نَفَعَ (ا) نَفْعٌ. فَتَحَ عُلْبَةَ الْمَحْفُوظَاتِ. 

قِسْمٌ, أَقْسَامٌ. شَهْرٌ, شُهُورٌ. فَرَحٌ. دَهْشَةٌ. كَانَ فِى فَرَحٍ.

\_\_\_\_\_\_\_\_\_\_\_\_\_\_\_\_\_\_\_\_\_\_\_\_\_\_\_\_\_\_\_\_\_\_\_

وَالِدُ حَازِمٍ يَعْمَلُ فِى الْمَصْنَعِ. هَذَا الْمَصْنَعُ يَحْفَظُ الطَّعَامَ وَ الْفَوَاكِهَ. يَحْفَظُ الطَّعَامَ فِى عُلَبٍ وَ يَحْفَظُ الْفَوَاكِهَ فِى عُلَبٍ. حَضَرَ وَالِدُ حَازِمٍ اِلَى الْمَدْرَسَةِ وَ قَالَ لِلْمُعَلِّمَةِ: لِمَاذَا لاَ تَزُورِينَ مَصْنَعَنَا مَعَ تَلاَمِيذِكِ؟ اَجَابَتِ الْمُعَلِّمَةُ: فِى الشَّهْرِ الْقَادِمِ سَنَزُورُ مَصْنَعَكُمْ إِنْ شَاءَ اللَّهُ. وَ فِى الشَّهْرِ الْقَادِمِ زَارَتِ الْمُعَلِّمَةُ مَعَ تَلاَمِيذِهَا الْمَصْنَعَ. كَانَ الْمَصْنَعُ كَبِيرًا وَ وَاسِعًا فَيهِ مَكِنَاتٌ كَثِيرَةٌ وَ فِيهِ اَقْسَامٌ كَثِيرَةٌ. قِسْمٌ يَحْفَظُ الْفَوَاكِهَ فَيَعْمَلُ مِنْهُ الْمُرَبَّى وَ الْخُشَافَ, وَ قِسْمٌ يَحْفَظُ الْخُضَرَ. وَ فِيهِ عَدَدٌ كَبِيرٌ مِنَ الْعُمَّالِ. زَارَ التَّلاَمِيذُ أَقْسَامَ الْمَصْنَعِ كُلَّهَا وَ تَفَرَّجُوا عَلَى عُلَبِ الْمَحْفُوظَاتِ. كَانَ التَّلاَمِيذُ فِى فَرَحٍ وَ دَهْشَةٍ. اَلطَّعَامُ يَدْخُلُ الْمَكِنَاتِ وَ يَخْرُجُ فِى عُلَبٍ وَ الْفَوَاكِهُ تَدْخُلُ الْمَكِنَاتِ وَ تَخْرُجُ فِى عُلَبٍ. اَلْمَكِنَاتُ تَدُورُ وَ الْعُمَّالُ يَدُورُونَ حَوْلَهَا وَ يَعْمَلُونَ. اَلْمَصَانِعُ تَخْدِمُ النَّاسَ وَ تَنْفَعُهُمْ. لاَبُدَّ لِدَوْلَةِ اْلإِسْلاَمِ وَ الْمُسْلِمِينَ مِنْ مَصَانِعَ وَ مَعَامِلَ. اُحِبُّ اَنْ اَشْرَبَ الشَّاىَ مَعَ الْمُرَبَّى لِلْفُطُورِ وَ اُحِبُّ اَنْ اَشْرَبَ الْخُشَافَ بَعْدَ الْغَدَاءِ. اِفْتَحْ عُلْبَةَ خُشَافٍ.

\subsubsection{К уроку 33}
Почему ты у меня спрашиваешь? Почему ты не спрашиваешь другого человека? Почему ты сейчас спрашиваешь? Почему вчера не спросил? На каком языке эта книга? — Эта книга на арабском языке. Ты знаешь арабский язык? — Да, знаю. Хорошо ты знаешь арабский язык? — Да, хорошо. Когда ты его изучил? — Когда учился в институте. Этот шкаф для арабских книг, а этот шкаф для персидских книг, а этот шкаф для газет и журналов. Где ты был перед обедом? Где ты был? После обеда куда поедешь? Что у тебя в руке? — У меня в руке сосуд. Что в сосуде? — В нём молоко. Ты слушаешься своего учителя? Ты слушаешься своего отца и своей матери? — Да я слушаюсь их всех. Какую книгу ты хочешь? Какой иностранный язык ты знаешь? Какой магазин около вас? В какой комнате ты живёшь?

\subsection{اَلدَّرْسُ الرَّابِعُ وَ الْخَمْسُونَ 54}
\  \includegraphics[width=1.2602in,height=1in]{images/MuhammadBagauddinprettified-img176.png}   \includegraphics[width=2.1665in,height=1.3646in]{images/MuhammadBagauddinprettified-img177.png} 

ثَعْلَبٌ, ثَعَالِبُ. حَظِيرَةٌ, حَظَائِرُ. ذَاتَ مَرَّةٍ. سَعِيدٌ, سُعَدَاءُ. ذَيْلٌ, اَذْيَالٌ. حِيلَةٌ, حِيَلٌ. عَمِلَ حِيلَةً. كَلاَمٌ. مِنْ فَضْلِكَ. مَكَّارٌ. عَالَجَ, يُعَالِجُ, عِلاَجٌ. مَرِضَ (ا) مَرَضٌ. 

بَحَثَ عَنْ... (ا) بَحْثٌ. أَغْلَقَ. هَا. هَا هُوَ الْبَيْتُ. 

مَنْ كَانَ يُرِيدُ. لِيَكْتُبْ. لِيَقْرَأْ. عَاشَ (ى) عَيْشٌ. 

شَكَا (و) شِكَايَةٌ. أَنَّ, يَئِنُّ, أَنِينٌ. أَشَارَ اِلَى...

\_\_\_\_\_\_\_\_\_\_\_\_\_\_\_\_\_\_\_\_\_\_\_\_\_\_\_\_\_\_\_\_\_\_\_

كَانَتِ الدَّجَاجَةُ فِى الْحَظِيرَةِ, كَانَتْ تَعِيشُ سَعِيدَةً مَعَ الدَّجَاجِ وَ الدِّيكِ. وَ ذَاتَ مَرَّةٍ مَرِضَتِ الدَّجَاجَةُ فَكَانَتْ تَقُولُ: آه آه اَنَا مَرِيضَةٌ. قَالَ الدِّيكُ: اَنَا اَبْحَثُ لَكِ عَنْ طَبِيبٍ. سَمِعَ الثَّعْلَبُ كَلاَمَ الدِّيكِ مَعَ الدَّجَاجَةِ فَفَرِحَ وَ قَالَ: اَنَا اَعْمَلُ حِيلَةً وَ اَدْخُلُ الْحَظِيرَةَ وَ آكُلُ الدَّجَاجَةَ وَ آكُلُ الدِّيك. لَبِسَ الثَّعْلَبُ ثِيَابَ طَبِيبٍ وَ مَشَى يَقُولُ: اَنَا طَبِيبٌ, اَنَا طَبِيبٌ, اَنَا أُعَالِجُ الْمَرْضَى, مَنْ كَانَ يُرِيدُ الْعِلاَجَ فَلْيَدْعُنِى, اَنَا طَبِيبٌ مَاهِرٌ. قَالَ الدِّيكُ: مِنْ فَضْلِكَ يَا طَبِيبُ تَعَالَ بِسُرْعَةٍ عِنْدَنَا مَرِيضَةٌ فِى الْبَيْتِ هِىَ تَشْكُو وَ تَئِنُّ. قَالَ الثَّعْلَبُ: اَيْنَ الْبَيْتُ؟ قَالَ الدِّيكُ: هَا هُوَ الْبَيْتُ غَيْرُ بَعِيدٍ وَ اَشَارَ اِلَى حَظِيرَتِهِ. سَارَ الدِّيكُ مَعَ "الطَّبِيبِ" وَ عِنْدَمَا وَصَلَ اِلَى الْحَظِيرَةِ عَرَفَ اَنَّهُ لَيْسَ بِطَبِيبٍ وَ اَنَّهُ الثَّعْلَبُ. دَخَلَ الدِّيكُ بِسُرْعَةٍ وَ اَغْلَقَ الْبَابَ وَ قَالَ: اَنَا رَأَيْتُ ذَيْلَكَ, اَنْتَ الثَّعْلَبُ الْمَكَّارُ, اَنْتَ الثَّعْلَبُ الْمَكَّارُ. عَرَفَ الدِيكُ حِيلَةَ الثَّعْلَبِ الْمَكَّارِ.

\subsubsection{К уроку 34}
Какой его дом? Какая твоя комната? В комнате

много мух, возьми хлопушку и выгони их ею. Это английская газета. В этой газете новые известия о жизни мусульман в мире. Ты умеешь читать и писать? — Да, умею. Кто тебя учил? — Я учился в школе. Ты сегодня ходил на реку ловить рыбу? — Нет, я сегодня не ходил туда, вчера тоже не ходил, я ходил туда позавчера, я пойду туда ещё послезавтра. Где вы продаёте рыбу? — Мы продаем рыбу на базаре. Когда вы обычно играете? — С утра мы идем в школу, после уроков возвращаемся домой, обедаем дома и после обеда выходим во двор, чтобы поиграть. Какие у вас новости? ( مَا هُوَ الْخَبَرُ الْجَدِيدُ عِنْدَكُمْ؟ ) — У нас много новостей. Где вы их слышали? — Мы каждый день читаем газеты и журналы и узнаём новости из них. Около нашего дома есть озеро, я и мой брат плаваем в озере в тёплый день, ловим рыбу, бегаем, играем, потом все садимся ( جَمِيعًا ) и читаем уроки.

\subsection{اَلدَّرْسُ الْخَامِسُ وَ الْخَمْسُونَ 55}
عُطْلَةٌ. عُطْلَةُ نِصْفِ السَّنَةِ. نِصْفٌ. بَعْدَ وَقْتٍ قَصِيرٍ. نُورٌ, اَنْوَارٌ. كَهْرَبَاءٌ. نُورُ الْكَهْرَبَاءِ. مَحَطَّةٌ (ات). 

 \includegraphics[width=2.1354in,height=1.3957in]{images/MuhammadBagauddinprettified-img178.png} 

\ مَحَطَّةُ الْكَهْرَبَاءِ. طَاهِرٌ. صَافٍ. مَاءٌ صَافٍ. 

كُلُّ عَامٍ وَ اَنْتُمْ بِخَيْرٍ! وَ اَنْتُمْ بِالْخَيْرِ وَ السَّعَادَةِ! 

حَكَى (ى) حِكَايَةٌ. اِنْتَهَى.

\_\_\_\_\_\_\_\_\_\_\_\_\_\_\_\_\_\_\_\_\_\_\_\_\_\_\_\_\_\_\_\_\_\_\_\_\_\_

اِنْتَهَتْ عُطْلَةُ نِصْفِ السَّنَةِ وَ رَجَعَ التَّلاَمِذَةُ اِلَى الْمَدْرَسَةِ. قَابَلَ زَيْدٌ اَصْحَابَهُ وَ سَلَّمَ عَلَيْهِمْ وَ فَرِحَ بِهِمْ فَرَدُّوا عَلَيْهِ السَّلاَمَ. وَ بَعْدَ وَقْتٍ قَصِيرٍ دَقَّ الْجَرَسُ وَ دَخَلَ التَّلاَمِذَةُ الْفَصْلَ ثُمَّ دَخَلَ الْمُعَلِّمُ. سَلَّمَ الْمُعَلِّمُ عَلَى التَّلاَمِذَةِ وَ سَلَّمُوا عَلَى الْمُعَلِّمِ ثُمَّ سَأَلَهُمُ الْمُعَلِّمُ: مَاذَا عَمِلْتُمْ فِى الْعُطْلَةِ؟ وَ اَيْنَ ذَهَبْتُمْ؟ لِيُخْبِرْ كُلُّ وَاحِدٍ مِنْكُمْ بِمَا عَمِلَ وَ مَا رَأَى وَ مَا سَمِعَ مِنَ اْلأَخْبَارِ. حَكَى زَيْدٌ عَنْ رِحْلَتِهِ اِلَى الرِّيفِ, حَكَى عَنِ الْغَيْطِ وَ مَكِنَةِ الرَّىِّ وَ عَنِ الْمِحْرَاثِ وَ الْجَرَّارَةِ وَ عَنِ الْمَاءِ الصَّافِى فِى الْقَنَاةِ وَ عَنْ نُّورِ الْكَهْرَبَاءِ وَ مَحَطَّةِ الْكَهْرَبَاءِ فِى الرِّيفِ, وَ عَنْ سُوقِ الرِّيفِ وُ عَنِ الْمَسْجِدِ فِى الرِّيفِ وَ كَيْفَ كَانَ يَذْهَبُ مَعَ اَوْلاَدِ الرِّيفِ الَى الْمَسْجِدِ لِيُصَلُّوا مَعَ الْجَمَاعَةِ. قَالَتِ التَّلاَمِذَةُ: رِحْلَةٌ جَمِيلَةٌ يَا زَيْدُ. وَ هَكَذَا حَكَى كُلٌّ مِنَ التَّلاَمِذَةِ بِمَا عَمِلَ فِى الْعُطْلَةِ. ثُمَّ قَالَ الْمُعَلِّمُ: كُلُّ عَامٍ وَ اَنْتُمْ بِخَيْرٍ يَا اَوْلاَدِى! - وَ اَنْتَ بِالْخَيْرِ وَ السَّعَادَةِ يَا مُعَلِّمَنَا.

\subsubsection{К уроку 35}
Кто ты? — Я студент. Где ты учишься? — Я учусь в университете. Кто она? — Она моя жена, она тоже студентка. Где находится ста­дион? — Стадион далеко отсюда, что ты хочешь? — Хочу пойти туда поиграть в футбол. Я люблю футбол. Мы каждый день играем в футбол на школьном стадионе. Стадион нашей школы новый и красивый. С праздником, Расул! — С праздником, Магомед! Куда ты идёшь? — Я иду на праздничный намаз в мечеть, а ты пойдёшь со мной туда? — Да, я тоже пойду туда. Какая у тебя религия? — Моя религия — Ислам. Ты любишь свою религию? — Да, я люблю свою религию Ислам и люблю джихад. Наш дом деревянный, и мы живём в деревянном доме. После окончания намаза пойдём к своим братьям для поздравления их с праздником? — Да, пойдём.

\subsection{الدَّرْسُ السَّادِسُ وَ الْخَمْسُونَ 56}
\  \includegraphics[width=1.1563in,height=0.9165in]{images/MuhammadBagauddinprettified-img179.png}   \includegraphics[width=1.6354in,height=0.9898in]{images/MuhammadBagauddinprettified-img180.png}   \includegraphics[width=0.9791in,height=0.9374in]{images/MuhammadBagauddinprettified-img181.png} 

مِطْرَقَةٌ, مَطَارِقُ. فَأْسٌ, فُؤُوسٌ. مِنْشَارٌ, مَنَاشِيرُ. 

\  \includegraphics[width=1.0311in,height=0.7811in]{images/MuhammadBagauddinprettified-img182.png}   \includegraphics[width=1.2291in,height=0.7602in]{images/MuhammadBagauddinprettified-img183.png}   \includegraphics[width=1.0937in,height=0.7291in]{images/MuhammadBagauddinprettified-img184.png} 

شَوْكَةٌ (ات). صَحْنٌ, صُحُونٌ. مِلْعَقَةٌ, مَلاَعِقُ. 

جَدٌّ, اَجْدَادٌ. جَدَّةٌ (ات). اَبَوَانِ. اَقَارِبُ. مُطِيعٌ. عَاصٍ. مُؤَدَّبٌ. غَيْرُ مُؤَدَّبٍ. كُنْ. لاَ تَكُنْ. شُورْبَةٌ. صَيْفٌ. خَشَبٌ, اَخْشَابٌ. مِسْمَارٌ, مَسَامِيرُ. اِلْتَفَتَ. اِحْتَرَمَ. دَائِمًا. 

\  \includegraphics[width=1.948in,height=1.0626in]{images/MuhammadBagauddinprettified-img185.png} 

اَثْنَاءَ... اَثْنَاءَ الدَّرْسِ. نَجَّارٌ (ون). آلَةٌ (ات). 

غَسَلَ (ى) غَسْلٌ. قَطَعَ (ا) قَطْعٌ. نَشَرَ (و) نَشْرٌ.

\_\_\_\_\_\_\_\_\_\_\_\_\_\_\_\_\_\_\_\_\_\_\_\_\_\_\_\_\_\_\_\_\_\_\_\_

أَخَذْتُ الشَّوْكَةَ وَ أَكَلْتُ طَعَامِى بِهَا. خُذِ الْمِلْعَقَةَ وَ اشْرَبْ شُورْبَتَكَ بِهَا. َاْلأُمُّ تَطْبُخُ الطَّعَامَ فِى الْمَطْبَخِ وَ بِنْتُهَا تُسَاعِدُهَا فِى ذَلِكَ وَ تَغْسِلُ الصُّحُونَ بَعْدَ اْلأَكْلِ. هَلْ غَسَلْتِ الشَّوْكَاتِ وَ الْمَلاَعِقَ وَ الصُّحُونَ يَا بِنْتِى؟ اِحْتَرِمْ مُعَلِّمَكَ دَائِمًا وَ احْتَرِمْ اَبَاكَ وَ اُمَّكَ وَ جَدَّكَ وَ جَدَّتَكَ وَ اِخْوَتَكَ وَ اَخَوَاتِكَ. اُنْظُرْ اَثْنَاءَ الدَّرْسِ اَمَامَكَ دَائِمًا وَ لاَ تَلْتَفِتْ يَمِينًا وَ لا شِمَالاً لاَ تَضْحَكُوا اَثْنَاءَ الدَّرْسِ وَ لاَ اَثْنَاءَ الصَّلاَةِ. اَلْفَأْسُ وَ الْمِطْرَقَةُ وَ الْمِنْشَارُ آلاَتُ النَّجَّارِ. بِالْفَأْسِ يَقْطَعُ النَّجَّارُ الْخَشَبَ وَ بِالْمِنْشَارِ يَنْشُرُهُ وَ بِالْمِطْرَقَةِ يَضْرِبُ الْمِسْمَارَ فِى الْخَشَبِ. اَلْمُعَلِّمُ يُحِبُّ التِّلْمِيذَ الْمُطِيعَ فَكُنْ مُطِيعًا. اَلْوَلَدُ الْعَاصِى لاَ يَحِبُّهُ اَحَدٌ: لاَ مُعَلِّمُهُ وَ لاَ اَبَوَاهُ وَ لاَ اَقَارِبُهُ. كُنْ مُؤَدَّبًا وَ مُطِيعًا وَ لاَ تَكُنْ عَاصِيًا ِلأَنَّ الْعَاصِىَ لاَ يُحِبُّهُ اَحَدٌ. اَيْنَ كُنْتَ فِى الصَّيْفِ؟ - كُنْتُ عِنْدَ اَقَارِبِى فِى الرِّيفِ.

\subsubsection{К уроку 36}
Когда ты поступил в институт? — Я поступил в институт в прошлом году. Ты хочешь поступить в институт? — В этом году не хочу; если Богу будет угодно, я поступлю в институт в будущем году. В эти дни ты ходил в школу? — -Я был болен, и поэтому вчера, позавчера и сегодня я не ходил в школу. Читай эту книгу, в ней много рассказов. Наш отец больной и слабый, поэтому он не выходит из дома. Кто тебя посетил сегодня утром? — Мой брат из другого города. А ты когда посетишь его? — Я посещу его в будущем месяце ( فِى الشَّهْرِ القَادِمِ ). Добрый день, Сайд. — Добрый день, Дауд. Ты был дома? — Да, я был дома. Это подарок от меня тебе. Где твои часы? Откуда они у тебя? Часы — подарок одного араба. У меня утренняя газета, в ней новости; ты будешь её читать? Мой дядя врач. Мои дяди врачи. Ты слышал этот рассказ? — Нет, не слышал Хочешь послушать? — Да, хочу. — Садись около меня.

\subsection{اَلدَّرْسُ السَّابِعُ وَ الْخَمْسُونَ 57 }
\  \includegraphics[width=0.6146in,height=0.6665in]{images/MuhammadBagauddinprettified-img186.png}   \includegraphics[width=0.7083in,height=0.8335in]{images/MuhammadBagauddinprettified-img187.png}   \includegraphics[width=1.5417in,height=0.6665in]{images/MuhammadBagauddinprettified-img188.png} 

كُوبِيكٌ (ات). رُوبِلٌ (ات). بَاصٌ (ات). 

وَاحِدٌ (1). اِثْنَانِ (2). ثَلاَثَةٌ (3). اَرْبَعَةٌ (4). 

خَمْسَةٌ (5). سِتَّةٌ (6). سَبْعَةٌ (7). ثَمَانِيَةٌ (8). 

\  \includegraphics[width=1.8854in,height=0.5in]{images/MuhammadBagauddinprettified-img189.png} 

تِسْعَةٌ (9). عَشَرَةٌ (10). سِكِّينٌ, سَكَاكِينٌ. 

صَحْنُ الْمَسْجِدِ. ضَيْفٌ, ضُيُوفٌ. غَالٍ. رَخِيصٌ. قِيمَةٌ. جَيْبٌ, جُيُوبٌ. بَاكِرًا. مُتَأَخِّرًا. أَقَامَ. تَخَلَّفَ عَنْ... كَمْ؟ بِكَمْ؟ يُسَاوِى. كَمْ يُسَاوِى؟ نَهَضَ (ا) نُهُوضٌ. نَهَضَ مِنْ نَّوْمِهِ. نَمْ. لاَ تَنَمْ. 

رَجُلٌ - رَجُلاَنِ. بَيْتٌ - بَيْتَانِ.

\_\_\_\_\_\_\_\_\_\_\_\_\_\_\_\_\_\_\_\_\_\_\_\_\_\_\_\_\_\_\_\_\_\_\_\_


\bigskip

وَاحِدٌ وَ وَاحِدٌ اِثْنَانِ. اِثْنَانِ وَ اِثْنَانِ اَرْبَعَةٌ. كَانَ فِى صَحْنِ الْمَسْجِدِ ثَلاَثَةُ رِجَالٍ يُصَلُّونَ. نَزَلَتْ مِنَ الْبَاصِ اَرْبَعُ فَتَيَاتٍ. تَخَلَّفَ عَنِ الدَّرْسِ خَمْسَةُ تَلاَمِذَةٍ وَ خَمْسُ تِلْمِيذَاتٍ. رَأَيْتُ ثَلاَثَةَ فَلاَّحِينَ يَزْرَعُونَ الْحِنْطَةَ فِى حُقُولِهِمْ. كَانَ عِنْدَنَا اَمْسِ ضَيْفٌ فَشَرِبَ سِتَّةَ فَنَاجِينَ شَايًا, وَ اَنَا شَرِبْتُ مَعَهُ سَبْعَةَ فَنَاجِينَ. فِى مَدِينَتِنَا يُوجَدُ ثَمَانِيَةُ مَسَاجِدَ. هَذَا الثَّوْبُ غَالٍ جِدًّا قِيمَتُهُ تِسْعَةُ رُوبِلاَتٍ وَ ذَاكَ الثَّوْبُ رَخِيصٌ قِيمَتُهُ رُوبِلاَنِ. كَمْ رُوبِلاً فِى جَيْبِكَ؟ - فِى جَيْبِى رُوبِلٌ وَاحِدٌ وَ كُوبِيكَانِ. كَمْ يَوْمًا تُقِيمُ عِنْدَنَا؟ - سَأُقِيمُ عِنْدَكُمْ عَشَرَةَ اَيَّامٍ. كَمْ اَقَمْتَ عِنْدَ اَقَارِبِكَ فِى الصَّيْفِ؟ اَقَمْتُ عِنْدَهُمْ شَهْرًا وَاحِدًا. بِكَمِ اشْتَرَيْتَ سِكِّينَكَ؟ - اِشْتَرَيْتُهُ بِأَرْبَعَةِ رُوبِلاَتٍ وَ نِصْفٍ. نَمْ بَاكِرًا وَ انْهَضْ مِنْ نَّوْمِكَ وَ اذْهَبْ اِلَى الدُّرُوسِ بَاكِرًا وَ لاَ تَتَخَلَّفْ عَنِ الدُّرُوسِ اَبَدًا. نَهَضْتُ مِنْ نَّوْمِى بَاكِرًا وَ نَهَضَ اَخِى مُتَأَخِّرًا. كَمْ يُسَاوِى هَذَا الْبَيْتُ؟ - يُسَاوِى غَالِيًا, ِلأَنَّ الْبُيُوتَ فِى الْمَدِينَةِ غَالِيَةٌ.

\subsubsection{К уроку 37}
Кто ты? — Я спортсмен. Ты любишь спорт? — Да, очень ( كَثِيرًا) люблю. Что находится на полу? — На полу ковёр, ковёр большой и широкий, и очень красивый. Мы каждый день сидим на этом ковре, читаем свои уроки, а после уроков пьём чай, кофе или молоко. Что на стене в твоей комнате? — На стене в моей комнате карта. Карта — географическая. Мой дядя рыболов. У него много удочек. И у меня тоже есть одна удочка. У тебя есть ножницы? Пол в моей комнате деревянный. Давай сядем сюда. Давай напишем письмо своему брату. Давай возьмём удочку и пойдём к реке ловить рыбу. На круглом столе ваза, а в вазе — красивые розы. Эти розы из нашего сада. Все студенты на городском стадионе, некоторые из них играют в волейбол, а некоторые сидят на скамейках и смотрят волейбол.

\subsection{اَلدَّرْسُ الثَّامِنُ وَ الْخَمْسُونَ 58}
\  \includegraphics[width=1.9272in,height=1.4689in]{images/MuhammadBagauddinprettified-img190.png}   \includegraphics[width=1.5417in,height=0.8335in]{images/MuhammadBagauddinprettified-img191.png} 

\ دِهْلِيزٌ, دَهَالِيزُ. سِينَمَا. مِنْ زَمَانٍ. نَادٍ, نَوَادٍ. كَيْفَ حَالُكُمْ؟ - طَيِّبٌ. اَشْكُرُكُمْ. اِلَى اللِّقَاءِ. - مَعَ السَّلاَمَةِ. حَاضِرٌ (ون). غَائِبٌ (ون). اَمَامِىٌّ. بِجَانِبِ... مُخْلِصٌ. بَعْدَ قَلِيلٍ. مَعًا. تَذَاكَرَ. اِكْرَامٌ. اِكْرَامًا لَهُ. عَلَى الْفَوْرِ. غُرْفَةُ الدَّرْسِ. مَحَلَّةٌ (ات). مَسْجِدُ الْمَحَلَّةِ. تَفَرَّقَ. يُعْرَضُ. ذَهَبَا. يَذْهَبَانِ.

\_\_\_\_\_\_\_\_\_\_\_\_\_\_\_\_\_\_\_\_\_\_\_\_\_\_\_\_\_\_\_

اَدَقَّ الْجَرَسُ؟ - نَعَمْ, دَقَّ مِنْ زَمَانٍ. اَيْنَ زَيْنَبُ؟ هِىَ فِى النَّادِى مَعَ صَدِيقَاتِهَا. كَيْفَ حَالُكُمْ؟ شُكْرًا, طَيِّبٌ, اَلْحَمْدُ لِلَّهِ, وَ كَيْفَ اَنْتُمْ؟ اَشْكُرُكُمْ طَيِّبٌ اَيْضًا. اِلَى الِّقَاءِ. - مَعَ السَّلاَمَةِ. مَحْمُدٌ لَيْسَ حَاضِرًا, هُوَ غَائِبٌ ِلأَنَّهُ مَرِيضٌ, وَ فَاطِمَةُ اَيْضًا غَائِبَةٌ ِلأَنَّ عِنْدَهَا اَعْمَالاً كَثِيرَةً فِى الْبَيْتِ. حَسَنٌ عَلَى الْمَقْعَدِ اْلأَمَامِىِّ بِجَانِبِ سَلِيمٍ. حَسَنٌ وَ سَلِيمٌ صَدِيقَانِ مُخْلِصَانَ يَذْهَبَانِ اِلَى الْمَدْرَسَةِ مَعًا وَ يَدْرُسَانِ فِى فَصْلٍ وَاحِدٍ وَ يَجْلِسَانِ عَلَى مَقْعَدٍ وَاحِدٍ وَ يَلْعَبَانِ بِالْكُرَةِ مَعًا وَ يَتَذَاكَرَانِ الدُّرُوسَ مَعًا. بَعْدَ قَلِيلٍ دَخَلَ الْمُعَلِّمُ فَقَامَ التَّلاَمِيذُ اِكْرَامًا لَهُ. خَرَجَ مَحْمُدٌ اِلَى اللَّوْحِ فَسَأَلَهُ الْمُعَلِّمُ عَنِ مَعْنَى كَلِمَةٍ عَرَبِيَّةٍ فَأَجَابَهُ مَحْمُودٌ عَلَى الْفَوْرِ. بَعْدَ انْتِهَاءِ الدَّرْسِ خَرَجَ الطُّلاَّبُ مِنْ غُرْفَةِ الدَّرْسِ اِلَى الدِّهْلِيزِ ثُمَّ تَفَرَّقُوا, فَذَهَبَ بَعْضُهُمْ اِلَى السِّينَمَا وَ بَعْضُهُمْ اِلَى مَسْجِدِ الْمَحَلَّةِ وَ رَجَعَ بَعْضُهُمْ اِلَى مَنَازِلِهِمْ. مَاذَا يُعْرَضُ فِى السِّينَمَا الْيَوْمَ؟ - يُعْرَضُ فِى السِّينَمَا الْيَوْمَ فِلْمٌ عَرَبِىٌّ جَدِيدٌ.

\subsubsection{К уроку 38}
Скажи мне, кто твой Бог? — Мой Бог — Бог. А кто твой Про­рок? — Мой Пророк — Мухаммад. Кто отец Мухаммада? — Отец Мухаммада — Абдулла. Кто дедушка Мухаммада? — Дедушка Мухаммада — Абдулмуталиб. А кто дядя Мухаммада? — Дядя Мухаммада — Абуталиб. Как зовут жену Пророка? — Её зовут Хадиджа. Как зовут мать Пророка? — Её зовут Амина. Скажи, где родился наш Пророк и где его похоронили? — Он родился в Мекке, а похоронили его в Медине. Мои деньги в сундуке. Сундук мой закрыт, на нём большой замок, ключ от него у меня. В этом году мы будем сажать в поле картошку, помидоры и другие овощи. Мы любим овощи. Как тебя зовут? — Меня зовут Гарун. Как зовут твоего брата? — Его зовут Муса. У тебя есть деньги? — У меня мало денег, и они в сундуке.

\subsection{اَلدَّرْسُ التَّاسِعُ وَ الْخَمْسُونَ 59}
\  \includegraphics[width=1.6146in,height=1.1563in]{images/MuhammadBagauddinprettified-img192.png}   \includegraphics[width=1.1354in,height=1.0835in]{images/MuhammadBagauddinprettified-img193.png} 

مِصْبَاحٌ, مَصَابِيحُ. كَهْرَبَائِىٌّ. مِصْبَاحٌ كَهْرَبَائِىٌّ. 

\  \includegraphics[width=0.8957in,height=0.7083in]{images/MuhammadBagauddinprettified-img194.png}   \includegraphics[width=1.1874in,height=1.3228in]{images/MuhammadBagauddinprettified-img195.png} 

مِنْضَدَةٌ, مَنَاضِدُ. مِنْضَدَةُ الْكُتُبِ. سِيَاسِىٌّ. أَدَبِىٌّ. وَطَنِىٌّ. ضَرُورِىٌّ. وُدِّىٌّ. عَلاَقَةٌ (ات). عَلاَقَاتٌ وُدِّيَّةٌ. 

صَحِيفَةٌ, صُحُفٌ. مَقَالَةٌ (ات). شَبَابٌ. صَدَاقَةٌ (ات). سَلاَمٌ. بِالْقُرْبِ مِنْ... لاَسِيَّمَا. كَذَلِكَ. سَاعَتَيْنِ. لِكَىْ... لِكَىْ يَعِيشَ. طَوِيلاً. لاَيَكُونُ اِلاَّ... لاَ يَفْهَمُهُ اِلاَّ هُوَ. غَيْرَأَنَّ... مَطْبُوعٌ. مَخْطُوطٌ. مُطَالَعَةٌ. قَاعَةُ الْمُطَالَعَةِ. نَيِّرٌ. مَشْغُولٌ (ون). اَمَّا...فَ... بَيْنَ... بَيْنَهُمْ. بَيْنَكُمْ. بَيْنَنَا. مَكَثَ (و) مَكْثٌ.

\_\_\_\_\_\_\_\_\_\_\_\_\_\_\_\_\_\_\_\_\_\_\_\_\_\_\_\_\_\_\_\_\_\_

الْكُتُبُ اْلأَدَبِيَّةُ فِى هَذِهِ الْمَكْتَبَةِ كَثِيرَةٌ وَ اَمَّا الْكُتُبُ السِّيَاسِيَّةُ فَقَلِيلَةٌ. اَلطَّالِبَاتُ يَجْلِسْنَ بِالْقُرْبِ مِنْ مِنْضَدَةِ الْكُتُبِ. عَلَى الْمِنْضَدَةِ عَدَدٌ كَبِيرٌ مِنَ الصُّحُفِ وَ الْمَجَلاَّةِ بِاللُّغَةِ الْوَطَنِيَّةِ. فِى صَحِيفَةِ الْيَوْمِ مَقَالَةٌ طَوِيلَةٌ عَنِ انْتِشَارِ اْلإِسلاَمِ بَيْنَ شُعُوبِ الْعَالَمِ لاَسِيَّمَا بَيْنَ الشَّبَابِ. اَلصَّدَاقَةُ وَ الْعَلاَقَاتُ الْوُدِّيَّةُ ضَرُورِيَّةٌ لِكَىْ تَعِيشَ الشُّعُوبُ فِى سَلاَمٍ. غَيْرَأَنَّ السَّلاَمَ لاَ يَكُونُ اِلاَّ مِنَ اْلإِسْلاَمِ. مَكَثَ الطُّلاَّبُ فِى الْمَكْتَبَةِ سَاعَتَيْنِ, ثُمَّ ذَهَبُوا اِلَى الْجَامِعَةِ وَ كَذَلِكَ الطَّالِبَاتُ مَا مَكَثْنَ طَوِيلاً. عِنْدَ اَبِينَا كَثِيرٌ مِنَ الْكُتُبِ الْمَطْبُوعَةِ وَ الْمَخْطُوطَةِ. فِى مَكْتَبَةِ مَدْرَسَتِنَا قَاعَةٌ لِلْمُطَالَعَةِ, وَ هِىَ وَاسِعَةٌ وَ نَيِّرَةٌ. فِيهَا طَاوِلاَتٌ وَ كَرَاسِىُّ وَ رُفُوفٌ لِلْكُتُبِ وَ عَلَى الطَّاوِلاَتِ مَصَابِيحُ كَهْرَبَائِيَّةٌ. اَلطُّلاَّبُ اْلآنَ فِى قَاعَةِ المُطَالَعَةِ, وَ هُمْ مَشْغُولُونَ بِالْقِرَاءَةِ وَ الْكِتَابَةِ.

\subsubsection{К уроку 39}
У нас есть конь. Он работает на поле, возит мешки, и мы садимся верхом на него. Конь — очень полезное животное. Дочь моя, возьми верёвку, привяжи коня и дай ему ячменя. У тебя есть коробка красок? — Да, у меня есть коробка красок. Бассейн в вашем дворе глубокий? — Да, очень глубокий. В нём много воды. Я плаваю в бассейне. Плавать я научился в детстве. Меня учил плавать мой старший брат. Дети, учитесь плавать, плавание очень полезно. Я клеил бумагу клеем. Зухра нарисовала рисунок, затем раскрасила его в разные цвета и наклеила на стену. Что ты делаешь в свободное вре­мя? — Обычно в свободное время я слушаю радиопередачи. Что в этих мешках? — В них пшеница. Эта верёвка длинная. Этой верёвкой мы привязываем осла, быка и корову.

\subsection{اَلدَّرْسُ السِّتُّونُ 60}
عِشْرُونَ - 20 ثَلاَثُونَ - 30 أَرْبَعُونَ - 40 

خَمْسُونَ - 50 سِتُّونَ - 60 سَبْعُونَ - 70 

ثَمَانُونَ - 80 تِسْعُونَ - 90 مِائَةٌ - 100 

مِائَتَانِ - 200 ثَلاَثُمِائَةٍ - 300 أَلْفٌ - 1000 

اَلْفَانِ – 2000 ثَلاَثَةُ آلاَفٍ - 3000 مِلْيُونٌ - 1000000 أَحَدَ عَشَرَ – 11 اِثْنَا عَشَرَ - 12 ثَلاَثَةَ عَشَرَ - 13 

مِائَةٌ وَ اَحَدَ عَشَرَ - 111. 

رَبِيعٌ. صَيْفٌ. خَرِيفٌ. شِتَاءٌ. سَاعَةٌ (ات). 

دَقِيقَةٌ, دَقَائِقُ. ثَانِيَةٌ, ثَوَانٍ. فَصْلٌ, فُصُولٌ. فُصُولُ السَّنَةِ. كُتَّابٌ, كَتَاتِيبُ.

\_\_\_\_\_\_\_\_\_\_\_\_\_\_\_\_\_\_\_\_\_\_\_\_\_\_\_\_\_

اَنَا مُعَلِّمٌ فِى الْكُتَّابِ وَ عِنْدِى عِشْرُونَ تِلْمِيذًا. اِشْتَرَيْتُ الْيَوْمَ ِلأَوْلاَدِى ثَلاَثِينَ قَلَمًا وَ أَرْبَعِينَ كُرَّاسًا وَ خَمْسِينَ رِيشَةً وَ خَمْسَ مَقَالِمَ. بِكَمِ اشْتَرَيْتَ كُلَّ ذَلِكَ؟ - اِشْتَرَيْتُهَا بِاَحَدَ عَشَرَ رُوبِلاً وَ تِسْعَةٍ وَ سِتِّينَ كُوبِيكًا. فِى الْيَوْمِ اَرْبَعٌ وَ عِشْرُونَ سَاعَةً وَ فِى السَّاعَةِ سِتُّونَ دَقِيقَةً وَ فِى الدَّقِيقَةِ سِتُّونَ ثَانِيَةً. فِى الشَّهْرِ أَرْبَعَةُ أَسَابِيعَ وَ فِى اْلأُسْبُوعِ سَبْعَةُ اَيَّامٍ: يَوْمُ الْجُمُعَةِ وَ يَوْمُ السَّبْتِ وَ يَوْمُ اْلأَحَدِ وَ يَوْمُ اْلإِثْنَيْنِ وَ يَوْمُ الثُّلاَثَاءِ وَ يَوْمُ اْلأَرْبَعَاءِ وَ يَوْمُ الْخَمِيسِ. فُصُولُ السَّنَةِ هِىَ: الرَّبِيعُ وَ الصَّيْفُ وَ الْخَرِيفُ وَ الشِّتَاءُ. فِى السَّنَةِ اِثْنَا عَشَرَ شَهْرًا: الْمُحَرَّمُ, صَفَرٌ, رَبِيعٌ الأَوَّلُ, رَبِيعٌ اْلآخِرُ, جُمَادَى اْلأُولَى, جُمَادَى الأُخْرَى, رَجَبٌ, شَعْبَانُ, رَمَضَانُ, شَوَّالٌ, ذُو الْقَعْدَةِ, ذُو الْحِجَّةِ. فِى مَكْتَبَتِنَا مِائَتَا كِتَابٍ وَ اَلْفَا مَجَلَّةٍ. اَلرَّبِيعُ فَصْلُ اْلأَمْطَارِ وَ الشِّتَاءُ فَصْلُ الثُّلُوجِ وَ الْخَرِيفُ فَصْلُ الْفَوَاكِهِ.

\subsubsection{К уроку 40}
Когда вы посетили зоопарк? Какие животные в зоопарке? Газель, лев, тигр и обезьяна — дикие животные. Сайда, покорми кур ячменём. Я кормил быков сеном. Ты кошку кормила хлебом? — Нет, ещё не кормила. Мой ребёнок в детском саду. Это интересный рассказ, я хочу услышать от тебя тот рассказ ещё. Я люблю интересные рассказы. Это хорошая книга, в ней интересные рассказы о диких животных: рассказ о львах, рассказ о тиграх и другие рассказы-Куда попала муха? — Муха попала в паутину. Тогда кто набросилс на муху? — На муху набросился паук. Почему ты смеёшься? Почему ты (ж.р.) плачешь? В моей комнате нет пауков. Я ткач и моя сестра тоже ткачиха. Мы ткём сети для ловли рыбы. Паук насекомое и муха тоже насекомое. Где находится зоопарк? Давай посетим его и возьмем младшего брата тоже с нами. Слушай, что я тебе скажу: сегодня праздник, сегодня зоопарк не работает, понял? – Да, понял. 

\subsection{الدَّرْسُ الْحَادِى وَ السِّتُّونَ 61}
 \includegraphics[width=1.3126in,height=1.3335in]{images/MuhammadBagauddinprettified-img196.png}   \includegraphics[width=1.3126in,height=1.3335in]{images/MuhammadBagauddinprettified-img197.png}   \includegraphics[width=1.1354in,height=1.4583in]{images/MuhammadBagauddinprettified-img198.png} 

خَيَّاطٌ (ون). خَيَّاطَةٌ (ات). كُرَةُ السَّلَّةِ. 

اَلَّّذِى, اَلَّذِينَ. اَلَّتِى, اَللاَّتِى. يُسَمَّى. دَوَاءٌ, اَدْوِيَةٌ. 

عَزَبٌ, اَعْزَابٌ. عَزَبَةٌ (ات). مَكْتُوبٌ, مَكَاتِيبُ. 

جُنَيْنَةٌ, جَنَائِنُ. اُسْتَاذَةٌ. فَرِيقٌ. لِبَاسٌ, اَلْبِسَةٌ. نَهَارًا. آنِفًا. مُنْذُ اُسْبُوعٍ. رَاجِعٌ (ون). خَاطَ (ى) خِيَاطَةٌ. لَقِىَ (ا) لِقَاءٌ. فَازَ (و) فَوْزٌ. اِسْتَعْمَلَ. أَعْطَى, يُعْطِى. 

شُفِىَ, يُشْفَى. اِذَا لَمْ تَفْعَلْ. 

\_\_\_\_\_\_\_\_\_\_\_\_\_\_\_\_\_\_\_\_\_\_\_\_\_\_\_\_\_\_\_\_

هَلْ رَأَيْتَ الضَّيْفَ الَّذى جَاءَنَا اَمْسِ نَهَارًا؟ هُوَ عَمُّنَا الَّذِى يَسْكُنُ فِى الرِّيفِ وَ قَدْ جَاءَنَا يَزُورُنَا. اَلْمَرْاَةُ الَّتِى تُعَلِّمُ التِّلْمِيذَاتِ تُسَمَّى مُعَلِّمَةً. اَلنِّسَاءُ اللاَّتِى يَخِطْنَ الثِّيَابَ يُسَمَّيْنَ خَيَّاطَاتٍ. اَلآلَةُ الَّتِى يَقْطَعُ بِهَا النَّجَّارُ الْخَشَبَ تُسَمَّى فَأْسًا وَ الآلَةُ الَّتِى يَسْتَعْمِلُهَا لِنَشْرِ الْخَشَبِ تُسَمَّى مِنْشَارًا. اِذَا لَمْ تَسْتَعْمِلِ الدَّوَاءَ الَّذِى اَعْطَاكَهُ الطَّبِيبُ لاَ تُشْفَى. شُفِىَ بِحَمْدِ اللَّهِ اَبُونَا اْلآنَ وَ كَانَ مَرِيضًا مُنْذُ شَهْرٍ. مَا أَخْبَارُ ابْنِكَ فِى الْجَامِعَةِ؟ - مَا تَسَلَّمْتُ مِنْهُ رِسَالَةً وَ مَا سَمِعْتُ عَنْهُ خَبَرًا مُنْذُ شَهْرَيْنِ. جَاءَتْنِى رِسَالَةٌ مِنْ مُعَلِّمِى الَّذِى اُحِبُّهُ كَثِيرًا. يُعْطَى الْمَرِيضُ الدَّوَاءَ لِيُشْفَى. يُقَالُ لِلرَّجُلِ الَّذِى لَيْسَتْ لَهُ زَوْجَةٌ عَزَبٌ وَ لِلْمَرْأَةِ الَّتِى لَيْسَ لَهَا زَوْجٌ عَزَبَةٌ. هَذِهِ الْمَرْأَةُ اُسْتَاذَةُ جَامِعَتِنَا وَ هِىَ الَّتِى فَتَحَتْ جَامِعَةً لِلْفَتَيَاتِ فِى الْمَدِينَةِ. الْفِتْيَانُ الَّذِينَ لَقِينَاهُمْ آنِفًا هُمْ طُلاَّبُ مَعْهَدِنَا وَ هُمُ اْلآنَ رَاجِعُونَ مِنَ الْمَلْعَبِ. كَانُوا يَلْعَبُونَ مَعَ فَرِيقِ الْمَعْهَدِ الآخَرِ فَفَازَ فَرِيقُنَا. اَاَنْتَ الَّذِى قَطَعْتَ الشَّجَرَ فِى جُنَيْنَتِنَا؟ اَهَذَا الَّذِى صَادَ السَّمَكَ فِى بُحَيْرَتِنَا؟ - نَعَم, هُوَ الَّذِى صَادَ السَّمَكَ فِى بُحَيْرَتِكُمْ. اَلْفَلاَّحُونَ هُمُ الَّذِينَ يَحْرُثُونَ اْلأَرْضَ وَ يَزْرَعُونَهَا.

\subsubsection{К уроку 41}
В зоопарке лев, тигр, волк, медведь, обезьяна а другие дикие животные. Что из мебели имеется в твоей квартире? — В моей квартире немного мебели: один круглый стол, две кровати, один диван, два стула. И на столе радиопримёник и телефонный аппарат. А на кухне один четырёхугольный обеденный стол, две скамейки. А что на полу? На полу два ковра. Где находится мечеть в вашем городе? — Мечеть находится на улице им. Али, в центре города. Твой дом на какой улице? — Мой дом на улице им. Умара. Ты сегодня ходил на базар продавать рыбу? — Нет, не ходил, у меня не было рыбы, всю рыбу я продал вчера. Почему ты бьёшь этого маленького мальчика? Не бей его, он же (اِنَّهُ ) плачет. Потому что он взял мой пенал и не

спросил меня. Ты знаешь, почему я здесь? — Да, знаю. Ты знаешь, почему сегодня лифт не работает? — Нет, не знаю. Если бы я был на твоём месте, я бил бы его. Если бы ты был на моём месте, ты бы плакал. Мой ученик маленький, но он старательный.

\subsection[اَلدَّرْسُ الثَّانِى وَ السِّتُّونَ 62]{اَلدَّرْسُ الثَّانِى وَ السِّتُّونَ 62}
 \includegraphics[width=1.0937in,height=1.5in]{images/MuhammadBagauddinprettified-img199.png}   \includegraphics[width=1.1665in,height=1.3957in]{images/MuhammadBagauddinprettified-img200.png}  

\ ذُرَةٌ. طَاحُونٌ, طَوَاحِينُ. 

 \includegraphics[width=1.8957in,height=1.3854in]{images/MuhammadBagauddinprettified-img201.png}   \includegraphics[width=1.2709in,height=0.9898in]{images/MuhammadBagauddinprettified-img202.png}   \includegraphics[width=1.5311in,height=1.2811in]{images/MuhammadBagauddinprettified-img203.png} 

\ مِحَشٌّ (ات). مِنْجَلٌ, مَنَاجِلُ. عُشْبٌ, اَعْشَابٌ. 

مَاشِيَةٌ, مَوَاشٍ. شُونَةٌ, شُوَنٌ. جَارَةٌ (ات). شَمْسٌ, شُمُوسٌ. حَادٌّ. كَهَامٌ. حُبُوبٌ. دَقِيقٌ. نُخَالَةٌ. اِشْتَغَلَ. 

اِشْتَغَلَ فِى الْحَقْلِ. اِشْتَغَلَ بِالْكُتُبِ. اِسْتَرَاحَ. كَدَّسَ. 

حَشَّ (و) حَشٌّ. حَصَدَ (و) حَصْدٌ. طَحَنَ (ا) طَحْنٌ. 

خَزَنَ (و) خَزْنٌ. نَقَلَ (و) نَقْلٌ. يَبِسَ (ا) يُبْسٌ. اَحَدَّ. يَحْصُلُ الدَّقِيقُ. اِذَا جَاءَ. حَتَّى يَجِئَ. رَدَّ (و) رَدٌّ.

\_\_\_\_\_\_\_\_\_\_\_\_\_\_\_\_\_\_\_\_\_\_\_\_\_\_\_\_\_

الْفَلاَّحُ يُحِشُّ الْحَشِيشَ بِالْمِحَشِّ ثُمَّ يَتْرُكُهُ فِى الشَّمسِ حَتَّى يَيْبَسَ فَاِذَا يَبِسَ يَنْقُلُهُ اِلَى الشُّونَةِ وَ يَخْزُنُهُ فِيهَا لِمَاشِيَتِهِ ِلأَيَّامِ الشِّتَاءِ. الْفَلاَّحَةُ تَحْصُدُ الزَّرْعَ بِالْمِنْجَلِ. يَاجَارَتِى هَلْ زَوْجُكِ فِى الْبَيْتِ؟ - نَعَمْ, وَ مَاذَا تُرِيدِينَ؟ - اُرِيدُ اَنْ اُعْطِيَهُ مِنْجَلِى لِيُحِدَّهُ فَاِنَّهُ كَهَامٌ جِدًّا لاَ يَقْطَعُ شَيْئًا, خُذِيهِ وَ قُولِى لَهُ لِيُحِدَّهُ, وَ اَعْطِينِى مِنْجَلَكِ ِلأَحْصُدَ بِهِ زَرْعِى الْيَوْمَ وَ فِى الْغَدِ اَرُدُّهُ اِلَيْكِ. - حَسَنًا, وَ لَكِنْ لاَ تَنْسَىْ اَنْ تَرُدِّيهِ. الْفَرَسُ يَأْكُلُ الْعُشْبَ وَ الشَّعِيرَ وَ النُّخَالَةَ. اَلدَّقِيقُ يَحْصُلُ مِنْ طَحْنِ الْقَمْحِ وَ الذُّرَةِ وَ الشَّعِيرِ وَ غَيْرِهَا مِنَ الْحُبُوبِ وَ مِنَ الدَّقِيقِ نَعْمَلُ الْخُبْزَ الَّذِى نَأْكُلُهُ. اَيْنَ طَحَنْتَ هَذَا الدَّقِيقَ؟ - طَحَنْتُهُ فِى طَاحُونِ عَلِىٍّ. اَلْفَلاَّحُونَ يَشْتَغِلُونَ فِى الصَّيْفِ كَثِيرًا: يَحْرُثُونَ اْلأَرْضَ وَ يَزْرَعُونَ الْحُبُوبَ وَ الْخُضْرَوَاتِ وَ يَحُشُّونَ الْحَشِيشَ وَ يُكَدِّسُونَهُ وَ يَحْصُدُونَ الزَّرْعَ ثُمَّ اِذَا جَاءَ الشِّتَاءُ يَسْتَرِيحُونَ اِلَى الرَّبِيعِ. فِى الشِّتَاءِ نَتَعَلَّمُ وَ فِى الصَّيْفِ نَسْتَرِيحُ.

\subsubsection{К уроку 42}
В вашей стране есть горы? — Да, в нашей стране есть горы, они высокие, и в них курорты. Я живу в горах, в деревне. Мой брат тоже жил с нами в горах, а теперь он не живёт с нами, он в городе-В каком году ты был на курорте? — Я был на курорте и в этом году, и в прошлом году, я почти каждый год езжу на курорт, что это за деревья? — Это ореховые деревья. Первый урок лёгкий, второй урок тоже лёгкий, а третий урок трудный. Когда ты вышел из комнаты, вошёл твой друг. Когда я вышел из школы, я застал учителя во дворе. Это твой друг? — Я его не знаю, кто он? Где мама? — Мам на кухне, готовит пищу на газовой плите. Ты молился? — Да, молился. Где ты молился? — Я молился в мечети. Кто ещё с тобой молился? — Со мной молился мой друг Тарик. Что мама делает на кухне? — Она готовит нам завтрак. Кто для вас готовит завтрак, обед, ужин? — Мама готовит, а когда она уходит в поле работать, готовит старшая сестра. Сегодня завтрак очень вкусный, кто готовил его. - Мама готовила его. Мама, спасибо тебе, ты накормила нас хорошо.

\subsection{اَلدَّرْسُ الثَّالِثُ وَ السِّتُّونَ 63}
\  \includegraphics[width=0.5728in,height=1.0102in]{images/MuhammadBagauddinprettified-img204.png}   \includegraphics[width=1.2709in,height=1.0626in]{images/MuhammadBagauddinprettified-img205.png}   \includegraphics[width=1.6043in,height=1.1354in]{images/MuhammadBagauddinprettified-img206.png} 

اِبْرِيقٌ, اَبَارِيقُ. صَابُونٌ. حَنَفِيَّةٌ (ات). 

\  \includegraphics[width=0.8228in,height=0.4791in]{images/MuhammadBagauddinprettified-img207.png}   \includegraphics[width=1.9063in,height=1.0728in]{images/MuhammadBagauddinprettified-img208.png}   \includegraphics[width=1.2189in,height=0.7602in]{images/MuhammadBagauddinprettified-img209.png} 

طَسْتٌ, طُسُوتٌ. مِنْشَفَةٌ, مَنَاشِفُ. فُرْشَةُ اْلأَسْنَانِ. 

\  \includegraphics[width=0.7709in,height=0.6457in]{images/MuhammadBagauddinprettified-img210.png}   \includegraphics[width=1.2811in,height=1.2398in]{images/MuhammadBagauddinprettified-img211.png}   \includegraphics[width=1.7083in,height=1.2811in]{images/MuhammadBagauddinprettified-img212.png} 

سِنٌّ, اَسْنَانٌ. مِشْجَبٌ, مَشَاجِبٌ. سَرِيرٌ, سُرُرٌ. 

وَجْهٌ, وُجُوهٌ. رَكْعَةٌ, رَكَعَاتٌ. وُضُوءٌ. تَوَضَّأَ. أَذَانٌ. فَجْرٌ. اَذَانُ الْفَجْرِ. نَشَّفَ. اِسْتَيْقَظَ, يَسْتَيْقِظُ. أَخَّرَ, يُؤَخِّرُ. مُعَلَّقٌ. صُبْحٌ. ظُهْرٌ. عَصْرٌ. مَغْرِبٌ. عِشَاءٌ.

\_\_\_\_\_\_\_\_\_\_\_\_\_\_\_\_\_\_\_\_\_\_\_\_\_\_\_\_\_\_

خُذِ الصَّابُونَ وَ هُنَاكَ الْمَاءُ فِى الْحَنَفِيَّةِ وَ اغْسِلْ وَجْهَكَ وَ يَدَيْكَ ثُمَّ اذْهَبْ اِلَى حُجْرَتِى وَ نَشِّفْ وَجْهَكَ وَ يَدَيْكَ بِالْمِنْشَفَةِ الْجَدِيدَةِ الْمُعَلَّقَةِ عَلَى الْمِشْجَبِ. اِسْتَيْقَظْتُ الْيَوْمَ بَاكِرًا فَسَمِعْتُ اَذَانَ الْفَجْرِ فَتَوَضَّأْتُ بِسُرْعَةٍ وَ رَكَضْتُ اِلَى الْمَسْجِدِ ِلأُصَلِّىَ الصُّبْحَ مَعَ الْجَمَاعَةِ. يَا وَلَدِى صُبَّ فِى اِبْرِيقِى مَاءً ِلأَتَوَضَّأَ. - نَعَم, يَا وَالِدِى, اَمَاءً بَارِدًا اَصُبُّ لَكَ اَمْ دَافِئًا؟ - بَلْ صُبَّ لِى مَاءً دَافِئًا. صَلاَةُ الْصُّبْحِ رَكْعَتَانِ. صَلاَةُ الظُّهْرِ اَرْبَعُ رَكَعَاتٍ. صَلاَةُ الْعَصْرِ اَرْبَعُ رَكَعَاتٍ. صَلاَةُ الْمَغْرِبِ ثَلاَثُ رَكَعَاتٍ. صَلاَةُ الْعِشَاءِ اَرْبَعُ رَكَعَاتٍ. تَوَضَّأَ عَبْدُ اللَّهِ فَنَشَّفَ وَجْهَهُ وَ يَدَيْهُ. صَلِّ جَمِيعَ الصَّلَوَاتِ فِى اَوْقَاتِهَا. وَ لاَ تُؤَخِّرْ صَلاَةً عَنْ وَقْتِهَا. اِسْتَعْمِلْ فُرْشَةَ اْلأَسْنَانِ دَائِمًا وَ نَظِّفْ بِهَا اَسْنَانَكَ قَبْلَ كُلِّ وُضُوءٍ. هَلْ اَنْتَ عَلَى الْوُضُوءِ؟ لاَ, لَسْتُ عَلَى الْوُضُوءِ. - اِذَنْ فَتَوَضَّأْ, هَذَا هُوَ اْلإِبْرِيقُ وَ الطَّسْتُ تَحْتَ السَّرِيرِ, عَلَيْنَا اَنْ نَحْضُرَ صَلاَةَ الْجَمَاعَةِ فِى الْمَسْجِدِ. اَيْنَ تَنَامُ؟ - اَنَامُ عَلَى سَرِيرِي فِى حُجْرَتِى. لِبَاسِى مُعَلَّقٌ عَلَى الْمِشْجَبِ فِى الدِّهْلِيزِ.

\subsubsection{К уроку 43}
Он рад. Они рады. Я рад. Мы рады. Меня били. Скажи мне, кто тебя бил, кто этот человек? Почему он тебя бил? Эта книга написана в прошлом году. Твой кофе выпили. Моё мясо съели, и я не знаю, кто его съел. Пора молиться, вставайте, ребята. Папа, где мы будем молиться? Где мы будем разбивать палатку? — Молиться мы будем под этим деревом, а палатку разбивать будем под теми высокими деревьями. Что имеется в твоём чемодане? — В нём игрушки для моих детей. Твой сын ходит в школу? — Нет, не ходит, он еше маленький. Он пойдёт в школу через год, а моя дочь уже выросла, и она пойдёт в школу в этом году. У неё есть тетради, книги, ручка и портфель; у неё ещё есть школьная форма. Мы сидим вокруг обеденного стола. Я вернулся из путешествия поездом утром. Отец всё ещё болен? — Да, он всё ещё болен. Он для намаза встаёт? Нет, он очень слаб, и поэтому для намаза он не встаёт. Когд встречаешь своих друзей, приветствуй их. Когда настаёт время намаза вставай для намаза. Это утренний поезд.

\subsection{الدَّرْسُ الرَّابِعُ وَ السِّتُّونَ 64}
 \includegraphics[width=1.2602in,height=0.3437in]{images/MuhammadBagauddinprettified-img213.png}   \includegraphics[width=1.0835in,height=0.7709in]{images/MuhammadBagauddinprettified-img214.png}   \includegraphics[width=1.3335in,height=1.2917in]{images/MuhammadBagauddinprettified-img215.png} 

مِبْرَاةٌ, مَبَارٍ. مِحْفَظَةُ نُقُودٍ. دِينَارٌ, دَنَانِيرُ. 

دِرْهَمٌ, دَرَاهِمُ. مَعْرِضٌ, مَعَارِضُ. يَا أَبَتِ. أَرْجُوكَ. 

لاَ ضَرُورَةَ لِذَلِكَ. بَدَلاً مِنْهُ. لَسْتُ بِحَاجَةٍ اِلَى... مُتَضَايِقٌ. يَكْفِينِى. جُمْلَةٌ, جُمَلٌ. قَضِيَّةٌ, قَضَايَا. أَمِينٌ, أُمَنَاءُ. أَمَانَةٌ. شَدِيدٌ. عَظِيمٌ. أَحْضَرَ, يُحْضِرُ. أَعْلَنَ, يُعْلِنُ, اِعْلاَنٌ. رَاجَعَ, يُرَاجِعُ. اِسْتَحَقَّ. نَسِىَ (ا) نِسْيَانٌ. فَقَدَ (ى) فَقْدٌ. حَزِنَ (ا) حُزْنٌ. عَثَرَ عَلَى... (و) عُثُورٌ. سُرَّ. سُرُورٌ. مُكَافَأَةٌ (ات). سَلَّمَ. ثَوَابٌ.

\_\_\_\_\_\_\_\_\_\_\_\_\_\_\_\_\_\_\_\_\_\_\_\_\_\_\_\_\_\_\_\_

الْوَلَدُ لِوَالِدِهِ: قَالَ لَنَا مُعَلِّمُنَا: "غَدًا سَنَزُورُ مَعْرِضَ كِتَابِ اْلأَطْفَالِ, لِيُحْضِرْ كُلٌّ مِنْكُمْ دِينَارًا". لِذَلِكَ اَرْجُوكَ يَا أَبَتِ اَنْ تُعْطِيَنِى دِينَارًا. اَلْوَالِدُ: - لاَ ضَرُورَةَ لِذَلِكَ, لَسْنَا بِحَاجَةٍ اِلَى الْكُتُبِ, اَلْكُتُبُ عِنْدَنَا كَثِيرَةٌ, تَكْفِينَا. فَقَالَتِ اْلأُخْتُ ِلأَخِيهَا: اَبُوكَ مُتَضَايِقٌ اِنْسَ قَضِيَّةَ الدِّينَارِ, اَنَا اُعْطِيكَ وَ لَكِنْ لَيْسَ عِنْدِى دِينَارٌ, اُعْطِيكَ دَرَاهِمَ مِنْ مِحْفَظَةِ نُقُودِى بَدَلاً مِنَ الدِّينَارِ فَسُرَّ الْوَلَدُ وَ شَكَرَ أُخْتَهُ. فَقَدْتُ الْمِبْرَاةَ الَّتِى اشْتَرَيْتُهَا يَوْمَ اْلأَحَدِ الْمَاضِى فِى الْمَخْزَنِ وَ كَانَتْ جَمِيلَةً وَ حَزِنْتُ لِذَلِكَ حُزْنًا شَدِيدًا. وَ الْيَوْمَ سَمِعْتُ اِذَاعَةَ الْمَدْرَسَةِ تُعْلِنُ: مَنْ فَقَدَ شَيْئًا فَلْيُرَاجِعِ اْلإِذَاعَةَ فَرَاجَعْتُهَا فَاِذَا مِبْرَاتِى فِيهَا فَسُرِرْتُ سُرُورًا عَظِيمًا. كَيْفَ ذَلِكَ؟ وَ كَيْفَ وَقَعَتْ فِيهَا؟ - عَثَرَ عَلَيْهَا وَلَدٌ اَمِينٌ فَسَلَّمَهَا اِلَى اْلإِذَاعَةِ لِْلإِعْلاَنِ عَنْهَا. - هَذَا وَلَدٌ اَمِينٌ, لَقَدِ اسْتَحَقَّ الثَّوَابَ مِنَ اللَّهِ وَ الْمُكَافَأَةَ مِنَ النَّاسِ. - نَعَمْ, اَلأَمَانَةُ عِنْدَنَا دِينٌ, وَ عَلَى كُلٍّ مِنَّا اَنْ يَكُونَ أَمِينًا. كَمْ كَلِمَةً فِى هَذِهِ الْجُمْلَةِ؟

\subsubsection{К уроку 44}
У тебя есть ко мне вопросы? Какие у тебя ко мне вопросы? Задавай свои (давай сюда твои) вопросы. У меня нет к тебе вопросов. Я хорошо понял урок. Мы должны быть солдатами Ислама. Мы должны учиться и в школах, и в институтах, и в университетах. И мы должны защищать свою религию и свою родину. Мы должны защищать своих братьев — мусульман. Да здравствует наше государство, государство Ислама! Да здравствует наша религия Ислам! Да здравствуют мусульмане! Да сохранит тебя Бог, наш Ислам! Да сохранит вас Бог, мусульмане! Когда прозвенел звонок, почему ты не заходишь в класс? Почему ты стоишь здесь? Все мусульмане — мои братья. Я люблю всех мусульман и защищаю их. Мой отец герой. Наши отцы герои. Это исламская страна? — Нет, это не исламская страна. Я становился в ряд. Мы становились в ряды. Иди, становись в ряд. Почему ты не становишься в ряд? Где преподаватели? — Они в университском дворе. Мой народ неарабский, и я неараб, но я люблю арабов, потому что мой Пророк араб и язык Корана — арабский. У меня много арабских друзей, я их люблю, и они меня любят, так говорят они.

\subsection[الدَّرْسُ الْخَامِسُ وَ السِّتُّونَ 65 ]{الدَّرْسُ الْخَامِسُ وَ السِّتُّونَ 65 }
\  \includegraphics[width=1.1563in,height=1.1457in]{images/MuhammadBagauddinprettified-img216.png}   \includegraphics[width=1.0626in,height=1.0311in]{images/MuhammadBagauddinprettified-img217.png} 

ضَابِطٌ, ضُبَّاطٌ. قَائِدٌ, قُوَّادٌ. قَائِدُ دَبَّابَةٍ

مَا لَكَ؟ - وَلاَحَاجَةَ. حَالٌ اَحْوَالٌ. صِحَّةٌ. لاَبَأْسَبِهِ

كُلُّ شَيْءٍ عَلَى مَا يُرَامُ. نَوْبَتْجِىٌّ. مِنْ جَدِيدٍ.

تَخَرَّجَ مِنْ ... حَرْبِىٌّ. مَدْرَسَةٌ حَرْبِيَّةٌ. مُلاَزِمٌ.

رُتْبَةٌ, رُتَبٌ. صَبَاحَ الْخَيْرِ. - صَبَاحَ النُّورِ. 

فِرْقَةٌ, فِرَقٌ. اَلفِرْقَةُ الْمُدَرَّعَةُ. اَكَادِيمِيَّةُ الْمُدَرَّعَاتِ.

مُحَارِبٌ (ون). فِى سَبِيلِ اللَّهِ. عَسْكَرِىٌّ. 

اِنْ وَفَّقَنِىَ اللَّهُ. حَتَّى اَكُونَ مُحَارِبًا. خِلاَلَ شَهْرٍ.

\ تَدْرِيبٌ (ات). تَدْرِيبَاتٌ عَسْكَرِيَّةٌ. بَسِيطٌ, بُسَطَاءُ.

مَبْسُوطٌ بِ... كَمْ رَأَيْتُ! حَدَّثَ. خَطَفَ (ى) خَطْفٌ.

\ تَخَرَّجَ مِنَ الْمَدْرَسَةِ الْحَرْبِيَّةِ. هَرَبَ (و) هُرُوبٌ

\_\_\_\_\_\_\_\_\_\_\_\_\_\_\_\_\_\_\_\_\_\_\_\_\_\_\_\_\_\_\_\_

مَا لَكَ؟ وَلاَحَاجَةَ. فَلِمَاذَا تَضْحَكُ؟ خَطَفَ الْكَلْبُ اللَّحْمَ ثُمَّ هَرَبَ.ِ كَيْفَ حَالُكُمْ؟ لاَبَأْسَ بِهَا. وَكَيْفَ صِحَّتُكُمْ أَنْتُمْ؟ أَلْحَمْدُ لِلَّهِ كُلُّ شَيْءٍ عَلَى مَا يُرَامُ. مَنْ كَانَ نَوْبَتْجِيًّا أَمْسِ؟ - أَنَا. وَ مََنِ النَّوْبَتْجِيُّ الْيَوْمَ؟ لآ أَعْرِفُ. لِمَاذَا لآ تَعْرِفُ؟ عَلَيْكَ أَنْ تَعْرِفَ مَنْ كَانَ نَوْبَتْجِيًّا قَبْلَكَ وَ مَنْ يَكُونُ بَعْدَكَ. تَعَالَ إِلَى اللَّوْحِ, إِقْرَأْ هَذِهِ الْجُمْلَةَ مِنْ جَدِيدٍ.أَخِى طَالِبٌ فِى الْمَدْرَسَةِ الْحَرْبِيَّةِ وَ سَيَكُونُ بَعْدَ التَّخَرُّجِ مِنْهَا ضَابِطًا بِرُتْبَةِ الْمُلاَزِمِ. بَعْدَ دُخُولِ الْمُعَلِّمِ قَامَ التَّلاَمِيذُ مِنْ مَقَاعِدِهِمْ إِكْرَامًا لَهُ وَ قَالُو لَهُ: صَبَاحَ الْخَيْرِ, فَاَجَابَهُمْ: صَبَاحَ النُّورِ. اُرِيدُ دُخُولَ اَكَادِيمِيَّةِ الْمُدَرَّعَاتِ حَتَّى اَخْدِمَ الإِسْلاَمَ وَ اَكُونَ مُحَارِبًا فِى سَبِيلِ اللَّهِ. خَدَمْتُ فِى الْفِرْقَةِ الْمُدَرَّعَةِ قَائِدَ دَبَّابَةٍ وَ كَمْ قُمْنَا خِلاَلَ شَهْرٍ بِتَدْرِيبَاتٍ عَسْكَرِيَّةٍ. حَدِّثْنِى عَنْ حَيَاتِكَ يَا صَدِيقِى؟ حَيَاتِى حَيَاةُ مُسلِمٍ بَسٍيطٍ. هَلْ اَنْتَ مَبْسُوطٌ بِحَيَاتِكَ؟ نَعَمْ, اَنَا مَبْسُوطٌ بِحَيَاتِى, اَشْكُرُ اللَّهَ.

\subsubsection{К уроку 45}
Мальчик, не делай так, а делай вот так. Не читай так, а читай вот так. Я тебе говорил, не пиши так, а пиши вот так. Тебе следует вставать, когда заходят твой отец или старший брат, или учитель. Когда ты встречаешь старших, ты должен приветствовать их, а когда тебя приветствуют они, ты должен отвечать на их приветствие. Я работаю на танковом заводе. Мой отец работает на автозаводе. Твой друг работает на авиазаводе. Я ходил по двору и увидел бумагу на земле. Я взял бумагу и положил ее в корзину для бумаги. Твоей комнате следует быть чистой, твоим книгам и тетрадям тоже следует быть чистыми, и тебе следует быть чистым. Эта бумага из твоей книги. Куда ты бросил бумагу? Нам следует поднять знамена Ислама над домами и над мечетью. Ахмед и его друзья совершили вчера экскурсию на авиазавод. Мы тоже совершим завтра экскурсию на танковый завод. Сегодня праздник: над мечетью мы установили зеленое исламское знамя. Над домами и мечетью развевается наше знамя.

\subsection{الدَّرْسُ السَّادِسُ وَ السِّتُّونَ 66}
وَاصَلَ. حَاوَلَ. اَيْقَظَ. اَعَادَ. اِسْتَيْقَظَ مَعَ الْفَجْرِ.

غَلَطٌ, اَغْلاَطٌ. بِلاَأَغْلاَطٍ. نَصٌّ, نُصُوصٌ. مَرَّةً. كَمْ مَرَّةً؟ 

اَحْصَى. تَرْجَمَ. اَغْلَقَ الْكِتَابَ. خَافِتٌ. بِصَوْتٍ خَافِتٍ

يَعْنِى. مَاذَا يَعْنِى؟ زَادَ (ى) زِيَادَةٌ. مَزِيدٌ. هَلْ اَزِيدُكَ؟

ظَلاَّسَةٌ (ات). كِفَايَه! لاحَاجَةَ اِلَى... نَزَلَ بِهِ. هَلْ لَّكَ؟

وَاجِبٌ (ات). وَاجِبُ الْمَنْزِلِ. تَمْرِينَاتٌ رِيَاضِيَّةٌ. بِانْتِظَامٍ

قَطَعَ (ا) قَطْعٌ. طَلَبَ (و) طَلَبٌ. اَمْلَى. اِمْلاَءٌ

\_\_\_\_\_\_\_\_\_\_\_\_\_\_\_\_\_\_\_\_\_\_\_\_\_\_\_\_\_\_\_

يَا اَحْمَدُ, وَاصِلِ الْقِرَاءَةَ وَ لا تَقْطَعْهَا وَ حَاوِلْ اَنْ تَقْرَأَ بِلاَ أَغْلاَطٍ. - قَدْ حَاوَلْتُ ذَلِكَ مِرَارًا وَ لَكِنْ لَمْ يَحْصُلْ مِنِّى. - حَاوِلْ مَرَّةً اُخْرَى وَ اُخْرَى سَيَحْصُلُ. سَأُحَاوِلُ, اِنْ شَاءَ االلَّهُ. كَمْ مَرَّةً قَرَأْتَ النَّصَّ؟ - قَرَأْتُهُ عِدَّةَ مَرَّاتٍ لَمْ اُحْصِهَا. اِنَّكَ تَقْرَاُ بِصَوْتٍ خَافِتٍ اِرْفَعْ صَوْتَكَ. مَاذَا تَعْنِى هَذِهِ الْكَلِمَةُ؟ تَرْجِمْهَا لِى. مَا يَعْنِى قَوْلُكَ هَذَا؟ اَلآنَ اَغْلِقُواالْكُتُبَ وَ اسْمَعُوا اَسْئِلَتِى. يَا حَسَنُ, خُذِ الظُّلاَسَةَ وَ امْسَحِ اللَّوْحَ, اُمْلِى عَلَيْكَ هَذِهِ الْعِبَارَةَ الآتِيَةَ. هَلْ اَزِيدُكَ؟ - كِفَايَه! لا حَاجَةَ اِلَى مَزِيدٍ. اِنْ شِئْتَ فَاطْلُبْ فَاِنْ طَلَبْتَ الْمَزِيدَ زِدْتُكَ. مَاذَا تَفْعَلُ هُنَا؟ اَقُومُ بِوَاجِبِ الْمَنْزِلِ. اَتُقُومُ بِالتَّمَارِينِ الرِّيَاضِيََّةِ كُلَّ يَوْمٍ؟ نَعَمْ, اَقُومُ بِاالتِّمَارِينِ الرِّيَاضِيِّةِ كُلَّ صَبَاحٍ بِانْتِظَامٍ. اَسْتَيْقِظُ مَعَ الْفَجْرِ وَ اُوقِظُ اِخْوَتِى الصِّغَارَ ثُمَّ نُصَلِّى الصُّبْحَ جَمَاعَةً فِى الْبَيْتِ اَوْ فِى مَسْجِدِ الْمَحَلَّةِ, وَ بَعْدَ انْتِهَاءِ الصَّلَّاةِ نَخْرُجُ الَى التَّمَارِينِ الرِّيَاضِيَّةِ الصَّبَاحِيَّةِ. نَسِيتُ مَا قُلْتَ لِى هَلْ لَّكَ تُعِيدَهُ, مِنْ فَضْلِكَ اَعِدْهُ. هَلْ لَّكَ اَنْ تَنْزِلَ بِى وَ تَكُونَ ضَيْفًا؟ هَلْ لَّكَ اَنْ تُسَاعِدَنِى؟ الْيَوْمَ عِنْدَنَا دَرْسُ الإِمْلاَءِ. 

\subsubsection{К уроку 46}
Курица побежала, и за ней побежал петух. По полю бегут быки и коровы. В вашем классе есть доска? — Да, есть. Какого она цвета? — Чёрного цвета. А мел? — Мел тоже есть. Какого цвета мел? — Мел белого цвета. Я нашёл резинку, что мне с ней делать? Я ходил по улице и вдруг увидел своего друга Махмуда. Я ему сказал: „Ты где был?" Он мне сказал: ,,Я был в театре". В нашем лесу много диких зверей ( الْوُحُوشٌ). Там есть волки и медведи. Стены моей комнаты белые, а пол красный. Читай эту фразу и объясни её мне. О чем ты хочешь спросить меня? — Я хочу спросить тебя, когда мы пойдём на экскурсию в лес. Эта задача была трудная, Сайд помог мне, и я поблагодарил его. Это книга Хасана, я хочу вернуть её, но его нет дома. Я её брал у него вчера. Куда он пошёл? — Он пошёл в лес. Я подумал, что с ней делать, и сделал, как ты мне говорил. Делай, как я тебе говорю. Через неделю вернётся наш отец из своей поездки.

\subsection{الدَّرْسُ السَّابِعُ وَ السِّتُّونَ 67}
زَعَمُوا. وَصَفَ (ى) وَصْفٌ. صِنَاعَةٌ (ات). زِرَاعَة (ات)ِ

مَنَاخٌ. لَوْلاَ. نََصِيحَةٌ, نَصَائِحُ. اِمْتِحَانٌ (ات). اَدَّى اْلإِمْتِحَانَ. رَسَبَ فِى اْلإِمْتِحَانِ (و) رُسُوبٌ. خَسِرَ (ا) خُسْرٌ. كَامِلاً

اِنْ لَمْ... اِنْ لَمْ تَفْعَلْ مَا آمُرُكَ. حَدِيثٌ, اَحَادِيثُ. اِسْتَمَعَ اِلَى...ِ

مُتَّسَعٌ مِنَ الْوَقْتِ .رَغْبَةٌ عَظِيمَةٌ فِى... كُنْتُ عَلَى رَغْبَةٍ عَظِيمَةٍ فِى... حَالَ دُونَ... (و) حَيْلُولَةٌ. مَكَانٌ, اَمْكِنَةٌ. مَكَانُ الْعَمَلِ.ِ

عَالِمٌ, عُلَمَاءُ. جَاهِلٌ, جُهَلاَءُ. مُمَثِّلٌ (ون). قَادِرٌ (ون).ِ

كَمَا تَعْرِفُ. أَدَاءٌ. خَطِيرٌ. اِسْتَعَدَّ. مِنَ الآنَ. دُونَ...ِ

صَحِيحٌ. هَلْ هَذَا صَحِيحٌ؟ تَرَكَ عَمَلَهُ اِلَى وَقْتٍ آخَرَ. 

\_\_\_\_\_\_\_\_\_\_\_\_\_\_\_\_\_\_\_\_\_\_\_\_\_

زَعَمُوا اَنَّكَ زُرْتَ اَمْرِيكَا ثَلاَثَ مَرَّاتٍ. فَهَلْ هَذَا صَحِيحٌ؟ نَعَمْ, صَحِيحٌ. وَ لَكِنْ لا ثَلاَثَ مَرَّاتٍ بَلْ مَرَّتَيْنِ. فَهَلْ لَّكَ اَنْ تَصِفَهَا لَنَا؟ صِفْ لَنَا صِنَاعَتَهَا وَ زِرَاعَتَهَا وَ مَنَاخَهََا وَ الْمُسْلِيمِنَ فِيهَا. لَوْ لا مُسَاعَدَتُكَ لِى وَ لَوْ لا نَصَائِحُكَ لَرَسَبْتُ فِى الإِمْتِحَانِ وَ لَخَسِرْتُ هَذِهِ السَّنَةَ كَامِلَةً. عِنْدَنَا فِى الشَّهْرِ الْقَادِمِ اِمْتِحَانٌ وَ سَأَرْسُبُ فِيهِ اِنْ لَمْ اَسْتَعِدَّ لِأَدَائِهِ مِنَ الآنَ. نَتْرُكُ هَذَا الْحَدِيثَ الَى وَقْتٍ آخَرَ لِاَنِّى مَشْغُولٌ الآنَ بِعَمَلٍ آخَرَ وَ سَيَكُونُ عِنْدِى مُتَّسَعٌ مِنَ الْوَقْتِ لِلإِسْتِمَاعِ الَى حَدِيثِكَ. هَلْ قُمْتَ بِوَاجِبِ الْبَيْتِ؟ لا لَمْ اَقُمْ بِهِ لَمْ يَكُنْ عِنْدِى مُتَّسَعٌ مِنَ الْوَقْتِ لِذَلِكَ. اَرَادَ لِقَائِى فَلَقِيَنِى فِى مَكَانِ عَمَلِى, وَ اَرَدْتُ لِقَاءَهُ فَلَقِيْتُهُ وَ هُوَ ذَاهِبٌ اِلَى الْمَسْجِدِ. هَذَا الرَّجُلُ عَالِمٌ وَ هَذَا جَاهِلٌ. مِنْ اَيْنَ عَلِمْتُ اَنَّهُ جَاهِلٌ؟ عَلِمْتُهُ مِنْ قَوْلِهِ: اَنَا عَالِمٌ, مَنْ قَالَ: اَنَا عَالِمٌ فَهُوَ جَاهِلٌ. هُوَ, كَمَا تَعْرِفُ, غَيْرُ قَادِرٍ عَلَى اَدَاءِ هَذَا الْوَاجِبِ. اَللَّهُ قَادِرٌ عَلَى كُلِّ شَيْءٍ. اُنْظُرْ هَلْ تُوجَدُ اَغْلاَطٌ فِى هَذِهِ الْجُمْلَةِ؟ نَعَمْ, لَكَ فِى هَذِهِ الْجُمْلَةِ ثَلاَثَةُ اَغْلاَطٍ: اَمَّ الْغَلَطُ الأَوَّلُ فَخَطِيرٌ دُونَ الثَّانِى وَ الثَّالِثِ.

\subsubsection{К уроку 47}
На вашей улице есть почто§ое отделение? — Да, на нашей улице недалеко от нашего дома, риом с мечетью, есть почтовое отделение. Твой отец почтальон? — Нет, мой отец не почтальон, а дядя мой — почтальон. Я хочу одну марку, чтобы написать письмо своей дочери. Вот почтовая марка, на, бери. Почтовое отделение сейчас работает? — Сейчас не знаю, но утром работало. Я пойду туда узнать. Этот камень тяжёлый, а ты слабый и ещё болел, ты не поднимешь его. Ты не поднимай тяжелые камни, а когда захочешь поднять, позови своих друзей. Наш отец болен, иди позови врача. Не бери его с собой, оставь его здесь. Мой друг — ученик с хорошим характером, полный энергии. Его характер хороший. Твой характер плохой. Я учусь на историческом факультете. Ты учишься на географическом факультете. Она учится на факультете востоковедения. Ты видел моего друга Саида? — Да, видел недавно ( مُنْذُ وَقْتٍ قَرِيبٍ ) около твоей машины. Я убрал с дороги несколько камней. Убери с дороги вон тот камень. Давай пойдём в лес вместе. Давай поиграем в мяч. Пойдём сядем.

الدَّرسُ الثَّامِنُ وَ السِّتُّونَ 68

رَسُولٌ, رُسُلٌ. صَلَّى اللَّهُ عَلَيْهِ وَ سَلَّمَ. يَتِيمٌ, اَيْتَامٌ. 

عُمْرٌ, اَعْمَارٌ. اَرْسَلَ. اَنْزَلَ. نَشَأَ (ا) نَشْءٌ. بَلَغَ (و) بُلُوغٌ. 

\  \includegraphics[width=1.3783in,height=1.3335in]{images/MuhammadBagauddinprettified-img218.png} 

اِبْنُ الْعَمِّ. بَشَرٌ. شُرْفَةٌ شُرَفٌ. ذَاكَرَ. قَاعِدَةٌ, قَوَاعِدُ. حَفِظَ. حَفِظَ الْقَاعِدَةَ. اَلنَّحْوُ.ِ اَلصَّرْفُ. اِهْتِمَامٌ (ات). لَفَتَ اْلإِهْتِمَامَ اِلَى...(ى) لَفْتٌ. تَلْحِينٌ. نَطَقَ (ى) نُطْقٌ. بَلِيَّةٌ, بَلاَيَا. رَصِيفُ النَّهْرِ. هُوَ الآخَرُ. ذَاهِبٌ (ون). مَا يَزَالُ هُنَا. غَادَرَ. مَضَى (ى) مُضِيٌّ. حَدِيثٌ, اَحَادِيثُ. مَا هُوَ؟ مُنْذُ اَنْ...

\_\_\_\_\_\_\_\_\_\_\_\_\_\_\_\_\_\_\_\_\_\_\_\_\_\_\_\_\_\_\_ 

مَا تَعْلَمُ عَنِ الرَّسُولِ مُحَمَّدٍ, صَلَّى اللَّهُ عَلَيْهِ وَ سَلَّمَ؟ - اَعْلَمُ اَنَّهُ وُلِدَ يَتِيمًا وَ نَشَأَ يَتِيمًا لاَنَّ اَبَاهُ مَاتَ قَبْلَ وِلاَدَتِهِ وَ مَاتَتْ اُمُّهُ عِنْدَمَا بَلَغَ سِتَّ سَنَوَاتٍ. وُلِدَ فِى مَكَّةَ وَ اِذَا بَلَغَ اَرْبَعِينَ سَنَةً مِنْ عُمْرِهِ اَرْسَلَهُ اللَّهُ رَسُولاً اِلَى النَّاسِ وَ اَنْزَلَ عَلَيْهِ الْقُرْآنَ. الْقُرْآنُ هُوَ كَلاَمُ اللَّهِ وَ لَيْسَ كَلاَمَ الرَّسُولِ وَ لا كَلاَمَ غَيْرِهِ مِنَ الْبَشَرِ. اَمَّا كَلاَمُ الرَّسُولِ فَهُوَ الْحَدِيثُ. عَلَيْنَا اَنْ نَّحْفَظَ الْقُرْآنَ وَ نَحْفَظَ الاَحَادِيثَ. لاَبُدَّ لِلْعَالِمِ مِنْ اَنْ يَعْلَمَ الْقُرْآنَ وَ يَعْلَمَ الاَحَادِيثَ. مَنْ فِى الشُّرْفَةِ؟ هُوَ ابْنُ عَمِّى يُذَاكِرُ دُرُوسَهُ. يَااَيُّهَا الطَّلَبَةُ ذَاكِرُوا دُرُوسَكُمْ وَاحْفَظُوا قَوَاعِدَ النَّحْوِ وَ الصَّرْفِ وَ الْفِتُوا الإِهْتِمَامَ اِلَى التَّلْحِينِ عِنْدَمَا تَنْطِقُونَ بِالْكَلِمَاتِ. هَلْ عِنْدَكَ وَقْتٌ لِتَسْمَعَ كَيْفَ اَقْرَاُ؟ - نَعَمْ, اِقْرَاْ, قِرَاءَتُكَ صَحِيحَةٌ لَكِنَّ التَّلْحِينَ غَيْرُ صَحِيحٍ. اَخُونَا وَقَعَ فِى بَلِيَّةٍ عَلَيْنَا اَنْ نُفَكِّرَ كَيْفَ نُسَاعِدُهُ. قُلْ لِّى مَا هُوَ الطَّرِيقُ اِلَى رَصِيفِ النَّهْرِ؟ - سِرْمَعِى, اَنَا الآخَرُ ذَاهِبٌ اِلَى رَصِيفِ النَّهْرِ. هَلْ اَخُونَا الْعَرَبِيُّ مِنْ مِصْرَ مَا يُزَالُ هُنَا؟ - لَقَدْ مَضَتْ سَنَةٌ مُنْذُ اَنْ غَادَرَ أَخُونَا بِلاَدَنَا.

\subsubsection{К уроку 48}
Что здесь случилось? Почему здесь люди? Что случилось на улице? У нас есть трактор, трактор тащит плуг и пашет землю. Крестьянину необходимы плуг, трактор или быки. Когда идёшь в школу, тебе надо чистить свою одежду и выходить чистым. Нам необходима чистота. Каждому из нас следует быть чистым. Бог любит чистого мальчика. Крестьянин пашет землю плугом. Где мой циркуль? — Я не знаю, куда ты положил его. Я тоже не знаю. Он читает быстро. Ты читай быстро. Не читай быстро. Мой сосед вошёл в мой дом, а потом быстро вышел. Почему он быстро вернулся? — У него отец болен дома, поэтому он вернулся быстро. Сегодня все крестьяне на полях пашут, сеют. Каждый из нас выполняет одну работу, и все работы важные. Повернись направо. Повернись налево. Смотри вперёд. Моя тетрадь чистая, а твоя тетрадь не чистая. У вас есть трактор? — Да, есть. Когда вы его купили? — В прошлом году.

\subsection{الدَّرْسُ التَّاسِعُ وَ السِّتُّونَ 69}
سَرَّ (و) سُرُورٌ. تَمَنَّى. يَا لَهُ مِنْ... يَا لَهُ مِنْ رَجُلٍ! مُتَوَقَّعٌ.ِ غَيْرُ مُتَوَقَّعٍ. شَخْصٌ اَشْخَاصٌ. ذَكِيٌّ اَذْكِيَاءُ. مُشْمِسٌ. بَعْدَ رَنِّ الْجَرَسِ. بِدُونِ اَنْ... اِسْتِرَاحَةٌ. اِسْمَحْ لِى؟ أَلَيْسَ كَذَلِكَ؟ - بَلَى. فُرْسَةٌ سَعِيدَةٌ! - اَنَا اَسْعَدُ. يَا صَاحِ! يَوْمًامَّا. تُصْبِحُ عَلَى خَيْرٍ. وَ اَنْتَ بِخَيْرٍ. سَلِّمْ لِى عَلَى... سَيِّدَةٌ (ات). اَبْلَغَ. مُشْتَاقٌ اِلَى...ِ كَمِ السَّاعَةُ؟ - اَلسَّاعَةُ تُشِيرُ اِلَى... مُنْتَصَفُ اللَّيْلِ. بَعْدَ مُنْتَصَفِ اللَّيْلِ. اِبْنُ الْخَالِ. سَاعَةُ الْمِعْصَمِ. سَاعَةُ الْجَيْبِ. سَاعَةُ الْجِدَارِ. سَاعَةٌْ مُنَبِّهَةٌ. نَفْسُهُ. لِنَفْسِهِ. هُرِعَ, يُهْرَعُ. وَدَّ لَوْ... وَدِدْتُ لَوْ... قَصَدَ(ى) قََصْدٌ. شَدَّمَا... (لَشَدَّمَا). اِشْتَاقَ اِلَى...ِ

\_\_\_\_\_\_\_\_\_\_\_\_\_\_\_\_\_\_\_\_\_\_\_\_\_

كَمْ تَمَنَّيْتُ رُؤْيَتَكُمْ وَ كَمْ سَرَّنِى لِقَاؤُكُمْ! يَالَهُ مِنْ لِقَاءٍ! هَذَا اللِّقَاءُ كَانَ غَيْرَ مُتَوَقَّعٍ. يَالَهُ مِنْ شَخْصٍ ذَكِيٍّ! يَالَهُ مِنْ يَوْمٍ مُشْمِسٍ دَافِئٍ! بَعْدَ رَنِّ الْجَرَسِ يُهْرَعُ الطُّلاَّبُ الَى غُرْفَةِ الدَّرْسِ وَ لا يَتَخَلَّفُ اَحَدٌ مِنْهُمْ. اَجِبْ عَلَى السُّؤَالِ بِدُونِ النَّظَرِ فِى الْكِتَابِ. بَعْدَ كَمْ دَقِيقَةً الإِسْتِرَاحَةُ؟ الإِسْتِرَاحَةُ بَعْدَ خَمْسِ دَقَائِقَ, اَلَيْسَ كَذَلِكَ؟ - بَلَى. اِسْمَحْ لِى بِالدُّخُولِ. اِسْمَحْ لِى بِالْخُرُوجِ. فُرْصَةٌ سَعِيدَةٌ! اَيْنَ كُنْتَ وَ اَيْنَ قَصَدْتَ؟ - اَنَا اَسْعَدُ, جِئْتُكُمْ اَزُورُكُمْ, لَشَدَّمَا اَشْتَقْتُ اِلَيْكُمْ فَجِئْتُكُمْ وَ وَدِدْتُ لَوْ تَزُورُونَنَا اَنْتُمْ اَيْضًا بِاهْلِكُمْ جَمِيعًا. - اِنْ شَاءَ اللَّهُ سَنَزُورُكُمْ يَوْمًا مَّا, سَنَجِدُ لِذَلِكَ وَقْتًا. اِلَى اللِّقَاءِ يَا اَخِى, وَ تُصْبِحُونَ عَلَى خَيْرٍ. - وَ اَنْتُمْ بِخَيْرٍ. مِنْ فَضْلِكَ, سَلِّمْ لِى عَلَى السَّيِّدَةِ عَائِشَةَ, هِىَ كَانَتْ مُعَلِّمَتِى وَ كَانَتْ تُحِبُّنِى وَ اُحِبُّهَا وَ اَبْلِغْهَا اَنِّى مُشْتَاقَةٌ اِلَى رُؤْيَتِهَا. اَلسَّاعَةُ كَمْ؟ - اَلسَّاعَةُ تُشِيرُ اِلَى الْوَاحِدَةِ بَعْدَ مُنْتَصَفِ اللَّيْلِ. اِشْتَرَيْتُ اَرْبَعَ سَاعَاتٍ: سَاعَةَ الْمِعْصَمِ لِنَفْسِى وَ سَاعَةَ الْجَيْبِ لِأَبِى وَ سَاعَةَ الْجِدَارِ لِأُخْتِى وَ السَّاعَةَ الْمُنَبِّهَةَ لِأِبْنِ خَالِى.ِ

\subsubsection{К уроку 49}
У нас в доме есть гнездо воробья, и в гнезде несколько птенцов. Воробей каждое утро летает вокруг дома и чирикает. Я рассказал своему другу о том, что я делал на даче, и кто со мной еще там был. Расскажи мне, где ты был с утра. Расскажи мне, что там случилось. Он не услышал твои слова. Кто к тебе пришёл? — Никт ко мне не пришёл. Когда ты ко мне придёшь? Мой друг книготорговец, а его отец — старый садовник. Отвечай на мои вопросы. Я его спросил о многом, и он ответил на мои вопросы. Когда я отпустил птенца, он улетел далеко от меня и сел на дерево. Мусульманин любит свободу. Все мусульмане любят свободу. Я мусульманин поэтому я люблю свободу. Свобода только в Исламе ( لاَ حُرِيَّةَ اِلاَّ فِى الإِسْلاَمِ ). Птицы летают в воздухе. Рыбы плавают в воде. А люди ходят по земле. Ты кого-нибудь видел? — Нет, я никого не видел. Ты кого-нибудь спросил? — Да, я спросил одного человека. Оставь кошку и не хватай её. Ребёнок упал на землю и заплакал. Мальчик упал на землю, но он встал и не плакал.

\subsection[الدَّرْسُ السَّبْعُونَ 70]{الدَّرْسُ السَّبْعُونَ 70}
 \includegraphics[width=2.0835in,height=1.2602in]{images/MuhammadBagauddinprettified-img219.png}   \includegraphics[width=2.1146in,height=1.2811in]{images/MuhammadBagauddinprettified-img220.png}  

\ سِكَّةٌ حَدِيدِيَّةٌ. مَحَطَّةُ السِّكَّةِ الْحَدِيدِيَّةِ. 

 \includegraphics[width=1.9272in,height=1.4689in]{images/MuhammadBagauddinprettified-img221.png}   \includegraphics[width=2.0626in,height=1.1043in]{images/MuhammadBagauddinprettified-img222.png} 

\ قُطْنٌ, اَقْطَانٌ. قِطَافٌ. مَاكِنَةُ قِطَافِ الْقُطْنِ. 

نَجَاحٌ. اَتَمَنَّى لَكُمُ النَّجَاحَ. مَصَحٌّّ (ات).

مَوْجُودٌ. حَالِيًّا. تَفْصِيلاً. اَخْبَرَ تَفْصِيلاً. كَلِمَةً كَلِمَةً. جُمْلَةً جُمْلَةً. شَاهَدَ. بِكُلِّ اَرْتِيَاحٍ. بَاتَ (ى) مَبِيتٌ. تَوَجَّهَ. تَاَخَّرَ.ِ مِيعَادٌ. تَاَخَّرَ عَنِ الْمِيعَادِ. تَكَلَّمَ. كَفَى التَّكَلُّمُ! اِسْتَعْجَلَ. فَاتَهُ الْقِطَارُ ِ(و) فَوْتٌ. حَفْلَةٌ (ات). اَلثَّامِنَةُ تَمَامًا. تَوَقَّفَ. مُدَّةٌ، مُدَدٌ. مُدَّةُ تَوَقُّفِ الْقِطَارِ. اِسْمَحْ لِى اَنْ اَشْكُرَكَ. لا شُكْرَ عَلَى وَاجِبٍ. دَعْوَةٌ. لَبَّى الدَّعْوَةَ. وَصَلَ الْقِطَارُ. لِئَلاَّ... لِكَيْلاَ... حَوَالَىْ... سَيِّدٌ سَادَةٌ. يَا سَيِّدِى. كَهَذَا

\_\_\_\_\_\_\_\_\_\_\_\_\_\_\_\_\_\_\_\_\_\_\_\_\_\_

اَتَمَنّىَ لَكَ النَّجَاحَ فِى دِرَاسَتِكَ وَ فِى اَعْمَالِكَ كُلِّهَا. - شُكْرًا لَكَ يَا سَيِّدِى. جَاءَتْنِى مِنْ اَخِى الْمَوْجُودِ حَالِيًّا فِى الْمَصَحِّ رِسَالَةٌ وَ حَدَّثَنِى فِيهَا عَنْ حَالِهِ تَفْصِيلاً فَقَرَاْتُ الرِّسَالَةَ كَلِمَةً كَلِمَةً حَلَّلْتُ عِبَارَاتِهَا جُمْلَةً جُمْلَةً. هَلْ تَذْهُبُ مَعِى اِلَى الْحَقْلِ؟ الْيَوْمَ قِطَافُ الْقُطْنِ وَ سَتُشَاهِدُ مَاكِنَاتِ قِطَافِ الْقُطْنِ؟ - نَعَمْ, بِكُلِّ اَرْتِيَاحٍ. بَاتَ الضَّيْفُ عِنْدَنَا لَيْلَتَيْنِ وَ فِى صَبَاحِ الْيَوْمِ الثَّالِثِ خَرَجَ وَ تَوَجَّهَ اِلَى مَحَطَّةِ السِّكَّةِ الْحَدِيدِيَّةِ. نَظَرَ الشَّابُّ الَى سَاعَتِهِ لِيَعْرِفَ كَمِ السَّاعَةُ لِئَلاَّ يَتَأَخَّرَ عَنِ الْمِيعَادِ مَعَ صَدِيقِهِ. كَفَى التَّكَلُّمُ! لَقَدْ تَكَلَّمْتَ كَثِيرًا! اِسْتَعْجِلْ لِكَيْلاَ يَفُوتَكَ الْقِطَارُ اَلَسْتَ تُرِيدُ السَّفَرَ؟ - بَلَى. ذَهَبَ لِمُشَاهَدَةِ الْحَفْلَةِ وَ عَادَ مِنْهَا فِى السَّاعَةِ الثَّامِنَةِ تَمَامًا. مَتَى يَصِلُ الْقِطَارُ؟ - يَصِلُ الْقِطَارُ حَوَالَىِ الرَّابِعَةِ وَ مُدَّةُ تَوَقُّفِهِ فِى هَذِهِ الْمَحَطَّةِ عَشْرُ دَقَائِقَ. اِسْمَحُوا لِى اَنْ اَشْكُرَكُمْ عَلَى دَعْوَتِكُمْ. قَدْ لَبَّيْتُ دَعْوَتَكُمْ وَجِئْتُ فِى الْمِيعَادِ. - لا شُكْرَ عَلَى الْوَاجِبِ يَا سَيِّدِى. سَتَكُونُ عِنْدَنَا فِى الشَّهْرِ الْقَادِمِ حَفْلَةٌ كَهَذِهِ فَنَحْنُ نَدْعُوكُمْ اِلَيْهَا وَ عَلَيْكُمْ اَنْ تَحْضُرُوهَا. - اِنْ شَاءَ اللَّهُ سَنُلَبِّى دَعْوَتَكُمْ وَ نَحْضُرُهَا وَ اِلَى اللِّقَاءِ.

\subsubsection{К уроку 50}
Пойдём посмотрим на поле, на сад и на розы в саду. Около нашег села есть глубокое озеро, пойдём посмотреть его? Где продаётся мя со? Где продатёся зелень? Где продаются сыр, масло, яйца и други продукты? — Всё это продаётся на рынке. Где вы получаете моло ко? — Мы его получаем от коровы. У вас много коров? — У на несколько коров, а вы где получаете молоко? — Мы молоко покупаем в магазине. Аиша, ты корову доила? — Нет, не доила; она вернулась с пастбища? — Да, вернулась. Во что я подою её? — Дои её в этот сосуд. Мы поели хлеба с маслом, затем выпили чай с сахаром. Сегодня сельский базар, все люди идут на базар. Одни идут что-то купить; другие идут что-то продать. Эта старуха несёт в своей корзине яйца а та старуха несёт сыр, третья старуха несёт в мешке ячмень. Мь сидим за обеденным столом, на столе хлеб, масло и соль, и пере; каждым из нас стакан кипячёного молока. Я люблю пить кипячёное молоко после еды, а мой брат не любит кипячёное молоко, он всегда пьёт некипячёное молоко.

\subsection{الدَّرْسُ الْحَادِى وَ السَّبْعُونَ 71}
 \includegraphics[width=1.6772in,height=1.1146in]{images/MuhammadBagauddinprettified-img223.png}   \includegraphics[width=1.0311in,height=1in]{images/MuhammadBagauddinprettified-img224.png} 

\ زُجَاجَةٌ (ات). كَأْسٌ, كُؤُوسٌ. مُرَطِّبَاتٌ. زُجَاجَةُ الْمُرَطِّبَاتِ. عَطِشَ (ا) عَطَشٌ. قَطُّ. \newline
شَعَرَبِ... (و) شُعُورٌ. شَعَرَ بِعَطَشٍ. مَاْكُولاَتٌ شَرْقِيَّةٌ. اَلْفُ شُكْرٍ. - عَفْوًا. لا اَعْرِفُ كَيْفَ اَشْكُرُكُمْ. نَحْوَ اْلإِسْلاَمِ. قَوَّمَ. كِتَابٌ لا يُقَوَّمُ بِثَمَنٍ. اَلْقَى سُؤَالاً. أَهْمَلَ. بِلاَجَوَابٍ. أَزْعَجَ. تَرَكَهُ وَ شَأْنَهُ. تَمَنِّيَاتٌ طَيِّبَةٌ. \newline
نَصَرَ عَلَى... (و) نَصْرٌ. نَصَرَهُ عَلَى أَعْدَائِهِ. وَدَّعَ. نُوَدِّعُكُمْ. عَلَى أَمَلِ اللِّقَاءِ. عَنْ قَرِيبٍ. نَسْتَوْدِعُكُمُ اللَّهَ. رَحْمَةُ اللَّهِ. السَّلاَمُ عَلَيْكُمْ. - وَ عَلَيْكُمُ السَّلاَمُ وَ رَحْمَةُ اللَّهِ. سَأَلَ اللَّهَ. مَلَأَ (ا) مَلْءٌ. مَلَأَ السَّطْلَ مَاءً.

\_\_\_\_\_\_\_\_\_\_\_\_\_\_\_\_\_\_\_\_\_\_\_\_\_\_\_\_\_\_\_\_\_\_

اَلْفُ شُكْرٍ يَا اَخِى لَقَدْ نَصَرْتَنِى وَ عَمِلْتَ عَمَلاً لا يُقَوَّمُ بِثَمَنٍ وَ لا اَعْرِفُ كَيْفَ اَشْكُرُكَ. - عَفْوًا, هَذَا وَاجِبِى وَ وَاجِبُ كُلِّ وَاحِدٍ مِنَ الْمُسْلِمِينَ نَحْوَ الإِسْلاَمِ وَ الْمُسْلِمِينَ. اِذَا وَقَعَ اَحَدُنَا فِى بَلِيَّةٍ فَعَلَى الآخَرِينَ اَنْ يَنْصُرُوهُ. اَلْقَيْتُ عَلَيْكَ سُؤَالاً لَكِنَّكَ اَهْمَلَتَهُ وَ تَرَكْتَهُ بِلاَ جَوَابٍ. هُوَ مَرِيضٌ, لِمَاذَا تُزْعِجُهُ, اُتْرُكْهُ وَ شَأْنَهُ. نَتَمَنَّى لَكُمْ كُلَّ خَيْرٍ.- نَشْكُرُكُمْ عَلَى تَمَنِّيَاتِكُمُ الطَّيِّبَةِ وَ نَسْأَلُ اللَّهَ اَنْ يَنْصُرَنَا عَلَى اَعْدَائِنَا وَ اَعْدَاءِ الإِسْلاَمِ. هُوَ عَدُوٌّ لِى ِلاَنَّهُ غَيْرُ مُسْلِمٍ وَ اَنَا عَدُوُّهُ ِلاَنَّى مُسْلِمٌ. عَطِشْتُ كَثِيرًا فَسَاَلْتُ زَوْجَتِى اَنْ تَفْتَحَ زُجَاجَةً مِنَ الْمُرَطِّبَاتِ فَمَلَأَتْ كَأْسًا فَشَرِبْتُهُ. اِمْلَئِى لِى كَأْسًا مِنْ زُجَاجَةِ الْمُرَطِّبَاتِ فَاِنِّى اَشْعُرُ بِعَطَشٍ شَدِيدٍ. قَالَ لِى صَدِيقِى: اُحِبُّ الْمَأْكُولاَتِ الشَّرْقِيَّةَ جِدًّا فَقَلْتُ لَهُ: اَمَّا اَنَا فَلَمْ آكُلْهَا قَطُّ وَ وَدِدْتُ لَوْ تَطْبَخُهَا. نُوَدِّعُكُمْ عَلَى اَمَلِ اللِّقَاءِ بِكُمْ عَنْ قَرِيبٍ وَ نَسْتَوْدِعُكُمُ اللَّهَ وَ السَّلاَمُ عَلَيْكُمْ وَ رَحْمَةُ اللَّهِ.

\subsubsection{К уроку 51}
Где большая мечеть? Когда пятница? — Сегодня пятница. В пятницу все мужчины идут в большую мечеть, чтобы совершить пятничный намаз. Кто работает в мечети? — В мечети работают имам, муэдзин и уборщик. Что делает каждый из них? — Имам совершает намаз (стоя) впереди людей, муэдзин призывает людей на намазы, а уборщик подметает мечеть и убирает её. Ты ходишь в мечеть? — Да, я хожу. Тебе следует на каждый намаз ходить в мечеть. Имам становится впереди всех, и люди становятся в ряды за имамом, затем имам молится, и все люди молятся за имамом. Вот так совершают люди групповой намаз. Когда урок кончился, ученики вернулись по своим домам. В пятницу мусульмане не ходят (не идут) на работу, не ходят на заводы, на фабрики, в поля; в пятницу мусульмане ходят только в мечеть. В пятницу мусульмане не работают. Сальма, ты двор подме­ла? — Нет, не подмела. Надо было подмести, иди возьми веник и подмети. — Хорошо. Мама, я веника не нахожу, куда ты его положи­ла? — Спроси Лейлу, веник утром был у неё в руках.

الدَّرسُ الثَّانِى وَ السَّبْعُونَ 72

نُكْتَةٌ, نُكَةٌ. قَصَّ (و) قَصٌّ. قَصَّ نُكْتَةً. نَعَمْ؟ كَذِبٌ. مُزَاحٌ. جَامِدٌ. مُتَحَجِّرٌ. تَحَدَّثَ. اَمْرٌ, اُمُورٌ. فِيمَا بَعْدُ. سَنَتَحَدَّثُ فِى اْلأَمْرِ. حِصَّةٌ, حِصَصٌ. مَا فَعَلْتَهُ. اِعْتَبَرَ. اِنْتِهَاكٌ. حُرْمَةٌ. اِنْتِهَاكٌ لِحُرْمَةِ... عِلْمٌ, عُلُومٌ. تَسْلِيَةٌ. \newline
قَصَدَ الْمُزَاحَ. مَحَلٌّ, مَحَالُّ (ات). يَانَصِيبٌ. \newline
وَرَقُ اليَانَصِيبِ. وَسْوَسَ. شَيْطَانٌ, شَيَاطِينُ. \newline
طَلَبَ (و) طَلَبٌ. اَخْرَجَ. تَالٍ. قَلَقٌ. سِرٌّ, اَسْرَارٌ. \newline
مَا سِرُّ ذَلِكَ؟ نَتِيجَةٌ, نَتَائِجُ. ظَهَرَ (ا) ظُهُورٌ. \newline
مَاذَا فِى الأَمْرِ؟ حَرَامٌ. نَوْعٌ, اَنْوَاعٌ. قِمَارٌ. زُقَاقٌ, اَزِقَّةٌ. بَيْعٌ. جَائِزٌ. غَيْرُ جَائِزٍ.

\_\_\_\_\_\_\_\_\_\_\_\_\_\_\_\_\_\_\_\_\_\_\_\_\_\_\_\_\_\_\_\_\_\_

\ - اَلْمُعَلِّمُ لا يَارَانَا أَقُصُّ نُكْتَةً عَلَيْكَ يَا مُجَاهِدُ. - نَعَمْ؟

- اِسْمَعْ مِنِّى هَذِهِ النُّكْتَةَ.

- الْوَقْتُ لَيْسَ وَقْتَ كَذِبٍ وَ مُزَاحٍ يَا خَالِدُ.

- اَنْتَ جَامِدٌ مُتَحَجِّرٌ.

- سَنَتَحَدَّثُ فِى اْلأَمْرِ فِى مَا بَعْدُ.

بَعْدَ اِنْتِهَاءِ الْحِصَّةِ قَالَ مُجَاهِدٌ لِخَالِدٍ: مَا فَعَلْتَهُ اَثْنَاءَ الدَّرْسِ يَا خَالِدُ يُعْتَبَرُ انْتِهَاكًا لِحُرْمَةِ الْعِلْمِ وَ الْمُعَلِّمِ.

- اَنَا قَصَدْتُ التَّسْلِيَةَ.ِ

رَأَى جَمَالٌ وَ هُوَ يَسِيرُ فَى اَحَدِ أَزِقَّةِ الْمَدِينَةِ مَحَلَّ بَيْعِ أَوْرَاقِ الْيَانَصِيبِ فَوَسْوَسَ الشَّيْطَانُ لَهُ فَقَالَ: مَا هَذَا؟ اِنَّهُ مَحَلُّ بَيْعِ اَوْرَاقِ الْيَانَصِيبِ فَدَخَلَهُ, وَ بِسُرْعَةٍ اَخْرَجَ مِنْ جَيْبِهِ النُّقُودَ وَ طَلَبَ مِنَ الْبَيَّاعِ: أَعْطِنِى وَرَقَةَ يَانَصِيبِ. وَ فِى الْيَوْمِ التَّالِى قَالَ لِصَاحِبِهِ:

- اَنَا الْيَوْمَ فِى قَلَقٍ يَا مُجاهِدُ.ِ

- وَ مَا سِرُّ قَلَقِكَ هَذَا يَا جَمَالُ؟

- الْيَوْمَ تَظْهَرُ نَتَائِجُ الْيَانَصِيبِ.ِ

- هَذَا يَعْنِى اَنَّكَ اشْتَرَيْتَ وَرَقَةَ الْيَانَصِيبِ؟

- نَعَم, وَ مَاذَا فِى اْلأَمْرِ؟

- هَذَا حَرَامٌ يَا جَمَالُ هَذَا غَيْرُ جَائِزٍ, هَذَا نَوْعٌ مِنْ اَنْوَاعِ الْقِمَارِ, وَ قَدْ نَهَانَا اْلإِسْلاَمُ عَنِ الْقِمَارِ.

\subsubsection{К уроку 52}
У нас жирный барашек. У барашка длинная шерсть. Барашек ест люцерну. Я слышу звук, а ты слышишь звук? — Да, я тоже слышу, а что это за звук? — Это гул (звук) трактора. TpatTop в поле, он пашет землю. На тракторе работает мой сосед. Трактор работает на поле целый день и каждый день с утра до вечера. Рядом с полем есть канал, в канале вода. Что делает Фарид целый день в библиотеке? Он целый день сидит в читальном зале и читает старые арабские книги. В канале течёт вода, в воде плавает рыба, и ребята сидят там и ловят рыбу. У каждого из них есть удочка. В нашем городе много мечетей и школ, чистые улицы, парки и сады. Эти рабочие и работницы из Москвы, а те — из Киева. Где ты живёшь? — Я живу на этой улице. Дверь открыта? — Да, дверь открыта. Все стулья в этой комнате чёрные. На всех столах арабские книги. Все дома в городе белые. Мы читали эту книгу, а вы тоже читали её? Они взяли письмо и прочитали его. Ты знаешь этого писателя? — Да, я его книги читал. Вы знаете этих писателей? — Да, мы их книги читали.

\subsection{اَلدَّرْسُ الثَّالِثُ وَ السِّبْعُونَ 73}
نَجَحَ (ا) نَجَاحٌ. حَتْمًا. مَغْرُورٌ. غُرُورٌ. أَعْمَى. بَصِيرَةٌ, بَصَائِرُ. حَقِيقَةٌ, حَقَائِقُ. سَلَبَ (و) سَلْبٌ. \newline
جَعَلَ (ا) جَعْلٌ. زُمْرَةٌ, زُمَرٌ. فَاشِلٌ (ون). قَرِيبٌ. رَأَى بِأُمِّ عَيْنَيْهِ. صِدْقٌ. بَقِىَ (ا) بَقَاءٌ. طَرِيحُ الْفِرَاشِ. \newline
ضَاعَ (ى) ضَيَاعٌ. أَدْرَكَ. خَطَأٌ. تَفْكِيرٌ. مَرَضٌ, أَمْرَاضٌ. قَاتِلٌ. يَجِبُ عَلَى... تَخَلَّصَ مِنْ... رَدَّدَ. يَا لَلْخَسَارَةِ! يَا لَلأَسَفِ! مُتَأَسِّفٌ! - اَلْعَفْوَ. اَنَا مُتَأَسِّفٌ عَلَى اْلإِزْعَاجِ. مِنْ كُلِّ بُدٍّ. مَا زَالَ يَقُولُ. فَاِنَّ...

\_\_\_\_\_\_\_\_\_\_\_\_\_\_\_\_\_\_\_\_\_\_\_\_\_\_\_\_\_\_\_

- أَنَا ذَكِىٌّ وَ كُلُّ الْمُعَلِّمِينَ وَ الطُّلاَّبِ يَعْرِفُونَ ذَلِكَ سَأَنْجَحُ فِى الإِمْتِحَانِ حَتْمًا.

- لاَ تَقُلْ هَذَا يَا عَادِلُ فَأَنْتَ مَغْرُورٌ, وَ الْغُرُورُ أَعْمَى بَصِيرَتَكَ عَنْ رُؤْيَةِ الْحَقِيقَةِ, اِنَّكَ لاَ تَعْرِفُ مَا سَيَكُونُ فِيمَا بَعْدُ, اِنَّ اللَّهَ قَادِرٌ عَلَى اَنْ يَسْلُبَكَ النَّجَاحَ وَ يَجْعَلَكَ فِى زُمْرَةِ الْفَاشِلِينَ.

- الإِمْتِحَانُ قَرِيبٌ يَا مُجَاهِدُ وَ سَتَرَى بِأُمِّ عَيْنَيْكَ صِدْقَ مَا اَقُولُ وَ اَنَّهَا هِىَ الْحَقِيقَةُ.

مَرِضَ عَادِلٌ وَ بَقِىَ طَرِيحَ الْفِرَاشِ اُسْبُوعَيْنِ وَ فَاتَهُ الإِمْتِحَانُ وَ ضَاعَتْ عَلَيْهِ السَّنَةُ كَامِلاَةً وَ اَدْرَكَ خَطَأَ تَفْكِيرِهِ وَ اَنَّهُ كَانَ حَقًّا فِى الْغُرُورِ وَ اَدْرَكَ اَنَّ الْغُرُورَ مَرَضٌ قَاتِلٌ يَجِبُ عَلَى الْمُسْلِمِ اَنْ يَتَخَلَّصَ مِنْهُ, وَ مَازَالَ يُرَدِّدُ: يَا لَلْخَسَارَةِ! يَا لَلأَسَفِ! 

اِذَا كُنْتَ فِى مِصْرَ فَزُرْ آثَارَهَا التَّارِيخِيَّةَ مِنْ كُلِّ بُدٍّ فَاِنَّ فِيهَا آثَارًا تَارِيخِيَّةً كَثِيرَةً. اَنَا مُتَأَسِّفٌ عَلَى الأَزْعَاجِ! اَنَا مُتَأَسِّفٌ عَلَى التَّأَخُّرِ! اَنَا مُتَأَسِّفٌ نَسِيتُ ذَلِكَ.

\subsubsection{К уроку 53}
В прошлом месяце я был на заводе. Этот завод консервирует фрукты и овощи в банках. На заводе много цехов. В каждом отделе много станков. Станки крутятся целый день, и рабочие ходят (крутятся) вокруг станков. Мусульманин любит Ислам и служит ему! Мусульманин служит государству Ислама. Я сегодня радостный (в радости). Почему ты сегодня радостный (в радости)? — Потому что папа вернулся из путешествия. Он посетил зоопарк и видел там разных диких зверей, поэтому он был удивлён (в удивлении). Быки дают пользу людям, они пашут землю, и лошади тоже дают пользу людям, люди ездят на них. Заводы и фабрики необходимы нам. Я преподаватель арабского языка. Моя жена преподавательница русского языка. Ты преподаватель персидского языка. Наш завод консервирует рыбу и мясо. В начале месяца. В начале года. В начале книги. Оросительная машина находится в начале поля. Кто это? — Это рабочие с завода, они сейчас идут на работу. Персики, абрикосы, виноград, яблоки, финики и инжир — всё это вкусные фрукты, и все они полезны нам.

\subsection[الدَّرْسُ الرَّابِعُ وَ السَّبْعُونَ 74]{الدَّرْسُ الرَّابِعُ وَ السَّبْعُونَ 74}
سَكَتَ (و) سُكُوتٌ. مَادَامَ. اِنْصَرَفَ. فِى سَاعَةٍ مُتَأَخِّرَةٍ مِنَ اللَّيْلِ. تَلْفَنَ. طَالَمَا. غَرَّ (و) غَرٌّ. شُيُوعِيَّةٌ. اِنْكَشَفَ. أَسْوَأُ. نِظَامٌ, اَنْظِمَةٌ, نُظُمٌ. اِخْتَرَعَ. هَوًى, أَهْوَاءٌ. بَشَرِيٌّ. أَحْسَنُ. نَازِيَّةٌ. فَاشِيَّةٌ. اِصْلاَحٌ. خَالَفَ. تَصَرُّفٌ (ات). نَدِمَ عَلَى... (ا) نَدَمٌ. وَبَالٌ. فِى آخِرِ اْلأَمْرِ. رَئِيسٌ, رُؤَسَاءُ. كُفْرٌ. ذَمَّ (و) ذَمٌّ. سَابِقٌ. لاَحِقٌ. قَلَبَ (ى) قَلْبٌ. رَأْسًا عَلَى عَقِبٍ. مُسْتَعْرِبٌ (ون). تَارِيخٌ, تَوَاريخُ. بُلْدَانٌ. جُغْرَافِيَا. كَافِرٌ, كُفَّارٌ. كَانَ قَدْ ذَهَبَ. يَكُونُ قَدْ ذَهَبَ. لَيْسَ اَحَدٌ مِنْهُمْ اِلاَّ وَ يَعْرِفُ. لَيْسَ اَحَدٌ مِنْ بُلْدَانِ الْعَالَمِ اِلاَّ وَ فِيهِ الْمُسْلِمُونَ. 

\_\_\_\_\_\_\_\_\_\_\_\_\_\_\_\_\_\_\_\_\_\_\_\_\_\_\_\_\_\_\_\_\_\_

يَجِبُ عَلَىَّ اَنْ اَسْكُتَ مَادَامَ هُوَ هُنَا وَ اِذَا انْصَرَفَ اَخْبَرْتُكَ بِالْحَقِيقَةِ تَفْصِلاً. اِنْ وَصَلْتَ اِلَى الْمَدِينَةِ فِى سَاعَةٍ مُتَأَخِّرَةٍ مِنَ اللَّيْلِ فَلاَ تُتَلْفِنْ لِى فَاِنِّى بَعْدَ الْعَاشِرَةِ أَكُونُ قَدْ نِمْتُ. لَمَّا وَصَلْتُ اِلَى الْمَسْجِدِ كَانَ النَّاسُ قَدْ صَلَّوُا الْجُمُعَةَ وَ خَرَخُوا مِنْهُ. تَلْفَنْتُ لَهَ مِرَارًا وَ لَكِنِّى لَمْ اَلْقَهُ فِى الْبَيْتِ. طَالَمَا غَرَّتِ الشُّيُوعِيَّةُ النَّاسَ وَ الآنَ قَدِ انْكَشَفَ لِلنَّاسِ حَقِيقَتُهَا السَّوْدَاءُ. اَلشُّيُوعِيَّةُ اَسْوَأُ نِظَامٍ اِخْتَرَعَهُ الْهَوَى الْبَشَرِىُّ وَ لَيْسَتْ اَحْسَنَ حَالاً مِنَ النَّازِيَّةِ وَ الْفَاشِيَّةِ. اَلإِسْلاَمُ هُوَ نِظَامُ اللَّهِ ِلإِصْلاَحِ الْبَشَرِ وَ كُلُّ نِظَامٍ يُخَالِفُهُ فَهُوَ نِظَامُ الشَّيْطَانِ. لَيْسَ شَيْءٌ مِنْ اَعْمَالِ الْكُفَّارِ وَ تَصَرُّفَاتِهِمْ اِلاَّ نَدِمُوا عَلَيْهَا وَ كَانَتْ وَبَالاً عَلَيْهِمْ فِى آخِرِ اْلأَمْرِ, وَ لَيْسَ اَحَدٌ مِنْ رُؤَسَاءِ الْكُفْرِ اِلاَّ ذَمَّ اللاَّحِقُ مِنْهُمُ السَّابِقَ وَ قَلَبَهُ رَأْسًا عَلَى عَقِبٍ.

مَنْ هُوَ الْمُسْتَعْرِبُ؟ اَلْمُسْتَعْرِبُ هُوَ الَّذِى يَدْرُسُ اللُّغَةَ الْعَرَبِيَّةَ وَ يَدْرُسُ تَارِيخَ وَ جُغْرَافِيَا الْبُلْدَانِ الْعَرَبِيَّةِ.

\subsubsection{К уроку 54}
У нас загон. В загоне много кур и петухов. Однажды я зашёл в загон и поймал там петуха. Петух был нежирный, я его отпустил и поймал другого петуха. Я счастлив. Она счастлива. Однажды наша мать заболела, и мы её повели (взяли) к врачу. Врач лечил её несколько дней. Однажды наша курица вышла из загона и пошла далеко от него в поле, а там лиса поймала её и съела. Это его речь. Это речь профессора. Коран — речь Бога. Это хитрость, не хитри (не делай со мной хитрости). Я лечу больных. Доктор лечит больного. Лечи меня, я больной. Пожалуйста, закрой дверь, на улице холодно. Пожалуйста, открой окна, в комнате жарко. Ты искал

свою книгу среди моих книг? — Да, искал. Нашёл? — Не нашёл. Ищи её на этой полке. Я поймал лису и отрезал ей хвост. Лиса хитрая. У нас нет загона для кур, нам надо сделать загон для них. Надо быстро искать врача, больной сильно кричит. Лиса

каждую ночь охотится на кур и съедает их, надо сделать какую-то хитрость.

\subsection[اَلدَّرْسُ الْخَامِسُ وَ السَّبْعُونَ 75]{اَلدَّرْسُ الْخَامِسُ وَ السَّبْعُونَ 75}
مِسَاحَةٌ (ات). اَكْثَرُ. اَكْثَرُ مِنْ مِائَةٍ. مُرَبَّعٌ. \newline
اَلإِتِّحَادُ السُّوفْيَيْتِىُّ. طُولٌ. كِيلُومَتْرٌ (ات). مَسَافَةٌ (ات). 

 \includegraphics[width=1.552in,height=1.5311in]{images/MuhammadBagauddinprettified-img225.png} 

\ سَدٌّ, سُدُودٌ. سَدُّ أَسْوَانَ. غَرْبِىٌّ. حَدٌّ, حُدُودٌ. أَهَمُّ.\newline
فَحْمٌ حَجَرِىٌّ. نِفْطٌ. طَبِيعِىٌّ. غَازٌ طَبِيعِىٌّ. ثَرْوَةٌ (ات). ثَرَوَاتٌ طَبِيعِيَّةٌ. مَعْدِنٌ, مَعَادِنُ. أَطْوَلُ. نَهْرُ النِّيلِ. نَهْرُ الْفُولْغَا. يَقُومُ. مِنَ الْمُؤْسِفِ. كَلاَمٌ سَلِيمٌ. مَا هُوَ؟

\_\_\_\_\_\_\_\_\_\_\_\_\_\_\_\_\_\_\_\_\_\_\_\_\_\_\_\_\_\_\_\_\_\_\_\_\_\_\_

اَيْنَ تَسْكُنُ؟ - اَسْكُنُ فِى الاِتِّحَادِ السُّوفْيَيْتِىِّ. هَلْ الاِتِّحَادُ السُّوفْيَيْتِىُّ بَلَدٌ اِسْلاَمِيٌّ؟ - لاَ, الاِتِّحَادُ السُّوفْيَيْتِىُّ لَيْسَ بَلَدًا اِسْلاَمِيًّا. يَعْنِى اَنَّكَ تَعِيشُ فِى بَلَدٍ غَيْرِ اِسْلاَمِىٍ؟ - نَعَمْ, اَنَا اَعِيشُ فِى بَلَدٍ غَيْرِ اِسْلاَمِىٍّ. اَتَعْرِفُ مَا هِىَ مِسَاحَةُ الاِتِّحَادِ السُّوفْيَيْتِىِّ؟ - مِسَاحَةُ الاِتِّحَادِ السُّوفْيَيْتِىِّ وَاسِعَةٌ جِدًّا وَ تَبْلُغُ اَكْثَرَ مِنَ اثْنَيْنِ وَ عِشْرِينَ مِلْيُونَ كِيلُومَتْرٍ مُرَبَّعٍ. اَتَعْرِفُ طُولَ الْمَسَافَةِ مِنَ الْحُدُودِ الشَّرْقِيَّةِ لِلاِتِّحَادِ السُّوفْيَيْتِىِّ اِلَى حُدُودِهِ الْغَرْبِيَّةِ؟ - تَبْلُغُ هَذِهِ الْمَسَافَةُ عَشَرَةَ اَلاَفِ كِيلُومَتْرٍ. حَدِّثْنِى مَا هِىَ ثَرَوَاتُ الاِتِّحَادِ السُّوفْيَيْتِىِّ الطَّبِيعِيَّةُ؟ اِنَّ الاِتِّحَادَ السُّوفْيَيْتِىَّ غَنِىٌّ جِدًّا بِالثَّرَوَاتِ الطَّبِيعِيَّةِ وَ اَهَمُّهَا الْفَحْمُ الْحَجَرِىُّ وَ النِّفْطُ وَ الْغَازُ الطَّبِيعِىُّ وَ الْمَعَادِنُ الْمُخْتَلِفَةُ الأُخْرَى وَ لَكِنْ مِنَ الْمُؤْسِفِ اَنَّ كُلَّ هَذَا تَحْتَ اَيْدِى الْكُفَّارِ. - كَلاَمٌ سَلِيمٌ. اُنْظُرْ اِلَى الْخَرِيطَةِ وَ قُلْ لِّى اَىُّ النَّهْرَيْنِ اَطْوَلُ؟ نَهْرُ النِّيلِ اَمْ نَهْرُ الْفُولْغَا؟ - اَلنِّيلُ اَطْوَلُ مِنْ فُولْغَا وَ مِنْ كُلِّ نَهْرٍ فِى الْعَالَمِ. يَقُومُ عَلَى نَهْرِ النِّيلِ سَدٌّ عَالٍ فَمَا اسْمُ هَذَا السَّدِّ؟ - هُوَ سَدُّ اَسْوَانَ.

\subsubsection{К уроку 55}
На каникулах мы с младшим братом были в селе. Когда у вас кончаются каникулы? — У вас сейчас каникулы? — У меня сейчас каникулы полугодия. Ты читал эту книгу? — Я читал только половину книги. А вторую половину когда будешь читать? — Если Богу будет угодно, завтра прочитаю. Съешь половину хлеба. Половина этих денег — твоя. Первую половину поля я вспахал с утра до обеда, теперь обеденное время, надо обедать, а вторую половину поля я буду пахать после обеда. Ты был в городе, расскажи мне о том, что ты видел там. Эта вода чистая? Вода в реке в это время чистая и прозрачная. В канале течёт прозрачная вода. В нашем доме есть электрический свет. Около нашего города электростанция. Не пей нечистую воду, а пей чистую воду. Звонок прозвенел (раздался)?-Я не слышал. Я сел на поезд на этой станции. Какая ( مَا هِىَ؟ ) следующая станция? Электростанции приносят людям много пользы. Людям необходимы электростанции. Когда кончится урок? - Урок кончится через некоторое время. С Новым годом, ученики! С праздником, ученики!

\subsection{اَلدَّرْسُ السَّادِسُ وَ السَّبْعُونَ 76}
اِسْمَحْ بِالتِّلِفُونِ. - تَفَضَّلْ. مَبْرُوكْ. - اَللَّهُ يُبَارِكُ فِيكَ. بَلَغَنِى. سَلاَمَةٌ. وُصُولٌ. سَلاَمَةُ الْوُصُولِ. أُهَنِّئُكَ عَلَى سَلاَمَةِ الْوُصُولِ. يَا بُنَيَّ. طَقْسٌ. رَدِىءٌ. جَوْلَةٌ (ات). عَامٌ, اَعْوَامٌ. قَضَاءٌ. صَيْفِىٌّ. عُطْلَةٌ صَيْفِيَّةٌ. \newline
نَوَى (ى) نِيَّةٌ. سَاحِلٌ, سَوَاحِلُ. حَيْثُ... وَالِدَانِ. قَارِئٌ, قُرَّاءٌ. مُجِيدٌ. مُنْذُ وَقْتٍ قَرِيبٍ. تَرْتِيلٌ. رَتَّلَ الْقُرْآنَ. أَصْغَى اِلَى... تِلاَوَةٌ. طَبْعًا. وَ لِمَ لاَ؟ عَالٍ! وَفَّقَ. وَفَّقَكَ اللَّهُ! حَتَّى اْلآنَ. لَوْ... لَوْ عَرَفْتُ. كَمَا قُلْتُ لَكَ. رَجَانِى. أَرْجُوكَ. اللَّهُ يُسَلِّمُكَ.

\_\_\_\_\_\_\_\_\_\_\_\_\_\_\_\_\_\_\_\_\_\_\_\_\_\_\_\_\_\_\_

اِسْمَحْ بِالتِّلِفُونِ. - تَفَضَّلْ, وَ اَيْنَ تَقْصِدُ اَنْ تُتَلْفِنَ؟ - اَقْصِدُ اَنْ أُتَلْفِنَ لِصَدِيقِى وَ قَدْ رَجَانِى اَنْ أُتَلْفِنَ لَهُ بَعْدَ عَوْدَتِى مِنْ جَوْلَتِى.

مَبْرُوكٌ يَا اَحْمَدُ! بَلَغَنِى نَجَاحُكَ فِى اَدَاءِ الإِمْتِحَانِ. - اَللَّهُ يُبَارِكُ فِيكَ! اَلْحَمْدُ لِلَّهِ لَقَدْ وَفَّقَنِىَ اللَّهُ. هَنَّأَتِ الأُمُّ اِبْنَاهَا بِسَلاَمَةِ الْوُصُولِ قَائِلَةً: اَلْحَمْدُ لِلَّهِ عَلَى السَّلاَمَةِ يَا بُنَىَّ, أُهَنِّئُكَ عَلَى سَلاَمَةِ الْوُصُولِ, فَأَجَابَهَا: اَللَّهُ يُسَلِّمُكِ يَا اُمِّى. لَوْ عَرَفْنَا اَنَّ الطَّقْسَ يَكُونُ رَدِيئًا مَا قُمْنَا بِالْجَوْلَةِ اِلَى الْغَابَةِ. فِى هَذَا الْعَامِ أَنْوِى قَضَاءَ الْعُطْلَةِ الصَّيْفِيَّةِ لاَ عَلَى سَاحِلِ الْبَحْرِ كَمَا كُنْتُ اَفْعَلُ حَتَّى الآنَ, بَلْ فِى الرِّيفِ حَيْثُ يَسْكُنُ وَالِدَاىَ. كَمْ اَنَا مَسْرُورٌ بِرُؤْيَتِكَ! وَ كَمْ اَنَا مَبْسُوطٌ بِالِّقَاءِ مَعَكَ! أَنْتَ قَارِءٌ مُجِيدٌ لِلْقُرْآنِ وَ قَدْ تَعَلَّمْتُ مُنْذُ وَقْتٍ قَرِيبٍ قَوَاعِدَ التَّرْتِيلِ فَهَلْ لَّكَ أَنْ تُصْغِى اِلَى تِلاَوَتِى لِتَنْظُرَ كَيْفَ اَقْرَأُ؟ - طَبْعًا, طَبْعًا, و لِمَ لاَ؟ اِقْرَأْ اَسْمَعْ؟ - كَيْفَ؟ - كَيْفَ أَقْرَأُ؟ - عَالٍ! عَالٍ! 

\subsubsection{К уроку 56}
Это вилка, а это ложка. Вилка и ложка в тарелке, а тарелка на обеденном столе. Сегодня суп вкусный, кто готовил его? Я поел, а потом помыл свою тарелку. Всегда уважай старших. Всегда уважай урок и учителя. Не оборачивайся во время урока направо и налево. Я взял топор и срубил им старое дерево в саду. Бери молоток и забивай гвозди в эту доску. Этот гвоздь большой. Мне нужны (я хочу) маленькие гвозди. Мальчик взял пилу и распилил бревно на две половины. Где сейчас живут твои родители? Ваши родные с вами? Некоторые ученики в нашем классе послушные, а некоторые непослушные. Учитель любит послушного ученика. Бог любит послушного человека. Я люблю своих родных и всех мусульман. Мой дедушка старик, он мне всегда говорит: „Сынок мой ( يَا بُنَىَّ), будь всегда активным, старательным, послушным и воспитанным, вот тогда тебя полюбят все". Где продаются инструменты плотника? — Инструменты плотника продаются в магазине. Моя бабушка больна, и она говорит: „Ах, ах".

لدَّرْسُ السَّابِعُ وَ السَّبْعُونَ 77

\  \includegraphics[width=1.2291in,height=0.4791in]{images/MuhammadBagauddinprettified-img226.png}   \includegraphics[width=1.8646in,height=1.5102in]{images/MuhammadBagauddinprettified-img227.png}   \includegraphics[width=0.9689in,height=0.6457in]{images/MuhammadBagauddinprettified-img228.png} 

زَوْرَقٌ, زَوَارِقُ. سَاحَةُ التَّزَلُّجِ. طَنْجَرَةٌ, طَنَاجِرُ. 

سَاحَةٌ (ات). هُتَافٌ (ات). عَاشَ الإِسْلاَمُ. لاَفِتَةٌ (ات). قَامَتْ دَوْلَةُ الْمُسْلِمِينَ. لِيَسْقُطْ. اِسْتِعْمَارٌ. اِمْبِرِيَالِيَّةٌ. وَ عَلَى رَأْسِهَا. مَلْعُونٌ. نَظَرًا لِ... لَدَى... لَدَىَّ. اِنْتَظَرَ. اُتْرُكْ هَذَا الأَمْرَ اِلَىَّ. عَصِيدَةٌ, عَصَائِدُ. عَصِيدَةٌ بِالَّبَنِ. تَنَزَّهَ. تَنَزَّهَ فِى الزَّوْرَقِ. قَسَّمَ. فِئَةٌ (ات). حَسَبَ... عَلَى حِدَةٍ. اِشْرَافٌ. تَحْتَ اِشْرَافِ. مُدَرِّبٌ (ون). اِنْتَقَلَ. حَىٌّ, اَحْيَاءٌ. مُجَاوِرٌ. مَا لاَ يَقِلُّ عَنْ...

\_\_\_\_\_\_\_\_\_\_\_\_\_\_\_\_\_\_\_\_\_\_\_\_\_

سَمِعْنَا يَوْمَ الْعِيدِ فِى السَّاحَةِ هُتَافَاتٍ ِلأَبْنَاءِ الْمُسْلِمِينَ: عَاشَ الإِسْلاَمُ! عَاشَ الْمُسْلِمُونَ! قَامَتْ دَوْلَةُ الْمُسْلِمِينَ! لِتَسْقُطْ دَوْلَةُ الْكُفَّارِ! وَ بَعْضُهُمْ يَحْمِلُ فِى يَدَيْهِ لاَفِتَاتٍ مَكْتُوبٍ عَلَيْهَا: لِيَسْقُطِ الإِسْتِعْمَارُ! لِتَسْقُطِ الإِمْبِرِيَالِيَّةُ! لِتَسْقُطْ كُلُّ الأَنْظِمَةِ غَيْرِ الإِسْلاَمِيَّةِ وَ عَلَى رَأْسِهَا الشُّيُوعِيَّةُ الْمَلْعُونَةُ! نَظَرًا ِلأَنَّهُ لَيْسَ لَدَىَّ وَقْتٌ لِلإِنْتِظَارِ. اَتْرُكُ هَذَا الأَمْرَ اِلَيْكَ لِتَقُومَ اَنْتَ بِهِ بَدَلاً مِنِّى. يَا بِنْتِى ضَعِى طَنْجَرَةً عَلَى الْمَوْقِدِ وَ اطْبَخِى عَصِيدَةً بِالَّبَنِ لِغَدَائِنَا فَسَنَخْرُجُ بَعْدَ تَنَاوُلِ الْغَدَاءِ لِلتَّنَزُّهِ فِى الزَّوْرَقِ. فِى الشِّتَاءِ نُقَسِّمُ التَّلاَمِيذَ اِلَى عِدَّةِ فِئَاتٍ حَسَبَ اَعْمَارِهِمْ ثُمَّ تَذْهُبُ كُلُّ فِئَةٍ عَلَى حِدَةٍ اِلَى سَاحَةِ التَّزَلُّجِ تَحْتَ اِشْرَافِ مُدَرِّبٍ. هُوَ انْتَقَلَ اِلَى شِقَّةٍ أُخْرَى حَصَلَ عَلَيْهَا مُنْذُ وَقْتٍ قَرِيبٍ فِى الْحَىِّ الْمُجَاوِرِ. اِنْتَظَرْتُكَ الْيَوْمَ مَا لاَ يَقِلُّ عَنْ سَاعَتَيْنِ وَ انْظَرْتُكَ اَمْسِ مَا لاَ يَقِلُّ عَنْ سَاعَةٍ لِمَاذَا تَتَأَخَّرُ عَنْ مِيعَادِكَ هَكَذَا دَائِمًا؟

\subsubsection{К уроку 57}
Сколько у тебя рублей? За сколько ты купил нож? За сколько вы купили этот дом? А вы почём продаёте его? Этот дорогой, а этот дешёвый. Это стоит 10 рублей, а это стоит 1 рубль. Сколько учеников отсутствовали на уроке? — На уроке отсутствовали 2 ученика. На (возьми) деньги и положи их в карман. Я всегда ложусь рано и встаю с постели рано, и никогда не отсутствую на уроке. У нас сегодня важный урок — урок арабского языка. Два преподавателя. Три студента. Четыре книги. Пять досок. Шесть газет. Семь уроков. Восемь товарищей. Девять комнат. Десять быков. Пять лет. Два стакана холодной воды. Три чашки горячего кофе. Я прочитал шесть книг. В нашем городе восемь институтов и десять мечетей. На утреннем намазе в мечети было 9 рядов и в каждом ряду (по) 8 человек. В моём кармане 2 рубля и 2 копейки. Сколько стоит книга? — Книга стоит два с половиной рубля. Она дешёвая? — Да, недорогая. Когда ты уезжаешь в горы? — Через 3 дня. Сколько дней ты там бу­дешь? — 5 дней.

\subsection{اَلدَّرْسُ الثَّامِنُ وَ السَّبْعُونَ 78}
\  \includegraphics[width=0.6874in,height=0.8957in]{images/MuhammadBagauddinprettified-img229.png}   \includegraphics[width=0.4272in,height=0.948in]{images/MuhammadBagauddinprettified-img230.png}   \includegraphics[width=0.448in,height=0.8957in]{images/MuhammadBagauddinprettified-img231.png} 

كَفٌّ, كُفُوفٌ. اِصْبَعٌ, اَصَابِعُ. ظُفْرٌ, اَظْفَارٌ. 

\  \includegraphics[width=0.5626in,height=0.9272in]{images/MuhammadBagauddinprettified-img232.png}   \includegraphics[width=0.552in,height=1.052in]{images/MuhammadBagauddinprettified-img233.png}   \includegraphics[width=0.5728in,height=0.8228in]{images/MuhammadBagauddinprettified-img234.png} 

فَمٌ, اَفْوَاهٌ. لِسَانٌ, اَلْسِنَةٌ. اُذُنٌ, آذَانٌ.

كَذَبَ (ى) كَذِبٌ. حَفِظَ لِسَانَهُ. تَكَلَّمَ. مُسْتَحِيلٌ! تَغَيَّرَ. صَرَفَ (ى) صَرْفٌ. صَرَفَ الْوَقْتَ. مُعْظَمٌ. لَهْوٌ. لَعِبٌ. تَمَامًا. عَلَى مَا يَبْدُو. غَرِيبٌ! أَسَاءَ اِلَى... تَوَقَّعَ. فَضْلٌ. عَيْبٌ عَلَيْكَ. يُمْكِنُ. كَيْفَ يُمْكِنُ؟ مِثْلٌ, اَمْثَالٌ. مِثْلُ هَذَا. دَخَّنَ, يُدَخِّنُ . مَمْنُوعٌ. شَرَعَ فِى... (ا) شُرُوعٌ. اَهْلٌ, اَهَالٍ. اَهْلُ الْقَرْيَةِ. بِنَاءٌ. عَلَى اَثَرِ... اِجْتِمَاعٌ (ات). بِشَأْنِ... عَقَدَ الإِجْتِمَاعَ (ى) عَقْدٌ. كَسُولٌ, كُسْلٌ. فِيمَا أَعْلَمُ. 

\_\_\_\_\_\_\_\_\_\_\_\_\_\_\_\_\_\_\_\_\_\_\_\_\_\_\_\_\_\_\_\_\_\_

يَااَيُّهَا الْوَلَدُ الْمُسْلِمُ اِحْفَظْ لِسَانَكَ مِنَ الْكَذِبِ. يَا اَيُّهَا الأَوْلاَدُ الْمُسْلِمُونَ اِحْفَظُوا اَلْسِنَتَكُمْ مِنَ الْكَذِبِ. الْكَاذِبُ مَلْعُونٌ فِى كِتَابِ اللَّهِ. التِّلْمِيذُ الْمُؤَدَّبُ لاَ يَكْذِبُ اَبَدًا. بِالأُذُنِ نَسْمَعُ وَ بِاللِّسَانِ نَتَكَلَّمُ وَ بِالْفَمِ نَأْكُلُ. لِكُلِّ كَفٍّ خَمْسُ أَصَابِعَ وَ لِكُلِّ اِصْبَعٍ ظُفْرٌ. أَدَّى عَادِلٌ اَلإِمْتِحَانَ بِنَجَاحٍ. - هَذَا مُسْتَحِيلٌ, اِنَّهُ كَانَ, فِيمَا اَعْلَمُ, تِلْمِيذًا كَسُولاً كَثِيرًا مَّا يَتَخَلَّفُ عَنِ الدُّرُوسِ وَ يَصْرِفُ مَعْظَمَ اَوْقَاتِهِ فِى اللَّهْوِ وَ اللَّعِبِ, غَرِيبٌ. - تَغَيَّرَ تَمَامًا, عَلَى مَا يَبْدُو. يَا تِلْمِيذُ اِصْرَفْ مُعْظَمَ اَوْقَاتِكَ فِى الدَّرْسِ وَ الْعِلْمِ وَ لاَ تَصْرِفْهُ فِى اللَّهْوِ وَ اللَّعِبِ. لَمْ اَكُنْ اَتَوَقَّعُ هَذَا مِنْكَ, لَقَدْ اَسَأْتَ اِلَىَّ بِتَصَرُّفِكَ هَذَا وَ نَسِيتَ فَضْلِى عَلَيْكَ, عَيْبٌ عَلَيْكَ, مَاذَا تَعْمَلُ؟ وَ مَاذَا تَقُولُ؟ كَيْفَ يُمْكِنُ عَمَلُ مِثْلِ هَذَا وَ قَوْلُ مِثْلِ هَذَا؟

مَنْ دَخَّنَ هُنَا؟ اَاَنْتَ دَخَّنْتَ هُنَا؟ لاَ تُدَخِّنْ هُنَا, اَلتَّدْخِينُ هُنَا مَمْنُوعٌ. شَرَعَ اَهْلُ الْقَرْيَةِ فِى بِنَاءِ الْمَسْجِدِ عَلَى اَثَرِ الإِجْتِمَاعِ الَّذِى عُقِدَ بِشَأْنِ بِنَاءِ الْمَسْجِدِ.

\subsubsection{К уроку 58}
Почему ты сегодня поздно пришёл в школу? Что тебя задержало? — Я лёг поздно и поэтому встал с постели утром поздно. Как ваши дела? — Спасибо, хорошо. Где ты сейчас живёшь (проживаешь)? — С родными в селении. До свидания! — До свидания! Кто сегодня присутствует? Кто отсутствует? Когда заходят отец или учитель, или Другой старший человек, (ты) вставай в знак уважения к ним. Где студенты? — Они в аудитории. Что идёт в кинотеатре? ( مَاذَا يُعْرَضُ فِى دَارِ السِّينَمَا ) В кинотеатре сегодня идёт новый фильм. В каком ряду ты сидел (был) в кино? — Я сидел (был) в первом ряду. А кто рядом с тобой сидел? — Рядом со мной сидел мой верный друг Хасан. Мы с ним вместе учим уроки, вместе играем. Я задал ему ( اَلْقَيْتُ عَلَيْهِ) вопрос, и он сразу ответил на мой вопрос. По окончании

\ намаза в мечети мы разошлись по домам. В клубе есть люди? В вашем квартале одна мечеть? — Нет, в нашем квартале две мечети, потому что наш квартал большой. Махмуд не знает новые слова этого Урока, потому что он отсутствовал на нём. Почему Мухаммед отсутствует на намазе? Почему Фатима отсутствует на уроках?

\subsection{اَلدَّرْسُ التَّاسِعُ وَ السَّبْعُونَ 79}
مَضَى عَلَ... (ى) مُضِىٌّ. مَضَى عَلَى مَا مَاتَ. مَا يَزِيدُ عَلَى مِائَةٍ. اِفْتَرَقَ. مَا اَسْرَعَ مَا... صَالِحٌ, صُلَحَاءُ. مُحْسِنٌ اِلَى... يَغْفِرُ اللَّهُ لَهُ. هُوَ عَلَى قَيْدِ الْحَيَاةِ. دَرَّسَ. اَكْثَرُهُمْ. بِمَا فِى ذَلِكَ. يَا خَسَارَه! يَا مُصِيبَه! عُنْفُوَانُ الشَّبَابِ. فَارَقَ الْحَيَاةَ. اِنْتَقَلَ اِلَى جِوَارِ رَبِّهِ. رَحِمَهُ اللَّهُ. مَوْتٌ. حَتْمٌ. صَبَرَ (ى) صَبْرٌ. شُدَّ حَيْلَكَ. اِسْتَسْلَمَ. نَصْرٌ. صَابِرٌ (ون). أَجْزَاخَانَةٌ (ات). مِنْ قَبْلُ. عَادَ الْمَرِيضَ (و) عِيَادَةٌ. مُسْتَشْفًى, مُسْتَشْفَيَاتٌ. اَلشِّفَاءَ الْعَاجِلَ. مُجَاهِدٌ (ون). حَادِثٌ, حَوَادِثُ. حَادِثُ طَرِيقٍ. فِى سَبِيلِ اللَّهِ.

\_\_\_\_\_\_\_\_\_\_\_\_\_\_\_\_\_\_\_\_\_\_\_\_\_\_\_\_\_\_\_\_\_\_\_

مَضَى عَلَى مَا افْتَرَقْنَا اَنَا وَ اَبُوكَ مَا يَزِيدُ عَلَى عَشْرِ سَنَوَاتٍ فَمَاذَا فَعَلَ بَعْدَ ذَلِكَ؟ وَ اَيْنَ هُوَ الآنَ؟ - قَدْ مَاتَ. - يَا لَلأَسَفِ! مَا اَسْرَعَ مَا مَاتَ! لَقَدْ كَانَ رَجُلاً صَالِحًا مُحْسِنًا اِلَى الْفُقَرَاءِ يَغْفِرُ اللَّهُ لَهُ. - وَ اَبُوكَ كَيْفَ حَالُهُ؟ وَ اَيْنَ هُوَ؟ هُوَ مَا يَزَالُ عَلَى قَيْدِ الْحَيَاةِ وَ يَسْكُنُ حَيْثُ سَكَنَ مِنْ قَبْلُ.

تُدَرَّسُ فِى جَامِعَتِنَا اَكْثَرُ اللُّغَاتِ الشَّرْقِيَّةِ بِمَا فِى ذَلِكَ اُورْدُو. يَا خَسَارَه! يَا مُصِيبَه! فَتًى فِى عُنْفُوَانِ الشَّبَابِ وَقَعَ فِى حَادِثِ طَرِيقٍ فَفَارَقَ الْحَيَاةَ وَ انْتَقَلَ اِلَى جِوَارِ رَبِّهِ, رَحِمَهُ اللَّهُ! مَاذَا نَعْمَلُ؟ الْمَوْتُ حَتْمٌ عَلَى كُلِّ اِنْسَانٍ. اِصْبِرْ وَ شُدَّ حَيْلَكَ وَ لاَ تَسْتَسْلِمْ فَنَحْنُ مَعَكَ وَ النَّصْرُ قَرِيبٌ وَ اللَّهُ مَعَ الصَّابِرِينَ. مَرِضَ جَارُنَا مُنْذُ اَيَّامٍ فَاُخِذَ اِلَى الْمُسْتَشْفَى فَلَمَّا اَرَدْنَا اَنْ نَّعُودَهُ عَلِمْنَا اَنَّهُ بِحَاجَةٍ اِلَى أَدْوِيَةٍ فَذَهَبْنَا اِلَى الأَجْزَاخَانَةِ وَ اشْتَرَيْنَا مِنْهَا اَدْوِيَةً ثُمَّ عُدْنَا جَارَنَا الْمَرِيضَ فِى الْمُسْتَشْفَى وَ عِنْدَ الْخُرُوجِ قُلْنَا لَهُ: الشِّفَاءَ الْعَاجِلَ يَا جَارَنَا فَشَكَرَنَا. نَعْلَمُ اَنَّهُ كَانَ مُجَاهِدًا فِى سَبِيلِ اللَّهِ وَ مَاتَ فِى سَبِيلِ اللَّهِ يَغْفِرُ اللَّهُ لَهُ.

\subsubsection{К уроку 59}
Эта газета политическая. Эта книга литературная. Этот журнал литературно-политический. Сколько политических и сколько литера­турных книг на этажерке? Этажерка новая, я её купил вчера за 100 рублей. Где сидят студенты? — Студенты сидят на стульях около окна. Ты эту статью читал? — Нет, не читал, о чём она? — Статья написана о распространении Ислама среди народов мира, особенно среди молодёжи. Почему среди молодёжи? — Потому что она (молодёжь) быстро всё понимает, а что касается стариков, то они никогда не понимают. В клубе идёт хороший фильм, пойдёшь со мной смотреть фильм? — Нет, не пойду, я очень занят, у меня много работы. Где джамаат? — Это полдень, сейчас джамаат в мечети, джамаат занят намазом. Чем занят ученик? — Ученик занят уроками. Моя комната просторная и светлая. В моей комнате 3 окна. На столе электролампа. На полках много печатных и рукописных книг. У тебя есть сегодняшняя газета? — Есть, на, читай.

\subsection{الدَّرْسُ الثَّمَانُونَ 80}
اَثْبَتَ. زَمَنٌ, اَزْمَانٌ. فِى الْعَالَمِ بِأَسْرِهِ. دُوَيْلَةٌ (ات). اِتَّحَدَ. كَوَّنَ. دَوْلَةٌ اِسْلاَمِيَّةٌ عُظْمَى. مُنَظَّمَة (ات). الإِخْوَانُ الْمُسْلِمُونَ. قِيَادَةٌ. تَحْتَ قِيَادَةِ... اِمَامٌ, اَئِمَّةٌ. ثَوْرَةٌ (ات). نَهْضَةٌ. نَهْضَةٌ اِسْلاَمِيَّةٌ. ثَوْرَةٌ اِسْلاَمِيَّةٌ. سَبَبٌ, اَسْبَابٌ. أَشْعَلَ. عَصْرٌ, عُصَورٌ. فِى عَصْرِنَا هَذَا. ضَوْءٌ, أَضْوَاءٌ. اَشْعَلَ الضَّوْءَ. عَكَفَ عَلَى... (و) عُكُوفٌ. حَتَّى النَّوْمِ. قَبْلَ النَّوْمِ. أَطْفَأَ. أَحْسَنَ. أَحْسَنَ السِّبَاحَةَ. أَحْسَنَ اللُّغَةَ. فِيمَا بَيْنَهُمْ. أَصْلاً. جُبْنَةٌ. خَوْفًا مِنْهُ. تَسَلَّقَ. لَيْتَ. صَحِيحٌ, اَصِحَّاءُ. جَاهَدَ. قَوِيٌّ, أَقْوِيَاءُ. صَحْوَةٌ. صَحْوَةُ الشَّبَابِ. يَلِيهِ. يَلِي هَذَا.

\_\_\_\_\_\_\_\_\_\_\_\_\_\_\_\_\_\_\_\_\_\_\_\_\_\_\_\_\_

اَثْبَتَ الزَّمَنُ اَنَّهُ لاَبُدَّ لِلنَّاسِ فِى الْعَالَمِ بِأَسْرِهِ مِنَ الرُّجُوعِ اِلَى الإِسْلاَمِ وَ لاَبُدَّ لِلدُّوَلِ وَ الدُّوَيْلاَتِ اَنْ تَتَّحِدَ وَ تُكَوِّنَ دَوْلَةً اِسْلاَمِيَّةً عُظْمَى. مُنَظَّمَةُ "الإِخْوَانُ الْمُسْلِمُونَ" تَحْتَ قِيَادَةِ الإِمَامِ حَسَنِ الْبَنَّا وَ الثَّوْرَةُ الإِسْلاَمِيَّةُ فِى اِيرَانِ تَحْتَ قِيَادَةِ الإِمَامِ الْخُمَيْنِىِّ كَانَتَا مِنْ اَهَمِّ اَسْبَابِ النَّهْضَةِ الإِسْلاَمِيَّةِ فِى عَصْرِنَا هَذَا. عَصْرُنَا هَذَا عَصْرُ النَّهْضَةِ الإِسْلاَمِيَّةِ وَ الصَّحْوَةِ الإِسْلاَمِيَّةِ وَ الْعَصْرُ الَّذِى يَلِيهِ هُوَ عَصْرُ الإِسْلاَمِ كَامِلاً اِنْ شَاءَ اللَّهُ.

اُشْعِلُ الضَّوْءَ فِى غُرْفَتِى مَسَاءَ كُلِّ يَوْمٍ وَ اَعْكُفُ عَلَى قِرَاءَةِ كُتُبِى وَ كِتَابَةِ دُرُوسِى حَتَّى النَّوْمِ ثُمَّ اُصَلِّى الْعِشَاءَ وَ قَبْلَ النَّوْمِ اُطْفِئُ الضَّوْءَ. مَنْ اَطْفَأَ الضَّوْءَ؟ حَانَ وَقْتُ النَّوْمِ قُمْ وَ اَطْفِئِ الضَوْءَ وَ اَذْهَبْ اِلَى غُرْفَةِ نَوْمِكَ.

اَوْلاَدِى يُحْسِنُونَ اللُّغَةَ الْعَرَبِيَّةَ وَ يَتَكَلَّمُونَ فِيمَا بَيْنَهُمْ بِهَا.

خَطَفَ هِرٌّ جُبْنَةً وَ هَرَبَ بِهَا خَوْفًا مِنَ النَّاسِ فَتَسَلَّقَ شَجَرَةً قَرِيبَةً مِنَ الدَّارِ. لَيْتَنِى كُنْتُ عَالِمًا فَاُعَلِّمَ الأَوْلاَدَ وَ لَيْتَنِى كُنْتُ صَحِيحًا قَوِيًّا فَاُجَاهِدَ فِى سَبِيلِ اللَّهِ. يَا فَتَى اِنْ كُنْتَ تُحِبُّ اللَّهَ فَجَاهِدْ فِى سَبِيلِ اللَّهِ فَمَنْ اَحَبَّ اللَّهَ جَاهَدَ فِى سَبِيلِهِ.

\subsubsection[К уроку 60]{К уроку 60}
Сколько секунд в минуте? — В минуте 60 секунд. Сколько минут в часе? — В часе 60 минут. Сколько часов в сутках? — В сутках 24 часа. Сколько дней в году? — В году 365 дней. Сколько кварталов в году? — В году 4 квартала: весна, лето, осень и зима. Сколько дней в неделе? — В неделе 7 дней: понедельник, вторник, среда, четверг, пятница, суббота, воскресенье. 11 мужчин, 12 женщин, 13 книг. 20 журналов. 45 дней. 51 статей. 21 студент. 21 студентка. 17 портфелей. 18 рабочих. 20 газет. 88 столов. 67 комнат. 12 статей. 15 друзей. 16 домов. 54 двери. 62 города. 30 лет. 22 карандаша. 25 газет. 32 журнала. 81 школьница. 18 товарищей. Сколько стоит бык? — Бык стоит 2000 рублей. Сколько стоит машина? — Машина стоит 200 000 рублей. Сколько книг в большой библиотеке? — Там 2 000 000 книг. Сколько студентов учится с тобой на одном курсе? — Со мной на одном курсе учатся 44 студента, 20 из них — иностранцы, 24 — советские. Они все мусульмане? — Хвала Богу, они все мусульмане. Я рад им.

\subsection{اَلدَّرْسُ الْحَادِى وَ الثَّمَانُونَ 81}
مُنْتَبِهٌ. اَحْسَنْتَ! زُجَاجٌ. لَوْحُ الزُّجَاجِ. خَطَأً. عَنَّفَ. بِبُطْءٍ. فِى طَرِيقِهِ اِلَى... اَهْلاً وَ سَهْلاً. مَرْحَبًا بِكُمْ. بَيْنَمَا. اِذْ. غَرِيبٌ, غُرَبَاءُ. جَانِبٌ, جَوَانِبُ. لَمْ اَرَهُ. عَلَى عَجَلٍ. عَزِيزٌ, اَعِزَّاءُ. صَادَفَ. تَبَادَلَ. تَحِيَّةٌ. رَأْىٌ, آرَاءٌ. تَبَادَلْنَا الرَّأْىَ. هَا قَدْ جَاءَ. مَنْجَمٌ, مَنَاجِمُ. عَامِلُ مَنْجَمٍ. لَعِبٌ, اَلْعَابٌ. رِيَاضِىٌّ. أَلْعَابٌ رِيَاضِيَّةٌ. فَتْرَةُ الإِسْتِرَاحَةِ. تَمَازَحَ. شُغْلٌ, اَشْغَالٌ. بِلاَ شُغْلٍ. فَرْضٌ, فُرُوضٌ. اَلْعَالَمُ الإِسْلاَمِىُّ. 

\_\_\_\_\_\_\_\_\_\_\_\_\_\_\_\_\_\_\_\_\_\_\_\_\_\_\_\_\_\_\_\_\_\_\_

فَاطِمَةٌ تِلْمِيذَةٌ نَشِيطَةٌ تَجْلِسُ وَقْتَ الدَّرْسِ مُنْتَبِهَةً. قَالَتْ لَهَا الْمُعَلِّمَةُ أَمْسِ لَمَّا اَجَابَتْ عَلَى سُؤَالِهَا: اَحْسَنْتِ يَا فَاطِمَةُ! كَسَرْتُ لَوْحَ الزُّجَاجِ خَطَأً فَعَنَّفَتْنِى أُمِّى. رَأَيْتُ شَيْخًا يَسِيرُ بِبُطْءٍ فِى طَرِيقِهِ اِلَى الْمَسْجِدِ لِلصَّلاَةِ. بَيْنَمَا اَنَا اَسِيرُ عَلَى عَجَلٍ فِى طَرِيقِى اِلَى الْمَحَطَّةِ اِذْ صَاحَ بِى مِنْ جَانِبِ الطَّرِيقِ رَجُلٌ غَرِيبٌ لاَ اَعْرِفُهُ وَ لَمْ أَرَهُ قَطُّ. عُدْنَا مُعَلِّمَنَا الْمَرِيضَ فَقَالَ لَنَا: اَهْلاً وَ سَهْلاً يَا أَوْلاَدِى, مَرْحَبًا بِكُمْ, تَفَضَّلُوا اُدْخُلُوا وَ صَاحَ بِزَوْجَتِهِ: اَحْضِرِى لَنَا الشَّاىَ وَ قَدِّمِى لَنَا الطَّعَامَ هَا قَدْ جَاءَنِى أَوْلاَدِىَ الأَعِزَّاءُ.

الْيَوْمَ صَادَفْتُ عُمَرَ وَ تَبَادَلْتُ مَعَهُ التَّحِيَّةَ ثُمَّ تَبَادَلْتُ مَعَهُ الرَّأْىَ فِيمَا سَنَدْرُسُهُ مِنَ الْكُتُبِ بَعْدَ هَذَا الْكِتَابِ. عُمَّالُ الْمَنَاجِمِ فِى بَلْدَتِنَا كُلُّهُمْ شُبَّانٌ وَ كُلُّهُمْ يُحِبُّونَ الأَلْعَابَ الرِّيَاضِيَّةَ لاَسِيَّمَا الْكُرَةَ الطَّائِرَةَ وَ كُرَةَ السَّلَّةِ وَ يَتَبَادَلُونَ النُّكَةَ فِى فَتْرَةِ الإِسْتِرَاحَةِ وَ يَتَمَازَحُونَ. لاَ تَجْلِسْ هُنَا بِلاَ شُغْلٍ بَلِ اذْهَبْ اِلَى حُجْرَتِكَ وَ اكْتُبْ فُرُوضَكَ اَوِ اسْتَمِعْ اِلَى الإِذَاعَةِ الْعَرَبِيَّةِ لِتَعْلَمَ مَا هُوَ الْجَدِيدُ فِى الْعَالَمِ الإِسْلاَمِىِّ.

\subsubsection{К уроку 61}
Человек, которого мы видели во дворе мечети, — ( هُوَ ) наш имам. А женщина, которая стояла рядом с ним, — ( هِىَ ) его жена. Стадион, на котором студенты играли, — ( هُوَ) городской стадион, и парни, которые там играли, — это студенты Исламской школы. Кто такой имам? — Имам — это тот, кто молится (стоя) впереди людей. А кто такой муэдзин? — Муэдзин — это тот, кто призывает людей на намазы в их время. Юноши и девушки, уважайте матерей, которые кормят вас. Ученики и ученицы, уважайте учителей и учительниц, которые обучают вас. Ты болел уже месяц, когда ты выздоро­вел? — Я выздоровел неделю тому назад. Откуда ты сейчас возвращаешься? — Я возвращаюсь со стадиона. Там играла наша команда с командой другого института. Какая команда выиграла? — Выиграла их команда. Где письмо, которое ты получил от отца? — В портфеле, который находится в шкафу. Как называется женщина, у которой нет мужа? И как называется мужчина, у которого нет жены? — Её называют незамужней, а его называют холостым.

\subsection{الدَّرْسُ الثَّانِى وَ الثَّمَانُونَ 82}
سَهِرَ (ا) سَهَرٌ. اَلْبَارِحَةَ. مِنَ الثَّابِتِ أَنَّ... تَبْدِيلٌ. تَبْدِيلُ الْهَوَاءِ. مَسْكَنٌ, مَسَاكِنُ. حِينٌ, اَحْيَانٌ. حِينًا بَعْدَ حِينٍ. 

\  \includegraphics[width=1.4165in,height=1.2811in]{images/MuhammadBagauddinprettified-img235.png} 

تَعَرَّضَ لِ‍... مِظَلَّةٌ (ات). مَجْرَى الْهَوَاءِ. تَعَرَّضَ لِمَجْرَى الْهَوَاءِ. ضَيِّقٌ. مُلْتَوٍ. مُسْتَقِيمٌ. اَصْبَحَ. كِلاَهُمَا. كِلْتَاهُمَا. فَرَغَ مِنْ... (ا) فَرَاغٌ. تَفَرَّغَ لِ‍... كَانَ عَلَيْنَا اَنْ... هَمْسًا. فِى الْمَنَامِ. فِى اسْتِطَاعَتِى. 

لَيْسَ فِى اسْتِطَاعَتِى. مُتَرْجِمٌ. يَوْمٌ مُصْحٍ. مَا كَادَ يَذْهَبُ حَتَّى رَجَعَ. مَا كِدْتُ أَجْلِسُ حَتَّى نَظَرَاِلَىَّ. اَخَذَ يَقْرَأُ. سَقَطَ الْمَطَرُ. بَلَّ (و) بَلٌّ. سَوْفَ يَذْهَبُ.

\_\_\_\_\_\_\_\_\_\_\_\_\_\_\_\_\_\_\_\_\_\_\_\_\_\_

سَهِرْتُ الْبَارِحَةَ وَ لَمْ أَنَمْ طُولَ اللَّيْلَةِ ِلأَنَّ الدُّرُوسَ كَانَتْ كَثِيرَةً. مِنَ الثَّابِتِ اَنَّ تَبْدِيلَ الْهَوَاءِ فِى الْمَسَاكِنِ حِينًا بَعْدَ حِينٍ مِنَ الأُمُورِ الضَّرُورِيَّةِ لِلصِّحَّةِ. مَنْ اَرَادَ الْعِلْمَ سَهِرَ اللَّيَالِىَ. تَعَرَّضْتُ وَ اَنَا رَاكِبٌ سَيَّارَةَ سَيِّدْ اَحْمَدَ لِمَجْرَى الْهَوَاءِ فَعَلِمْتُ اَنَّنِى سَأَمْرَضُ حَتْمًا. كَانَتْ شَوَارِعُ مَدِينَتِنَا مِنْ قَبْلُ ضَيِّقَةً وَ مُلْتَوِيَةً فَاَصْبَحَتِ الآنَ عَرِيضَةً وَ مُسْتَقِيمَةً. كَانَ هَذَانِ الْكِتَابَانِ كِلاَ هُمَا فِى الْخِزَانَةِ. هَاتَانِ الْفَتَاتَانِ كِلْتَاهُمَا طَالِبَتَانِ فِى الْمَعْهَدِ الإِسْلاَمِىِّ. فَرَغْنَا مِنَ الْكِتَابَةِ وَ الآنَ نَشْرَعُ فِى الْقِرَاءَةِ وَ بَعْدَ الْقِرَاءَةِ سَوْفَ نَتَفَرَّغُ لِلَّعِبِ. كَانَ عَلَيْنَا اَنْ نَتَكَلَّمَ هَمْسًا لِكَيْلاَ نُزْعِجَ الْمَرِيضَ فِى مَنَامِهِ. هُوَ اَجْنَبِىٌّ لَيْسَ مِنْ بَلَدِى وَ لَيْسَ فِى اسْتِطَاعَتِى أَنْ أَفْهَمَ كَلاَمَهُ فَلاَبُدَّ اَنْ يَكُونَ بَيْنَنَا مُتَرْجِمٌ. كَانَ الْيَوْمُ مُصْحِيًا عِنْدَ دُخُولِىَ الْمَسْجِدَ وَ مَا كِدْتُ اَخْرُجُ مِنَ الْمَسْجِدِ حَتَّى اَخَذَ يَسْقُطُ الْمَطَرُ وَ كَانَتْ بِيَدِى مِظَلَّةٌ فَفَتَحْتُهَا لِئَلاَّ يَبُلَّنِىَ الْمَطَرُ. بَيْنَمَا اَنَا رَاجِعٌ مِنَ الْجَامِعِ مِنْ صَلاَةِ الصُّبْحِ اِذْ سَقَطَ الْمَطَرُ فَبَلَّ ثِيَابِى وَ لَمْ يَكُنْ بِيَدِى مِظَلَّةٌ.

\subsubsection{К уроку 62}
Ты приготовил сено для своего скота на зимнее время? — Да, я приготовил много сена на зиму. А где ты его хранишь? — Я его. храню в сарае. Когда приходит зима, кончаются работы крестьян и крестьянок, а когда приходит весна, они выходят на поля, чтобы пахать землю и сеять зерно. Когда работают крестьяне и когда они отдыхают? — Крестьяне работают весной, летом и осенью, а зимой отдыхают. А что касается учеников и студентов, то они занимаются осенью, зимой и весной, а отдыхают летом. Летом скот пасётся на пастбище и ест зелёную траву, а зимой ест сено. Мой топор тупой, я пришёл к тебе, чтобы ты наточил его. — Оставь его у меня, я поточу его вечером, сейчас я очень занят. Сено, которое скосили вчера, высохло? — Да, высохло. — Когда перенесём его в сарай? — Не знаю, посмотрим, когда у нас будет время. На чём перевезём? На машине или на тракторе. Оставим эту задачу до тех пор, пока не придёт учитель, а когда он придёт, спросим его. — Хорошо, оставим. От чего получается мука? — Мука получается от размола, зерна. Где дедушка? — Он спит в своей комнате, отдыхает.

\subsection{اَلدَّرْسُ الثَّالِثُ وَ الثَّمَانُونَ 83}
 \includegraphics[width=1.5311in,height=1.1874in]{images/MuhammadBagauddinprettified-img236.png}   \includegraphics[width=1.9063in,height=0.9063in]{images/MuhammadBagauddinprettified-img237.png}  \includegraphics[width=1.2083in,height=1.7709in]{images/MuhammadBagauddinprettified-img238.png} 

\ بِطِّيخٌ. شَمَّامٌ. لِفْتٌ. 

\  \includegraphics[width=1.0311in,height=1.6252in]{images/MuhammadBagauddinprettified-img239.png}   \includegraphics[width=1.4791in,height=1.1043in]{images/MuhammadBagauddinprettified-img240.png}   \includegraphics[width=1.7291in,height=1.1665in]{images/MuhammadBagauddinprettified-img241.png} 

\ بَصَلٌ. كُرُنْبٌ. خِيَارٌ. 

 \includegraphics[width=1.4583in,height=1.6354in]{images/MuhammadBagauddinprettified-img242.png}  \includegraphics[width=1.8228in,height=1.1772in]{images/MuhammadBagauddinprettified-img243.png}   \includegraphics[width=0.7291in,height=0.8854in]{images/MuhammadBagauddinprettified-img244.png} 

\ فِجْلٌ. جَزَرٌ. ثُومٌ.

رَفِيقٌ, رُفَقَاءُ. اَسْرَعَ اِلَى... دُونَ اَنْ يَجْلِسَ. وَ لَوْ... 

\  \includegraphics[width=2.052in,height=1.0937in]{images/MuhammadBagauddinprettified-img245.png}   \includegraphics[width=0.5102in,height=2.0102in]{images/MuhammadBagauddinprettified-img246.png} 

اَعْطِِنِى وَ لَوْ دِرْهَمًا. مَبْقَلَةٌ, مَبَاقِلُ. مَنَارَةٌ, مَنَاوِرُ. 

مِنْ عَلَى الطَّاوِلَةِ. مَنْ. رَأَيْتُ مَنْ جَاءَكَ. نَحْوَ الْبَيْتِ. مُتَوَجِّهًا نَحْوَ الْبَيْتِ. اِنْطَلَقَ. لاَ يَلْوِى عَلَى شَىْءٍ. دَاخِلَ... دَاخِلَ الْمَدِينَةِ. خَالِدٌ. نَبَتَ (و) نَبْتٌ. هَمَّ, هُمُومٌ. غَالٍ - اَغْلَى. كَمْ سَنَةً عُمْرُكَ؟ نَاهَزْتُ خَمْسِينَ سَنَةً. بُلُوغٌ. نَاهَزْتُ الْبُلُوغَ. اِذذَّاكَ. عَلَى الأَكْثَرِ. عَلَى الأَقَلِّ. رَاتِبٌ, رَوَاتِبُ. تَقَاضَى. قُوَّةٌ, قُوًى. مِقْدَارٌ, مَقَادِيرُ. يَوْمُ الْقِيَامَةِ. اِلْتَحَقَ بِ... اِلْتَحَقَ بِالْجَامِعَةِ.

\_\_\_\_\_\_\_\_\_\_\_\_\_\_\_\_\_\_\_\_\_\_\_\_\_\_\_\_\_\_\_\_

لِى رَفِيقٌ نَشِيطٌ لاَ يَكَادُ يَصِلُ اِلَى الْبَيْتِ مِنَ الْجَامِعَةِ وَ يَتَنَاوَلُ الطَّعَامَ حَتَّى يُسْرِعُ اِلَى الْمَكْتَبَةِ دُونَ اَنْ يَسْتَرِيحَ وَلَوْ قَلِيلاً. وَ لاَ يَكَادُ يَسْمَعُ صَوْتَ الْمُؤَذِّنِ مِنْ عَلَى مَنَارَةِ الْمَسْجِدِ حَتَّى يَخْرُجُ مِنَ الْبَيْتِ مُتَوَجِّهًا نَحْوَ الْمَسْجِدِ فَيُسَلِّمُ عَلَى مَنْ بِصَحْنِ الْمَسْجِدِ وَ يَنْطَلِقُ اِلَى دَاخِلِهِ لاَ يَلْوِى عَلَى شَىْءٍ.

فِى مَبْقَلَتِنَا يَنْبُتُ الْخِيَارُ وَ الْكُرُنْبُ وَ الْبَصَلُ وَ الثُّومُ وَ الْجَزَرُ وَ اللِّفْتُ وَ الْبِطِّيخُ وَ الشَّمَامُ وَ الْفِجْلُ.

وَقْتُكَ اَغْلَى مَا فِى حَيَاتِكَ فَاصْرِفْهُ فِيمَا يَنْفَعُكَ وَ يَنْفَعُ الإِسْلاَمَ. اَلإِسْلاَمُ دِينٌ خَالِدٌ يَبْقَى اِلَى يَوْمِ الْقِيَامَةِ وَ لَكِنْ لاَ قُوَّةَ لَهُ اِلاَّ بِالْعِلْمِ وَ الْجِهَادِ فَاجْعَلْهُمَا أَوَّلَ هُمُومِكَ. الْمُسْلِمُونَ اِذَا دَرَسُوا الْعُلُومَ وَ جَاهَدُوا اَصْبَحُوا أَقْوِيَاءَ وَ اِذَا تَرَكُوا الْعُلُومَ وَ الْجِهَادَ أَصْبَحُوا ضُعَفَاءَ وَ ذَهَبَتْ قُوَّتُهُمْ.

كَمْ سَنَةً عُمْرُكَ؟ - نَاهَزْتُ الأَرْبَعِينَ. كَمْ كَانَ عُمْرُكَ عِنْدَمَا الْتَحَقْتَ بِالْجَامِعَةِ؟ - كَانَ عُمْرِى اِذْذَّاكَ خَمْسًا وَ عِشْرِينَ سَنَةً عَلَى الأَكْثَرِ. وَ كَمْ سَنَةً عُمْرُ وَلَدِكَ؟ - هُوَ نَاهَزَ الْبُلُوغَ. يَتَقَاضَى فِى كُلِّ شَهْرٍ رَاتِبًا مِقْدَارُهُ مِائَتَا رُوبِلٍ عَلَى الأَقَلِّ.

\subsubsection{К уроку 63}
Налей в кувшин воды. Ты мыл лицо? — Да, я вымыл лицо и вытер его полотенцем. (Ты) всегда вытирай лицо после мытья чистым полотенцем. Полотенце висит на вешалке. Что у тебя на лице ( عَلَى وَجْهِكَ )? Лицо у тебя грязное (нечистое), вымой его хорошенько

с мылом. В кувшине есть вода? — В кувшине воды нет, а в кране есть. Где твой ребёнок? — Он спит на своей кровати. Когда ты проснулся? — Я проснулся недавно. Ты утренний азан слышал? — Да, слышал. Сколько ракатов в утреннем намазе? Сколько ракатов в полуденном намазе7 Квартальная мечеть далеко от вас? — Совсем рядом ( قَرِيبٌ جِدًّا ) Тогда ты должен все намазы совершать только в

мечети, А ты каждый азан слышишь? — Да, когда бываю дома. Отец, купи мне вешалку, у меня нет вешалки, и зубную щё'тку. Не откладывай сегодняшнюю работу на завтра. Делай омовение быстро, люди давно ушли в мечеть. Мои зубы чистые, потому что я при каждом омовении ( عِنْدَ كُلِّ وُضُوءِ) чищу их зубной щёткой.

\subsection{اَلدَّرْسُ الرَّابِعُ وَ الثَّمَانُونَ 84}
تَزَوَّجَ. تَزَوَّجَتْ. اَنْجَبَ. مَمَرٌّ (ات). لَعَلَّ. اَىْ. 

\  \includegraphics[width=0.7917in,height=1.0209in]{images/MuhammadBagauddinprettified-img247.png}   \includegraphics[width=0.6043in,height=1.1252in]{images/MuhammadBagauddinprettified-img248.png} 

رَأْسٌ, رُؤُوسٌ. اِصْطَدَمَ بِ... لاَ يَسَعُنِى اِلاَّ... رِجْلٌ, أَرْجُلٌ. سَاكِنٌ, سُكَّانٌ. سُكَّانُ الأَرْيَافِ. مِنَ الْمَعْلُومِ أَنَّ. بِوَاسِطَةِ... اَنْقَذَ. رَدَّ عَلَى...(و) رَدٌّ. هَزَّ (و) هَزٌّ. هَزَّ رَأْسَهُ. قَائِلاً. قَبِلَ (ا) قَبُولٌ. وَحِيدٌ. سَائِرُ... جَاءَ بِ... رَضِىَ (ا) رِضًا. جَعَلَ يَضْحَكُ. يَدُ الْعَوْنِ. مَدَّ يَدَ الْعَوْنِ. وَقَعَ عَلَى الْأَرْضِ. اَللَّهُ تَعَالَى. أَقَامَ. أَقَامَ الْقَاعِدَ.

\_\_\_\_\_\_\_\_\_\_\_\_\_\_\_\_\_\_\_\_\_\_\_\_\_

تَزَوَّجَتْ بِنْتِى فِى الْعِشْرِينَ مِنْ عُمْرِهَا اَىْ مُنْذُ عَشْرِ سَنَوَاتٍ وَ مُنْذُ ذَلِكَ اَنْجَبَتْ اَىْ وَلَدَتْ سِتَّةً مِنَ الأَوْلاَدِ. تَزَوَّجْ يَا فَتَى وَ لاَ تَكُنْ عَزَبًا, وَ أَنْتِ يَا فَتَاةُ تَزَوَّجِى وَ لاَ تَكُونِى عَزَبَةً. قِفْ هُنَا وَ لاَ تَقِفْ فِى الْمَمَرِّ وَ لاَ تَمُدَّ رِجْلَكَ اِلَى طَرِيقِ النَّاسِ لَعَلَّ اَحَدًا يَصْطَدِمُ بِهَا فَيَقَعُ عَلَى الأَرْضِ فَتَكُونُ اَنْتَ السَّبَبَ فِى ذَلِكَ. اِنَّكَ اَنْقَذْتَ حَيَاتِى فَلاَ يَسَعُنِى اِلاَّ اَنْ اَشْكُرَكَ عَلَى ذَلِكَ. مِنَ الْمَعْلُومِ اَنَّ الْجَرَائِدَ وَ الْمَجَلاَّتِ وَ الْخِطَابَاتِ تَصِلُ اِلَى سُكَّانِ الْمُدُنِ وَ الأَرْيَافِ بِوَاسِطَةِ الْبَرِيدِ. لِمَاذَا رَدَّ عَلَى قَوْلِكَ؟ - لَمْ يَرُدَّ عَلَى قَوْلِى بِكَلاَمٍ طَوِيلٍ بَلْ هَزَّ رَأْسَهُ قَائِلاً: طَيِّبٌ, طَيِّبٌ. اِنِّى اُحِبُّ اَنْ يَنْتَشِرَ الإِسْلاَمُ فِى الْعَالَمِ بِأَسْرِهِ وَ يَقْبَلَهُ جَمِيعُ النَّاسِ وَ يَتْرُكُوا سَائِرَ الأَدْيَانِ. اَلدِّينُ الْوَحِيدُ الَّذِى يَرْضَاهُ اللَّهُ هُوَ دِينُ الإِسْلاَمِ. اَلَّذِى جَاءَ بِهِ مُحَمَّدُ بْنُ عَبْدِ اللَّهِ مِنْ عِنْدِ اللَّهِ. قَالَ اللَّهُ تَعَالَى فِى كِتَابِهِ: "اِنَّ الدِّينَ عِنْدَ اللَّهِ الإِسْلاَمُ". اَنَا غَرِيبٌ هُنَا وَ لَيْسَ لِى مَنْ يُسَاعِدُنِى وَ يَمُدُّ لِى يَدَ الْعَوْنِ لَيْسَ لِى مَنْ يَمُدُّ لِى يَدَ الْعَوْنِ. اِذَا وَقَعَ اَخُوكَ فِى بَلِيَّةٍ فَمُدَّ لَهُ يَدَ الْعَوْنِ وَ اَنْقِذْهُ مِنْهَا. وَقَعَ طِفْلٌ صَغِيرٌ عَلَى الأَرْضِ أَمَامِى فَجَعَلَ يَبْكِى فَاَقَمْتُهُ وَ قُلْتُ لَهُ: لاَ تَبْكِ, اَلَسْتَ فَتًى؟ \ \ \ \ 

\subsubsection{К уроку 64}
Пусть каждый из вас убирает свою комнату. Пусть Хусейн быстро делает свои уроки. Папа (отец), дай мне рубль. Почему ты сердит? Почему учитель сегодня не в настроении? — Потому что многие его ученики отсутствуют на уроке. Сегодня пойдём в лес? Книги возьмём с собой? — Не возьмём, в них нет необходимости. Купим новый дом? — Нет необходимости в новом доме. Этот дом хоть и (وَاِنْ كَانَ) старый, нам хватает. Я забыл книгу на столе. Где ты забыл свой нож? Ученица забыла свой портфель в парте и ушла домой, а когда пришла домой, мать спросила её: „Где твой портфель?" Она сказала: "Забыла в классе", и сразу вернулась. В радиопередаче объявили утром, что профессор уже умер. Когда мы услышали это известие, мы сильно огорчились. Я потерял свой нож, но вместо него нашел другой нож. Сколько денег (كَمْ مِّنَ النُّقُودِ ) в твоём кошельке?

— Немного, но нам хватит. Я еду в город, купить тебе что-нибудь? ( شَيْئًا مَّا )• Спасибо, мне ничего не нужно. Ты честный мальчик, ты заслуживаешь вознаграждение от Бога, да сохранит тебя Бог, Бог любит честных.

\subsection[اَلدَّرْسُ الْخَامِسُ وَ الثَّمَانُونَ 85]{اَلدَّرْسُ الْخَامِسُ وَ الثَّمَانُونَ 85}
لَمْ تَعُدْ طِفْلاً. لاَ يَلِيقُ بِ... سَفَاسِفُ الأُمُورِ. اِشْتَغَلَ بِسَفَاسِفِ الأُمُورِ. طَالِبُ عِلْمٍ. يَنْبَغِى اَنْ لاَّ... خَيْرٌ. قِيَامُ اللَّيْلِ. صَلاَةُ اللَّيْلِ. صَلاَةُ التَّهَجُّدِ. خَافَ (ا) خَوْفٌ. سِوَى... وَحْدَهُ. وَحْدَكَ. وَحْدِى. كَفَاهُ اللَّهُ. مُحَاضَرَةٌ (ات). اَلْقَى مُحَاضَرَةً. حَاجَةٌ. هَوَ بِحَاجَةٍ اِلَى... تَعِبٌ. عَادِلٌ, عُدُولٌ. ظَالِمٌ, ظَلَمَةٌ. عَدْلٌ. ظُلْمٌ. جَهْلٌ. اَصَابَ. اُصِيبَ بِمَرَضٍ. ضُعْفٌ. غَلَبَ (ى) غَلَبَةٌ. زَادَ عَنْ... كَمَا هُوَ الْحَالُ. اَلاَ تُرِيدُ؟

\_\_\_\_\_\_\_\_\_\_\_\_\_\_\_\_\_\_\_\_\_\_\_\_\_\_

اِنَّكَ الآنَ شَابٌّ وَ لَمْ تَعُدْ طِفْلاً فَلاَ يَلِيقُ بِكَ أَنْ تَشْتَغِلَ بِسَفَاسِفِ الأُمُورِ, وَ اَنْتَ طَالِبُ عِلْمٍ وَ تَنَامُ كَثِيرًا يَنْبَغِى اَنْ لاَّ يَزِيدَ نَوْمُكَ عَنْ سِتِّ سَاعَاتٍ. وَ لاَ تَنْسَ قِيَامَ اللَّيْلِ لِصَلاَةِ اللَّيْلِ فَاِنَّ الْخَيْرَ كُلَّهُ فِى صَلاَةِ اللَّيْلِ. مَا هِىَ صَلاَةُ التَّهَجُّدِ؟ - صَلاَةُ التَّهَجُّدِ هِىَ الصَّلاَةُ بَعْدَ النَّوْمِ وَ هِىَ صَلاَةُ اللَّيْلِ. يَا اَيُّهَا الْوَلَدُ الْمُسْلِمُ خَفِ اللَّهَ وَ لاَ تَخَفْ سِوَى اللَّهِ فَمَنْ خَافَ اللَّهَ وَحْدَهُ وَ لَمْ يَخَفْ سِوَاهُ كَفَاهُ اللَّهُ .سَيُلْقِى أُسْتَاذُنَا غَدًا مُحَاضَرَةً عَنْ حَاجَةِ النَّاسِ اِلَى الإِسْلاَمِ, اَلاَ تُرِيدُ اَنْ تَحْضُرَ الْمُحَاضَرَةَ؟ بَلَى, بِكُلِّ ارْتِيَاحٍ, اِنَّنِى بِحَاجَةٍ اِلَى اَنْ اَعْرِفَ مَزِيدًا عَنِ الإِسْلاَمِ. اَلإِسْلاَمُ دِينُ الْعَدْلِ وَ الْعِلْمِ فَلاَ يَرْضَى الظُّلْمَ وَ لاَ يَرْضَى الْجَهْلَ. الإِسْلاَمُ لاَيَقُومُ اِلاَّ بِالْجِهَادِ فَكُنْ مُجَاهِدًا فِى كُلِّ حِينٍ. هُوَ تَعِبٌ جِدًّا وَ بِحَاجَةٍ اِلَى النَّوْمِ ِلاَنَّهُ سَهِرَ الْبَارِحَةَ عَاكِفًا عَلَى دُرُوسِهِ. اِبْنِى تَعَرَّضَ لِلْبَرْدِ فَاُصِيبَ بِمَرَضٍ فَهُوَ الآنَ طَرِيحُ الْفِرَاشِ. اِذَا تَرَكَ الْمُسْلِمُونَ الْجِهَادَ وَ خَافَوا الْكُفَّارَ وَ استَسْلَمُوا لَهُمْ أَصَابَهُمْ الضُّعْفُ وَ غَلَبَهُمُ الْكُفَّارُ وَ ذَهَبَتْ دَوْلَةُ الْمُسْلِمِينَ وَ قَامَتْ دَوْلَةُ الْكُفَّارِ كَمَا هُوَ الْحَالُ الآنَ.

\subsubsection{К уроку 65}
Что тебе нужно? — Ничего. Почему ты тогда плачешь? Кошка схватила мясо и убежала под кровать. Лиса схватила нашего единственного ( الْوَحِيدَ) петуха и убежала в лес. Как ваши дела? — Спасибо, нормально. Как учёба ( الدِّرَاسَة)? — Всё в порядке. Мой брат окончил военное училище в прошлом году, а я оканчиваю в этом году, а сестра окончит институт в будущем году. Мой отец — военный офицер в звании лейтенанта, и он служит Исламу. Я учился в военном училище, чтобы служить Исламу. Кто завтра дежурный? — Ты опять дежурный Если Бог поможет, поступлю в военное училище и послужу Исламу. Паша дивизия – танковая) فِرْقَةُ الْمُدَرَّعاتِ), и я — танкист. В течение двух месяцев мы проводили военные учения в горах. Я воин на ifvth Бога, будь и ты воином на пути Бога, и пусть каждый будет воином на пути Бога. Мы все мусульмане и мы все должны быть воинами на пути Бога. Путь Бога — это наш путь.

\subsection[اَلدَّرْسُ السَّادِسُ وَ الثَّمَانُونَ 86]{اَلدَّرْسُ السَّادِسُ وَ الثَّمَانُونَ 86}
سَمَاوِىٌّ. دِينٌ سَمَاوِىٌّ. سَعَادَةٌ. نَالَ (ا) نَيْلٌ. اَلدُّنْيَا. اَلآخِرَةُ. فِى الدُّنْيَا وَ الآخِرَةِ. رَفَضَ (و) رَفْضٌ. شَقِىَ (ا) شَقَاءٌ. آخِرٌ. اِحْتَاجَ اِلَى... مُحْتَاجٌ اِلَى... اَلصِّينُ. عِبَادَةٌ (ات). مَعِيشَةٌ, مَعَايِشُ. اَيْنَمَا... حَيْثُمَا... وُجِدَ. اَقْرَبُ. أَحَبُّ. حَبِيبٌ, أَحِبَّاءُ. طَلَبَ الْعِلْمَ. مَالٌ, أَمْوَالٌ. صَادِقٌ. كَامِلٌ. أَنْفُسُهُمْ. فِى اَىِّ بَلَدٍ آخَرَ. قَدَرَ عَلَى... (ى) قُدْرَةٌ. خَيْرٌ. شَرٌّ. شَقِىٌّ, أَشْقِيَاءُ.

\_\_\_\_\_\_\_\_\_\_\_\_\_\_\_\_\_\_\_\_\_\_\_\_\_\_

اَلإِسْلاَمُ دِينٌ سَمَاوِىٌّ جَاءَ مِنَ اللَّهِ تَعَالَى لِسَعَادَةِ النَّاسِ جَمِيعًا فَمَنْ قَبِلَهُ نَالَ السَّعَادَةَ فِى الدُّنْيَا وَ الآخِرَةِ وَ مَنْ رَفَضَهُ شَقِىَ فِى الدُّنْيَا وَ الآخِرَةِ. اَلنَّاسُ مَعَ الإِسْلاَمِ سُعَدَاءُ وَ هُمْ بِدُونِ الإِسْلاَمِ اَشْقِيَاءُ. اَلنَّاسُ مَعَ الإِسْلاَمِ لاَ يَحْتَاجُونَ اِلَى شَىْءٍ آخَرَ فَفِيهِ اُمُورُ عِبَادَاتِهِمْ وَ فِيهِ اُمُورُ مَعَايِشِهِمْ وَ فِيهِ نِظَامُ حَيَاتِهِمْ. اَلإِتِّحَادُ السُّوفْيَيْتِىُّ مُحْتَاجٌ اِلَى الإِسْلاَمِ وَ الصِّينُ مُحْتَاجَةٌ اِلَى الإِسْلاَمِ وَ الْيَابَانُ وَ أُسْتُرَالِيَا وَ الْبَرَازِيلُ وَ كَنَدَا وَ غَيْرُهَا مِنَ الدُّوَلِ كُلُّهَا مُحْتَاجٌ اِلَى الإِسْلاَمِ, حَتَّى الْمُسْلِمُونَ اَنْفُسُهُمْ مُحْتَاجُونَ اِلَى الإِسْلاَمِ الصَّحِيحِ الْكَامِلِ.

اَلْمُسْلِمُ فِى الإِتِّحَادِ السُّوفْيَيْتِىِّ اَخُو الْمُسْلِمِ فِى أَمْرِيكَا اَوْ فِى اَىِّ بَلَدٍ آخَرَ مِنْ بُلْدَانِ الْعَالَمِ. اَلْمُسْلِمُ اَيْنَمَا وُجِدَ وَ حَيْثُمَا ذَهَبَ اَقْرَبُ اِلَى الْمُسْلِمِ وَ اَحَبُّ اِلَيْهِ مِنْ أَخِيهِ وَ أُمِّهِ وَ أَبِيهِ مِنْ غَيْرِ الْمُسْلِمِينَ فَعِشْ اَيُّهَا الْوَلَدُ الْحَبِيبُ مَعَ الإِسْلاَمِ وَ لِلإِسْلاَمِ. اُطْلُبِ الْعِلْمَ لِلإِسْلاَمِ وَ جَاهِدْ فِى سَبِيلِ الإِسْلاَمِ تَكُنْ مُسْلِمًا صَادِقًا. عَلَى الْمُسْلِمِ أَنْ يَعْرِفَ أَخْبَارَ الْعَالَمِ الإِسْلاَمِىِّ, وَ عَلَيْهِ اَنْ يَفْرَحَ اِذَا سَمِعَ عَنْهُمْ خَيْرًا وَ اَنْ يَحْزَنَ اِذَا سَمِعَ عَنْهُمْ شَرًّا وَ يَقُومَ لِمُسَاعَدَتِهِمْ بِنَفْسِهِ وَ مَالِهِ اِنْ قَدَرَ عَلَى ذَلِكَ. 

\subsubsection{К уроку 66}
Ты доволен своей учёбой? — Да, я доволен. Всё хорошо. Продолжай урок и не прерывай его, я сейчас ухожу. У тебя много ошибок. Попробуй писать без ошибок. Ты сосчитал учеников, которые присутствовали на уроке? — Да, сосчитал. Сколько их было? Их было 35. Сколько раз я тебе говорил об этом? Сколько

раз ты получал письма от своего друга? Сколько раз мусульмане молятся в сутки? Не повышай голос, читай тихо, я слышу. Читай текст и переведи его мне. Закрой двери и окна, на улице холодно. Переведи это арабское выражение. Что это значит? Что означает это слово? Он мне сказал: "Добавить тебе?" — и я ему ответил: "Не надо". Мальчик, делай гимнастику каждый день регулярно. Когда просыпаешься, буди и своих товарищей. Всегда просыпайся на заре.

Мой мальчик регулярно ходит в мечеть, особенно на утренний намаз а когда возвращается оттуда, занимается утренней гимнастикой, поэтому он всегда активный, он просыпается вместе с петухами и не спит после этого. Пришёл бы ты к нам? Ты мог бы посидеть немного со мной? Рассказал бы ты мне немного о своей жизни? У кого гости остановились? — Гости остановились у Махмуда.

\subsection{اَلدَّرْسُ السَّابِعُ وَ الثَّمَانُونَ 87}
ذُو... ذَوُو, أُولُو. ذَاتُ... ذَوَاتُ, أُولاَتُ. ذُو مَالٍ. عِبَارَةٌ عَنْ... مَبْنًى, مَبَانٍ. طَابِقٌ, طَوَابِقُ. سِتَارَةٌ, سَتَائِرُ. حَرِيرٌ. سَخِرَ (ا) سُخْرٌ. أَيُّهُمْ؟ عِيَالٌ. مُحْتَرَمٌ (ون). لاَ...اِلاَّ... اَنْفَقَ. اِزْدِهَارٌ. اِزْدِهَارُ الْعِلْمِ. سِلاَحٌ, أَسْلِحَةٌ. حَارَبَ. عَبَدَ (و) عِبَادَةٌ. نِظَامٌ. بِنِظَامٍ. خُطْبَةٌ, خُطَبٌ. مَوْعِظَةٌ, مَوَاعِظُ. تَلَقَّى. بَنَى (ى) بِنَاءٌ. شَتَّى. بَعْدَ اَنْ... اِسْتَقَرَّ. سَعَى اِلَى... (ا) سَعْىٌ. صَمْتٌ. بِصَمْتٍ. بَطْنٌ, بُطُونٌ. أَرْضَعَ. هَيَّأَ. قَبَّلَ. وَجُودٌ. عَظِيمٌ – مَا أَعْظَمَهُ. جَمِيلٌ – مَا أَجْمَلَهُ. أَمَرَ (و) أَمْرٌ. كَمَا أَمَرَ اللَّهُ. حَفِظَكَ اللَّهُ. حَىٌّ, اَحْيَاءٌ. صَبَاحَ مَسَاءَ. خَيْرٌ.

\_\_\_\_\_\_\_\_\_\_\_\_\_\_\_\_\_\_\_\_\_\_

اَلْمَكْتَبَةُ عِبَارَةٌ عَنْ مَبْنًى ذِى طَابِقَيْنِ, فِى كُلِّ طَابِقٍ غُرَفٌ ذَاتُ نَوَافِذَ وَاسِعَةٍ وَ عَلَى النَّوَافِذِ سَتَائِرُ مِنَ الْحَرِيرِ. اَنْتَ رَجُلٌ ذُو مَالٍ وَ هَذَا الَّذِى تَسْخَرُ مِنْهُ رَجُلٌ ذُو عِلْمٍ فَاَيُّكُمَا خَيْرٌ. اَنَا ذُو عِيَالٍ كَثِيرٍ. ذَوُو الْعِلْمِ مُحْتَرَمُونَ وَ ذَوُو الأَمْوَالِ اَيْضًا مُحْتَرَمُونَ وَ لَكِنْ نَحْنُ – الشَّبَابَ الْمُسْلِمِينَ لاَ نَحْتَرِمُهُمْ اِلاَّ اِذَا كَانُوا يُنْفِقُونَهَا فِى سَبِيلِ اِزْدِهَارِ الْعِلْمِ وَ الإِسْلاَمِ. نَحْنُ اُولُو قُوَّةٍ وَ اُولُو سِلاَحٍ كَثِيرٍ فَانْظُرْ هَلْ نُحَارِبُ. مَا هُوَ الْمَسْجِدُ؟ - اَلْمَسْجِدُ هُوَ مَكَانُ اجْتِمَاعِ الْمُسْلِمِينَ لِلصَّلاَةِ. فِيهِ يَعْبُدُونَ اللَّهَ وَ فِيهِ يُصَلُّونَ بِنِظَامٍ وَ فِيهِ يَسْمَعُونَ الْخُطَبَ وَ الْمَوَاعِظَ وَ فِيهِ يَتَلَقَّوْنَ شَتَّى الدُّرُوسِ, وَ هُوَ اَوَّلُ مَا بَنَى النَّبِىُّ – صَلَّى اللَّهُ عَلَيْهِ وَ سَلَّمَ – بَعْدَ اَنْ اِسْتَقَرَّ فِى الْمَدِينَةِ. اَلْمُسْلِمُ يُحِبُّ الْمَسْجِدَ وَ يَسْعَى اِلَيْهِ وَ يَجْلِسُ فِيهِ بِصَمْتٍ وَ احْتِرَامٍ.

حَمَلَتْنِى اُمِّى تِسْعَةَ أَشْهُرٍ فِى بَطْنِهَا وَ اَرْضَعَتْنِى حَلِيبَهَا وَ سَهِرَتِ اللَّيَالِىَ بِجَانِبِى. هِىَ تُحِبُّنِى تُهَيِّئُ لِى طَعَامِى. أُمِّى سَبَبُ وُجُودِى مَا اَعْظَمَ اُمِّى! وَ مَا اَجْمَلَهَا! اُقَبِّلُ يَدَهَا صَبَاحَ مَسَاءَ. اُحِبُّ اُمِّى وَ اُطِيعُهَا كَمَا اَمَرَ اللَّهُ. حَفِظَ اللَّهُ اُمِّى وَ حَفِظَ مَعَهَا اَبِى وَ عَاشَا فِى الصِّحَّةِ وَ السَّلاَمَةِ مَادَامَا حَيَّيْنِ. 

\subsubsection{К уроку 67}
У нас есть петух. Он каждый день до зари кричит и будит нас. Мы благодарим петуха. Говорят, что лев, волк и лиса вышли на охоту ( لِلصَّيْدِ ). Когда я зашёл, он прервал свой разговор. Я посетил зоопарк и, когда вернулся, я описал своему младшему брату диких зверей (الْوُحُوش), которых я видел там. Тогда он тоже захотел посетить его. Завтра экзамены, надо готовиться к ним. Ты сдал экзамен? — Да, сдал. Кто-нибудь провалился? — Да, трое провалились. У меня было большое желание сесть с тобой, попить чаю и в это время ( وَ فِى اثناء ذلكَ) рассказать тебе о нашей стране, но я, как видишь, очень занят, и у тебя тоже нет свободного времени. Ты слышал его разговор? — Да, слышал. О чём он рассказывал? — О своей прошлой жизни. У меня тебе совет. Хочешь, я дам совет. Я его искал везде: и дома, и в мечети, и на работе (на рабочем месте), но я его не нашёл. Твоя работа (твоё рабочее место) далеко отсюда? — Да, далеко, я еду туда автобусом и возвращаюсь тоже автобусом. Будь учёным и не будь невеждой, учёного все любят, а невежду никто не любит.

\subsection[اَلدَّرْسُ الثَّامِنُ وَ الثَّمَانُونَ 88]{اَلدَّرْسُ الثَّامِنُ وَ الثَّمَانُونَ 88}
 \includegraphics[width=1.6146in,height=0.9583in]{images/MuhammadBagauddinprettified-img249.png}   \includegraphics[width=1.7811in,height=0.8752in]{images/MuhammadBagauddinprettified-img250.png} 

\ دَرَّاجَةٌ (ات). دَرَّاجَةٌ نَارِيَّةٌ. حَلَّ (و) حُلُولٌ. قَطَفَ (ى) قَطْفٌ. ثَمَرٌ, ثِمَارٌ. حِرَاثَةٌ. حَقًّا مَّا تَقُولُ. آلِىٌّ. مِحْرَاثٌ آلِىٌّ. لاَ تُزْعِجْ نَفْسَكَ. جَاهِزٌ. كَفَى كُفْرَانًا. كُفْرَانٌ بِالنِّعَمِ. نِعْمَةٌ, نِعَمٌ. نِعَمُ اللَّهِ. شَرَابٌ, أَشْرِبَةٌ. لاَ يُحْصَى. حَمِدَ (ا) حَمْدٌ. نَبَأٌ, أَنْبَاءٌ. سَلْ. لاَ تَسَلْ. اَنَا فِى انْتِظَارِهِ. قَدِمَ (ا) قُدُومٌ. اَحْرَزَ. اَحْرَزَ النَّجَاحَ. بِفَارِغِ الصَّبْرِ. إِلَى أَنْ... قَرِيبًا. جَادٌّ. هَازِلٌ. أَتَرْضَى؟ مَنَحَ (ا) مَنْحٌ. مَبْلَغٌ, مَبَالِغُ. عَلَى الرَّأْسِ وَ الْعَيْنِ. يَا بُنَىَّ. مَلَكَ (ى) مِلْكٌ. يَمْلِكُ.

\_\_\_\_\_\_\_\_\_\_\_\_\_\_\_\_\_\_\_\_\_\_\_\_

حَلَّ الْخَرِيفُ وَ قُطِفَتِ الثِّمَارُ وَ حُصِدَتِ الزُّرُوعُ وَ عَلَيْنَا مِنَ الآنَ اَنْ نَسْتَعِدَّ لِلْحِرَاثَةِ – حَقًّا مَّا تَقُولُ, لاَبُدَّ مِنَ الْحُصُولِ عَلَى مِحْرَاثٍ آلِىٍّ أَوِ اثْنَيْنِ عَلَى الأَقَلِّ. – لاَ تُزْعِجْ نَفْسَكَ, اَلآلاَتُ وَ الْمَكِنَاتُ عِنْدَنَا جَاهِزَةٌ.

كَفَى كُفْرَانًا بِنِعَمِ اللَّهِ! مِنْ اَيْنَ طَعَامُكَ؟ وَ مِنْ اَيْنَ شَرَابُكَ؟ وَ مِنْ اَيْنَ لِبَاسُكَ؟ وَ مِنْ اَيْنَ صِحَّتُكَ؟ اَللَّهُ الَّذِى مَنَحَكَ كُلَّ هَذَا وَ غَيْرَهَا مِنْ نِعَمٍ لاَ تُحْصَى, فَعَلَيْكَ اَنْ تَحْمَدَهُ وَ تَشْكُرَهُ تَنْصُرَ دِينَهُ وَ تُجَاهِدَ فِى سَبِيلِهِ.

بَلَغَنَا نَبَأُ نَجَاحِكَ فِى تَأْدِيَةِ الاِمْتِحَانَاتِ وَ لاَ تَسَلْ عَنْ فَرَحِنَا لَمَّا عَرَفْنَا الْخَبَرَ وَ اِنَّنَا جَمِيعًا لَفِى انْتِظَارِ قُدُومِكَ بِفَارِغِ الصَّبْرِ, وَ اِنَّا لَنُهَنِّئُكَ بِالنَّجَاحِ الَّذِى اَحْرَزْتَهُ وَ اِلَى اَنْ نَرَاكَ بَيْنَنَا قَرِيبًا. اَجَادٌّ اَنْتَ اَمْ هَازِلٌ؟ - بَلْ جَادٌّ. أَتَرْضَى أَنْ أُوَاصِلَ دِرَاسَتِى فِى الْمَعْهَدِ؟ - وَ لِمَ لاَ؟ - اِذًا فَعَلَيْكَ اَنْ تَمْنَحَنِى مَبْلَغًا مِنَ النُّقُودِ يَا أَبَتِ؟ - عَلَ الرَّأْسِ وَ الْعَيْنِ يَا بُنَىَّ. اِشْتَرَى اَبُونَا دَرَّاجَتَيْنِ: دَرَّاجَةً لِى وَ أُخْرَى ِلأَخِى, وَ اَبُونَا نَفْسُهُ يَمْلِكُ دَرَّاجَةً نَارِيَّةً.

\subsubsection{К уроку 68}
Что ты знаешь о нашем Пророке Мухаммаде — да благословит Его Бог и приветствует? Что ты знаешь о его детях? О его жёнах? Он вырос сиротой. Он не видел своею отца. Мать он потерял в детстве, когда ему было 6 лет. Родился он в Мекке, умер Он и похоронили Его в Медине. Когда Бог послал его к людям Посланником? — Бог послал Его к людям Посланником, когда Он достиг 40 лет. Чья речь — Коран? — Коран — это речь Бога. Что такое ха-дис? — Хадис — это речь Пророка. Я выучил наизусть тысячу хадисов. Этот учёный знает Коран наизусть. Я знаю половину Корана наизусть — с начала до середины. В сегодняшнем уроке 3 правила. Выучите их наизусть Мы неарабы, вот поэтому нам необходима грамматика. Произнеси это правило правильно и обращай внимание на интонацию. Прошло два месяца с тех пор. как он ушёл от нас. Прошёл год с тех пор, как я окончил институт. Учитель ещё в шко­ле? — Нет, он ушёл домой. Что ты делаешь на балконе? — Учу свои уроки. Мы должны изучить арабский язык, потому что арабский язык — это язык Корана и нашего Пророка, и мы мусульмане, а мусульманам арабский язык необходим.

\subsection[اَلدَّرْسُ التَّاسِعُ وَ الثَّمَانُونَ 89]{اَلدَّرْسُ التَّاسِعُ وَ الثَّمَانُونَ 89}
 \includegraphics[width=1.448in,height=1.5in]{images/MuhammadBagauddinprettified-img251.png}   \includegraphics[width=1.2602in,height=0.8957in]{images/MuhammadBagauddinprettified-img252.png} 

\ شَطْرَنْجٌ. دَامَا. بَطَلٌ, أَبْطَالٌ. بِسُهُولَةٍ. 

شَبْعَانُ, شِبَاعٌ. جَوْعَانُ, جِيَاعٌ. طَعِمَ (ا) طَعْمٌ. أَرْضَى. بَيَّنَ. خَلَقَ (و) خَلْقٌ. عَدَمٌ. أَحْيَا. أَمَاتَ. رَزَقَ (و) رِزْقٌ. اَدْخَلَ. جَنَّةٌ, جِنَانٌ. نَارٌ, نِيرَانٌ. عَصَى (ى) عِصْيَانٌ. أَشَدُّ. حُمْرَةٌ. أَشَدُّ حُمْرَةً. أَقَلُّ اجْتِهَادًا. بِفَضْلِ اللَّهِ. بِرَحْمَةِ اللَّهِ. عَدَدٌ, أَعْدَادٌ. يُصْبِحُ أَقْوَى فَأَقْوَى. يُصْبِحُ اَكْثَرَ فَاَكْثَرَ. قَبْلَ كُلِّ شَىْءٍ.

\_\_\_\_\_\_\_\_\_\_\_\_\_\_\_\_\_\_\_\_\_\_\_

هُوَ يَتَكَلَّمُ بِاللُّغَةِ الْعَرَبِيَّةِ بِسُرْعَةٍ وَ بِسُهُولَةٍ. اَتَلْعَبُ بِالشَّطْرَنْجِ؟ - نَعَمْ, اَلْعَبُ بِالشَّطْرَنْجِ وَ بِالدَّامَا وَ اَنَا بَطَلُ الْمَدْرَسَةِ فِى الشَّطْرَنْجِ. أَتَاْكُلُ مَعِى؟ - لاَ, اَنَا شَبْعَانُ – لَكِنِّى آكُلُ ِلاَنِّى جَوْعَانُ لَمْ اَطْعَمْ مُنْذُ الْبَارِحَةِ شَيْئًا. لاَ تَشْرَبُ وَ لَوْ شَايًا؟ - اَمَّا الشَّاىُ فَأَشْرَبُ.

عَلَى الْوَلَدِ الْمُسْلِمِ قَبْلَ كُلِّ شَىْءٍ اَنْ يُفَكِّرَ كَيْفَ يَعْمَلُ عَمَلاً يُرْضِى اللَّهَ وَ يَنْصُرُ دِينَهُ وَ اَنْ يَدْعُوَ اِلَى الإِسْلاَمِ زُمَلاَءَهُ الَّذِينَ يَدْرُسُونَ مَعَهُ وَ يُعَلِّمَهُمُ الصَّلاَةَ, وَ يُبَيِّنَ لَهُمْ اَنَّ اللَّهَ هُوَ الَّذِى خَلَقَ النَّاسَ وَ خَلَقَ الْعَالَمَ مِنَ الْعَدَمِ وَ اَنَّهُ هُوَ الَّذِى يُحْيِينَا وَ يُمِيتُنَا وَ يَرْزُقُنَا. وَ اِذَا مِتْنَا فَهُوَ يُحْيِينَا ثَانِيَةً فَمَنْ اَطَاعَهُ وَ نَصَرَ دِينَهُ اَدْخَلَهُ الْجَنَّةَ وَ مَنْ عَصَاهُ وَ لَمْ يَنْصُرْ دِينَهُ اَدْخَلَهُ النَّارَ. هَذَا الْحِبْرُ اَشَدُّ حُمْرَةً مِنْ ذَاكَ. أَخِى اَقَلُّ اجْتِهَادًا مِنِّى. اَلطَّقْسُ الْيَوْمَ أَجْمَلُ مِمَّا كَانَ أَمْسِ. يُصْبِحُ الإِسْلاَمُ بِفَضْلِ اللَّهِ وَ بِرَحْمَتِهِ كُلَّ يَوْمٍ أَقْوَى فَأَقْوَى وَ يُصْبِحُ عَدَدُ الْمُسْلِمِينَ لاَ سِيَّمَا الشَّبَابِ اَكْثَرَ فَاَكْثَرَ.

\subsubsection{К уроку 69}
Какой он прекрасный человек! Как ( كَمْ ) меня обрадовала встреча с ним! Когда мы вышли на экскурсию в лес, день был тёплый, солнечный. Моя комната светлая, солнечная, окна её большие. После звонка ученики спешат в классы, и вскоре заходит учитель и приветствует их, и они встают со своих скамеек в знак уважения к нему и отвечают на его приветствие. Сколько времени? Пора совершить намаз? Я сейчас возвращаюсь домой, спокойной ночи! — Спокойной ночи! Когда перерыв? Через сколько минут? — Перерыв будет скоро, примерно через полчаса. Как там мои родители в селе и как мои братья? — Все они тоскуют по тебе. Сообщи им, что у меня на будущей неделе экзамен; если Богу будет угодно, сдам его и сразу вернусь к ним. Когда ты посетишь нас? — На будущей неделе. Когда посетишь нас, не забудь взять с собой жену и детей. Кому ты купил эти часы? — Я их купил себе. Мухаммед сам пошёл туда или он другого человека послал? — Он сам пошёл, а не послал другого человека. Этот урок ты сам понял или учитель объяснил тебе? — Слава Богу, я сам понял, и никто мне не объяснял.

\subsection{اَلدَّرْسُ التِّسْعُونَ 90}
أَصْبَاحًا أَمْ مَسَاءً؟ أَدَبٌ, آدَابٌ. مِنْ أَقْدَمِ الْعُصُورِ إِلَى يَوْمِنَا هَذَا. أَتَى (ى) اِتْيَانٌ. وَاللَّهِ. بِالدِّقَّةِ. اِقْتَرَبَ. اِقْتَرَبَ أَجَلُهُ. جُمْهُورٌ, جَمَاهِيرُ. مَقْهًى, مَقَاهٍ. يُمْكِنُكَ. يُمْكِنُنِى. كَالْعَادَةِ. بَاقٍ. صَفْحَةٌ (ات). اِحْتَشَدَ. أَظُنُّ. مِينَاءٌ, 

\  \includegraphics[width=1.8854in,height=1.0311in]{images/MuhammadBagauddinprettified-img253.png} 

مَوَانِئُ. رَصِيفٌ, أَرْصِفَةٌ, رُصُفٌ. بَاخِرَةٌ, بَوَاخِرُ. شِفَاهِيًّا. كِتَابِيًّا. أَنَا عَلَى يَقِينٍ. خَذَلَ (و) خِذْلاَنٌ. أَمَلٌ, آمَالٌ. وَثِقَ بِ… (ى) ثِقَةٌ. صَرَخَ (و) صُرَاخٌ. بِأَعْلَى صَوْتِهِ. أَوَّاهْ! هَيَّا بِنَا. مَطْعَمٌ, مَطَاعِمُ.

\_\_\_\_\_\_\_\_\_\_\_\_\_\_\_\_\_\_\_\_\_\_\_\_\_

سَتُلْقَى فِى الأُسْبُوعِ الْقَادِمِ فِى قَاعَةِ جَامِعَتِنَا مُحَاضَرَتَانِ عَنْ تَارِيخِ الْعَرَبِ وَ أَدَبِهِمْ مِنْ أَقْدَمِ الْعُصُورِ اِلَى يَوْمِنَا هَذَا. فِى أَيَّةٍ سَاعَةٍ يَأْتِنَا سُلَيْمَانُ غَدًا؟ - وَاللَّهِ لاَ أَعْرِفُ بِالدِّقَّةِ. كَيْفَ حَالُ يُوسُفَ الْيَوْمَ؟ - رَدِيئَةٌ جِدًّا لاَ يَنَامُ وَ لاَ يَاْكُلُ. لَمَّا اقْتَرَبَتْ بَاخِرَتُنَا مِنَ الْمِينَاءِ رَأَيْنَا جُمْهُورًا كَثِيرًا قَدِ احْتَشَدَ عَلَى الرَّصِيفِ فِى انْتِظَارِ وُصُولِنَا. عِنْدَمَا نَرْجِعُ اِلَى الْبَيْتِ فِى سَاعَةٍ مُتَأَخِّرَةٍ مِنَ اللَّيْلِ يَكُونُ ابْنُنَا الصَّغِيرُ قَدْ نَامَ. سَأَلْتُ مُحَمَّدًا عَنْ أَبِيهِ فَقَالَ لِى: يُمْكِنُكَ اَنْ تَجِدَهُ فِى الْمَقْهَى ِلاَنَّهُ فِى هَذِهِ السَّاعَةِ كَالْعَادَةِ يَجْلِسُ فِيهِ.

اَلصَّفَحَاتُ الْبَاقِيَّةُ سَأَقْرَأُهَا غَدًا. هَذِهِ الأَسْئِلَةُ اَجِيبُوا عَلَيْهَا شِفَاهِيًّا, وَ هَذِهِ تُجِيبُونَ عَلَيْهَا كِتَابِيًّا. اَيْنَ أَخُوكَ سَعِيدٌ الآنَ؟ - أَظُنُّ فِى قَاعَةِ الْمُطَالَعَةِ. اَنَا عَلَى يَقِينٍ بِاَنَّكَ صَدِيقِى وَ اَنَّكَ تَنْصُرُنِى وَ لاَ تَخْذُلُنِى وَ عِنْدِى الأَمَلُ اَنَّكَ تَثِقُ بِى. صَرَخْتُ بِاَعْلَى صَوْتِى قَائِلاً: أَوَّاهْ, قَدِ اقْتَرَبَ اَجَلِى. وَاللَّهِ اَنَا جَوْعَانُ. هَيَّا بِنَا اِلَى الْمَطْعَمِ.

\subsubsection{К уроку 70}
Ассаляму алайкум! — Ваалайкуму-ссалям! Как ваши дела? Как ваше здоровье? — Спасибо, ничего. Пойдешь со мной в клуб? — Что там? — Там идёт фильм ,,Рисалят" о жизни Посланника Бога — С удовольствием, я давно хотел посмотреть его, мне о нём рассказали мои друзья. Эта одежда сделана из хлопка. Летом надевай хлопчато-­бумажную одежду ( ثوب قطن )> а зимой — шерстяную одежду ( ثوب صوف ). В Узбекистане я видел на полях хлопкоуборочные машины. Их там много, и хлопка там много. Узбекистан — страна хлопка. Это железная дорога. По железной дороге ходят поезда. Поезда перевозят людей с одной станции на другую. Ты должен сесть на. поезд на этой станции и сойти на следующей станции. Торопись, чтобы не опоздать на джамаат. Торопись, чтобы успеть на поезд ( لكيلا يفوتك القطار). Он мне подробно рассказал о своих делах ( عَنْ حَالِهِ)• Где находится в настоящее время твой отец? — Он в санатории. Хватит говорить! Тебе не следует так много говорить! Воспитанный ученик много не разговаривает, он много думает и много делает, и мало говорит. Сейчас который час? — Сейчас ровно 8. — Значит, ты должен идти на уроки, иди и не опаздывай.

\subsection[اَلدَّرْسُ الْحَادِى وَ التِّسْعُونَ 91]{اَلدَّرْسُ الْحَادِى وَ التِّسْعُونَ 91}
مَا اَحْسَنَ بِكَ! مَا اَقْبَحَ بِكَ! صَدَقَ (و) صِدْقٌ. بِالتَّاْكِيدِ. دُونَ شَكٍّ. أَجَادَ. فِى السَّنَوَاتِ مَا بَعْدَ الثَّوْرَةِ. اَلشَّرْقُ. اَلْغَرْبُ. سَوَاءٌ اَقُلْتَ اَمْ لَمْ تَقُلْ. سَوَاءٌ فِى الْغَرْبِ أَمْ فِى الشَّرْقِ. فِى طَرِيقِ النَّهْضَةِ. صَحَا (و) صَحْوٌ. مُؤَامَرَةٌ (ات). ضِدَّ... بَاءَ بِالْفَشَلِ (و) بَوْءٌ. لاَبُدَّ وَ أَنْ يَعُودَ. وَعَدَ (ى) وَعْدٌ. ذَا. مُتَّهَمٌ (ون). حَكَمَ عَلَى... (و) حُكْمٌ. سَجَنَ (و) سَجْنٌ. لِمُدَّةِ أُسْبُوعٍ. كَاَنَّمَا... هَذَا لاَ يَعْنِينِى. عَاقَبَ. بِلاَدُ الإِنْجِلِيزِ. بِلاَدُ الأَمْرِيكَانِ. جَرِيمَةٌ, جَرَائِمُ. مَدَى الْحَيَاةِ. اَلسَّجْنُ مَدَى الْحَيَاةِ. بَيْنَمَا... اِعْدَامٌ. شَنْقٌ. بَشَرِيَّةٌ. حُكْمٌ. صَلُحَ (و) صَلاَحٌ. بِمِثْلِ الإِسْلاَمِ. 

سُنَّةٌ, سُنَنٌ. غَيَّرَ. اُمَّةٌ, اُمَمٌ. ذُلٌّ. عِزٌّ. سُنَّةُ اللَّهِ. 

\  \includegraphics[width=0.948in,height=2.0626in]{images/MuhammadBagauddinprettified-img254.png} 

جَرَتْ سُنَّةُ اللَّهِ. اَلإِعْدَامُ شَنْقًا. بَذَلَ وُسْعَهُ (و) بَذْلٌ. سَعَى فِى نُصْرَةِ الدِّينِ.

\_\_\_\_\_\_\_\_\_\_\_\_\_\_\_\_\_\_\_\_\_\_\_\_\_

مَا اَحْسَنَ بِالرَّجُلِ اَنْ يَصْدُقَ وَ مَا اَقْبَحَ بِهِ أَنْ يَكْذِبَ. اَتُحِبُّ الْعِلْمَ؟ - طَبْعًا, بِالتَّاْكِيدِ أُحِبُّهُ. – اِذًا فَابْذُلْ فِى طَلَبِهِ كُلَّ وُسْعِكَ. هُوَ دُونَ شَكٍّ يُجِيدُ ثَلاَثَ لُغَاتٍ اَجْنَبِيَّةٍ. فِى السَّنَوَاتِ مَا بَعْدَ الثَّوْرَةِ فِى اِيرَانَ اَدْرَكَ الْكُفَّارُ سَوَاءٌ فِى الْغَرْبِ اَمْ فِى الشَّرْقِ اَنَّ الإِسْلاَمَ فِى طَرِيقِ النَّهْضَةِ وَ اَنَّ الْمُسْلِمِينَ يَصْحُونَ وَ اَنَّهُمْ سَيَتَّحِدُونَ كَمَا اَدْرَكُوا اَنَّ مُؤَامَرَاتِهِمْ ضِدَّ الإِسْلاَمِ سَتَبُوءُ بِالْفَشَلِ. مَتَى يَعُودُ أَخُوكُمْ؟ - لاَبُدَّ وَ اَنْ يَعُودَ قَبْلَ الْمَسَاءِ, بِذَا وَعَدَنَا. اِسْتَمَعَ الْمُتَّهَمُ اِلَى الْحُكْمِ عَلَيْهِ بِالسَّجْنِ مَدَى الْحَيَاةِ كَاَنَّمَا لاَ يَعْنِيهِ. أُتْرُكْ مَا لاَ يَعْنِيكَ. يُعَاقَبُ فِى بِلاَدِ الإِنْجِلِيزِ عَلَى هَذِهِ الْجَرِيمَةِ بِالسَّجْنِ لِمُدَّةِ خَمْسَ عَشْرَةَ سَنَةً بَيْنَمَا يُعَاقَبُ عَلَى مِثْلِهَا فِى بِلاَدِ الأَمْرِيكَانِ بِالإِعْدَامِ شَنْقًا. اَلإِسْلاَمُ أَعْدَلُ الأَدْيَانِ وَ أَعْدَلُ الأَنْظِمَةِ حُكْمًا وَ لَنْ تَصْلُحَ حَالُ الْبَشَرِيَّةِ بِمِثْلِ الإِسْلاَمِ. جَرَتْ سُنَّةُ اللَّهِ اَنْ لاَ يُغَيِّرَ حَالَةَ فَرْدٍ اَوْ حَالَةَ أُمَّةٍ اِلاَّ اِذَا أَرَادُوا ذَلِكَ وَ بَذَلُوا فِيهِ وُسْعَهُمْ فَعَلَى الْمُسْلِمِينَ اِذًا اَنْ يَسْعَوْا فِى تَغْيِيرِ حَالِهِمْ مِنْ ذُلٍّ اِلَى عِزٍّ.

\subsubsection{К уроку 71}
Я должен поблагодарить тебя за твою помощь мне, когда я попал в беду. — Не стоит благодарности, это мой долг перед братом-мусуль­манином. Братья-мусульмане всегда помогают друг другу (بعضهم بعضًا). Не тревожь его, оставь его в покое, он сегодня не в настроении. Сегодня перед вами выступит профессор университета (يَخْطُبُكُمْ), задавайте ему свои вопросы. Бог дарует (даст) нам победу над нашими врагами. Бог дарует (даст) победу своей религии. Бог всегда дарует (даёт) победу мусульманам, когда они подчиняются Ему. Открой мне бутылку прохладительного напитка, очень хочется пить (انا عطشان جدًّا). В нашем городе много исторических памятников, особенно в окрестностях города. Где ты провёл каникулы? — Я провел каникулы в горах и часто отдыхал в лесу на свежем воздухе. Прощаемся с вами в надежде встретиться снова. Выйди на свежий воздух и отдохни, ты очень много занимался. Читай тихо, чтобы не разбудить спящих и не потревожить их. Иди, разбуди его, ему пора вставать, время (السَّاعة) — ровно шесть. Я часто провожу свободные часы в мечети, а иногда ночую там.

\subsection{اَلدَّرْسُ الثَّانِى وَ التِّسْعُونَ 92}
 \includegraphics[width=1.3335in,height=0.7602in]{images/MuhammadBagauddinprettified-img255.png}   \includegraphics[width=1.2917in,height=0.9165in]{images/MuhammadBagauddinprettified-img256.png}   \includegraphics[width=1.8854in,height=1.052in]{images/MuhammadBagauddinprettified-img257.png} 

\ نَعْلٌ, نِعَالٌ. مِخْرَزٌ, مَخَارِزُ. شَالٌ, شِيلاَنٌ. 

دُكَّانٌ, دَكَاكِينُ. صَادِقٌ, صَادِقُونَ. غَدِيرٌ, غُدْرَانٌ. 

نَقَّ (ى) نَقِيقٌ. بَرَّمَائِىٌّ. بَرَى (ى) بَرْىٌ. خَاطَبَ. 

\  \includegraphics[width=1.052in,height=1.0102in]{images/MuhammadBagauddinprettified-img258.png} 

مُخَاطِبًا. اِسْكَافٌ, اَسَاكِفَةٌ. ضِفْدَعٌ, ضَفَادِعُ. 

ثَقَبَ (و) ثَقْبٌ. حَضَرَتِ الْقَهْوَةُ. حَضَرَتِ الصَّلاَةُ. يَسُرُّنِى. جَلَبَ (ى) جَلْبٌ. بِكُلِّ سُرُورٍ. تَقِىٌّ, اَتْقِيَاءُ.

مُحِبٌّ لِلْعِلْمِ. اَللَّهُ يَرْضَى عَنْكَ. حَرَصَ (ى) حِرْصٌ. صَاحَبَ

\_\_\_\_\_\_\_\_\_\_\_\_\_\_\_\_\_\_\_\_\_

شَاهِينٌ يَشْرَبُ كَأْسَ شَاىٍ. شَاهِينٌ وَلَدٌ صَادِقٌ لاَ يَكْذِبُ اَبَدًا ِلاَنَّهُ يَعْلَمُ اَنَّ الْكَاذِبَ مَلْعُونٌ فِى كِتَابِ اللَّهِ. لَهُ شَالٌ مِنْ صُوفٍ اِشْتَرَاهُ لَهُ أَبُوهُ فِى دُكَّانِ رَشِيدٍ. اَلضَّفَادِعُ فِى الغَدِيرِ كَثِيرَةٌ. وَ هِىَ تَنِقُّ كَثِيرًا وَ يُسْمَعُ نَقِيقُهَا فِى اللَّيَالِى عَلَى مَسَافَةٍ بَعِيدَةٍ. وَ الضِّفْدَعُ مِنَ الْحَيْوَانَاتِ الْبَرَّمَائِيَّةِ وَ لَهُ اَنْوَاعٌ كَثِيرَةٌ. اَلْمِبْرَاةُ آلَةُ بَرْىِ الْقَلَمِ. بَرَيْتُ الْقَلَمَ بِالْمِبْرَاةِ. يَقُولُ الْمُعَلِّمُ مُخَاطِبًا التَّلاَمِيذَ: اَيَّهَا التَّلاَمِيذُ اُتْرُكُو اللَّعِبَ فَيَسْمَعُ التَّلاَمِيذُ قَوْلَ مُعَلِّمِهِمْ وَ يُطِيعُونَهُ وَ يَقْرَأُونَ دُرُوسَهُمْ. اَلْمِخْرَزُ آلَةُ الإِسْكَافِ بِهِ يَثْقُبُ النَّعْلَ الَّذِى يَخِيطُهُ. حَضَرَتِ الْقَهْوَةُ بِاللَّبَنِ فَهَلْ تَشْرَبُهَا؟ - نَعَمْ, بِكُلِّ سُرُورٍ, وَ مِنْ اَيْنَ لَكُمُ الْقَهْوَةُ؟ مُنْذُ وَقْتٍ قَرِيبٍ زَارَ أَبِى بِلاَدَ الْيَمَنِ فَجَلَبَهَا مِنْهَا. يَسُرُّنِى يَا وَلَدِى أَنْ اَرَاكَ هَكَذَا طَالِبًا مُجْتَهِدًا نَشِيطًا تَقِيًّا مُحِبًّا لِلاِسْلاَمِ مُجَاهِدًا فِى سَبِيلِ اللَّهِ, اَللَّهُ يَرْضَى عَنْكَ وَ احْرِصْ اَلاَّ تُصَاحِبَ اِلاَّ مَنْ كَانَ كَذَلِكَ.

\subsubsection{К уроку 72}
Рассказать тебе анекдот? Я знаю один интересный анекдот. Мой дедушка знал много анекдотов. Не надо анекдотов (لا ضرورة), я не люблю их. Когда мы поговорим по этому делу? — Когда у нас будет свободное время. Когда ты мне урок объяснишь? Я к тебе пришёл с несколькими вопросами. — Потом, когда (они) уедут от нас. Ты лотерею покупаешь? — Нет, я не покупаю лотерею, я знаю, что лотерея запрещена в Исламе, и я должен подчиняться правилам Ислама, а всякое запретное — это дело сатаны. Это сатана ( (الشيطان هو الذى) наушничает людям. Сатана — враг человека. Лотерея — один из видов азартных игр, а всякая (كلّ) азартная игра запрещена. Почему ты этим делом занимаешься? — А что тут такого? — Ты забыл, что нам говорил папа? — Да, забыл. Зафир, я вижу тебя сегодня радостным и не понимаю причины (секрета) твоей радости. — Ты не слышал, У меня родился сын! Один лотерейный билет стоит 60 копеек. Оставь Шутки! — Что? — Я говорю тебе: „Оставь шутки!" Положи нож в карман! Вытащи нож из кармана.

اَلدَّرسُ الثَّالِثُ وَ التِّسْعُونَ 93

\  \includegraphics[width=0.8335in,height=0.6772in]{images/MuhammadBagauddinprettified-img259.png}   \includegraphics[width=1in,height=0.7083in]{images/MuhammadBagauddinprettified-img260.png}   \includegraphics[width=0.9165in,height=0.5937in]{images/MuhammadBagauddinprettified-img261.png} 

هُدْبٌ, أَهْدَابٌ. جَفْنٌ, أَجْفَانٌ. حَاجِبٌ, حَوَاجِبُ.

\  \includegraphics[width=0.7602in,height=1.2291in]{images/MuhammadBagauddinprettified-img262.png}   \includegraphics[width=0.75in,height=0.7083in]{images/MuhammadBagauddinprettified-img263.png}   \includegraphics[width=0.7811in,height=0.9689in]{images/MuhammadBagauddinprettified-img264.png}   \includegraphics[width=1.9583in,height=1.302in]{images/MuhammadBagauddinprettified-img265.png} 

أَنْفٌ, أُنُوفٌ. شَارِبٌ, شَوَارِبُ. لِحْيَةٌ, لُحًى.

عُضْوٌ, اَعْضَاءٌ. وَظِيفَةٌ, وَظَائِفُ. أُحَادُ. مَثْنَى. 

مِنْهَا مَا هُوَ. شَمَّ (و) شَمٌّ. رَائِحَةٌ, رَوَائِحُ. لِمَ لاَ تَقْرَأُ؟

عَمَّا قَرِيبٍ. جَزَّ (و) جَزٌّ. جَزَّ الصُّوفَ. اَعْفَى اللِّحْيَةَ.

شَعْرَةٌ (ات). شَعْرٌ, شُعُورٌ. وَ لَوْ مَرَّةً. 

طَرَفٌ, اَطْرَافٌ. قَلَمَ (ى) قَلْمٌ. قَلَمَ الظُّفْرَ. 

دَهَنَ (و) دَهْنٌ. سَرَّحَ. سَرَّحَ شَعْرَهُ. أَحْفَى الشَّارِبَ.

مَجُوسِىٌّ, مَجُوسٌ.

\_\_\_\_\_\_\_\_\_\_\_\_\_\_\_\_\_\_\_\_\_\_\_\_\_\_\_\_

لِلاِنْسَانِ اَعْضَاءٌ كَثِيرَةٌ وَ لِكُلِّ عُضْوٍ مِنْهَا وَظِيفَتُهُ. فَلَهُ مِنْهَا مَا هُوَ اُحَادُ كَالرَّأْسِ وَ الاَنْفِ وَ الْفَمِ وَ اللِّسَانِ وَ غَيْرِهَا, وَ لَهُ مِنْهَا مَا هُوَ مَثْنَى كَاليَدَيْنِ وَ الرِّجْلَيْنِ وَ الاُذُنَيْنِ وَ الْعَيْنَيْنِ. بِالأُذُنِ يَسْمَعُ الاِنْسَانُ الاَصْوَاتَ وَ بِالِّسَانِ يَتَكَلَّمُ وَ بِالْفَمِ وَ الاَسْنَانِ يَاْكُلُ وَ يَشْرَبُ وَ بِالْيَدِ يُمْسِكُ الاَشْيَاءَ وَ بِالاَنْفِ يَشُمُّ الرَّائِحَةَ وَ بِالْعَيْنِ يَرَى. لِكُلِّ يَدٍ خَمْسُ اَصَابِعَ: اَلاِبْهَامُ وَ السَّبَّابَةُ وَ تُسَمَّى الْمُسَبِّحَةَ أَيْضًا وَ الْوُسْطَى وَ الْبِنْصِرُ وَ الْخِنْصِرُ. لِمَ لاَ تَكْتُبُ دُرُوسَكَ؟ لِمَ لاَ تَحْتَرِمُ الْكِبَارَ؟ لِمَ لاَ تُصَلِّى الصَّلَوَاتِ فِى أَوْقَاتِهَا؟ عَمَّا قَرِيبٍ سَيَنْبُتُ شَارِبُكَ وَ تَنْبُتُ لِحْيَتُكَ فَاِذَا نَبَتَا فَجُزَّ شَارِبَكَ وَ أَعْفِ لِحْيَتَكَ بِذَا اَمَرَنَا النَّبِىُّ صَلَّى اللَّهُ عَلَيْهِ وَ سَلَّمَ, وَ قَدْ جَاءَ فِى حَدِيثٍ آخَرَ: اَحْفُوا الشَّوَارِبَ وَ أَعْفُوا اللُّحَى وَ خَالِفُوا الْمَجُوسَ. اَلاَهْدَابُ هِىَ شَعَرَاتٌ نَبَتَتْ عَلَى أَطْرَافِ الاَجْفَانِ وَ الْحَوَاجِبُ هِىَ الشَّعَرُ النَّابِتُ فَوْقَ الْعَيْنَيْنِ. اِقْلِمْ ظُفْرَكَ وَ قُصَّ شَارِبَكَ وَ اغْسِلْ شَعْرَ رَأْسِكَ وَ ادْهُنْهُ وَ سَرِّحْهُ وَ لَوْ مَرَّةً فِى كُلِّ اُسْبُوعٍ.

\subsubsection{К уроку 73}
Не будь никогда самодовольным, Бог отнимет у тебя успех. Когда экзамен? — Экзамен скоро. Я своими глазами видел то, что случилось в автобусе среди парней. Из города были парни? — Нет, пришли с гор. Где наш товарищ Саид? — Он болен, прикован к постели, вот уже месяц ( منذ شهرٍ ) он не выходит из дома. — Тогда мы должны посетить его с гостинцами (подарками). Он понял ошибочность своего мышления. Она поняла ошибочность своего мышления. Вы поняли ошибочность своего мышления? — Да, поняли. — Это хорошо. Он провалился (на экзамене). Эту болезнь надо лечить ( لابدّ من) Это серьёзная болезнь. Когда сын вернётся из армии, то обязательно сообщите нам. Когда экзамен закончится, то обязательно напиши нам о результате своего экзамена. — Хорошо, напишу. Извините за беспокойство! — Пожалуйста, что вы хотели? — Я хотел спросить вас о том, когда прибывает поезд. Я хотел спросить вас о результатах экзаменов моего сына. И в какой он группе, в группе успешно сдавших (экзамены) или ( أم ) в группе провалившихся? — О результатах экзаменов написано на этом листе, пожалуйста, прочитайте его.

\subsection{اَلدَّرْسُ الرَّابِعُ وَ التِّسْعُونَ 94}
 \includegraphics[width=1.9791in,height=1.1665in]{images/MuhammadBagauddinprettified-img266.png}   \includegraphics[width=0.6563in,height=0.8854in]{images/MuhammadBagauddinprettified-img267.png} 

\ بُنْدُقِيَّةٌ, بَنَادِقُ. خَرْطُوشَةٌ (ات). كَرِيمٌ, كُرَمَاءُ. 

سَخِىٌّ, اَسْخِيَاءُ. بَخِيلٌ, بُخَلاَءُ. فِيمَا مَضَى. غَيْرُهُ. اِسْتَغْنَى عَنْ... سَائِلٌ, سُؤَّالٌ. رَدَّ (و) رَدٌّ. 

قِطْعَةٌ, قِطَعٌ. شَرْبَةٌ. بِخِلاَفِ... (عَلَى خِلاَفِ...). تَصَدَّقَ عَلَى... مَعْرُوفٌ بِ... كَرَمٌ. شُحٌّ. سَخَاءٌ. بُخْلٌ. بَقِىَ (ا) بَقَاءٌ. لَهُ اَنْ... لَيْسَ لَهُ اَنْ... اِفْتَخَرَ.

اِنْطَلَقَ. اِحْتَطَبَ. اِسْتَصْحَبَ. وَحْشٌ, وُحُوشٌ. 

صَادَ الْوَحْشَ. فَارِغٌ. عَبَّأَ.

\_\_\_\_\_\_\_\_\_\_\_\_\_\_\_\_\_\_\_\_\_\_\_\_\_\_\_\_

كُنَّا نَقْرَأُ فِيمَا مَضَى لِدِرَاسَةِ اللُّغَةِ الْعَرَبِيَّةِ كِتَابَ "مَبْدَأِ الْقِرَاءَةِ" أَوِ "الدُّرُوسِ الشِّفَاهِيَّةِ" وَ قَدْ ظَهَرَتْ عِنْدَنَا الآنَ كُتُبٌ غَيْرُهُمَا نَسْتَغْنِى بِهَا عَنْ تِلْكَ. خَالِدٌ بَخِيلٌ جِدًّا اِذَا جَاءَهُ سَائِلٌ يَرُدُّهُ وَ لاَ يَمْنَحُهُ قِطْعَةَ خُبْزٍ وَ لاَ شَرْبَةَ مَاءٍ. وَ اَمَّا جَارُهُ حَسَنٌ فَكَرِيمٌ سَخِىٌّ عَلَى خِلاَفِهِ يُحْسِنُ اِلَى الْفُقَرَاءِ وَ يَتَصَدَّقُ عَلَى الْمُحْتَاجِينَ فَذَلِكَ مَعْرُوفٌ بِبُخْلِهِ وَ شُحِّهِ وَ هَذَا مَعْرُوفٌ بِكَرَمِهِ وَ سَخَائِهِ. هَذِهِ اَقَارِبِى وَ تِلْكَ اَقَارِبُكَ فَانْظُرْ اَيُّهُمَا أَقْوَى اِيمَانًا وَ اَكْثَرُ جِهَادًا اِذًا فَلِى اَنْ اَفْتَخِرَ بِأَقَارِبِى وَ لَيْسَ لَكَ اَنْ تَفْتَخِرَ بِأَقَارِبِكَ. غَدًا صَبَاحًا نَنْطَلِقُ اِلَى الْغَابِ لِلاِحْتِطَابِ وَ نَسْتَصْحِبُ بَنَادِقَنَا لِنَصِيدَ بَعْضَ الْوُحُوشِ وَ نَبْقَى فِيهِ اِلَى الْمَسَاءِ ثُمَّ نَرْجِعُ اِلَى مَنْزِلِنَا. هَلْ عِنْدَكَ خَرْطُوشَاتٌ؟ - نَعَمْ عِنْدِى خَرْطُوشَاتٌ لَكِنَّهَا فَارِغَةٌ عَلَىَّ اَنْ أُعَبِّئَهَا.

\subsubsection{К уроку 74}
Этот мальчик способный, к тому же ( أضو الى ذلك ) и старательный. Из него выйдет, если Богу будет угодно, учёный. Долго обманывал коммунизм людей, теперь они поняли, что такое ( ما هى ) коммунизм. Отныне он не обманет их никогда. Долго ты обманывал меня. |Человеческий ум изобрёл много полезных вещей, он изобрел и много вредных вещей ( ضارة ), одной из таких является коммунизм. Ислам — строй Бога для людей. Ислам — лучший строй. А другие строи — строи сатаны. По окончании уроков студенты ушли домой. По окончании намаза мужчины ушли домой. Нет никого (ни одного из людей), который не знал бы эту истину. Не осталось ни одного целовека, который бы не понимал, что такое коммунизм. Теперь появился другой строй — хуже коммунизма; а ты знаешь, что это Такое? Это — демократия. Я совершил это дело, но я потом пожалел ندمت فيما بعد) ) о нём. Мусульмане — враги гяуров. Амр самый плохой человек в нашем селе, а Зайд — самый хороший. Изучай историю 'Ислама, чтобы ты полностью понимал его. Гости вернулись поздней ночью. Я занимался уроками до поздней ночи. Пока он здесь, я помолчу, он старше меня. Пока он дома, я не войду туда, я боюсь его. Учись, пока молодой.

\subsection{اَلدَّرْسُ الْخَامِسُ وَ التِّسْعُونَ 95}
ظَهَرَ الإِسْلاَمُ عَلَى سَائِرِ الأَدْيَانِ. فِى شَأْنِ... مَاضٍ.

بِمَا يَقْدِرُ. وَجَبَ (ى) وُجُوبٌ. وَاجِبٌ. وَ هَكَذَا دَوَالَيْكَ. 

\  \includegraphics[width=1.3543in,height=1.1874in]{images/MuhammadBagauddinprettified-img268.png} 

تَفَكَّرَ. حُكْمٌ. مِرْآةٌ, مَرَايَا. حَسَّنَ. كَىْ يَكُونَ. 

مَيِّتٌ, أَمْوَاتٌ. رَسْمٌ. تَبَعْثَرَ. ضَاعَ (ى) ضَيَاعٌ.

حَافَظَ عَلَى... حَافَظَ عَلَى النَّظَافَةِ. حَافَظَ عَلَى الصَّلاَةِ. 

صَامَ (و) صَوْمٌ, صِيَامٌ. وَسَّخَ. عَلَى وَجْهِ الأَرْضِ. حُكْمٌ.

\_\_\_\_\_\_\_\_\_\_\_\_\_\_\_\_\_\_\_\_\_\_\_\_\_\_\_\_\_\_\_\_\_

أَتَعْرِفُ مَا هُوَ الْجِهَادُ؟ - نَعَمْ, اَلْجِهَادُ هُوَ اَنْ تَبْذُلَ مَا فِى وُسْعِكَ فِى نُصْرَةِ الإِسْلاَمِ وَ فِى ظَهُورِهِ عَلَى سَائِرِ النُّظُمِ وَ الأَدْيَانِ وَ اَنْ لاَ يَبْقَى دِينٌ وَ لاَ حُكْمٌ وَ لاَ نِظَامٌ غَيْرُ الإِسْلاَمِ عَلَى وَجْهِ الأَرْضِ. اَتَعْرِفُ مَاذَا قَالَ النَّبِىُّ – صَلَّى اللَّهُ عَلَيْهِ وَ سَلَّمَ – فِى شَأْنِ الْجِهَادِ؟ - هُوَ قَالَ فِى شَأْنِ الْجِهَادِ: اَلْجِهَادُ مَاضٍ اِلَى يَوْمِ الْقِيَامَةِ, يَعْنِى اَنَّ الْجِهَادَ لاَ يَنْقَطِعُ وَ لاَ يَنْبَغِى اَنْ يَنْقَطِعَ مَادَامَ عَلَى وَجْهِ الأَرْضِ دِينٌ أَوْ نِظَامٌ اَوْ حُكْمٌ غَيْرُ الإِسْلاَمِ. اَيَجِبُ الْجِهَادُ عَلَى كُلِّ مُسْلِمٍ وَ مُسْلِمَةٍ؟ - نَعَمْ, اَلْجِهَادُ وَاجِبٌ عَلَى كَلِّ وَاحِدٍ بِمَا يَقْدِرُ: فَالْغَنِىُّ يُجَاهِدُ بِمَالِهِ وَ الْقَوِىُّ بِقُوَّتِهِ وَ الْعَالِمُ بِعِلْمِهِ وَ الرِّيَاضِىُّ بِرِيَاضَتِهِ وَ الْعَامِلُ بِعَمَلِهِ وَ هَكَذَا دَوَالَيْكَ. نَظَرْتُ فِى الْمِرْآةِ فَرَأَيْتُ وَجْهِى فِيهَا وَ تَفَكَّرْتُ فِى خَلْقِهِ وَ حَمِدْتُ اللَّهَ عَلَى أَنْ حَسَّنَهُ. تَعَلَّمْ يَاوَلَدُ كَىْ تُصْبِحَ عَالِمًا وَ لاَ تَبْقَى جَاهِلاً فَالْجُهَلاَءُ أَمْوَاتٌ قَبْلَ مَوْتِهِمْ وَ الْعُلَمَاءُ اَحْيَاءٌ بَعْدَ مَوْتِهِمْ. اَلْعُلَمَاءُ مُحْتَرَمُونَ دَائِمًا. بَعْدَ انْتِهَائِكَ مِنَ الرَّسْمِ ضَعِ الاَقْلاَمَ فِى الْمِقْلَمَةِ لِئَلاَّ تَتَبَعْثَرَ وَ تَضِيعَ. عَائِشَةُ تِلْمِيذَةٌ مُجْتَهِدَةٌ نَشِيطَةٌ وَ مُؤَدَّبَةٌ تُحَافِظُ عَلَى صَلاَتِهَا وَ صِيَامِهَا وَ تُحِبُّهَا مُعَلِّمَتُهَا كَثِيرًا وَ لاَ تُوَسِّخُ لِبَاسَهَا اَبَدًا وَ تَكُونُ دَائِمًا نَظِيفَةً, وَدِدْتُ لَوْ اَنَّهَا كَانَتْ زَوْجَتِى.

\subsubsection[К уроку 75]{К уроку 75}
Давай поговорим ( تعال ), пока у нас есть свободное время. Учись, ещё раз (ثمّ ) учись, пока ты молод. Площадь нашей страны миллион квадратных километров. Площадь нашего селения — один квадратный километр. Это западная граница, а это — восточная граница, и расстояние между ними — 2000 километров. Расстояние между двумя городами 300 километров. Он не только превосходный учёный, но и превосходный водитель. Посмотри на карту и скажи в какой стране самая глубокая река и самое глубокое озеро. Скажи мне, какая самая высокая гора в мире? У какой страны самая большая Территория в мире? Какой самый лучший строй? Река Нил — самая длинная река в мире. — Совершенно верно. Нам надо купить каменного угля на зиму. — В вашем доме природного газа нет? Жаль, что ты не понимаешь ( لا تدرك) ) суть. Жаль, что мусульмане пока ещё слабы. Жаль, что мусульмане ещё не пробудились (не проснулись) полностью. Скажи мне, где ваша граница, здесь или (أم ) здесь?

\subsection{اَلدَّرْسُ السَادِسُ وَ التِّسْعُونَ 96}
 \includegraphics[width=1.4272in,height=0.8646in]{images/MuhammadBagauddinprettified-img269.png}   \includegraphics[width=0.9791in,height=0.8957in]{images/MuhammadBagauddinprettified-img270.png}   \includegraphics[width=1.1252in,height=0.9689in]{images/MuhammadBagauddinprettified-img271.png} 

آلَةُ التَّصْوِيرِ. لِجَامٌ, لُجُمٌ. سَرْجٌ, سُرُوجٌ.

حَاضِنَةٌ, حَوَاضِنُ. وَلِيدٌ, وِلْدَانٌ. سَقَاهُ مَاءً. 

قَمَطَ (ى) قَمْطٌ. نَوَّمَ. نَوَّمَ الْوَلِيدَ. مُرْ. خَادِمٌ, خُدَّامٌ.

\  \includegraphics[width=0.9898in,height=0.9063in]{images/MuhammadBagauddinprettified-img272.png} 

خَادِمَةٌ (ات). مَهْدٌ, مُهُودٌ. لَحْدٌ, لُحُودٌ. أَلْجَمَ. أَسْرَجَ.

مَصْنُوعٌ. عَزَمَ عَلَى... (ى) عَزْمٌ. جِلْدٌ, جُلُودٌ.

عَلَى مَهْلٍ. مَشَى عَلَى مَهْلٍ. حُزْمَةٌ, حُزَمٌ. سَمِيكٌ.

لَوَازِمُ الْكِتَابَةِ. تَعَوَّدَ. صَوَّرَ. مُصَوِّرٌ (ون).

غَرِيبٌ. غَرِيبُ الشَّكْلِ. اَشْبَهَ. لاَ يُشْبِهُ. 

حَجْمٌ, حُجُومٌ. صَغِيرُ الْحَجْمِ. كَبِيرُ الْحَجْمِ. عَادِىٌّ.

اِتَّخَذَ صَدِيقًا. دَارٌ, دِيَارٌ. كَفَرَ (و) كُفْرٌ. كَفَرَ بِاللَّهِ.

رَضِيتُ بِاللَّهِ رَبًّا. رَضِيتُ بِالإِسْلاَمِ دِينًا.

\_\_\_\_\_\_\_\_\_\_\_\_\_\_\_\_\_\_\_\_\_\_\_\_\_

اَلْحَاضِنَةُ تُطْعِمُ الْوَلِيدَ وَ تَسْقِيهِ وَ تَقْمِطُهُ وَ تُنَوِّمُهُ فِى الْمَهْدِ. مُرِ الْخَادِمَ أَنْ يُلْجِمَ الْفَرَسَ وَ يُسْرِجَهُ فَقَدْ عَزَمْتُ عَلَى السَّفَرِ بَعْدَ الظُّهْرِ. بِأَىِّ اللِّجَامَيْنِ يُلْجِمُ؟ - لِيُلْجِمْ بِاللِّجَامِ الْجَدِيدِ الْمَصْنُوعِ مِنْ جِلْدٍ وَ اللَّذِى جِئْتُ بِهِ مِنَ السُّوقِ مُنْذُ أَيَّامٍ. مُوسَى يَمْشِى عَلَى مَهْلٍ وَ لَيْلَى تَمْشِى عَلَى مَهْلٍ. عَلَى مَكْتَبِى دَوَاةٌ وَ نَشَّافَةٌ وَ حُزْمَةٌ سَمِيكَةٌ مِنَ الْوَرَقِ وَ سَائِرُ لَوَازِمِ الْكِتَابَةِ. تَعَوَّدْ اَنْ تَنَامَ بَاكِرًا وَ تَنْهَضَ بَاكِرًا. عُثْمَانُ مُصَوِّرٌ مَاهِرٌ يُصَوِّرُ النَّاسَ بِآلَةِ التَّصْوِيرِ, وَ عِنْدَهُ آلَةُ تَصْوِيرٍ جُلِبَتْ مِنَ الْيَابَانِ, وَ هِىَ غَرِيبَةُ الشَّكْلِ لاَ تُشْبِهُ آلاَتِ التَّصْوِيرِ عِنْدَنَا وَ صَغِيرَةُ الْحَجْمِ يُمْكِنُ حَمْلُهَا فِى الْجَيْبِ الْعَادِىِّ. الإِسْلاَمُ دِينُ اللَّهِ الْخَالِدُ, عَلَى الْمُسْلِمِ أَنْ يَأْخُذَ مِنَ الإِسْلاَمِ كُلَّ مَا يَحْتَاجُ اِلَيْهِ, وَ عَلَى الْمُسْلِمِ اَنْ يَرْضَى بِهِ دِينًا وَ دَوْلَةً وَ حُكْمًا وَ نِظَامًا وَ خُلُقًا وَ يُحِبَّهُ وَ لَا يَرْضَى بِسِوَاهُ فَمَنْ رَضِيَ بِسِوَاهُ دِينًا وَ دَوْلَةً وَ حُكْمًا وَ نِظَامًا وَ خُلُقًا فَقَدْ عَبْدَهُ وَ اتَّخَذَهُ اِلَهًا وَ مَنِ اتَّخَذَ اِلَهًا غَيْرَ اللَّهِ فَقَدْ كَفَرَ, وَ لَيْسَ لِلْكَافِرِ دَارٌ يَسْكُنُهَا فِى الآخِرَةِ اِلاَّ النَّارُ. اُطْلُبِ الْعِلْمَ مِنَ الْمَهْدِ اِلَى اللَّحْدِ وَ مَعَ ذَلِكَ تَمُوتُ جَاهِلاً.

\subsubsection{К уроку 76}
Погода, как видишь, сегодня плохая, что будем делать? Пойдём в лес на охоту ( للصّيد )? Где ты проводил летние каникулы? — На курорте. Когда твои каникулы кончились? — Недавно. Ты хороший чтец Корана, слушай моё чтение. Я недавно вернулся от них ( من عندهم ) Как их дела? Как их здоровье? — Слава Богу, всё в порядке. Приходи ко мне с уроками, будем учить их вместе. Статью читал в сегодняшней газете? — Да, читал. Как пишет наш друг? — Мне понравилась его идея. До сих пор мы не учили своих детей читать и писать. — Это плохо, надо было вам учить их. На, читай эту книгу, она мне очень (كثيرا) понравилась и тебе понравится. Она небольшая. Как тебе нравится ( هل أعجبتك ) наша страна? — Великолепно, но жаль, что она в руках гяуров, а не мусульман. Почему ты здесь до сих пор-Почему ты остался до сих пор? Ты знаешь, сколько сейчас времени كم الساعة) )? До сих пор людям не раскрылась суть коммунизма.

\subsection{-97-}
وَصِيَّةٌ, وَصَايَا. أَوْصَى. خَالَفَ الْوَصِيَّةَ. اِغْتَسَلَ.

اِسْتَدْعَى. عَايَنَ. عَايَنَ الْمَرِيضَ. كَشَفَ (ى) كَشْفٌ.

دَاءٌ, أَدْوَاءٌ. وَصَفَ الدَّوَاءَ. مُعَالَجَةٌ. دَامَ (و) دَوَامٌ.

اَمْرٌ, أَوَامِرُ. أَوَامِرُ اللَّهِ. عِقَابٌ. تَابَ (و) تَوْبَةٌ.

عَادَ اِلَى... لَمْ يَعُدْ اِلَى... فَرْضٌ, فَرُوضٌ. مُعَقَّدٌ.

مَسْئَلَةٌ مُعَقَّدَةٌ. وَرَدَ (ى) وُرُودٌ. اَكْرَمَ. شَيْخُوخَةٌ.

رِفَاقٌ. غِيَابٌ. هَذِهِ الْمُدَّةَ. صَلَّى الْفَرْضَ.

\_\_\_\_\_\_\_\_\_\_\_\_\_\_\_\_\_\_\_\_\_\_\_\_\_\_\_

كَانَتِ الاُمُّ تُوصِى ابْنَهَا دَائِمًا اَنْ لاَ يَذْهَبَ فِى يَوْمٍ بَارِدٍ اِلَى النَّهْرِ وَ اَنْ لاَ يَغْتَسِلَ فِيهِ وَ لَكِنَّهُ خَالَفَ وَصِيَّةَ أُمِّهِ وَ ذَهَبَ اِلَى النَّهْرِ وَ اغْتَسَلَ فِيهِ فَاُصِيبَ بِمَرَضٍ فَاسْتَدْعَوْا طَبِيبًا فَعَايَنَهُ الطَّبِيبُ فَكَشَفَ دَاءَهُ فَوَصَفَ لَهُ دَوَاءً. بَقِىَ التِّلْمِيذُ مَرِيضًا عَلَى فِرَاشِهِ عِدَّةَ أَيَّامٍ. وَ بَعْدَ مُعَالَجَةٍ دَامَتْ أُسْبُوعَيْنِ شُفِىَ التِّلْمِيذُ مِنْ مَرَضِهِ وَ ذَهَبَ اِلَى الْمَدْرَسَةِ وَ اَخْبَرَ رِفَاقَهُ عَنْ سَبَبِ غِيَابِهِ هَذِهِ الْمُدَّةَ. أَدْرَكَ التِّلْمِيذُ اَنَّ مَا اَصَابَهُ لَمْ يَكُنْ اِلاَّ عِقَابًا مِنَ اللَّهِ لَهُ عَلَى مُخَالَفَتِهِ ِلاَوَامِرِ اُمِّهِ فَتَابَ مِمَّا عَمِلَ وَ لَمْ يَعُدْ اِلَى الْمُخَالَفَةِ قَطُّ. زَوْجَةُ عَبْدِ اللَّهِ اِمْرَاَةٌ صَالِحَةٌ تُصَلِّى فَرْضَهَا وَ تُطِيعُ زَوْجَهَا وَ تُحْسِنُ اِلَى جَارَاتِهَا بِخِلاَفِ جَارَتِهَا الَّتِى تَعْمَلُ كُلَّ شَىْءٍ عَلَى خِلاَفِهَا. هَلْ سَاَلْتَ الْمُعَلِّمَ عَنِ الْمَسْئَلَةِ الْمُعَقَّدَةِ الَّتِى وَرَدَتْ فِى دَرْسِ الأَمْسِ؟ - لاَ, مَا سَأَلْتُهُ. – لِمَ لَمْ تَسْأَلْهُ؟ كَانَ يَجِبُ عَلَيْكَ أَنْ تَسْأَلَهُ. اَلشَّيُوخُ يُكْرَمُونَ. اَكْرِمُوا الشُّيُوخَ اَيُّهَا الْفِتْيَانُ اِنْ كُنْتُمْ تُحِبُّونَ اَنْ يُكْرِمَكُمُ الصِّغَارُ فِى شَيْخُوخَتِكُمْ.

\subsubsection{К уроку 77}
Да здравствует Ислам! Нет мира без Ислама! ( لا سلامَ بغيرِ الإسلام ) Жди меня, пока я не приду к тебе с известием. Сколько я должен ждать? У меня нет времени ждать. Это дело я оставляю за тобой. Что на обед? — На обед каша с молоком. Выйдем покататься на лодке? И погода сегодня прекрасная, и день солнечный. Ты получил новую квартиру? — Да, получил. — В каком квартале? — В соседнем квартале. Когда вы переехали туда? — Две недели назад. Для новой квартиры и новую мебель купили. Чего ты ждёшь? — Я жду прибытия поезда; когда он прибывает? — Поезд опаздывает на три часа. Что было написано на плакатах? — Там было написано: „Да здравствует Ислам и долой коммунизм", ,,Мы хотим Ислам", ,,Ислам и комму­низм — враги" и другие плакаты. Мы пошли на каток, там было много спортсменов. Дочь моя, возьми кастрюлю, налей туда воды и поставь на плиту, я сейчас сварю кашу с молоком на завтрак. А что мы приготовим на обед? — На обед мы сварим суп с мясом. В каждом квартале города есть квартальная мечеть, и в центре города есть соборная мечеть. На пятничный намаз мы ходим в эту мечеть.

\subsection{-98-}
رَاكِبٌ, رُكَّابٌ. سِنٌّ, أَسْنَانٌ. كَبِيرُ السِّنِّ. تَعَجَّبَ مِنْ...\newline
غَرَسَ (ى) غَرْسٌ. غَرَسَ شَجَرَةً. أَمَّلَ. أَتُؤَمِّلُ.\newline
مَا لَكَ؟ مَا لَكَ تَضْحَكُ؟ مَا لَكَ لاَ تَقْرَأُ؟ أَعِدُكَ. طَبَّلَ.\newline
نَائِمٌ, نِيَامٌ (نُوَّامٌ). قَالَ قَائِلٌ. أُمْنِيَّةٌ, اَمَانِىُّ.\newline
أُمْنِيَّتِىَ الْوَحِيدَةُ. كَانَتْ أُمْنِيَّتِىَ الْوَحِيدَةُ. اِسْتَنْفَدَ.\newline
اِسْتَنْفَدَ الْمَالَ.

\_\_\_\_\_\_\_\_\_\_\_\_\_\_\_\_\_\_\_\_\_\_\_\_\_

قَالَتِ الْمُعَلِّمَةُ ِلاِحْدَى التِّلْمِيذَاتِ: اُرْسَمِى سَيَّارَةً فِيهَا رَاكِبَانِ فَفَكَّرَتِ التِّلْمِيذَةُ قَلِيلاً ثُمَّ رَسَمَتْ سَيَّارَةً وَ مَا رَسَمَتْ رَاكِبَيْنِ. فَقَالَتِ الْمُعَلِّمَةُ: اَيْنَ الرَّاكِبَانِ؟ فَأَجَابَتِ التِّلْمِيذَةُ: نَزَلاَ. كَانَ رَجُلٌ كَبِيرُ السِّنِّ يَغْرِسُ زَيْتُونًا فِى حَقْلِهِ فَرَآهُ شَابٌّ فَتَعَجَّبَ مِنْهُ وَ قَالَ لَهُ: أَتُؤَمِّلُ اَنْ تَاْكُلَ مِنْ ثَمَرِ هَذَا الزَّيْتُونِ وَ اَنْتَ شَيْخٌ كَبِيرُ السِّنِّ. فَأَجَابَ الشَّيْخُ: غَرَسَ مَنْ قَبْلَنَا فَاَكَلْنَا وَ نَغْرِسُ فَيَاْكُلُ مَنْ بَعْدَنَا. قَالَ الْوَلَدُ ِلاَبِيهِ: مَالَكَ لاَ تَشْتَرِى لِى يَا أَبِى طَبْلاً صَغِيرًا؟ قَالَ الْوَالِدُ: اَخَافُ اَنْ تُزْعِجَنِى يَا بُنَىَّ بِصَوْتِهِ. فَقَالَ الْوَلَدُ: لاَ تَخَفْ يَا أَبِى, اِنِّى اَعِدُكَ اَنْ لاَ اُطَبِّلَ بِهِ اِلاَّ وَ اَنْتَ نَائِمٌ. قَالَ قَائِلٌ لِصَاحِبِهِ: كَانَتْ اُمْنِيَّتِىَ الْوَحِيدَةُ اَنْ اَشْتَرِىَ سَيَّارَةً. فَقَالَ لَهُ صَاحِبُهُ: وَ مَا اُمْنِيَّتُكَ الْوَحِيدَةُ بَعْدَ اَنِ اشْتَرَيْتَ سَيَّارَةً؟ قَالَ: أَبِيعَهَا. قَالَ: وَ لِمَاذَا؟ قَالَ: ِلاَنَّهَا اسْتَنْفَدَتْ أَمْوَالِى.

\subsubsection{К уроку 78}
Береги свой (ж.р.) язык от лжи. Берегите свой (ж.р.) язык от лжи. Воспитанная ученица никогда не врёт. Человек слышит ухом и говорит языком, и видит глазом, и держит рукой. Язык находится во рту за зубами, и зубы тоже находятся во рту. Ты ленивый ученик, часто отсутствуешь на уроках. Ученики, занимайтесь своими уроками много и не тратьте своё дорогое время на развлечение и игры. Дочка проснулась? — Да, проснулась. Видно, ты не хочешь пойти с нами в плавательный бассейн. Учитель пришёл на уроки? — Кажется, пришёл. Книгу Зейд задержал у себя, видимо, она ему понравилась. Ты кончил читать? Ты кончила писать? Вы кончили молиться? Мама кончила стирать одежду? Рабочие кончили работать? — Нет, ещё не кончили, скоро кончат. Я не хотел обижать тебя. Он не хотел обижать вас. Он сидит в зале ожидания и ждёт твоего прихода. Сегодня вечером После ночного намаза в мечети состоится собрание джамаата, всем надо присутствовать на собрании. Сегодня мы закончили строительство мечети в центре города.

\subsection[-99-]{-99-}
مَرْفَأٌ, مَرَافِئُ. قَلَعَ (ا) قَلْعٌ. ضَارٌّ. أَصْلَحَ. تَعَاوَنَ عَلَى... جَمَّلَ. نَظَّمَ. بَارٌّ, بَرَرَةٌ.

\  \includegraphics[width=1.5311in,height=1.1772in]{images/MuhammadBagauddinprettified-img273.png} 

رَسَا (و) رُسُوٌّ. رَسَتِ السَّفِينَةُ. عَرَبَةُ الأَطْفَالِ.

عَرَبَةٌ (ات). مَرَّةً. قَسَمَ (ى) قَسْمٌ. عَلَى سَوَاءٍ.

نَادَى. لاَعَبَ. لاَعَبَ الْهِرَّةَ. اَكْمَلَ. حَالاً. ضَاحَكَ.

ضَاحَكَ الطِّفْلَ. غَنَّى.

\_\_\_\_\_\_\_\_\_\_\_\_\_\_\_\_\_\_\_\_\_\_\_\_\_\_

عِنْدَمَا يَعُودُ الاَطْفَالُ مِنَ الْمَدْرَسَةِ يَقْلَعُ سَعِيدٌ الأَعْشَابَ الضَّارَّةَ وَ تَكْنِسُ سَلْمَى الْمَنْزِلَ كَمَا يُصْلِحُ مَحْمُدٌ قُفْلَ الْبَابِ, لَقَدْ تَعَاوَنَ الْجَمِيعُ عَلَى تَجْمِيلِ الْمَنْزِلِ وَ تَنْظِيمِ الْحَدِيقَةِ, اِنَّهُمْ أَبْنَاءٌ بَرَرَةٌ. اَلْجَوَّابُ: لَقَدْ نَزَلْتُ فِى أَثْنَاءِ سَفَرِنَا فِى الْبَحْرِ عَلَى جَمِيعِ مَرَافِئِ الْعَالَمِ فَاَنَا اَعْرِفُهَا جَمِيعًا. اَلصَّدِيقُ: اَنْتَ اِذًا تَعْرِفُ الْجُغْرَافِيَّةَ جَيِّدًا. اَلْجَوَّابُ: نَعَمْ, فَقَدْ رَسَوْنَا عَلَيْهَا مَرَّةً لِنَأْخُذَ فَحْمًا. سَأَلَ الْمُعَلِّمُ اَحَدَ التَّلاَمِيذِ فَقَالَ: اُمٌّ لَهَا سَبْعَةُ أَوْلاَدٍ وَ لَيْسَ عِنْدَهَا سِوَى خَمْسِ تُفَّاحَاتٍ فَكَيْفَ تَقْسِمُهَا الاُمُّ بَيْنَ الْجَمِيعِ عَلَى سَوَاءٍ. فَكَّرَ التِّلْمِيذُ قَلِيلاً ثُمَّ أَجَابَ: تَعْمَلُ مِنَ التُّفَّاحَاتِ مُرَبًّى وَ تَقْسِمُهُ. كَانَ وَلَدٌ صَغِيرٌ يَلْعَبُ فِى سَاحَةِ الدَّارِ فَنَادَتْهُ اُمُّهُ وَ قَالَتْ: تَعَالَ لاَعِبْ أُخْتَكَ حَتَّى اُكْمِلَ عَمَلِى فَتَرَكَ اللَّعِبَ وَ ذَهَبَ حَالاً فَوَضَعَ أُخْتَهُ فِى عَرَبَتِهَا وَ خَرَجَ بِهَا اِلَى الْحَدِيقَةِ وَ جَعَلَ يُضَاحِكُهَا وَ يُغَنِّى لَهَا حَتَّى اَكْمَلَتِ الاُمُّ عَمَلَهَا.

\subsubsection{К уроку 79}
С тех пор, как я окончил университет, прошло не менее 10 лет, а я нигде ( فى اى مكان ) не работаю. Когда вы с ним расстались ( افترقتم معهم )? — Вот уже полтора года (منذ عام و نصف). Более 100 студентов провалились на экзаменах в этом году. Какой язык ты преподаёшь в университете? — Я преподаватель арабского языка. Где продаются лекарства? — Лекарства продаются в аптеке. Где лечат больных? — Больных лечат в больнице. У вас одна больница? — У нас три больницы: одна для мужчин, вторая для женщин, третья для детей. До свидания, друг, скорейшего выздоровления! — До свидания, друг, рад буду видеть вас, навещайте ещё раз. Да простит Бог мою мать, она научила меня и читать, и писать, и молиться, и всему (كلّ شيء ) Какая потеря, он умер в расцвете молодости, готовился к поступлению в институт. В вашем институте преподаются восточные языки? — Преподаются, но не все ( لا جميعها ), а большинство их, в том числе и арабский язык. Мусульманин должен быть муджахидом на пути Бога.

\subsection[-١٠٠-]{-١٠٠-}
خَطٌّ، خُطُوطٌ. زَائِِرٌ، زُوَّارٌ. سَافَرَ. حَارِسٌ، حُرَّاسٌ. حِسَابٌ. بَاضَ (ي) بَيْضٌ. بَيْضَةٌ (ات). فِضَّةٌ. كَثَّرَ. عَلَفٌ. اِنْشَقَّ. حَوْصَلَةٌ (ات)، حَوَاصِلُ.

ـــــــــــــــــــــــــ

\ كَانَ وَلَدٌ يَرْسُمُ فَقَالَ لَهُ أَبُوهُ: أَتَقْدِرُ أَنْ تَرْسُمَ قِطَارًا؟ قَالَ الاِبْنُ: نَعَمْ. ثُمَّ رَسَمَ خَطَّيْنِ مُسْتَقِمَيْنِ وَ مَا رَسَمَ قِطَارًا. فَقَالَ الأَبُ: مَا هَذَا؟ قَالَ الاِبْنُ: سِكَّةُ الحَدِيدِ. فَقَالَ الأَبُ: وَ أَيْنَ القِطَارُ؟ قَالَ الاِبْنُ: سَافَرَ.

وَصَلَ زَائِرٌ فِي حَدِيقَةِ الحَيَوَانَاتِ إِلَى قَفَصِ الأُسُودِ فَسَأَلَ الحَارِسَ: هَلْ أَنْتَ حَارِسُ هَذِهِ الأُسُودِ؟ قَالَ: نَعَمْ. قَالَ: أَتَخَافُهَا؟ قَالَ: لاَ. قَالَ: وَ كَيْفَ ذَلِكَ ؟ قَالَ: عِنْدَمَا أَدْخُلُ القَفَصَ تَكُونُ الأُسُودُ فِي قَفَصٍ آخَرَ.

رَأَى رَجُلٌ وَلَدًا صَغِيرًا فَسَأَلَهُ: إِلَى أَيْنَ أَنْتَ ذَاهِبٌ يَا بُنَيَّ؟- إِلَى المَدْرَسَةِ يَا عَمِّي.- مَا ذَا تَتَعَلَّمُ فِيهَا؟- أَتَعَلَّمُ القِرَاءَةَ وَ الكِتَابَةَ وَ الحِسَابَ وَ الرَّسْمَ وَ أَتَعَلَّمُ دُرُوسًا أُخْرَى.- وَ مَاذَا تَعْمَلُونَ فِي أَوْقَاتِ الفَرَاغِ؟- نَخْرُجُ إِلَى المَلْعَبِ فَنَلْعَبُ .

اِمْرَأَةٌ كَانَتْ لَهَا دَجَاجَةٌ تَبِيضُ بَيْضَةَ فِضَّةٍ فَقَالَتْ: إِنْ أَنَا كَثَّرْتُ عَلَفَهَا بَاضَتْ بَيْضَتَيْنِ، فَلَمَّا كَثَّرَتْ عَلَفَهَا اِنْشَقَّتْ حَوْصَلَتُهَا فَمَاتَتْ.

\subsubsection{К уроку 80}
Время докажет мою правоту. Прошлые века доказали правоту Пророка. Ты можешь доказать это? Я доказал ему, что это запрещено. Большим и маленьким государствам необходимо объединиться во всём мире и создать Великую Исламскую державу ( \textstylePolicepardfaut{دولة} ). Кто руководил организацией „Братья-мусульмане"? — Ею руководил Имам Хасан аль-Банна. Кто руководил Исламской революцией в Иране? — Ею руководил Имам Хумайни. Когда произошла Исламская революция в Иране и кто руководил ею? Сегодня мы видим во всех странах мира подъём Ислама. Зажги свет. Погаси свет. Ты играл до сих пор, хватит. , теперь налегай на уроки до сна, а перед сном выйди на свежий воздух и поделай немного зарядку ( \textstylePolicepardfaut{بالتّمَارين الرِّيَاضِية} ) Мы хорошо знаем арабский язык, потому что мы постоянно между собой разговариваем на арабском языке. О, если бы я был молодым, чтобы учиться! Если бы я была мужчиной, чтобы воевать! О, если бы я понимал Kopaн и хадисы!

\subsection{-١٠١-}
\  \includegraphics[width=1.0728in,height=1.0728in]{images/MuhammadBagauddinprettified-img274.png}   \includegraphics[width=1.8437in,height=1.2291in]{images/MuhammadBagauddinprettified-img275.png}   \includegraphics[width=1.1146in,height=1.3126in]{images/MuhammadBagauddinprettified-img276.png} 

صُوصٌ، صِيصَانٌ. حُزْمَةٌ، حُزَمٌ. خُمٌّ، أَخْمَامٌ.\newline
خُمُّ الدَّجَاجِ. سُنْبُلٌ، سَنَابِلُ. اِصْفَرَّ. مُصَفَّرٌ. عَزِيمَةٌ. بِعَزِيمَةٍ. نَشَاطٌ. بِنَشَاطٍ. سَاقٌ، سِيقَانٌ. سَاقُ نَبَاتٍ. دَابَّةٌ، دَوَابُّ. بَيْدَرٌ، بَيَادِرُ. تَعِبَ (ا) تَعَبٌ. أَقْعَى. عَتَبَةٌ (ات). غَلَبَهُ النُّعَاسُ. حَرَسَ (و) حِرَاسَةٌ. يَا حَبِيبِي. كِسْرَةٌ، كِسَرٌ. كَسْرَةُ خُبْزٍ. اِبْتَعَدَ عَنْ... خَبِيثٌ، خُبَثَاءُ. فِي الخَارِجِ. تَرَصَّدَ. عَمِلَ بِـ... عَمِلَ بِالنَّصِيحَةِ. عَمِلَ بِكِتَابِ اللَّهِ. تَسَاءَلَ. فَتَّشَ عَنْ... فِي كُلِّ مَكَانٍ. أَخِيرًا. جَزَاءٌ.

ـــــــــــــــــــــــــ

الصَّيْفُ فَصْلُ حَصَادِ الزَّرْعِ فَالسَّنَابِلُ مُصَفَّرَةٌ وَ هِيَ مَمْلُوءَةٌ قَمْحًا وَ العُمَّالُ يَسْتَيْقِظُونَ بَاكِرًا لِيَقْصِدُوا الحُقُولَ بِعَزِيمَةٍ وَ نَشَاطٍ وَ مَنَاجِلُهُمْ الحَادَّةُ تَقْطَعُ سِيقَانَ السَنَابِلِ وَ بَنَاتُهُمْ يَجْمَعْنَ الحُزَمَ فِي شِبَاكٍ وَ دَوَابُّهُمْ تَنْقُلُهَا إِلَى البَيَادِرِ. 

لاَعَبَ وَلَدٌ صَغِيرٌ كَلْبَهُ أَمَامَ دَارِهِ حَتَّى تَعِبَ فَقَعَدَ عَلَى عَتَبَةِ البَابِ يَسْتَرِيحُ فَغَلَبَهُ النُّعَاسُ فَنَامَ فَجَاءَ الكَلْبُ وَ أَقْعَى بِجَانِبِهِ يَحْرُسُهُ وَ بَعْدَ قَلِيلٍ اِسْتَيْقَظَ الوَلَدُ وَ لَمَّا رَأَى الكَلْبَ مُقْعِيًا بِجَانِبِهِ سَرَّهُ ذَلِكَ كَثِيرًا ثُمَّ دَخَلَ الدَّارَ وَ جَاءَهُ بِكِسْرَةِ خُبْزٍ.

أَوْصَتِ الدَّجَاجَةُ صَغِيرَهَا فَقَالَتْ: "لاَ تَخْرُجْ يَا حَبِيبِي إِلَى الغَابَةِ وَ لاَ تَبْتَعِدْ عَنِ الخُمِّ ِلأَنَّ فِي الخَارِجِ ثَعْلَبًا خَبِيثًا يَتَرَصَّدُكَ" وَ لَكِنَّ الصُّوصُ لَمْ يَعْمَلْ بِنَصِيحَةِ أُمِّهِ وَ لاَ مَكَثَ. وَ لَمَّا عَادَتِ الدَّجَاجَةُ نَادَتْهُ فَمَا سَمِعَتْ صَوْتَهُ وَ تَسَاءَلَتْ: "أَخَرَجَ؟ أَنَسِيَ نَصِيحَتِي؟" وَ فَتَّشَتْ عَنْهُ فِي كُلِّ مَكَانٍ. وَ أَخِيرًا عَثَرَتْ عَلَى شَيْءٍ مِنْ رِيشِهِ فَمَا بَكَتْ وَ لاَ حَزِنَتْ بَلْ قَالَتْ:"هَذَا جَزَاءُ الَّذِي لاَ يَطِيعُ أُمَّهُ".

\subsubsection{К уроку 81}
Я с утра вообще не ел. Он совсем не понимает иностранный язык. Идя в мечеть, я встретил своего друга Умара. Сколько лет было нашему Пророку, когда Бог послал его Посланником? — Ему было тогда 40 лет. — Молодец, Хасан, правильно ( \textstylePolicepardfaut{صَحِيحاً} ) ответил, а теперь садись. Будь сильным, будь здоровым, Бог любит сильного верующего. Бабушка ходит медленно, потому что она слаба после болезни. Дорогая мама, пишу тебе из санатория о своём состоянии и сообщаю тебе, что я прибываю домой через полторы недели, если Богу будет угодно. В каком доме живут шахтёры? — Шахтёры живут в том новом доме. Эти книги положи вместе, а эту книгу положи отдельно. Ты почему здесь без дела сидишь? Не проводи свое время без дела. Откуда я узнаю известия исламского мира? — Для этого ты должен каждый день регулярно читать исламские газеты и журналы или слушать передачи по радио. Не надо ему писать об этом ( ان يكتب اليه هذا), это обеспокоит его и сильно обидит. Не обижай людей. Не обижай соседей.

\subsection[-١٠٢-]{-١٠٢-}
حَفَرَ (ي) حَفْرٌ. دَفَنَ (ي) دَفْنٌ. أَظَلَّ. أَقْشَعَ. أَقْشَعَتِ السَّحَابَةُ. سَحَابَةٌ (ات). عَلاَمَةٌ (ات). أَضَاعَ. مِسْكِينٌ، مَسَاكِينُ. جَسَّ (و) جَسٌّ. نَبْضٌ. جَسَّ النَّبْضَ. قَاسَ (ي) قِيَاسٌ. دَرَجَةُ الحَرَارَةِ. قَاسَ دَرَجَةَ الحَرَارَةِ. اِسْتَطَاعَ. مَرْبُوطٌ. مِنَ الدَّاخِلِ. مَشْغُولُ البَالِ. اِنْتَبَهَ لِـ... غُلاَمٌ، غِلْمَانٌ. لَقَطَ (و) لَقْطٌ. لَحِقَ (ا) لُحُوقٌ، لََحَاقٌ. كَافَأَ. أَبَى (ا) إِبَاءٌ.

ـــــــــــــــــــــــــ

- لِمَاذَا تَحْفِرُ؟

- دَفَنْتُ مَالاً وَ لَسْتُ أَعْرِفُ أَيْنَ دَفَنْتُهُ.

- أَجَعَلْتَ لَهُ عَلاَمَةً؟

- قَدْ فَعَلْتُ.

- وَ مَا هِيَ؟

- سَحَابَةٌ كَانَتْ تُظِلُّنِي وَقْتَ دَفْنِهِ وَ قَدْ أَقْشَعَتْ وَ لَمْ تَبْقَ عَلاَمَةٌ لِلأَرْضِ الَّتِي دَفَنْتُ المَالَ فِيهَا.

- يَا مِسْكِينُ لَقَدْ أََضَعْتَ مَالَكَ.

مَرِضَ طِفْلٌ فَجَاءَ الطَّبِيبُ يَعُودُهُ وَ بَعْدَ أَنْ جَسَّ نَبْضَهُ وَ قَاسَ دَرَجَةَ حَرَارَتِهِ قَالَ لَهُ أَخْرِجْ لِسَانَكَ، أَخْرِجْهُ كُلَّهُ. فَقَالَ الطِّفْلُ: لاَ يَسْتَطِيعُ أَحَدٌ ذَلِكَ. قَالَ الطَّبِيبُ: لِمَاذَا؟ قَالَ الطِّفْلُ: ِلأَنَّهُ مَرْبُوطٌ مِنَ الدَّاخِلِ فَضَحِكَ الطَّبِيبُ وَ سُرَّ بِهِ. 

كَانَ رَجُلٌ يَمْشِي فِي الشَّارِعِ وَ هُوَ مَشْغُولُ البَالِ فَسَقَطَتْ مِحْفَظَةُ نُقُودِهِ وَ لَمْ يَنْتَبِهِ الرَّجُلُ لِذَلِكَ وَ كَانَ وَرَاءَهُ غُلاَمٌ فَرَأَى المِحْفَظَةَ قَدْ سَقَطَتْ فَلَقَطَهَا. رَكَضَ الغُلاَمُ مُسْرِعًا وَرَاءَ الرَّجُلِ حَتَّى لَحِقَهُ وَ قَالَ: هَذِهِ مِحْفَظَتُكَ قَدْ سَقَطَتْ مِنْكَ فَأَخَذَ الرَّجُلُ المِحْفَظَةَ وَ شَكَرَ الغُلاَمَ وَ أَرَادَ أَنْ يُكَافِئَهُ بِشَيْءٍ مِنَ النُّقُودِ لَكِنَّ الغُلاَمُ شَكَرَهُ وَ أَبَى أَنْ يَأْخُذَ شَيْئًا.

\subsubsection{К уроку 82}
Я всю ночь не спал, думал о предстоящих ( \textstylePolicepardfaut{المقبلة} ) экзаменах. Надо проветривать комнаты (менять воздух в комнатах) время от времени. Не спи здесь, здесь сквозняк, заболеешь. До сих пор я была занята на кухне варкой пищи, а теперь освободилась для твоего вопроса (لقضيّتك). Что ты видел во сне? — Сейчас я занят, потом я расскажу тебе, что я видел во сне. Друг мой, помоги мне, я попал в беду. — Пожалуйста, возьми эти деньги, но больше дать не могу. Я не могу (не в силах) поехать туда и вернуться в течение (َ خِلال ) одного дня. Ты был маленьким, а теперь стал большим, молодым, скоро станешь стариком, а потом умрёшь, — это дорога каждого человека. Я там работаю переводчиком. — Ты хорошо переводишь? — Нормально. Улицы селения кривые, а улицы города прямые. Говори шёпотом. Он мне шёпотом сказал: ,,Я тебя люблю". И я ему также ответил шёпотом: ,,Я тебя тоже люблю". Магомед, как только встал с постели, сразу вышел на зарядку. Как только он встретит меня, так сразу начинает задавать мне разные вопросы.

\subsection{-١٠٣-}
جَوْزَةٌ (ات). ظَنَّ (و) ظَنٌّ. عَضَّ (ا) عَضٌّ. مُرٌّ. مَا أَمَرَّ هَذِهِ التُّفَّاحَةَ. أَكْبَرُ مِنْهُ سِنًّا. قَشَرَ (ي، و) قَشْرٌ. قِشْرٌ، قُشُورٌ. لُبٌّ، لُبُوبٌ. أَبْقَى. ضَجِرَ مِنْ ... (ا) ضَجَرٌ. وَحْدَةٌ. ضَجِرَ مِنْ وَحْدَتِهِ. أَلْبَسَ. أَلْبَسَ الطِّفْلَ لِبَاسًا. صَارَ (ي) صَيْرُورَةٌ. صَارَ كَبِيرًا. لَوْزٌ. شَجَرَةُ اللَّوْزِ. مَقْطُوعٌ. تَأَلَّمَ. صَدَقَ (و) صِدْقٌ. آسِفٌ. إِنَّنِي آسِفٌ جِدًّا.

ـــــــــــــــــــــــــ

وَجَدَتْ بِنْتٌ جَوْزَةً خَضَرَاءَ فَظَنَّتْهَا تُفَّاحَةً وَ أَرَادَتْ أَنْ تَأْكُلَهَا فَلَمَّا عَضَّتْهَا وَجَدْتَهَا مُرَّةً فَرَمَتْهَا وَ قَالَتْ: مَا أَمَرَّ هَذِهِ التُّفَّاحَةََ! وَ لَكِنَّ أُخْتَهَا وَ قَدْ كَانَتْ أَكْبَرَ مِنْهَا سِنًّا لَقَطَتِ الجَوْزَةَ وَ قَشَرَتْهَا وَ قَالَتْ: نَعَمْ، إِنَّ قِشْرَ الجَوْزَةِ مُرٌّ لَكِنَّ لُبَّهَا لَذِيذٌ.

كَانَ ِلاِمْرَأَةٍ وَلَدَانِ أَرْسَلَتِ الكَبِيرَ مِنْهُمَا إِلَى المَدْرَسَةِ وَ أَبْقَتِ الصَّغِيرَ فِي البَيْتِ فَضَجِرَ هَذَا يَوْمًا مِنْ وَحْدَتِهِ وَ قَالَ ِلأُمِّهِ: لِمَاذَا لاَ تُرْسِلِينَنِي إِلَى المَدْرَسَةِ فَقَالَتْ لَهُ: أَنْتَ صَغِيرٌ وَ المَدْرَسَةُ لاَ تَقْبَلُ الصِّغَارَ الَّذِينَ فِي سِنِّكَ، فَقَالَ لَهَا: أَلْبِسِينِي ثِيَابَ أَخِي فَأَصِيرَ كَبِيرًا.

أَعْطَى رَجُلٌ وَلَدَهُ فَأْسًا صَغِيرًا فَفَرِحَ بِهَا كَثِيرًا ثُمَّ حَمَلَهَا وَ دَخَلَ البُسْتَانَ وَ قَطَعَ بِهَا شَجَرَةَ لَوْزٍ. وَ كَانَ أَبُوهُ يُحِبُّ تِلْكَ الشَّجَرَةَ ِلأَنَّهُ غَرَسَهَا بِيَدِهِ. وَ فِي اليَوْمِ التَّالِي دَخَلَ الأَبُ البُسْتَانَ فَوَجَدَ شَجَرَةَ اللَّوْزِ مَقْطُوعَةً فَتَأَلَّمَ كَثِيرًا وَ سَأَلَ وَلَدَهُ: مَنْ قَطَعَ الشَّجَرَةَ يَا خَالِدُ؟ فَقَالَ لَهُ خَالِدٌ: يَا أَبِي إِنِّي لاَ أَقْدِرُ أَنْ أَكْذِبَ، أَنَا الَّذِي قَطَعْتُ الشَّجَرَةَ وَ إِنِّي آسِفٌ جِدًّا. فَسُرَّ بِهِ أَبُوهُ ِلأَنَّهُ صَدَقَ.

\subsubsection{К уроку 83}
Я шёл по дороге домой, пошёл дождь и намочил мою одежду. — У тебя не было зонтика? — Зонтик у меня был, но я забыл его в гостях. Почитай хоть немного, я послушаю тебя. Напиши (ж.р.) матери хоть одно письмо, она тоскует по тебе. Задай мне хоть один вопрос из этого урока, я его хорошо выучил. Он зашёл, но вышел, не присев. Мальчик поел утром? — Нет, он пошёл в школу не поев. Это время обеда, не выходи из дома не поев. Какая ( كَمْ) у тебя зарплата? – Каждый месяц я получаю сто рублей. А сколько получает жена? – Она не работает, она занимается домашними делами (بأمور البيت ).

У вас есть в городе огород? — Да, есть. — Значит, вы овощи' не покупаете на рынке, а растут они на вашем огороде. Как только он услышал голос азана, он поспешил в мечеть, чтобы не опоздать на групповой намаз. Арбузы растут на вашем огороде? — Нет, мы арбузы не сажали в этом году, мы посадили дыню вместо арбузов, потому что дыня на рынке дороже арбузов.

\subsection[-١٠٤-]{-١٠٤-}
أَصْفَرُ، صَفْرَاءُ، صُفْرٌ. وَ إِلاَّ فَلاَ. هُنَاكَ. طَرِيقَةٌ، طَرَائِقُ. هُنَاكَ طَرِيقَةٌ. رَسَبَ (و) رُسُوبٌ. طَفَا (و) طَفْوٌ. فَحَصَ (ا) 

\  \includegraphics[width=1in,height=1.1665in]{images/MuhammadBagauddinprettified-img277.png} 

فَحْصٌ. قَلْبٌ، قُلُوبٌ. حَلْقٌ، حُلُوقٌ. أَقْبَلَ. فَاسِدٌ. جِسْمٌ، جُسُومٌ، أَجْسَامٌ. صَحِيحُ الجِسْمِ. شُجَاعٌ، شُجْعَانٌ. هَا هُوَ ذَا. خَجِلَ (ا) خَجَلٌ. أَعْلَى... أَعْلَى الشَّجَرَةِ. تَنَاوَلَ. رَمَى بِحَجَرٍ. يَسُرُّنِي. يَسُرُّنِي أَنِّي مَا رَمَيْتُهُ بِحَجَرٍ. تَسَمَّعَ إِلَى...

ـــــــــــــــــــــــــ

سَأَلَتْ بِنْتٌ أُمَّهَا: كَيْفَ نَعْرِفُ البَيْضَ الجَدِيدَ مِنَ البَيْضِ القَدِيمِ؟ فَقَالَتْ لَهَا أُمُّهَا: ضَعِي البَيْضَةَ بَيْنَ عَيْنَيَكِ وَ النُّورِ فَإِنْ كَانَتْ صَفْرَاءَ فَهِيَ جَدِيدَةٌ وَ إِلاَّ فَلا. وَ هُنَاكَ طَرِيقَةٌ أُخْرَى وَ هِيَ أَنْ يُوضَعَ البَيْضُ فِي المَاءِ فَإِنْ رَسَبَ فَهُوَ جَيِّدٌ وَ إِنْ طَفَا فَهُوَ فَاسِدٌ.

جَاءَ الطَّبِيبُ إِلَى المَدْرَسَةِ فَنَادَى تِلْمِيذًا صَغِيرًا لِيَفْحَصَهُ فَأَقْبَلَ التِّلْمِيذُ مُسْرِعًا وَ وَقَفَ أَمَامَ الطَّبِيبِ فَفَحَصَ الطَّبِيبُ عَيْنَيْهِ وَ فَحَصَ أُذُنَيْهِ وَ حَلْقَهُ و أَنْفَهُ وَ قَلْبَهُ وَ فَحَصَ جِلْدَهُ وَ شَعْرَهُ وَ لَمَّا فَرَغَ الطَّبِيبُ مِنْ عَمَلِهِ قَالَ: هَذَا وَلَدٌ نَظِيفٌ صَحِيحُ الجِسْمِ شُجَاعٌ، اللَّهُ يُحِبُّ أَمْثَالَهُ.

مَشَى ثَلاَثَةُ أُوْلاَدٍ فِي بُسْتَانٍ فَسَمِعُوا عُصْفُورًا يُغَرِّدُ عَلَى شَجَرَةٍ، فَقَالَ أَحَدُهُمْ: مَا أَجْمَلَ صَوْتَهُ أُحِبُّ أَنْ أَرَاهُ. وَ قَالَ ثَانِي: اُنْظُرْ هَا هُوَ ذَا عَلَى ذَلِكَ الغُصْنِ فِي أَعْلَى الشَّجَرَةِ بِجَانِبِ عُشِّهِ. أَمَّا الثَّالِثُ فَتَنَاوَلَ حَجَرًا وَ أَرَادَ أَنْ يَرْمِيَهُ بِهِ فَقَالاَ لَهُ: أَهَذَا جَزَاءُ تَغْرِيدِهِ؟ فَخَجِلَ وَ رَمَى الحَجَرَ عَلَى الأَرْضِ وَ وَقَفَ يَتَسَمَّعَ إِلَى العُصْفُورِ. ثُمَّ قَالَ: مَا أَجْمَلَ صَوْتَهُ! وَ يَسُرُّنِي جِدًّا أَنِّي مَا رَمَيْتُهُ بِحَجَرٍ.

\subsubsection{К уроку 84}
Тебе пора жениться и создавать семью, тебе сейчас 25 лет. Ты служил в армии, вернулся оттуда, окончил институт, стал инженером, чего ещё ( بعد ) ждёшь? — Я ничего не жду, всё это я знаю, я думаю об этом. Я не могу не поблагодарить человека, помогшего мне в беде. Я не могу не говорить ему об этом ( هذا ), когда вижу, чем он занимается. Всем известно, что положение Ислама и мусульман в Советском Союзе теперь изменилось. Известно, что наш учитель занимается обучением детей уже 15 лет. Из его школы вышли десятки учеников ( عشراتٌ من التلامذة ), которые стали после учёными, профес­сорами или инженерами. Поэтому он стал известным, уважаемым человеком и заслуженным учителем. Жители деревень работают весной, летом и осенью, а жители городов работают постоянно. Когда я тебе задал вопрос, ты не ответил на мой вопрос словом, а только кивнул головой. Я не могу не дать (что-либо) ( أن أمنع ), когда кто-то приходит ко мне и просит меня. Я не могу не принять ваше приглашение, когда вы приглашаете меня на торжество.

\subsection[-١٠٥-]{-١٠٥-}
 \includegraphics[width=1.9063in,height=1.6252in]{images/MuhammadBagauddinprettified-img278.png}   \includegraphics[width=2.4063in,height=1.4689in]{images/MuhammadBagauddinprettified-img279.png} 

\ قُفَّةٌ، قُفَفٌ. شَاطِئٌ، شَوَاطِئُ. شَاطِئُ البَحْرِ. رَمْلٌ، رِمَالٌ. نَزَلَ إِلَى النَّهْرِ. خَلَعَ (ا) خَلْعٌ. خَلَعَ ثِيَابَهُ. اِسْتَحَمَّ. هَلْ أَحْبَبْتَ؟ يَظْهَرُ. أَفْهَمَ. مِلْحٌ. مَاءٌ مِلْحٌ. مُفَتِّشٌ. مُفَتِّشُ سِكَّةِ الحَدِيدِ. خَطَرٌ، أَخْطَارٌ. مُعَيَّنٌ. فِي الوَقْتِ المُعَيَّنِ. سَاعَةٌ (ات). قَبَضَ (ي) قَبْضٌ. قَبَضَ الأَجْرَ. أَجْرٌ، أُجُورٌ. مَحَطَّةٌ (ات). تَقَدَّمَ. تَأَخَّرَ. بَائِعٌ، بَاعَةٌ. بَاعَةُُ الصُّحُفِ. رَافَقَ. رَافَقَ وَالِدَهُ إِلَى السُّوقِ. رَطْلٌ، أَرْطَالٌ. حُزْمَةٌ، حُزَمٌ. حُزْمَةٌ مِنَ البَصَلِ. سِلْقٌ. اِمْتَلَأَ. أَفْرَغَ.

ـــــــــــــــــــــــــ

أَخَذَتْ أُمٌّ وَلَدَهَا إِلَى شَاطِئِ البَحْرِ فَجَعَلَ يَلْعَبُ بِالرَّمْلِ ثُمَّ خَلَعَ ثِيَابَهُ وَ نَزَلَ إِلَى المَاءِ. وَ بَعْدَ أَنِ اسْتَحَمَّ قَالَتْ لَهُ: هَلْ أَحْبَبْتَ الاِسْتِحْمَامَ فِي البَحْرِ؟ فَقَالَ الوَلَدُ: نَعَمْ يَا أُمِّي وَ لَكِنْ يَظْهَرُ أَنَّ النَّاسَ رَمَوْا فِيهِ مِلْحًا كَثِيرًا. فَضَحِكَتِ الأُمُّ ثُمَّ أَفْهَمَتْ وَلَدَهَا أَنَّ مِيَاهَ البَحْرِ لاَ تَكُونُ إِلاَّ مِلْحَةً.

قَالَ رَجُلٌ لِمُفَتِّشِ سِكَّةِ الحَدِيدِ: إِنَّ حَيَاتَكُمْ حَيَاةُ أَخْطَارٍ. فَقَالَ المُفَتِّشُ: نَعَمْ، وَ لَكِنَّ فِيهَا أَوْقَاتَ سُرُورٍ. فَقَالَ الرَّجُلُ: أَظُنُّ أَنَّ ذَلِكَ يَكُونُ فِي السَّاعَةِ الَّتِي تَقْبِضُونَ فِيهَا أَجْرَكُمْ. قَالَ المُفَتِّشُ: لاَ ، بَلْ فِي السَّاعَةِ الَّتِي يَصِلُ فِيهَا إِلَى كُلِّ مَحَطَّةٍ فِي الوَقْتِ المُعَيَّنِ لاَ نَتَقَدَّمُ وَ لاَ نَتَأَخَّرُ.

لَمَّا رَجَعَ مُنِيرٌ مِنَ المَدْرَسَةِ قَالَتْ لَهُ أُمُّهُ:"اِذْهَبْ إِلَى أَبِيكَ وَ اطْلُبْ مِنْهُ أَنْ يَشْتَرِيَ اللَّحْمَ وَ الثِّمَارَ وَ الخُضَرَ" رَافَقَ مُنِيرٌ وَالِدَهُ إِلَى السُّوقِ فَسَمِعَهُ يَقُولُ لِلْجَزَّارِ "أَعْطِنِي رَطْلَ لَحْمٍ وَ لِبَائِعِ الخُضَرِ وَ الثِّمَارِ بِعْنِي حُزْمَةً مِنَ البَصَلِ وَ حُزْمَةً مِنَ السِّلْقِ وَ رَطْلاً مِنَ التُّفَّاحِ" وَ لَمَّا امْتَلَأَتِ القُفَّةُ رَجَعَ مُنِيرٌ إِلَى المَنْزِلِ وَ قَالَ ِلأُمِّهِ: خُذِي السَّلَّةَ وَ أَفْرِغِيهَا ِلأَنَّ أَبِي يَحْتَاجُ إِلَيْهَا.

\subsubsection{К уроку 85}
Многие русские солдаты в Афганистане перешли к муджахидам и приняли Ислам. Это счастливые ребята. Всевышний Бог принимает любого человека ( أىّ رجل), принявшего Ислам, и прощает ему. Где книга, которую я только что принёс из библиотеки? Кто тебе спас жизнь? Тебе не следует забывать его. Когда ты посетишь меня? Ты давно не посещал меня ( مِنْ زَمَانٍ ). — Если Богу будет угодно, сегодня вечером. Ты придёшь один или со своей женой? — Я приду один, жене некогда ( ليس عندها وقت ). У нас сегодня в аудитории лекция, её читает профессор нашего института. О чем лекция? — Лекция о распространении Ислама среди молодёжи мира. Ты боишься выходить из дома ночью? — Не боюсь. — Правильно (صحيح), не надо бояться. Мусульманин не боится ничего, кроме Бога. Всевышний Бог всегда помогает в беде, спасает от неё. Кажется, ты забыл приказ Бога (أمر). Не забудь, что ты сегодня дежурный, готовься всё делать дома; пищу готовить, комнаты подметать, воду носить, стол чистить, убирать и, наконец (أخِيرًا), пол мыть. Мама моет пол, а я приношу ей воду.

\subsection{-١٠٦-}
عِصَابَةٌ (ات). دَوَّرَ. دَفَعَ (ا) دَفْعٌ. رَقَصَ (و) رَقْصٌ. صَفَّقَ. زِدْ. زِدِ اقْتَرِبْ. لَمْ تُصِبْ. هَيَّا أَمْسِكْنَا. اِلْتَحَقَ بِـ... مَدَّ (و) مَدٌّ. مَدَّ يَدَهُ. تَحَسَّسَ. زَلْزَلَةٌ. زُلْزِلَتِ الأَرْضُ. فَائِدَةٌ، فَوَائِدُ. بِدُونِ فَائِدَةٍ. بَيْنَمَا... كَانَ الفَصْلُ صَيْفًا. تَزَاحَمَ. تَصَدَّعَ. جِدَارٌ، جُدْرَانٌ. تَصَدَّعَتِ الجُدْرَانُ. تَسَاقَطَ. سَقْفٌ، سُقُوفٌ. أَذًى. مَا أُصِيبَ أَحَدٌ بِأَذًى. مَدْرَسِيٌّ. الكِتَابُ المَدْرَسِيُّ. خِلاَلَ الدَّرْسِ. حِجَارَةٌ.

ـــــــــــــــــــــــــ

اِجْتَمَعَ الأَطْفَالُ فِي السَّاحَةِ وَ أَحْضَرُوا عِصَابَةً وَ رَبَطُوهَا عَلَى عَيْنَيْ رَشِيدٍ ثُمَّ دَوَّرُوهُ وَ دَفَعُوهُ بَعِيدًا عَنْهُمْ وَ أَخَذُوا يُغَنُّونُ وَ يَرْقُصُونَ وَ يُصَفِّقُونَ حَوْلَهُ وَ هُمْ يُرَدِّدُونَ:"نَحْنُ هُنَا ، اِقْتَرِبْ! زِدْ اقْتَرِبْ! لَمْ تُصِبْ! هَيَّا أَمْسِكْنَا! اِلْتَحِقْ بِنَا" وَ كَانَ المِسْكِينُ يَمُدُّ يَدَيْهِ وَ يَتَحَسَّسُ وَ لَكِنْ بِدُونِ فَائِدَةٍ.

بَيْنَمَا كَانَ التَّلاَمِيذُ فِي غُرَفِ الدَّرْسِ وَ كَانَ الفَصْلُ صَيْفًا زُلْزِلَتِ الأَرْضُ. فَتَرَكَ التَّلاَمِيذُ كُلَّ شَيْءٍ وَ خَرَجُوا إِلَى سَاحَةِ المَدْرَسِةِ فِي نِظَامٍ فَمَا تَزَاحَمُوا وَ لاَ تَأَخَّرَ أَحَدٌ. وَ كَانَتِ الزَّلْزَلَةُ شَدِيدَةً جِدًّا فَتَصَدَّعَتِ الجُدْرَانُ وَ تَسَاقَطَتِ الحِجَارَةُ مِنَ السُّقُوفِ وَ لَكِنْ مَا أُصِيبَ أَحَدٌ بِأَذًى.

خِلاَلَ دُرُوسِ اللُّغَةِ العَرَبِيَّةِ يَكْتُبُ المُعَلِّمُ جُمَلاً عَرَبِيَّةً عَلَى اللَّوْحِ وَ يَسْأَلُ الطُّلاَّبَ: أَفَهِمْتُمُوهَا؟ فَيَقُولُونَ لَهُ: نَعَمْ، فَهِمْنَاهَا. ثُمَّ يَقْرَأُ الطُّلاَّبُ جُمَلاً عَرَبِيَّةً مِنَ الكِتَابِ المَدْرَسَيِّ وَ يُتَرْجِمُونَهَا إِلَى لُغَتِهِمْ. بَعْدَ الدُّرُوسِ يَذْهَبُ الزُّمَلاَءُ إِِلَى بُيُوتِهِمْ أَمَّا مُحَمَّدٌ وَ مَحْمُودٌ فَلاَ يَذْهَبَانِ إِلَى بُيُوتِهِمْ، عِنْدَهُمَا أَعْمَالٌ كَثِيرَةٌ فِي الجَامِعَةِ.

\subsubsection{К уроку 86}
Ты уже не маленький, не подобает тебе говорить такие вещи. Он уже не ребёнок, и не подобает ему тратить всё своё время на игру и развлечения, он уже достиг 15 лет. Счастливый на этом и на том свете — это тот, кто принял Ислам и отверг неверие. Ислам — наилучшая небесная религия. Ислам — наилучшая религия ( خير الأديان). В Исламе есть всё, в чём нуждается человек. Ислам – это религия знания, религия справедливости и религия джихада. Людям надо изучать Ислам, чтобы хорошо понять его. Ты мне ближе его. Он мне милее тебя. Самая любимая пища для меня — это каша с молоком и кипячёное молоко. Когда мусульманин бывает настоящим? — Мусульманин бывает настоящим, когда все работы он выполняет во имя Ислама и ради Ислама. Я беру только эту книгу, потому что эта книга самая любимая для меня, все остальные книги я оставляю тебе. Юсуф был самым любимым сыном своего отца. Где бы ты ни находился, я тебя не забуду.

\subsection{-١٠٧-}
مَرَّ بِالجِبَلِ. أَتُرِيدُ مِنِّي؟ عَارٍ، عُرَاةٌ. مَشَى عَارِيًا. تَبَارَى. فِرْقَةٌ، فِرَقٌ. فِرَقُ الأَلْعَابِ الرِّيَاضِيَّةِ. جُمْهُورٌ، جَمَاهِيرُ. إِبْرَةٌ، إِبَرٌ.\newline
عَيْنُ الإِبْرَةِ. قَمِيصٌ، قُمْصَانٌ، أَقْمِصَةٌ. جَانِبٌ، جَوَانِبُ. حَكَمٌ. صَفَرَ (ي) صَفِيرٌ. ابْتَدَأَ. اِبْتَدَأَ اللَّعِبَ. حَاوَلَ جُهْدَهُ. تَسَاوَى. تَسَاوَتِ الفِرْقَتَانِ. إِصَابَةُ الأَهْدَافِ. اِدَّعَى. هَتَفَ (ي) هُتَافٌ. عَجَزَ عَنْ... (ي) عَجْزٌ. خَلِيقَةٌ، خَلاَئِقُ. هَاتِ. أُنْبُوبَةٌ، أَنَابِيبُ. قَامَ عَلَى قَدَمَيْهِ. جَعَلَ يَصِيحُ. مَهَارَةٌ. قَامَةٌ (ات). مِنْ قَامَتِهِ. أَضَاعَ. عَبَثًا. أَفَادَ. عَمَلٌ لاَ يُفِيدُ.

\begin{center}
\includegraphics[width=1.5in,height=0.8209in]{images/MuhammadBagauddinprettified-img280.png}
\end{center}
ـــــــــــــــــــــــــ

كَانَ بَائِعٌ يَحْمِلُ حَقَائِبَ كَثِيرَةً وَ يَقِفُ فِي السُّوقِ وَ يُنَادِي: حَقَائِبُ جَمِيلَةٌ، حَقَائِبُ جَمِيلَةٌ، اِشْتَرُوهَا. فَمَرَّ بِهِ رَجُلٌ فَقَالَ البَائِعُ لِلرَّجُلِ: اِشْتَرِ حَقِيبَةً. فَقَالَ الرَّجُلُ: مَاذَا أَفْعَلُ بِهَا؟ قَالَ البَائِعُ: تَضَعُ فِيهَا ثِيَابَكَ. فَقَالَ الرَّجُلُ: لَيْسَ لِي غَيْرَ ثِيَابِي الَّتِي أَلْبَسُهَا، أَتُرِيدُ مِنِّي أَنْ أَمْشِيَ عَارِيًا. 

تَبَارَتْ فِرْقَتَانِ مِنْ فِرَقِ الأَلْعَابِ الرِّيَاضِيَّةِ بِكُرَةِ القَدَمِ. وَ كَانَتْ قُمْصَانُ الفِرْقَةِ الأُولَى حَمْرَاءَ وَ قُمْصَانُ الثَّانِيَةِ بَيْضَاءَ. وَ قَدْ وَقَفَ جُمْهُورٌ كَبِيرٌ عَلَى جَوَانِبِ المَلْعَبِ يَتَفَرَّجُ. وَ لَمَّا صَفَرَ الحَكَمُ اِبْتَدَأَتَا اللَّعِبَ. وَ كَانَتْ كُلُّ فِرْقَةٍ مِنَ الفِرْقَتَيْنِ تُحَاوِلُ جُهْدَهَا أَنْ تَغْلِبَ الأُخْرَى. وَ لَكِنَّهُمَا تَسَاوَتَا فِي إِصَابَةِ الأَهْدَافِ فَهَتَفَ الجُمْهُورُ لَهُمَا.

اِدَّعَى رَجُلٌ أَنَّهُ يَصْنَعُ مَا يَعْجِزُ الخَلاَئِقُ عَنْهُ. فَاجْتَمَعَ النَّاسُ حَوْلَهُ وَ قَالُوا: هَاتِ. فَأَخْرَجَ أُنْبُوبَةً فَصَبَّ مِنْهَا إِبَرًا عِدَّةً، ثُمَّ وَضَعَ إِبْرَةً فِي الأَرْضِ وَ قَامَ عَلَى قَدَمَيْهِ وَ جَعَلَ يِرْمِي إِبْرَةً مِنْ قَامَتِهِ فَتَقَعُ كُلُّ إِبْرَةٍ فِي عَيْنِ الأُخْرَى فَتَعَجَّبُوا مِنْ مَهَارَتِهِ وَ لَكِنَّهُمْ قَالُوا: نَعَمْ، إِنَّ فِي عَمَلِكَ هَذَا مَهَارَةً وَ لَكِنَّهَا مَهَارَةٌ لاَ تُفِيدُكَ وَ لاَ تُفِيدُ أَحَدًا فَأَضَعْتَ وَقْتَكَ عَبَثًا.

\subsubsection{К уроку 87}
Читальный зал представляет собой просторную, светлую комнату, в которой столы и стулья, и на столах электролампы. Он денежный человек. (Вы), денежные люди, расходуйте свои деньги на процветание Ислама. Мусульманин всегда стремится к поиску знания, потому что он знает, что Ислам — религия знания. Мусульманин, где бы он ни находился, защищает Ислам и всё свое имущество во имя Ислама-Мы воюем с гяурами, и гяуры воюют с нами постоянно. Сегодня они обладатели силы и обладатели многочисленного оружия, но мы -обладатели имана, вот поэтому мы непременно победим, у нас Бог, а у них — сатана. Бог всемогущ, а сатана слаб. Исламский мальчик, уважай свою мать, люби её и слушайся её. Кто тебя 9 месяцев носил в чреве (животе)? Кто тебя грудью кормил? Кто рядом с тобой ночами не спал? — Всё это делала моя мать, поэтому я люблю её и слушаюсь её. Шёлковая одежда запрещена мужчинам. Почему ты над ним издеваешься? Не издевайся. Мусульманин никогда не издевается над человеком, а наоборот, уважает человека, — так приказал ему Бог. Где мусульмане собираются для пятничного намаза? — Они собираются в соборной мечети.

\subsection{-١٠٨-}
طَرِيقُ السَّيَّارَاتِ. خَطِرٌ. صَارَ يَلْعَبُ. أَحَدُهُمْ. أَحَدُنَا. لَزِمَ (ا) لُزُومٌ. لَزِمَ جَانِبَ الطَّرِيقِ. آتٍ. شِمَالٌ. اِلْتَفَتَ شِمَالاً. وَقَفَ. قَدَمٌ، أَقْدَامٌ. رَيْثَمَا... عَبَرَ (و) عُبُورٌ. فَرَكَ (و) فَرْكٌ. مَشَّطَ. مَشَّطَ شَعَرَهُ. مِنْدِيلٌ، مَنَادِيلُ. حَافَظَ عَلَى النَّظَافَةِ. مَرَّنَ. يَرُوحُ وَ يَجِيءُ. كَانَ يَرُوحُ وَ يَجِيءُ. أَعْمَالُ البَيْتِ. كَوَى (ي) كَيٌّ. كَوَى الثِّيَابَ. عَلَفَ (ي) عَلْفٌ. عَلَفَ الدَّجَاجَ. صَاحِبَةٌ، صَوَاحِبُ. فِي سُرْعَةٍ.

\begin{center}
\includegraphics[width=1.311in,height=0.9055in]{images/MuhammadBagauddinprettified-img281.png}
\end{center}
ـــــــــــــــــــــــــ

كَانَتْ قَرْيَةٌ عَلَى طَرِيقِ السَّيَّارَاتِ فَرَأَى أَوْلاَدُ القَرْيَةِ أَنَّ اللَّعِبَ فِي الطَّرِيقِ خَطِرٌ فَصَارُوا يَلْعَبُونَ فِي الحُقَولِ. وَ كَانَ أَحَدُهُمْ إِذَا مَشَى فِي الطَّرِيقِ لَزِمَ جَانِبًا، وَ إَذَا أَرَادَ أَنْ يَنْتَقِلَ مِنْ جَانِبٍ إِلَى جَانِبٍ اِلْتَفَتَ يَمِينًا وَ شِمَالاً فَإِذَا رَأَى سَيَّارَةً آتِيَةً مِنْ بَعِيدٍ وَقَفَ عَلَى جَانِبِ الطَّرِيقِ رَيْثَمَا تَعْبُرُ. 

الوَلَدُ النَّظِيفُ يَغْسِلُ رَأْسَهُ كُلَّ يَوْمٍ. الوَلَدُ النَّظِيفُ يَفْرُكُ رَأْسَهُ وَ يُمَشِّطُ شَعَرَهُ. الوَلَدُ النَّظِيفُ يُقَلِّمُ أَظْفَارَهُ. الوَلَدُ النَّظِيفُ يَلْبَسُ ثِيَابًا نَظِيفَةً. الوَلَدُ النَّظِيفُ يَحْمِلُ مِنْدِيلاً نَظِيفًا. الوَلَدُ النَّظِيفُ يُحَافِظُ عَلَى نَظَافَةِ ثِيَابِهِ وَ كُتُبِهِ وَ دَفَاتِرِهِ وَ مَكْتَبَتِهِ وَ مَدْرَسَتِهِ.

البِنْتُ النَّشِيطَةُ تَسْتَيْقِظُ بَاكِرًا. البِنْتُ النَّشِيطَةُ تُمَرِّنُ جِسْمَهَا مِنَ الرَّأْسِ إِلَى القَدَمِ كُلَّ يَوْمٍ. البِنْتُ النَّشِيطَةُ تَرُوحُ وَ تَجِيءُ فِي سُرْعَةٍ. البِنْتُ النَّشِيطَةُ تَدْرُسُ فِي أَوْقَاتِ الدَّرْسِ وَ تَلْعَبُ مَعَ صَوَاحِبِهَا فِي أَوْقَاتِ اللَّعِبِ. البِنْتُ النَّشِيطَةُ تُسَاعِدُ أُمَّهَا فِي كُلِّ أَعْمَالِ البَيْتِ: تَكْنُسُ البَيْتَ، وَ تَغْسِلُ الصُّحُونَ، وَ تَكْوِي الثِّيَابَ وَ تَسْقِى الحَدِيقَةَ وَ تَعْلِفُ الدَّجَاجَ.

\subsubsection{К уроку 88}
Наступила весна, и люди со своими тракторами и плугами выходят на поля, чтобы пахать и сеять. Хватит быть неблагодарным Богу! Дары Бога нам неисчислимы: пища — дар Бога, вода — дар Бога, сон — дар Бога, есть и другие дары, и мы должны благодарить Его за ( على ) эти дары и восхвалять Его. Верно ты говоришь, отныне мы должны готовиться к экзаменам. Нам надо встать на заре, чтобы не опоздать на поезд. Мы с нетерпением ждём вашего ответа. Мы с нетерпением ждём возвращения сына из армии. У тебя есть велосипед? — У меня ещё нет велосипеда, но отец сказал, что купит для меня велосипед, когда я окончу школу. Рашид сел на мотоцикл и уехал в лес. Я хочу продолжить учебу, а не прерывать её. До нас дошло, что ты болен, и мы сильно огорчились. Потом дошло до нас, что ты выздоровел спустя немного времени, и мы очень обрадовались. Когда вы начали строить мечеть? — Мы начали строить мечеть сразу после собрания. По какому вопросу ( بشأن أىّ شىءٍ ) созвано собрание? — Собрание было созвано по вопросу построик мечети.

\subsection[-١٠٩-]{-١٠٩-}
\begin{center}
\includegraphics[width=1.3752in,height=0.9055in]{images/MuhammadBagauddinprettified-img282.png}
\end{center}
أَبْلَهُ، بَلْهَاءُ، بُلْهٌ. القَاهِرَةُ. اليَابَانُ. تَفَكَّرَ. سَافَرَ. سَافَرَ بِالقِطَارِ. اِتَّبَعَ. اِتَّبَعَ طَرِيقًا. سَفِينَةٌ، سُفُنٌ، سَفَائِنُ. السُّوَيْسُ. رُبَّانٌ، رَبَابِنَةٌ. رُبَّانُ السَّفِينَةِ. رَصِيفٌ، رُصُفٌ، أَرْصِفَةٌ. مِرَارًا. كُلَّمَا. كُلَّمَا أَرَادَ أَنْ يَخْرُجَ. أَخَذَ بِيَدِهِ. عَابِرٌ. سَيَّارَةٌ عَابِرَةٌ. عَلَى يَمِينِهِ. عَلَى يَسَارِهِ. حَرَكَةٌ (ات). حَرَكَةُ السَّيَّارَاتِ. أَوْصَلَ. دَفْتَرُ الإِمْلاَءِ. خَطٌّ، خُطُوطٌ. وَاضِحٌ. خَطٌّ وَاضِحٌ. أَثْنَى عَلَى... كَرِيمُ الأَخْلاَقِ. بَعْدَ أَنْ...

ـــــــــــــــــــــــــ

سَأَلَ مُدَرِّسُ الجُغْرَافِيَا أَحَدَ التَّلاَمِيذِ البُلْهِ: مَا هُوَ الطَّرِيقُ الَّذِي تَتَّبِعُهُ لِلسَّفَرِ مِنَ القَاهِرَةِ إِلَى اليَابَانِ؟ فَأَجَابَهُ التِّلْمِيذُ بَعْدَ أَنْ تَفَكَّرَ طَوِيلاً: نُسَافِرُ بِالقِطَارِ مِنَ القَاهِرَةِ إِلَى السُّوَيْسِ، وَ مِنَ السُّوَيْسِ نَرْكَبُ السَّفِينَةَ وَ نَتْرُكُ الأَمْرَ إِلَى رُبَّانِ السَّفِينَةِ الَّذِي سَافَرَ مِرَارًا فِي هَذَا الطَّرِيقِ وَ هُوَ يَعْرِفُهُ جَيِّدًا.

أَرَادَ شَيْخٌ ضَعِيفٌ أَنْ يَمُرَّ فِي الشَّارِعِ مِنْ رَصِيفٍ إِلَى رَصِيفٍ وَ لَكِنَّهُ لَمْ يَقْدِرْ ِلأَنَّ السَّيَّارَاتِ كَانَتْ تَرُوحُ وَ تَجِيءُ فِي سُرْعَةٍ. وَ كَانَ كُلَّمَا أَرَادَ أَنْ يَمُرَّ يَرَى سَيَّارَةً عَابِرَةً عَلَى يِمِينِهِ وَ يَرَى سَيَّارَةً أُخْرَى عَابِرَةً عَلَى يَسَارِهِ فَيَرْجِعُ إِلَى مَكَانِهِ. فَرَآهُ الشُّرْطِيُّ فَرَفَعَ يَدَهُ فَوَقَفَتْ حَرَكَةُ السَّيَّارَاتِ وَ أَخَذَ الشُّرْطِيُّ بِيَدِ الشَّيْخِ وَ أَوْصَلَهُ إِلَى الرَّصِيفِ الآخَرِ.

دَخَلَ الأُسْتَاذُ غُرْفَةَ الدَّرْسِ وَ طَلَبَ مِنْ أَحَدِ التَّلاَمِيذِ دَفْتَرَ الإِمْلاَءِ فَوَجَدَ الدَّفْتَرَ نَظِيفًا وَ وَجَدَ الخَطَّ فِيهِ وَاضِحًا جَمِيلاً وَ وَجَدَ الأَغْلاَطَ فِيهِ قَلِيلَةً فَأَثْنَى عَلَى التِّلْمِيذِ. فَقَالَ التِّلْمِيذُ: يَا أُسْتَاذُ! دَفْتَرُ جَارِي أَحْسَنُ مِنْ دَفْتَرِي فَسُرَّ الأُسْتَاذُ بِهِ وَ أَثْنَى عَلَيْهِ مَرَّةً ثَانِيَةً وَ قَالَ: إِنَّكَ يَا نَبِيلُ كَرِيمُ الأَخْلاَقِ.

\subsubsection[К уроку 89]{К уроку 89}
Он на русском языке говорит свободно, потому что он уже три года изучает русский язык. Давай поиграем в шашки или в шахматы. — Нет, у меня нет времени. Хочешь пить? — Да, хочу. Хочешь поку­шать? — Нет, я сыт, я недавно ел. Какая сегодня погода? — Погода сегодня хорошая (جميل) лучше, чем была вчера. Кто наш Бог? — Наш Бог — Бог. Кто нас оживляет? Кто нас умерщвляет? Кто нас кормит? Кто нам здоровье даёт? — Это Бог ( الله هو الّذِى ) нас кормит и оживляет, и умерщвляет, и даёт нам здоровье. Береги здоровье в молодости (فى شبابك ), оно тебе пригодится (فسوف ينفعك ). Чтение книг полезно. Приступайте, братья , к чтению. После чтения погуляйте в саду. Берегите свое здоровье, девушки. Ребята, соберитесь сегодня в мечети слушать проповедь. Кто подчиняется Богу, того Бог введет в рай, а кто ослушается Его, того введёт в огонь. Это дело не удовлетворяет Бога, оставь его и делай другое дело, удовлетворяющее Бога. Призови к Исламу всех, кто вокруг тебя и докажи им, что нет им счастья без Ислама.

\subsection{-١١٠-}
 \includegraphics[width=0.4374in,height=0.6252in]{images/MuhammadBagauddinprettified-img283.png}   \includegraphics[width=0.9689in,height=0.5in]{images/MuhammadBagauddinprettified-img284.png}   \includegraphics[width=1.7917in,height=1.2709in]{images/MuhammadBagauddinprettified-img285.png} 

قَمْحَةٌ. نَمْلَةٌ. قَرْيَةُ النَّمْلِ. صِنْفٌ، أَصْنَافٌ. \newline
رِزْقٌ، أَرْزَاقٌ. طَلَبَ الرِّزْقَ. سَعَى (ا) سَعْيٌ. سَعَى فِي طَلَبِ الرِّزْقِ. سَعَى فِي طَلَبِ العِلْمِ. رَفِيقَةٌ (ات). عَائِلَةٌ، (ات)، عَوَائِلُ. قَرْنٌ، قُرُونٌ \newline
حَكَّ (و) حَكٌّ. حَكَّ يَدَهُ بِشَيْءٍ. كَأَنَّ... كَأَنَّهُ يَقُولُ. بَرِّيَّةٌ، بَرَارِيُّ. فَرَشَ (و) فَرْشٌ. زُرْبِيَّةٌ، زَرَابِيُّ. تَعَهَّدَ بِـ... طَرَفٌ، أَطْرَافٌ. حَطَبٌ، أَحْطَابٌ. أَوْقَدَ. نَضِجَ (ا) نَضْجٌ. نَضِجَ الغَدَاءُ. شَهِيَّةٌ. أَكَلَ بِشَهِيَّةٍ. تَمَدَّدَ. تَمَدَّدَ لِلرَّاحَةِ. دَبَشٌ، أَدْبَاشٌ. جَمَعَ أَدْبَاشَهُ. تَمَتَّعَ بِـ... نُزْهَةٌ. تَمَتَّعَ بِالنُّزْهَةِ. قِرْشٌ، قُرُوشٌ. ثَمَنٌ، أَثْمَانٌ. كَمْ ثَمَنُهُ؟

\begin{center}
\includegraphics[width=1.1252in,height=1.0547in]{images/MuhammadBagauddinprettified-img286.png}
\end{center}
ـــــــــــــــــــــــــ

كَانَتْ نَمْلَةٌ تَسْعَى فِي طَلَبِ الرِّزْقِ فَلَقِيَتْ فِي طَرِيقِهَا قَمْحَةً فَحَاوَلَتْ أَنْ تَنْقُلَهَا إِلَى قَرْيَتِهَا فَلَمْ تَقْدِرْ ِلأَنَّ القَمْحَةَ كَانَتْ أَثْقَلَ مِنْهَا. وَ بَعْدَ قَلِيلٍ رَأَتْ نَمْلَتَيْنِ مِنْ رَفِيقَاتِهَا فَأَسْرَعَتْ إِلَيْهِمَا وَ حَكَّتْ قَرْنَهَا بِقَرْنَيْهِمَا كَأَنَّهَا تَقُولُ لَهُمَا: سَاعِدَانِي . فَذَهَبَتِ النَّمْلَتَانِ مَعَهَا إِلَى مَكَانِ القَمْحَةِ فَأَمْسَكْنَ بِهَا مِنْ جَمِيعِ أَطْرَافِهَا وَ نَقَلْنَهَا إِلَى القَرْيَةِ.

خَرَجَتِ العَائِلَةُ إِلَى البَرِّيَّةِ فَفَرَشَتْ زَيْنَبُ زُرْبِيَّةً عَلَى العُشْبِ الأَخْضَرِ وَ تَعَهَّدَ كُلُّ فَرْدٍ بِأَدَاءِ عَمَلٍ. جَمَعَ الأَبُ الحَطَبَ، وَ أَوْقَضَتِ البِنْتُ النَّارَ، وَ أَحْضَرَ الوَلَدَانِ المَاءَ، وَ طَبَخَتِ الأُمُّ الطَّعَامَ. وَ لَمَّا نَضِجَ الغَدَاءُ أَكَلُوا بِشَهِيَّةٍ ثُمَّ تَمَدَّدُوا لِلرَّاحَةِ. وَ لَمَّا دَخَلَ المَسَاءُ جَمَعُوا أَدْبَاشَهُمْ وَ عَادُوا إِلَى مَنْزِلِهِمْ مُتَمَتِّعِينَ بِهَذِهِ النُّزْهَةِ.

ذَهَبْتُ إِلَى مَكْتَبَةٍ مِنَ المَكْتَبَاتِ فَسَأَلْتُ البَائِعَ: هَلْ عِنْدَكَ كُرَّاسَاتٌ؟ قَالَ: نَعَمْ، عِنْدِي مِنْهَا أَصْنَافٌ كَثِيرَةٌ. قُلْتُ: وَ كَمْ قِرْشًا ثَمَنُ الكُرَّاسَةِ الجَيِّدَةِ؟ قَالَ: ثَمَنُهَا قِرْشَانِ. قَلْتُ وَ هَلْ عَنْدَكَ أَقْلاَمٌ؟ قَالَ: لاَ، لَيْسَ عِنْدِي اليَوْمَ أَقْلاَمٌ. قُلْتُ: وَ مَتَى تُحْضِرُهَا؟ قَالَ: بَعْدَ يَوْمَيْنِ إِنْ شَاءَ اللَّهُ. فَاشْتَرَيْتُ ثَلاَثَ كُرَّاسَاتٍ وَ انْصَرَفْتُ.

\subsubsection{К уроку 90}
Наш профессор хорошо знает историю арабов с древнейших времён до сегодняшнего дня. Эти вопросы лёгкие, поэтому ты должен отвечать на них устно, а эти чуть-чуть ( بقليل ) труднее, на них можешь отвечать письменно. Где Сайд? — Думаю, спит. — Так рано? — Он, обычно, так рано спит, зато и рано встаёт. На какой странице была та фра- за? — Забыл, я давно не заглядывал в книгу. Пойдём в столовую, я голодный. Он должен хорошо подумать и потом ответить. Ты должен перевести это выражение в течение ( خلال ) 10 минут. Он пошёл, чтобы напиться воды. Студент начал учить свои уроки. Он уже начал говорить по-арабски. Крестьяне начали возвращаться с поля. Сегодня мы не хотим возвращаться домой. Она открыла ему дверь, и он вошёл с приветом. Учитель объяснил ученику урок, но тот (و لكنه ) оборачивался туда-сюда, как будто это его не касается. Бог не меняет положение людей, пока люди самиأنفسهم)) не меняют своё положение. Вот тогда Бог помогает им. Да поможет Бог нам ( نصرنا الله على ) против врагов Ислама!

\subsection[-١١١-]{-١١١-}
 \includegraphics[width=1.1354in,height=1.5626in]{images/MuhammadBagauddinprettified-img287.png}   \includegraphics[width=2.1563in,height=1.3646in]{images/MuhammadBagauddinprettified-img288.png} 

حَدَّادٌ (ون). لاَعِبٌ (ون). مَدْرَجٌ، مَدَارِجُ. قَالَ لِنَفْسِهِ. عَجَبًا. صَوْتٌ عَالٍ. خَفِيٌّ. صَوْتٌ خَفِيٌّ. عَاوَنَ. مَضَغَ (ا) مَضْغٌ. مَضَغَ الطَّعَامَ. بَعْدَ بُرْهَةٍ مِنَ الزَّمَانِ. قَدِمَ (ا) قُدُومٌ. قَدِمَ المُسَافِرُ. وَاحِدًا إِثْرَ وَاحِدٍ. مُمَرِّنٌ (ون). مُقَابَلَةٌ (ات). اِبْتَدَأَتِ المُقَابَلَةُ. مُتَفَرِّجٌ (ون). فِنَاءٌ، أَفْنِيَةٌ. مُبَارَاةٌ، مُبَارَيَاتٌ. شَيِّقٌ. مُبَارَاةٌ شَيِّقَةٌ. تَقَاتَلَ. غَالِبٌ. مَغْلُوبٌ. فِنَاءُ الدَّارِ. مَضَى (ي) مُضِيٌّ. مِنْ سَاعَتِهِ. مِنْ لَيْلَتِهِ. مَأْوًى، مَآوٍ. صَعِدَ (ا) صُعُودٌ. صَعِدَ الجَبَلَ. جَارِحَةٌ، جَوَارِحُ. بَصُرَ بِـ... (و) بَصَرٌ. صَفَّقَ. اِنْقَضَّ عَلَى... جِسْرٌ، جُسُورٌ. ظِلٌّ، ظِلاَلٌ. اِخْتَطَفَ.

ـــــــــــــــــــــــــ

كَانَ لِحَدَّادٍ كَلْبٌ وَ قَدْ تَعَوَّدَ الكَلْبُ أَنْ يَنَامَ وَ الحَدَّادُ يَشْتَغِلُ فَإِذَا اسْتَرَاحَ الحَدَّادُ وَ قَعَدَ لِيَأْكُلَ اِسْتَيْقَظَ الكَلْبُ. فَتَعَجَّبَ الحَدَّادُ وَ قَالَ لِنَفْسِهِ: عَجَبًا! صَوْتُ المَطَارِقِ العَالِي لا بَسْمَعُهُ وَ أَمَّا صَوْتُ المَضْغِ الخَفِيُّ فَيَسْمَعُهُ وَ يُوقِظُهُ.

فَتَحَ الحَارِسُ بَابَ المَلْعَبِ وَ قَدْ عَاوَنَهُ وَلَدَاهُ وَ بَعْدَ بُرْهَةٍ قَدِمَ الاَّعِبُونَ وَاحِدًا إِثْرَ وَاحِدٍ ثُمَّ أَقْبَلَ المُمَرِّنُ فِي سَيَّارَتِهِ الصَّغِيرَةِ. وَ لَمَّا حَانَ الوَقْتُ وَ اِمْتَلَأَتِ المَدَارِسُ صَفَرَ الحَكَمُ وَ اِبْتَدَأَتِ المُقَابَلَةُ فَشَاهَدَ المُتَفَرِّجُونَ مُبَارَاةً شَيِّقَةً.

دِيكَانِ كَانَا يَتَقَاتَلاَنِ فِي فِنَاءِ الدَّارِ فَغَلَبَ الكَبِيرُ الصَّغِيرُ. أَمَّا المَغْلُوبُ فَمَضَى مِنْ وَقْتِهِ إِلَى مَأْوَاهُ. وَ أَمَّا الغَالِبُ فَصَعِدَ فَوْقَ السَّطْعِ وَ جَعَلَ يُصَفِّقُ فَوْقَهُ وَ يَصِيحُ مُفْتَخِرًا. فَبَصُرَ بِهِ أَحَدُ الجَوَارِحِ فَانْقَضَّ عَلَيْهِ وَ اخْتَطَفَهُ. 

مَرَّ كَلْبٌ عَلَى جِسْرٍ وَ فِي فَمِهِ قَطْعَةُ لَحْمٍ فَرَأَى ظِلَّهُ فِي المَاءِ فَظَنَّ أَنَّهُ كَلْبٌ آخَرُ فِي فَمِهِ قِطْعَةُ لَحْمٍ. فَأَرَادَ أَنْ يَخْتَطِفَهَا مِنْهُ وَ رَمَى نَفْسَهُ فِي المَاءِ فَسَقَطَتْ قِطْعَةُ اللَّحْمِ مِنْ فَمِهِ وَ لَمْ يَجِدْ شَيْءً بَدَلَهَا.

ـــــــــــــــــــــــــ

أَجِبْ عَلَى الأَسْئَلَةِ الآتِيَةِ:

كم حكاية في الدرس؟ و كم بطلا في كل حكاية؟ 

ماذا كان يفعل الكلب عندما يشتغل الحداد؟

متى كان يستيقظ الكلب؟

مِمَّ تعجّب الحدّاد؟

ماذا قال الحداد لنفسه؟

من فتح باب الملعب و من عاونه عليه؟

لماذا قدم اللاعبون و المتفرجون إلى الملعب؟

ماذا مر على الجسر؟ و ماذا كان في فمه عندذلك؟ 

هل كان في الماء كلب آخر حقيقة؟

لماذا رمى الكلب نفسه في الماء؟

هل وجد الكلب في الماء ما أراد؟

\subsection{-١١٢-}
 \includegraphics[width=1.6252in,height=1.3646in]{images/MuhammadBagauddinprettified-img289.png}   \includegraphics[width=1.552in,height=1.6146in]{images/MuhammadBagauddinprettified-img290.png}   \includegraphics[width=1.5728in,height=1.2083in]{images/MuhammadBagauddinprettified-img291.png} 

\ مُلاَكِمٌ (ون). مُصَارِعٌ (ون). كَرْمٌ، كُرُومٌ. \newline
شَجَرَةُ الكَرْمِ. بَحَثَ فِي... قَضِيَّةٌ، قَضَايَا. بَحَثَ فِي قَضِيَّةٍ. فَرَّحَ. مُصْحَفٌ، مَصَاحِفُ. أَهْدَى. جُزْءٌ، أَجْزَاءٌ. فَرَحًا. فَرَحًا بِمَا فَعَلَ بِطَلَبٍ مِنِّي. تَمَّ لَهُ خَمْسُونَ سَنَةً. طَعَنَ فِي الخَمْسِينَ (ا) طَعْنٌ. كَلاَّ! فِي رَبِيعِ هَذَا العَامِ. هَكَذَا؟ بِكَثِيرٍ. أَكْبَرُ بِكَثِيرٍ. أَصْغَرُ بِكَثِيرٍ. يَبْدُو. كَيْفَ تَرَى؟ لاَ بَأْسَ بِهِ. عِلْمٌ لاَ بَأْسَ بِهِ. رَغْمَ ذَلِكَ. كِبَرُ السِّنِّ. رَغْمَ كِبَرِ سِنِّهِ. مَحْبُوبٌ. أَقْسَمَ بِـ... أُقْسِمُ بِاللَّهِ. قَضَى حَاجَتَهُ. نَفِدَ (ا) نَفَادٌ. نَفِدَ مَالُهُ. مَا بَالُكَ؟ حَزِينٌ. مَا بَالُكَ حَزِينًا؟ قَرِيبٌ، أَقْرِبَاءُ. سَبِيلٌ، سُبُلٌ. بَدَنِيٌّ.

ـــــــــــــــــــــــــ

يَطِيبُ لِي أَنْ أَجْلَسَ مَعَ إِخْوَانِي تَحْتَ أَشْجَارِ الكُرُومِ نَشْرَبُ الشَّايَ وَ نَبْحَثُ فِي قَضَايَا الإِسْلاَمِ. يَطِيبُ لِي أَنْ أُصْغِيَ إِلَيْهِ وَ هُوَ يُحَدِّثُ بِمَا رَأَى وَ سَمِعَ فِي أَسْفَارِهِ، وَ يَطِيبُ لَهُ أَنْ يُفَرِّحَنِي بِحِكَايَاتِهِ. فِي جَيْبِي مُصْحَفٌ صَغِيرٌ أَهْدَاهُ لِي أَبِي لَمَّا أَخْبَرْتُهُ أَنِّي حَفِظْتُ مِنَ القُرْآنِ خَمْسَةَ أَجْزَاءٍ مِنْ أَوَاخِرِهِ فَرَحًا بِمَا فَعَلْتُ، وَ كَانَ قَدِ اشْتَرَاهُ بِطَلَبٍ مِنِّي. كَمْ سَنَةً عُمْرُكَ؟ تَمَّ لِي خَمْسُونَ سَنَةً وَ طَعَنْتُ فِي الحَادِيَةِ وَ الخَمْسِينَ فِي رَبِيعِ هَذَا العَامِ- هَكَذَ؟ أَنْتَ تَبْدُو أَصْغَرَ بِكَثِيرٍ مِنْ عُمْرِكَ. كَيْفَ تَرَى مُعَلِّمَنَا؟- لَهُ عِلْمٌ لاَ بَأْسَ بِهِ وَ قُوَّةٌ بَدَنِيَّةٌ لاَ بَأْسَ بِهَا رَغْمَ كِبَرِ سِنِّهِ.- نَعَمْ، هُوَ كَانَ فِيمَا مَضَى مُلاَكِمًا وَ مُصَارِعًا وَ مَا يَزَالُ يُحِبُّ الأَلْعَابَ الرِّيَاضِيَّةَ كُلَّ صَبَاحٍ وَ يُوصِينَا بِهَا، هُوَ أُسْتَاذُنَا المَحْبُوبُ يُحِبُّنَا وَ نُحِبُّهُ. أُقْسِمُ بِاللَّهِ أَنَّهُ لاَ يَأْتِينِي فَقِيرٌ يَسْأَلُنِي مَالاً إِلاَّ أَعْطَيْتُهُ وَ قَضَيْتُ حَاجَتَهُ حَتَّى تَنْفَدَ أَمْوَالِي. مَا بَالُكَ لاَ تَزَالُ حَزِينًا بَعْدَ مَوْتِ وَالِدِكَ؟ هَلْ أَنْتَ وَحِيدٌ فِي ذَلِكَ؟ كَلاَّ! كُلٌّ مِنَّا مَاتَ لَهُ إِمَّا أَبٌ أَوْ أُمٌّ أَوِ ابْنٌ أَوْ أَخٌ أَوْ قَرِيبٌ آخَرُ، المَوْتُ سَبِيلُ كُلِّ حَيٍّ لاَ بُدَّ لَهُ أَنْ يَمُرَّ فِيهِ.

ـــــــــــــــــــــــــ

اذكر مَنْ تَرَى في كلِّ صُورَةٍ مِنَ الآتية؟ وَ مَاذَا بِيَدِ كلٍّ مِنْهُمْ؟ وَ مَاذَا يَعْمَلُ كلٌّ منهُم؟

 \includegraphics[width=1.7189in,height=1.448in]{images/MuhammadBagauddinprettified-img292.png}   \includegraphics[width=1.4583in,height=1.4791in]{images/MuhammadBagauddinprettified-img293.png}   \includegraphics[width=1.3126in,height=1.4689in]{images/MuhammadBagauddinprettified-img294.png} 

1) 2) 3) 

 \includegraphics[width=1.2189in,height=1.3126in]{images/MuhammadBagauddinprettified-img295.png}   \includegraphics[width=2in,height=1.5in]{images/MuhammadBagauddinprettified-img296.png}   \includegraphics[width=1.0937in,height=1.1252in]{images/MuhammadBagauddinprettified-img297.png} 

\ 4) 5) 6)

 \includegraphics[width=2.0417in,height=1.9898in]{images/MuhammadBagauddinprettified-img298.png}   \includegraphics[width=1.0626in,height=1.698in]{images/MuhammadBagauddinprettified-img299.png}   \includegraphics[width=1.8228in,height=1.7709in]{images/MuhammadBagauddinprettified-img300.png} 

\ 7) 8) 9)

\subsection{-١١٣-}
\  \includegraphics[width=0.9165in,height=1.4374in]{images/MuhammadBagauddinprettified-img301.png}   \includegraphics[width=1.4374in,height=1.5728in]{images/MuhammadBagauddinprettified-img302.png}   \includegraphics[width=1.0728in,height=1.6146in]{images/MuhammadBagauddinprettified-img303.png}   \includegraphics[width=1.7398in,height=1.0209in]{images/MuhammadBagauddinprettified-img304.png}  

فُسْتَانٌ ، فَسَاتِينُ. مِعْطَفٌ، مَعَاطِفُ. مَنَامَةٌ. مِقْلاَةٌ، مَقَالٍ. 

رُبَّمَا. رُبَّمَا يُرِيدُ. رُبَّمَا جَاءَ. هَاكَ. هَاكَ القَلَمَ. لاَ أَحَدَ. حَافِلَةٌ (ات)، حَوَافِلُ. قَادِمٌ (ون). رَاحَ (و) رَوَاحٌ. حَمَّامٌ (ات). اِخْتَفَى. فَتَّشَ عَنْ... بِالبَابِ. قُدَّامَ... دُولاَبٌ ، دَوَالِيبُ. دُولاَبُ المَلاَبِسِ. شَوَى (ي) شَيٌّ. شُوَاءٌ. صَبَرَ (ي) صَبْرٌ. حَضَرَ الطَّعَامُ. جَفْنَةٌ، جِفَانٌ. كُسْكُسٌ. عِيدُ مِيلاَدٍ. مُسْتَعِدٌّ (ون). عِيدٌ سَعِيدْ!

ـــــــــــــــــــــــــ

أَبِي، أَبِي! لَيْلَى تَبْكِي.- مَا بِهَا يَا خَالِدُ؟ رُبَّمَا تُرِيدُ دُمْيَةً؟- لاَ أَدْرِي، رُبَّمَا تُرِيدُ، هَلْ أَفْتَحُ الدُّولاَبَ يَا أَبِي ِلآخُذَ مِنْهُ الدُّمْيَةَ وَ أَعْطِيَهَا لَيْلَى؟- نَعَمْ، هَاكَ المِفْتَاحَ. نَزَلَ أَبِي مِنَ الحَافِلَةِ هَا هُوَ ذَا قَادِمٌ يَحْمِلُ حَقِيبَةً، فِي الحَقِيبَةِ فُسْتَانٌ ِلأُخْتِي وَ مِعْطَفٌ لِي. جَاءَ أَحْمَدُ إِلَى الدَّارِ وَ سَأَلَتْ أُمُّهُ: أَيْنَ مُرَادٌ وَ مَحْمُودٌ وَ سَالِمٌ؟- لاَ أَحَدَ، خَرَجُوا، رُبَّمَا رَاحُوا إِلَى الحَمَّامِ. فَرِيدَةُ تَخْتَفِي وَرَاءَ الشَّجَرَةِ وَ عُمَرُ يَخْتَفِي فَوْقَ الشَّجَرَةِ بَيْنَ أَغْصَانِهَا وَ سُعَادُ تُفَتِّشُ عَنْهُمَا. مَنْ بِالبَابِ؟- لاَ أَحَدَ.- فَمَا هَذَا الصَّوْتُ الَّذِي أَسْمَعُهُ؟ أَتُرِيدُ أَنْ تَرْقُدَ؟- نَعَمْ، لَكِنِّي لاَ أَجِدُ مَنَامَتِي- أَنَا رَأَيْتُ مَنَامَتَكَ فِي دُولاَبِ المَلاَبِسِ. الأُمُّ قُدَّامَ المَوْقِدِ تَشْوِي الشُّوَاءَ فِي المِقْلاَةِ وَ تَقُولُ لِمُصْطَفَى: اِصْبِرْ قَلِيلاً سَيَحْضُرُ الطَّعَامُ، وَ تَأْكُلُ. زُرْنَا رَفِيقَنَا فَجَاءَنَا بِجَفْنَةٍ مِنَ الكُسْكُسِ فَأَكَلَ وَ أَكَلْنَا وَ شَكَرْنَاهُ. قَالَتْ لَنَا هِنْدٌ: عِيدُ مِيلاَدِي غَدًا سَأَدْعُوكُمْ إِلَى الحَفْلَةِ. قُلْنَا كُلُّنَا: "عِيدٌ سَعِيدْ!"

ـــــــــــــــــــــــــ

اذكر اسمَاء الأشيَاء الَّتي تراهَا في الصور الآتيَةِ ؟ و هَلْ يحتَاجُ النّاسُ اليها؟ وَ لمَاذَا يحتَاجُونَ؟ و في أي دَرْسٍ وَرَد كلٌّ منهَا؟ 

 \includegraphics[width=1.2811in,height=0.7398in]{images/MuhammadBagauddinprettified-img305.png}   \includegraphics[width=1.698in,height=0.802in]{images/MuhammadBagauddinprettified-img306.png}   \includegraphics[width=1.2083in,height=1.0102in]{images/MuhammadBagauddinprettified-img307.png} 

 \includegraphics[width=1.3957in,height=1.0835in]{images/MuhammadBagauddinprettified-img308.png}   \includegraphics[width=1.5in,height=1.052in]{images/MuhammadBagauddinprettified-img309.png}   \includegraphics[width=1.0102in,height=0.9898in]{images/MuhammadBagauddinprettified-img310.png} 

\subsection{-١١٤-}
حَذَارِ. جَرَحَ (ا) جَرْحٌ. جَرَحَ يَدَهُ. سَمَّرَ. سَمَّرَ اللَّوْحَ. فُجْأَةً. أَطْلَقَ. صَرْخَةٌ. أَطْلَقَ صَرْحَةً. عَنِيدٌ. مُضْحِكٌ. نَعْسَانُ (نَعْسَى)، نِعَاسٌ. سَهِرَ (ا) سَهَرٌ. يَوْمُ عُطْلَةٍ. كَمَّلَ. وَ فَوْقَ ذَلِكَ. حِجْرٌ، حُجُورٌ. فِي حِجْرِهَا. حَالاً. نَعَسَ (ا) نَعْسٌ. رَقَدَ (و) رُقَادٌ. قُومُوا إِلَى النَّوْمِ. سَهْرَةٌ (ات). اِنْتَهَتِ السَّهْرَةُ. دَعْ. رَوَّحَ. وَ إِلاَّ. مُجْتَمِعٌ (ون). مُنْتَظِرٌ (ون). بَقِيَّةٌ، بَقَايَا. قِصَّةٌ، قِصَصٌ.

ـــــــــــــــــــــــــ

أَيْنَ المِنْشَارُ وَ المِطْرَقَةُ يَا أُمِّي؟- مَاذَا تُرِيدُ بِهِمَا؟- أُرِيدُ أَنْ أَصْنَعَ عَرَبَةً. حَذَارِ أَنْ تَجْرَحَ يَدَكَ! جَلَسَ مُصْطَفَى وَ بَدَأَ يُسَمِّرُ الأَلْوَاحَ (طْرَقْ). وَ فُجْأَةً أَطْلَقَ صَرْخَةً "آيْ" اُصْبُعِي، جَرَحْتُ اُصْبُعِي. عَنَّفَتْهُ الأُمُّ قَائِلَةً: إِنِّي قُلْتُ لَكَ: حَذَارِ، وَ أَنْتَ عَنِيدٌ. أَنْتَ نَعْسَانُ رِضَى، قُمْ إِلَى فِرَاشِكَ- أُرِيدُ أَنْ أَسْهَرَ وَ اسْتَمِعَ إِلَى حِكَايَةِ جَدَّتِنَا، إِنَّهَا بَدَأَتْ حِكَايَتَهَا أَمْسِ وَ لَمْ تُكَمِّلْهَا وَ كَانَتْ مُضْحِكَةً جِدًّا، وَ فَوْقَ ذَلِكَ غَدًا يَوْمُ عُطْلَةٍ لاَ نَذْهَبُ إِلَى المَدْرَسَةِ- آ، نَسِيتُ. بَدَأَتِ الجَدَّةُ بَقِيَّةَ قِصَّتِهَا فَكَانَ رِضَا جَالِسًا جَنْبَهَا وَ لَيْلَى جَالِسَةً فِي حِجْرِهَا. كَانَتْ الحِكَايَةُ مُضْحِكَةً حَقًّا لَكِنَّهَا طَوِيلَةٌ. نَعَسَتْ لَيْلَى فَرَقَدَتْ فِي حِجْرِهَا فَقَالَتِ الجَدَّةُ رَقَدَ النَّاسُ جَمِيعًا قُومُوا إِلَى النَّوْمِ، وَ أَمَّا القِصَّةُ فَسَأُكَمِّلُهَا غَدًا بِإِذْنِ اللَّهِ، وَ قَالَ الأَبُ إِنْتَهَتِ السَّهْرَةُ قُومُوا إِلَى النَّوْمِ. يَا فَرِيدُ دَعِ اللَّعِبَ وَ رَوِّحْ حَالاً، حَانَ وَقْتُ الصَّلاَةِ وَ إِلاَّ فَاتَتْكَ الجَمَاعَةُ فَإِخْوَانُكَ مِنْ جِيرَانِكَ مُجْتَمِعُونَ فِي المَسْجِدِ مُنْتَظِرُونَ حُضُورَ الإِمَامَ. 

ـــــــــــــــــــــــــ

إِنكَ تَعْرفُ الأشيَاءَ الَّتِي تَظْهَرُ في الصُّورِ وَ كُلُّهَا مَرَّتْ عَلَيْكَ رُبَّمَا يُوجَدُ بَعْضُهَا في غُرْفَتِكَ وَ عَلَى مَكْتَبَتِكَ. فَاذْكُرْ لي أسمَاءَهَا وَ في أَيِّ دَرْسٍ وَرَدَ كُلٌّ مِنْهَا؟

 \includegraphics[width=1.1043in,height=1.0937in]{images/MuhammadBagauddinprettified-img311.png}   \includegraphics[width=1.1665in,height=0.7709in]{images/MuhammadBagauddinprettified-img312.png}   \includegraphics[width=0.8854in,height=0.8646in]{images/MuhammadBagauddinprettified-img313.png}   \includegraphics[width=0.7917in,height=0.8437in]{images/MuhammadBagauddinprettified-img314.png} 

 \includegraphics[width=0.7709in,height=0.8437in]{images/MuhammadBagauddinprettified-img315.png}   \includegraphics[width=0.7917in,height=1.052in]{images/MuhammadBagauddinprettified-img316.png}   \includegraphics[width=0.5in,height=1.0835in]{images/MuhammadBagauddinprettified-img317.png}   \includegraphics[width=0.552in,height=0.8752in]{images/MuhammadBagauddinprettified-img318.png}   \includegraphics[width=0.5311in,height=1.0311in]{images/MuhammadBagauddinprettified-img319.png} 

\subsection{-١١٥-}
 \includegraphics[width=1.1252in,height=1.3646in]{images/MuhammadBagauddinprettified-img320.png}   \includegraphics[width=0.6457in,height=1.2398in]{images/MuhammadBagauddinprettified-img321.png}   \includegraphics[width=1.5937in,height=1.4272in]{images/MuhammadBagauddinprettified-img322.png} 

مُمَرِّضَةٌ (ات). سَمَّاعَةٌ (ات). مَطَرِيَّةٌ (ات). حَبَا (و) حَبْوٌ. مَا زَالَ. مَا زَالَ الوَلِيدُ يَحْبُو. تَنَاوَلَ.قَذِرٌ. جَاهِزٌ. الشَّايُ جَاهِزٌ. أَفْطَرَ. تَأَلَّمَ. أَحَسَّ بـ...

حُمَّى. أَحَسَّ بِالبَرْدِ. يُحِسُّ بِالحُمَّى. حَبَّةٌ (ات). نَزَعَ (ي) نَزْعٌ. نَزَعَ ثِيَابَهُ. طَفِيفٌ. زَالَ (و) زَوَالٌ. لَزِمَ (ا) لُزُومٌ. لَزِمَ الفِرَاشَ. لَزِمَ البَيْتَ. تَذَكَّرَ. تَذَكَّرْتُ! حُقْنَةٌ، حُقَنٌ. بِنَايَةٌ (ات). لاَ بَأْسَ عَلَيْكَ. فِي أَثْنَاءِ ذَلِكَ.

ــــــــــــــــــــــــ

هِشَامٌ لاَ يَمْشِي بَعْدُ هُوَ مَا زَالَ يَحْبُو. أُخْتُهُ تَجْلِسُ بِجَانِبِهِ وَ تَحْرُسُهُ دَائِمًا مِنْ أَنْ يَتَنَاوَلَ الأَشْيَاءَ القَذِرَةَ. الفُطُورُ جَاهِزٌ، تَعَالَوْا يَا أَطْفَالُ- بَلْ نُرِيدُ أَنْ نَخْرُجَ- لاَ، أَفْطِرُوا أَوَّلاً ثُمَّ اخْرُجُو، وَ إِذَا خَرَجْتُمْ فَالْبَسُوا المَعَاطِفَ وَ لاَ تَنْسَوُوا المَطَرِيَّاتِ فَالبَرْدُ شَدِيدٌ وَ المَطَرُ مَا زَالَ يَسْقُطُ.

خَالِدٌ يَتَأَلَّمُ هُوَ يُحِسُّ بِالحُمَّى. جَاءَ الطَّبِيبُ لِيُدَاوِيَ خَالِدًا. الطَّبِيبُ يَحْمِلُ سَمَّاعَةً. نَزَعَ خَالِدٌ ثِيَابَهُ فَفَحَصَهُ الطَّبِيبُ وَ قَالَ: لاَ بَأْسَ عَلَيْكَ، شَيْءٌ طَفِيفٌ، سَوْفَ يَزُولُ اِلْزَمِ الفِرَاشَ أُسْبُوعًا وَ تَنَاوَلْ فِي أَثْنَاءِ ذَلِكَ هَذَا الدَّوَاءَ حَبَّةً وَاحِدَةً كُلَّ صَبَاحٍ وَ مَسَاءٍ أَصِفُ لَكَ حُقْنَةً. مِسْكِينٌ خَالِدٌ مَا زَالَ مَرِيضًا ِلأَنَّهُ يَخَافُ الدَّوَاءَ وَ يَخَافُ الحُقْنَةَ، وَ إِذَا ظَهَرَتِ المُمَرِّضَةُ هَرَبَ وَ اخْتَفَى. هَيَّا نَذْهَبْ إِلَيْهِ، هُوَ فِي هَذِهِ البِنَايَةِ. أَيْنَ لَيْلَى يَا أُمِّي؟ خَرَجَتْ- أَيْنَ رَاحَتْ؟- لاَ أَدْرِي، لَعَلَّهَا رَاحَتْ تَشْتَرِي الدَّوَاءَ لِخَالِدٍ- لاَ، تَذَكَّرْتُ! إِنَّهَا رَاحَتْ عِنْدَ جَدَّتِهَا لِتَقُصَّ عَلَيْهَا بَقِيَّةَ قِصَّتِهَا.

ــــــــــــــــــــــــ

هَذِهِ الأَشْيَاءُ الَّتِي تَظْهَرُ في الصُّورِ الآتِيَةِ كُلُّهَا مَرَّتْ عَلَيْكَ فَمَا أَسْمَاءُهَا وَ مَا فَائِدَتُهَا؟ وَ بَعْضُهَا يُؤْكَلُ وَ بَعْضُهَا لاَ يُؤْكَلُ فَمَيِّزِ المَأْكُولَ مِنْهَا مِنْ غَيْرِ المَأْكُولِ.

 \includegraphics[width=1.1457in,height=0.698in]{images/MuhammadBagauddinprettified-img323.png}   \includegraphics[width=1.1252in,height=0.6252in]{images/MuhammadBagauddinprettified-img324.png}   \includegraphics[width=1.0835in,height=0.7811in]{images/MuhammadBagauddinprettified-img325.png}   \includegraphics[width=0.5626in,height=1.6457in]{images/MuhammadBagauddinprettified-img326.png} 

 \includegraphics[width=0.8543in,height=0.6772in]{images/MuhammadBagauddinprettified-img327.png}   \includegraphics[width=1.1354in,height=0.8646in]{images/MuhammadBagauddinprettified-img328.png}   \includegraphics[width=0.9689in,height=0.9791in]{images/MuhammadBagauddinprettified-img329.png}   \includegraphics[width=0.9898in,height=0.6457in]{images/MuhammadBagauddinprettified-img330.png} 

\subsection{-١١٦-}
 \includegraphics[width=1.3228in,height=1.1043in]{images/MuhammadBagauddinprettified-img331.png}   \includegraphics[width=1.3752in,height=1.0626in]{images/MuhammadBagauddinprettified-img332.png}   \includegraphics[width=1.0835in,height=0.6043in]{images/MuhammadBagauddinprettified-img333.png} 

نَظَّارَةٌ (ات). مِذْيَاعٌ، مَذَايِيعُ قُبَّعَةٌ (ات). أَخْفَى. \newline
ظَهْرٌ، ظُهُورٌ. حَرْفٌ، حُرُوفٌ. ضَعُفَ (و) ضُعْفٌ. بَصَرٌ. ضَعُفَ بَصَرُهُ. فِي السِّنِّ المُبَكِّرَةِ. أَنْتَ تَظْلِمُنِي! خَبَّأَ. أَظْلَمَ. أَظْلَمَ اللَّيْلُ. لَحْظَةً. فَتَحَ المِذْيَاعَ. عَشِيَّةٌ. كُلَّ عَشِيَّةٍ. الرُّسُومُ المُتَحَرِّكَةُ. شَخَرَ (ي) شَخِيرٌ. فَكَّكَ. رَكَّبَ. اِسْتَغَاثَ بِـ... قَضَى (ي) قَضَاءٌ. قَضَى الوَقْتَ. سَاعَاتُ الفَرَاغِ. مَتْحَفٌ، مَتَاحِفُ. الآثَارُ التَّارِيخِيَّةُ. ضَوَاحِى المَدِينَةِ. أَحْيَانًا. ضَاحِيَةٌ، ضَوَاحٍ. ذِكْرٌ، أَذْكَارٌ. دُعَاءٌ، أَدْعِيَةٌ. كَثِيرًامَّا. اِسْتَمْتَعَ بِـ... الهَوَاءُ الطَّلْقُ. فِي الهَوَاءِ الطَّلْقِ. تَلْفَزَةٌ.

ــــــــــــــــــــــــ

أَخْفَتْ نَوَالُ وَرَاءَ ظَهْرِهَا شَيْءً وَ قَالَتْ: طَوِيلٌ وَ مَبْرِيٌّ يَمْشِي عَلَى الوَرَقِ وَ يَتْرُكُ عَلَيْهِ الحُرُوفَ مَا هُوَ؟ أَيْنَ قُبَّعَتِي يَا مَحْفُوظُ؟- وَ لِمَاذَا تَسْأَلُنِي أَنَا؟- 

ِلأَنَّكَ خَبَّأْتَهَا- أَنَا مَارَأَيْتُهَا، أَنْتَ تَظْلِمُنِي، اُنْظُرْ فَوْقَ الشَّجَرَةِ، هَذَا عُمَرُ هُوَ الَّذِي خَبَّأَهَا. لَقَدْ أَظْلَمَ اللَّيْلُ هَيَّا نَرْجِعْ إِلَى الدَّارِ. 

اِنْتَظِرْ لَحْظَةً حَتَّى أَعْثُرَ عَلَى قُبَّعَتِي. هَلْ هَذِهِ نَظَّارَتُكَ؟- نَعَمْ- وَ تَلْبَسُ النَّظَّارَةَ؟- نَعَمْ- فِي السِّنِّ المُبَكِّرَةِ هَكَذَا ضَعُفَ بَصَرُكَ. 

فِي مَنْزِلِنَا مِذْيَاعٌ وَ تَلْفَزَةٌ. أَخِي يَفْتَحُ المِذْيَاعَ كُلَّ عَشِيَّةٍ وَ يَسْتَمِعُ أَمَّا أَنَا فَأَتَفَرَّجُ عَلَى التَّلْفَزَةِ وَ أُشَاهِدُ رُسُومًا مُتَحَرِّكَةً. كَانَتْ لَيْلَى رَاقِدَةً فِي فِرَاشِهَا وَ كَانَتْ تَشْخِرُ فَكَّكَ هِشَامٌ دُمْيَتَهَا فَاسْتَيْقَظَتْ لَيْلَى وَ رَأَتْ مَا فَعَلَ هِشَامٌ بِهَا فَأَخَذَتْ لَيْلَى تَبْكِي وَ تَسْتَغِيثُ بِأُمِّهَا، قَالَ هِشَامٌ لاَ تَبْكِي وَ لاَ تَحْزَنِي فَأَنَا أُرَكِّبُهَا كَمَا فَكَّكْتُهَا. أَيْنَ تَقْضِي سَاعَاتِ الفَرَاغِ؟- فِي سَاعَاتِ الفَرَاغِ أَزُورُ المَتْحَفَ وَ أَتَفَرَّجُ عَلَى الآثَارِ التَّارِخِييَّةِ وَ أَحْيَانًا أَخْرُجُ إِلَى ضَوَاحِي المَدِينَةِ ِلأَسْتَمِعَ بِالهَوَاءِ الطَّلْقِ، وَ كَثِيرًامَّا أَذْهَبُ إِلَى المَسْجِدِ وَ أَجْلِسُ فِيهِ سَاعَاتٍ فِي الذِّكْرِ وَ الدُّعَاءِ وَ الصَّلاَةِ.

ــــــــــــــــــــــــ

هَذَانِ حَيْوَانَانِ أَهْليَّانِ نَافِعَانِ وَ لِكُلٍّ مِنْهُمَا مَنَافِعُهُ الخَاصَّةُ، فَمَا مَنْفَعَةُ كُلٍّ مِنْهُمَا؟ وَ ِلأَيِّ شّيْءٍ يَسْتَخْدِمُهُ الإِنْسَانُ؟

 \includegraphics[width=1.8854in,height=1.6354in]{images/MuhammadBagauddinprettified-img334.png}   \includegraphics[width=1.75in,height=1.4583in]{images/MuhammadBagauddinprettified-img335.png} 

\subsection{-١١٧-}
 \includegraphics[width=1.6043in,height=1.1772in]{images/MuhammadBagauddinprettified-img336.png}   \includegraphics[width=1.5311in,height=1.1354in]{images/MuhammadBagauddinprettified-img337.png}   \includegraphics[width=1.7189in,height=1.1043in]{images/MuhammadBagauddinprettified-img338.png} 

\ حُفْرَةٌ، حُفَرٌ. دُكَّانٌ، دَكَاكِينُ. دُكَّانُ البَقَّالِ. 

بَقَّالٌ (ون). يَا تُرَى؟ طَالَ (و) طُولٌ. طَالَ انْتِظَارِي. غَنَّى. يَا عِفْرِيتُ! غَرَبَ (و) غُرُوبٌ. غَرَبَتِ الشَّمْسُ. حَتَّى أَمْلَأَ. مَا أَلَذَّهُ! وَاحِدٌ آخَرُ. رَاسَلَ. اِنْقَطَعَ. تَمَامًا. يَبْدُو. طِوَالَ سِنِي الدِّرَاسَةِ. عَلَى الدَّوَامِ. كُلُّ عَامٍ وَ أَنْتُمْ بِخَيْرٍ!- وَ أَنْتُمْ بِالصِّحَّةِ وَ السَّلاَمَةِ. كَيْفَ رَأَيْتَ؟ عَظِيمٌ! أَعْجَبَ. هَلْ أَعْجَبَكَ؟ يُعْجِبَنِي. اِحْذَرْ! أَجْرَى. لَمْ يَلْتَفِتْ إِلَى... حَذَّرَ. تَحْذِيرٌ. اِنْقَلَبَ. اِنْقَلَبَتْ بِهِ دَرَّاجَتُهُ. 

ــــــــــــــــــــــــ

أَظْلَمَ اللَّيْلُ وَ مَا رَجَعَ غَانِمٌ أَيْنَ هُوَ يَا تُرَى؟- أَنَا ذَاهِبٌ ِلأَبْحَثَ عَنْهُ. مَاذَا أَسْمَعُ؟ هَذَا صَوْتُهُ، إِنَّهُ يُغَنِّي، هَا قَدْ عَرَفْتُ مَكَانَكَ يَا عِفْرِيتُ. لَمَّا طَالَ انْتِظَارُ مُصْطَفَى فِي دُكَّانِ البَقَّالِ صَاحَ بِالبَقَّالِ: أُرِيدُ الصَّابُونَ. فَقَالَ: مَا لَكَ تَصِيحُ هَكَذَا؟- أُوْلَئِكَ أَصْحَابِي قَدْ وَصَلُوا إِلَى المَدْرَسَةِ وَ أَنَا مَا زِلْتُ هُنَا. أَيْنَ أَنْتَ، إِنِّي لاَ أَرَاكَ؟- أَنَا هُنَا فَوْقَ الشَّجَرَةِ أَقْطِفُ البُرْتُقَالَ- اِنْزِلْ لَقَدْ غَرَبَتِ الشَّمْسُ وَ عَلَيْنَا أَنْ نَصِلَ إِلَى المَسْجِدِ قَبْلَ فَوَاتِ الجَمَاعَةِ- حَتَّى أَمْلَأَ هَذِهِ السَّلَّةَ، كُلْ هَذِهِ البُرْتُقَالَةَ- "إِمْ" مَا أَلَذَّهَا أُرِيدُ وَاحِدَةً أُخْرَى. يَبْدُو أَنَّكَ نَسِيتَنِي تَمَامًا فَلَمْ تُرَاسِلْنِي، اِنْقَطَعَتْ عَنِّي رَسَائِلُكَ طِوَالَ سِنِي دِرَاسَتِكَ، وَ لَوْلاَ أَنَّكَ نَسِيتَنِي لَرَاسَلْتَنِي عَلَى الدَّوَامِ. كُلُّ عَامٍ وَ أَنْتُمْ بِخَيْرٍ يَا إِخْوَانِي!- وَ أَنْتَ بِالصِّحَّةِ وَ السَّلاَمَةِ يَا أَخَانَا، كَيْفَ رَأَيْتَ بَلَدَنَا؟ هَلْ أَعْجَبَكَ؟- عَظِيمٌ! كُلُّ شَيْءٍ فِيهِ أَعْجَبَنِي. رَكِبَ رِضَا دَرَّاجَتَهُ وَ رَاحَ يُجْرِيهَا، فَقِيلَ لَهُ: اِحْذَرْ يَا رِضَا، أَمَامَكَ حُفْرَةٌ، فَلَمْ يَلْتَفِتْ إِلَى تَحْذِيرِهِمْ فَوَقَعَ فِي الحُفْرَةِ وَ انْقَلَبَتْ بِهِ دَرَّاجَتُهُ. 

\subsection{-١١٨-}
الأُصُولُ الثَّلاَثَةِ

أَصْلٌ، أُصُولٌ. رَبَّى، يُرَبِّي، تَرْبِيَةٌ. عَالَمٌ (ون ، عَوَالِمُ). بِنِعْمَةِ اللَّهِ. مِنْ عَدَمٍ إِلَى وُجُودٍ. مَعْبُودٌ (ات). دَلِيلٌ، أَدِلَّةٌ. صِرَاطٌ. الصِّرَاطُ المُسْتَقِيمُ. كُلُّ مَا سِوَى اللَّهِ. ِلأَيِّ شَيْءٍ؟ اِتَّبَعَ. اِتَّبَعَ أَمْرَهُ. نَهْيٌ، مَنَاهٍ. اِجْتَنَبَ. اِجْتَنَبَ نَهْيَهُ. تَوْحِيدٌ. شِرْكٌ. طَاعَةٌ (ات). طَاعَةُ اللَّهِ. إِنْسِيٌّ، إِنْسٌ. اِنْقَادَ لِـ... مَنْهَجٌ، مَنَاهِجُ. شَرْعٌ. شَرْعُ اللَّهِ. أَقَامَ. أَقَامَ دَوْلَةَ الإِسْلاَمِ.

\ إِقَامَةُ حُكْمِ اللَّهِ فِي الأَرْضِ. بَرِئَ (ا) بَرَاءَةٌ مِنْ... بَرَاءَةٌ مِنَ الشِّرْكِ. وَضْعِيٌّ. قَانُونٌ وَضْعِيٌّ. رَفْضٌ بَاتٌّ. مُرْسَلٌ (ون). مِنْ لَدُنْ آدَمَ. طَلَوَاتُ اللَّهِ وَ سَلاَمُهُ. ذُرِّيَّةٌ (ات، ذَرَارِيُّ). تُرَابٌ، أَتْرِبَةٌ. جِنِيٌّ، جِنٌّ.

ــــــــــــــــــــــــ

الأُصُولُ الثَّلاَثَةُ الَّتِي يَجِبُ عَلَى كُلِّ مُسْلِمٍ وَ مُسْلِمَةٍ تَعَلُّمُهَا هِيَ مَعْرِفَةُ العَبْدِ رَبَّهُ وَ دِينَهُ وَ نَبِيَّهُ مُحَمَّدًا- صَلَّى اللَّهُ عَلَيْهِ وَ سَلَّمَ. 

١) الأَصْلُ الأَوَّلُ: إِذَا قِيلَ لَكَ: مَنْ رَبُّكَ؟ فَقُلْ رَبِّيَ اللَّهُ الَّذَي رَبَّانِي وَ رَبَّى العَالَمِينَ بِنِعْمَتِهِ، وَ خَلَقَنِي وَ خَلَقَ العَالَمِينَ مِنْ عَدَمٍ إِلَى وُجُودٍ وَ هُوَ مَعْبُودِي لَيْسَ لِي مَعْبُودٌ سِوَاهُ. وَ الدَّلِيلُ قَوْلُهُ: (وَ أَنَّ اللَّهَ رَبِّي وَ رَبُّكُمْ فَعْبُدُوهُ هَذَا صِرَاطٌ مُسْتَقِيمٌ) مريم-٣٢. وَ كُلَّ مَا سِوَى اللَّهِ عَالَمٌ وَ أَنَا وَاحِدٌ مِنْ ذَلِكَ العَالَمِ. وَ إِذَا قِيلَ لَكَ ِلأَيِّ شَيْءٍ خَلَقَكَ اللَّهُ؟ فَقُلْ: لِعِبَادَتِهِ وَ طَاعَتِهِ وَ اِتِّبَاعِ أَمْرِهِ وَ اِجْتِنَابِ نَهْيِهِ قَالَ تَعَالَى:

\ (وَ مَا خَلَقْتُ الجِنَّ وَ الإِنْسَ إِلاَّ لِيَعْبُدُونِ) النازيات- ٥٦ 

٢) الأصلُ الثّاني: إِذَا قِيلَ لَكَ: مَا دِينُكَ؟ فَقُلْ دِينِي الإِسْلاَمُ وَ هُوَ الإِسْتِسْلاَمُ لِلَّهِ بِالتَّوْحِيدِ وَ الاِنْقِيَادُ لَهُ بِالطَّاعَةِ وَ اِقَامَةُ حُكْمِهِ وَ شَرْعِهِ فِي الأَرْضِ كَنِظَامٍ وَ مَنْهَجٍ لِلْحَيَاةِ وَ البَرَاءَةُ مِنَ الشِّرْكِ وَ أَهْلِهِ وَ مِنَ النُّظُمِ وَ المَنَاهِجِ الوَضْعِيَّةِ وَ رَفْضُهَا رَفْضًا بَاتًّا. وَ هُوَ دِينُ الأَنْبِيَاءِ وَ المُرْسَلِينَ مِنْ لَدُنْ آدَمَ إِلَى مُحَمَّدٍ عَلَيْهِمْ صَلَوَاتُ اللَّهِ وَ سَلاَمُهُ. 

٣) الأَصْلُ الثالث: إِذَا قِيلَ لَكَ: مَنْ نَبِيُّكَ؟ فَقُلْ: نَبِيِّي مُحَمَّدُ ابْنُ عَبْدِ اللَّهِ ابْنِ عَبْدِ المُطَّلِبِ بْنِ هَاشِمٍ، وَ هَاشِمٌ مِنْ قُرَيْشٍ، وَ قُرَيْشٌ مِنْ كِنَانَةَ، وَ كِنَانَةُ مِنَ العَرَبِ، وَ العَرَبُ مِنْ ذُرِّيَّةِ إِسْمَاعِيلَ، وَ إِسْمَاعِيلُ مِنْ إِبْرَاهِيمَ الخَلِيلِ، وَ إِبْرَاهِيمُ مِنْ نُوحٍ وَ نُوحٌ مِنْ آدَمَ وَ آدَمُ مِنْ تُرَابٍ.

ـــــــــــــــــــــــــ

أسئلة للمناقشة:

- مَا هِيَ الأصول الثّلاثة الّتي يجب على كلّ مسلم و مسلمة تعلّمها؟

- إذا قيل لك من ربّك ، فماذا تجيب؟

- من ربك؟ من خلقك من عدم إلى وجود؟ من معبودك؟ 

- ما هو العالَمُ؟

- لأيّ شيءٍ خلقك الله؟

- لماذا خلق الله الجن و الإنس؟

- من أنت ؟ إنسيّ أم جنيّ؟

- ما دين الإسلام؟

- ما دين الأنبياء و المرسلين؟

- من أبو محمد ؟ من جدّ محمّد؟

- ذريّة من العرب؟

- ممّ خلق آدم؟

- هل للإنسان معبود غير الله تعالى؟

- اذكر الأصول الثلاثة التي يجب على كل مسلم و مسلمة تعلّمها؟ 

\subsection{-١١٩-}
التَّوْحِيدُ وَ الشِّرْكُ

أَعْظَمُ. أَشْرَكَ بِاللَّهِ. ذَنْبٌ، ذُنُوبٌ. عَلَى الإِطْلاَقِ. زِنًى. خَمْرٌ، خُمُورٌ. قَتَلَ (و) قَتْلٌ. قَطَعَ الطَّرِيقَ. قَطْعُ الطَّرِيقِ. دُونَ... عَقَّ (و) عُقُوقٌ. غَفَرَ (ي) مَغْفِرَةٌ، غُفْرَانٌ. شَاءَ (ا) مَشِيئَةٌ. لِمَنْ يَشَاءُ. خَفِيَ (ا) خَفَاءٌ عَلَى. يَخْفَى عَلَى كَثِيرٍ مِنَ النَّاسِ. بَلْ. أَخْفَى مِنْ. دَبَّ (ي) دَبِيبٌ. أَخْفَى مِنْ دَبِيبِ النَّمْلِ. صَخْرَةٌ، صَخْرٌ. صَخْرَةٌ صَمَّاءُ. لَيْلَةٌ ظَلْمَاءُ. أَبْطَلَ. الصَّالِحَاتُ. حَبِطَ (ا) حُبُوطٌ. لَئِنْ... لَئِنْ أَشْرَكْتَ لَيَحْبَطَنَّ عَمَلُكَ. َلاَقْتُلَنَّكَ. خَاسِرٌ (ون). أَحْبَطَ. تَابَ مِنْ ذَنْبِهِ. صَرَفَ (ي) صَرْفٌ. صَرَفَ العِبَادَةَ لِغَيْرِ اللَّهِ. صَرَفَ المَالَ فِي وُجُوهِ الخَيْرِ. عَلَيْكَ أَنْ تَحْيَا حَيَاتَكَ عَلَى وَجْهٍ يُرْضِي اللَّهَ. مَا حُكْمُهُ فِي الإِسْلاَمِ؟ إِنَّ اللَّهَ لاَ يَغْفِرُ أَنْ يُشْرَكَ بِهِ. 

ــــــــــــــــــــــــ

أَعْظَمُ أَمْرٍ أَمَرَ اللَّهُ الخَلْقَ بِهِ هُوَ التَّوْحِيدُ. وَ أَعْظَمُ نَهْيٍ نَهَى اللَّهُ عَنْهُ هُوَ الشِّرْكُ. فَالتَّوْحِيدُ هُوَ أَنْ نَعْبُدَ اللَّهَ وَحْدَهُ وَ لاَ نَعْبُدَ أَحَدًا غَيْرَهُ. وَ الشِّرْكُ هُوَ أَنْ نُشْرِكَ مَعَ اللَّهِ فِي عِبَادَتِهِ غَيْرَهُ. وَ الشِّرْكُ هُوَ أَعْظَمُ ذَنْبٍ عَلَى الإِطْلاَقِ، وَ كُلُّ ذَنْبٍ سِوَاهُ مِنْ قَتْلٍ أَوْ زِنًى أَوْ شُرْبِ خَمْرٍ أَوْ تَرْكِ صَلاَةٍ أَوْ قَطْعِ طَرِيقٍ أَوْ عُقُوقِ الوَالِدَيْنِ أَوْ غَيْرِهَا فَدُونَهُ. وَ الشِّرْكُ هُوَ الذَّنْبُ الَّذِي لاَ يَغْفِرُهُ اللَّهُ أَبَدًا. وَ كُلُّ ذَنْبٍ دُونَهُ يَغْفِرُهُ اللَّهُ لِمَنْ يَشَاءُ. قَالَ تَعَالَى: (إِنَّ اللَّهَ لاَ يَغْفِرُ أَنْ يُشْرَكَ بِهِ وَ يَغْفِرُ مَا دُونَ ذَلِكَ لِمَنْ يَشَاءُ) النساء-٤٨. وَ الشِّرْكُ أَنْوَاعٌ. وَ يَخْفَى عَلَى كَثِيرٍ مِنَ النَّاسِ بَلْ يَخْفَى عَلَى كَثِيرٍ مِنَ العُلَمَاءِ. حَتَّى قَالَ فِيهِ النَّبِيُّ- صلى الله عليه و سلم- "الشِّرْكُ أَخْفَى مِنْ دَبِيبٍ النَّمْلِ عَلَى الصَّخْرَةِ الصَّمَّاءِ فِي اللَّيْلَةِ الظَلْمَاءِ".

وَ الإِشْرَاكُ مُبْطِلٌ لِمَا عَمِلَهُ المُسْلِمُ قَبْلَ ذَلِكَ مِنَ الصَّالِحَاتِ لقوله تعالى:

(لَئِنْ أَشْرَكْتَ لَيَحْبَطَنَّ عَمَلُكَ وَ لَتَكُونَنَّ مِنَ الخَاسِرِينَ) الزمر-٦٥

فَإِنْ تَابَ مِنْ شِرْكِهِ بَعْدَ ذَلِكَ فَعَادَ إِلَى التَّوْحِيدِ فَلَهُ مَا عَمِلَهُ بَعْدَ ذَلِكَ وَ لاَ يَعُودُ إِلَيْهِ مَا عَمِلَهُ قَبْلَ ذَلِكَ وَ أَحْبَطَهُ الشِّرْكُ. وَ العِبَادَةُ أَنْوَاعٌ كَثِيرَةٌ يَجِبُ عَلَى المُسْلِمِ أَنْ يَعْرِفَهَا كُلَّهَا حَتَّى لاَ يَصْرِفَهَا ِلأَحَدٍ سِوَى اللَّهِ، فَيَقَعَ فِي الشِّرْكِ. بَلْ حَيَاةُ المُسْلِمِ كُلُّهَا عِبَادَةٌ فَعَلَيْهِ أَنْ يَحْيَاهَا عَلَى وَجْهٍ يُرْضِي اللَّهَ تَعَالَى. وَفَّقَنَا اللَّهُ لِمَا يُحِبُّ وَ يَرْضَى. 

ــــــــــــــــــــــــ

أسئلة للمناقشة:

- ما أعظم أمر أمر الله به؟ و ما أعظم نهي نهى الله عنه؟

- ما هو التوحيد ؟ ما هو الشرك؟

- ما منزلة الشرك بين الذنوب؟

- أيّ ذنب هو أعظم الذنوب؟

- هل يغفر الله الشرك؟ هل يغفر الله ما دون الشرك من سائر الذنوب؟

- هل يغفر الله الزنى أو شرب الخمر؟

- ماذا قال النبي- صلى عليه و سلم- في الشرك؟

- إذا أشرك الرجل فما حكمه في الإسلام؟ و ماذا يحدث له؟

- هل تبقى الحسنات السابقة لمن أشرك؟

- إذا تاب الرّجل من شركه فهل تعود إليه حسناته السّابقة؟

- هل يجوز لأحد أن يصرف عبادته لغير الله؟

- كيف يجب على المسلم أن يحيا حياته؟

- عَدِّدْ ما تعرف من أنواع العبادة؟

\subsection{-١٢٠-}
اَلدُّعَاءُ عِبَادَةٌ (١)

دُعَاءٌ، أَدْعِيَةٌ. دَعَا اللَّهَ. أَثْقَلَ. إِذَا أَثْقَلَكَ. تَحْمِيلٌ. بِضَاعَةٌ، بَضَائِعُ. رَاحِلَةٌ، رَوَاحِلُ. يَجُوزُ. لاَ يَجُوزُ. جَائِزٌ. غَيْرُ جَائِزٍ. لاَ بَأْسَ بِهِ. شَبَّ (و) شُبُوبٌ. حَرِيقٌ. إِصْطَبْلٌ (ات، أَصَاطِبُ). اِسْتَغَاثَ بِـ... أَطْْفَأَ الحَرِيقَ. فِرْقَةُ مَطَافِئَ. مِضَخَّةٌ (ات). خُرْطُومٌ، خَرَاطِيمُ. خَرَاطِيمُ المَاءِ. دَوَّارَةٌ (ات). لَمْ يَكُنْ يُحْسِنُ السِّبَاحَةَ. أَلْقَى. تَعَلَّقَ بِـ... أَشْرَفَ عَلَى... أَشْرَفَ عَلَى الغَرَقِ. غَرِقَ (ا) غَرَقٌ. نَجَا (و) نَجَاةٌ. دَيْنٌ، دُيُونٌ. حَلَّ أَجَلُهُ. قَضَى دَيْنَهُ. اِسْتَعَانَ بِـ... لاَ مَانِعَ مِنْهُ. عَادَةً. نَقَضَ (و) نَقْضٌ. نَقَضَ تَوْحِيدَهُ. وَ العِيَاذُ بِاللَّهِ. سَيَأْتِي. مِثَالٌ ، أَمْثِلَةٌ. 

ــــــــــــــــــــــــ

إِذَا أَثْقَلَكَ تَحْمِيلُ بِضَاعَتِكَ عَلَى رَاحِلَتِكَ فَدَعَوْتُ صَدِيقَكَ لِيُعِينَكَ عَلَى تَحْمِيلِهَا فَهَذَا جَائِزٌ لاَ بَأْسَ بِهِ، وَ إِذَا شَبَّ حَرِيقٌ فِي بَيْتِكَ أَوْ إِصْطَبْلِكَ فَاسْتَغَثْتَ بِجِيرَانِكَ لِيُطْفِئُوا الحَرِيقَ أَوِ اسْتَدْعَيْتَ بِالتِّلِفُونِ فِرْقَةَ المَطَافِئِ فَجَاؤُوا بِمِضَخَّاتِهِمْ وَ خَرَاطِيمِهِمْ فَأَطْفَؤُوا الحَرِيقَ فَهَذَا أَيْضًا جَائِزٌ. و إِنْ وَقَعْتَ فِي نَهْرٍ عَمِيقٍ ذِي دَوَّارَاتٍ وَ لَمْ تَكُنْ تُحْسِنُ السِّبَاحَةَ فَأَشْرَفْتَ عَلَى الغَرَقِ فَصِحْتَ بِأَحَدٍ مِنَ النَّاسِ وَاقِفٍ عَلَى شَاطِئِ النَّهْرِ وَ سَأَلْتَهُ أَنْ يَمُدَّ إِلَيْكَ يَدَهُ أَوْ يُلْقِيَ إِلَيْكَ حَبْلاً تَتَعَلَّقُ بِهِ فَتَنْجُو مِنَ الغَرَقِ فَهَذَا أَيْضًا لاَ بَأْسَ بِهِ. وَ إِنْ كَانَ عَلَيْكَ دَيْنٌ قَدْ حَلَّ أَجَلَهُ وَ لَمْ يَكُنْ عِنْدَكَ مَا تَقْضِي بِهِ دَيْنَكَ فَأَتَيْتَ رَجُلاً غَنِيًّا سَخِيًّا تَعْرِفُهُ وَ يَعْرِفُكَ فَاسْتَعَنْت بِهِ عَلَى قَضَاءِ دَيْنِكَ فَلاَ مَانِعَ مِنْ ذَلِكَ أَيْضًا، ِلأَنَّ هَذِهِ كُلَّهَا أُمُورٌ يَقْدِرُ عَلَيْهَا الإِنْسَانُ عَادَةً، وَ يَجُوزُ لَكَ أَنْ تَطْلُبَ مِنْ كُلِّ أَحَدٍ مَا يَقْدِرُ هُوَ عَلَيْهِ عَادَةً، وَ لَكِنْ لاَ يَجُوزُ ِلأَحَدٍ أَنْ يَدْعُو غَيْرَ اللَّهِ أَوْ يَسْتَعِينَ بِهِ أَوْ يَسْتَغِيثَ بِهِ فِي الأُمُورِ الَّتِي لاَ يَقْدِرُ عَلَيْهَا إِلاَّ اللَّهُ، فَمَنْ دَعَا غَيْرَ اللَّهِ أَوْ اسْتَعَانَ بِهِ أَوْ اسْتَغَاثَ بِهِ فِي هَذِهِ الأُمُورِ فَقَدْ عَبَدَهُ وَ مَنْ عَبَدَ غَيْرَ اللَّهِ فَقَدْ أَشْرَكَ وَ كَفَرَ وَ نَقَضَ تَوْحِيدَهُ. وَ العِيَاذُ بِاللَّهِ. وَ سَيَأْتِي فِي الدَّرْسِ الَّذِي يَلَى هَذَا بَعْضُ أَمْثَلَةٍ عَلَى ذَلِكَ.

ــــــــــــــــــــــــ

أسئلة للمناقشة:

- هل يجوز لك أن تدعو أحد أصدقائك أو جيرانك لمساعدتك؟ و في أيّ شيء يجوز؟

- ما هي الأمور التى يجوز لك أن تدعو فيها غير الله؟

- و ما هي الأمور التي لا يجوز لك أن تدعو غير الله فيها؟ 

- ايت بعدة أمثلة يجوز لك أن تدعو فيها غير الله أو تستعين به أو تستغيث؟

- ما حكم من دعا غير الله في الأمور التي لا يجوز له ذلك فيها؟

- هل سبق لك أن شبّ حريق في بيتك فاستغثت بأحدٍ من النّاس؟

- هل يكون من دعا غير الله في الأمور التي لا يقدر عليها عادةً إلا اللّه مسلمًا موجِّدًا؟

\subsection{-١٢١-}
الدُّعَاءُ عِبَادَةٌ (٢)

أَغَاثَ. أَغِثْنِي. كَشَفَ البَلاَءَ. كَشَفَ الغَمَّ. بَلاَءٌ. نَزَلَ بِهِ البَلاَءُ. شَفَى (ي) شِفَاءٌ.

كَثُرَ (و) كَثْرَةٌ. إِيمَانٌ. عِنْدَ... عِنْدَ المَوْتِ. قَرَّبَ. مُقَرَّبٌ. مُذْنِبٌ (ون). نَقِيٌّ. شَفَعَ (ا) شَفَاعَةٌ لِـ... شَفَعَ عِنْدَ اللَّهِ. نَحْوُ ذَلِكَ. مَلَكٌ، مَلاَئِكَةٌ. شَيْخٌ ، شُيُوخٌ ، مَشَايِخُ. أُسْتَاذٌ، أَسَاتِيذُ. قَبْرٌ، قُبُورٌ. مِنْ دُونِ اللَّهِ. إِيَّاكَ نَعْبُدُ وَ إِيَّاكَ نَسْتَعِينُ. اِسْتَجَابَ. اِسْتَجَابَ اللَّهُ دُعَاءَهُ. إِذًا. ظَالِمٌ (ون). إِنَّمَا. إِنَّمَا أَدْعُو رَبِّي. إِنَّمَا أَنَا بَشَرٌ. لاَ إِلَهَ إِلاَّ اللَّهِ. 

ــــــــــــــــــــــــ

قُلْنَا فِي الدَّرْسِ قَبْلَ هَذَا أَنَّ مَنْ دَعَا غَيْرَ اللَّهِ أَوِ اسْتَعَانَ بِهِ أَوِ اسْتَغَاثَ بِهِ فِي الأُمُورِ الَّتِي لاَ يَقْدِرُ عَلَيْهَا إِلاَّ اللَّهُ فَقَدْ أَشْرَكَ وَ كَفَرَ وَ نَقَضَ تَوْحِيدَهُ، وَ العِيَاذُ بِاللَّهِ. فَمَنْ قَالَ يَا عَبْدَ القَادِرِ الجِيلاَنِيَّ أَوْ يَا تِيجَانِيُّ، يَا دَسُوقِيُّ أَوْ يَا مَحْمُودُ أَفَنْدِيُّ، أَوْ يَا كُنْتَ حَاجِي، أَوْ يَا إِلْيَاسُ يَا خَضِرُ أَغْثِنِي، أَوْ اكْشِفْ عَنِّي هَذَا البَلاَءَ الَّذِي نَزَلَ بِي، أَوِ اشْفِ مَرِيضِي، أَوِ ارْزُقْنِي اِبْنًا فَقَدْ كَثُرَتْ بَنَاتِي، أَوِ انْصُرْنِي عَلَى عَدُوِّي، أَوِ اغْفِرْ ذَنْبِي، أَوِ احْفَظْ إِيمَانِي عِنْدَ المَوْتِ، أَوْ قَرِّبْنِي مِنَ اللَّهِ فَأَنَا مُذْنِبٌ بَعِيدٌ عَنِ اللَّهِ وَ أَنْتَ نَقِيٌّ قَرِيبٌ مِنَ اللَّهِ، أَوْ خُذْنِي مَعَكَ إِلَى الجَنَّةِ، أَوِ اشْفَعْ لِي عِنْدَ اللَّهِ، أَوْ نَحْوَ ذَلِكَ فَقَدْ أَشْرَكَ. سَوَاءٌ كَانَ هَذَا الَّذِي يَدْعُوهُ مَلَكًا مُقَرَّبًا أَوْ نَبِيًّا مُرْسَلاً أَوْ رَجُلاً حَيًّا أَوْ مَيِّتًا، شَيْخًا أَوْ أُسْتَاذًا، إِنْسِيًّا أَوْ جِنِّيًّا، قَبْرًا أَوْ شَجَرَةً، أَوْ غَيْرَ ذَلِكَ مِنْ دُونِ اللَّهِ، ِلأَنَّ الدُّعَاءَ فِي الأُمُورِ الَّتِي لاَ يَقْدِرُ عَلَيْهَا إِلاَّ اللَّهُ عِبَادَةٌ وَ العِبَادَةُ لاَ تَجُوزُ إِلاَّ لِلَّهِ. قَالَ تَعَالَى: (وَ اعْبُدُوا اللَّهَ وَ لاَ تُشْرِكُوا بِهِ شَيْئًا) النساء-٣٦ وَ قَالَ أَيْضًا: (وَ قَالَ رَبُّكُمْ ادْعُونِي أَسْتَجِبْ لَكُمْ) غافر-٦٠ وَ قَالَ أَيْضًا: (إِيَّاكَ نَعْبُدُ وَ إِيَّاكَ نَسْتَعِينُ) الفاتحة وَ قَالَ أَيْضًا: (وَ لاَ تَدْعُ مِنْ دُونِ اللَّهِ مَا لاَ يَنْفَعُكَ وَ لاَ يَضُرُّكَ فَإِنْ فَعَلْتَ فَإِنَّكَ إِذًا مِنَ الظَّالِمِينَ) يونس-١٠٦ وَ قَالَ أَيْضًا: (قُلْ إِنَّمَا أَدْعُو رَبِّي وَ لاَ أُشْرِكُ بِهِ أَحَدًا) الجن-٢٠ وَ قَالَ أَيْضًا: (وَ لاَ تَدْعُ مَعَ اللَّهِ إِلَهًا آخَرَ لاَ إِلَهَ إِلاَّ هُوَ) قصص-٨٨ وَ غَيْرُهَا كَثِيرٌ. وَ قَالَ رَسُولُ اللَّهِ- صلّى اللّه عَلَيْهِ و سلم " إِذَا سَأَلْتَ فَأَسْأَلِ اللَّهَ وَ إِذَا اسْتَعَنْتَ فَاسْتَعِنْ بِاللَّهِ" وَ قَالَ أَيْضًا: "إِنَّهُ لاَ يُسْتَغَاثُ بِي وَ إِنَّمَا يُسْتَغَاثُ بِاللَّهِ"

ــــــــــــــــــــــــ

أجب على الأسئلة:

- هل يجوز لنا أن نقول لرجل عالمٍ أو صالح اشفع لي عند الله؟ و لماذا؟

- هل يجوز لمريد أن يقول لشيخه أو أستاذه خذني معك إلى الجنة؟ و لماذا؟

- هل يجوز العبادة لغير الله؟

- فإن عبد الإنسان غير الله فما حكمه؟

- كم آية في الدرس؟ و هل حفظتها؟

- كم حديثا في الدرس؟

- هل حفظت الحديثين؟

\subsection[-١٢٢-]{-١٢٢-}
الوَاسِطَةَ بَيْنَ اللَّهِ وَ بَيْنَ العَبْدِ

اِتَّخَذَ. اِتَّخَذَ شَيْخًا. صُوفِيٌّ (ون). وَاسِطَةٌ، وَسَائِطُ. تَرَكَهُ وَرَاءَ ظَهْرِهِ. اِحْتَكَمَ إِلَى... عَمِلَ بِقَوْلِهِ. وَافَقَ. الكِتَابُ وَ السُّنَّةُ. نُورٌ، أَنْوَارٌ. عِلْمٌ لَدُنِيٌّ. عِلْمٌ بَاطِنِيٌّ. يَزْعُمُ. كَمَا يَزْعُمُ. كَمَا يَقُولُ. أُمُورُ الدِّينِ. بَاطِلٌ، أَبَاطِيلُ. عَبَأَ (ا) عَبْءٌ بِـ... لاَ نَعْبَأُ بِهِ. رَدَّ عَلَيْهِ قَوْلَهُ. صَاحِبٌ، أَصْحَابٌ. وَلِيٌّ، أَوْلِيَاءُ. دُونَ نِزَاعٍ. مُطَهَّرٌ. صِلَةٌ (ات). لَهُ صِلَةٌ بِـ... اَلْمَلَأُ الأَعْلَى. اِطَّلَعَ عَلَى... اللَّوْحُ المَحْفُوظُ. أَخَذَ بِيَدِهِ. غَيْبِيٌّ. نَجِسٌ. مَلِكٌ، مُلُوكٌ. عَامَّةٌ، عَوَامُّ. تَقَرَّبَ إِلَى... تَبْلِيغٌ. حَاشِيَةٌ، حَوَاشٍ. حَاشِيَةُ المَلِكِ. شَرِيعَةٌ، شَرَائِعُ. اللَّهُ عَزَّ وَ جَلَّ. إِنَّا لِلَّهِ وَ إِنَّا إِلَيْهِ رَاجِعُونَ! أَعَاذَنَا اللَّهُ. خَالَفَ.

ــــــــــــــــــــــــ

وَ مِنَ النَّاسِ مَنْ يَتَّخِذُ شَيْخًا صُوفِيًّا وَ أُسْتَاذًا وَ يَجْعَلُهُ وَاسِطَةً بَيْنَهُ وَ بَيْنَ اللَّهِ فِي الدُّعَاءِ وَ العِبَادَةِ يَحْتَكِمُونَ إِلَيْهِ وَ يَسْأَلُونَهُ عَنْ كُلِّ شَيْءٍ فِي الدِّينِ وَ الدُّنْيَا وَ يَتْرُكُونَ العُلَمَاءَ وَرَاءَ ظُهُورِهِمْ وَ يَعْمَلُونَ بِقَوْلِهِ سَوَاءٌ وَافَقَ كِتَابَ اللَّهِ وَ سُنَّةَ نَبِيِّهِ أَمْ لَمْ يُوَافِقْ وَ يَقُولُونَ أُسْتَاذُنَا إِنَّمَا يُجِيبُ بِنُورٍ مِنَ اللَّهِ وَ إِلْهَامٍ وَ عِنْدَهُ عِلْمٌ لَدُنِيٌّ أَوْ عِلْمٌ بَاطِنِيٌّ! كَمَا يَزْعُمُونَ أَنَّ مَنْ لَمْ يَتَّخِذْ شَيْخًا يُقَرِّبُهُ إِلَى اللَّهِ فَشَيْخَهُ الشَّيْطَانُ! فَهَذَا كُلُّهُ قَوْلٌ بَاطِلٌ لَيْسَ عَلَيْهِ دَلِيلٌ لاَ مِنَ القُرْآنِ وَ لاَ مِنَ الحَدِيثِ ، وَ كُلُّ مَا لَيْسَ عَلَيْهِ دَلِيلٌ مِنْ أُمُورِ الدِّينِ مِنْ هَذَيْنِ فَلاَ نَقْبَلُهُ وَ لاَ نَعْبَأُ بِهِ بَلْ نَرُدُّهُ عَلَى وُجُوهِ أَصْحَابِهِ. وَ كَمَا يَزْعُمُونَ أَنَّ شَيْخَهُمْ وَلِيٌّ مِنْ أَوْلِيَاءِ اللَّهِ دُونَ نِزَاعٍ قَرِيبٌ مِنَ اللَّهِ مُطَهَّرٌ مِنَ الذُّنُوبِ لَهُ صِلَةٌ بِالمَلَإِ الأَعْلَى وَِ يَطَّلِعُ عَلَى اللَّوْحِ المَحْفُوظِ وَ يَقْرَأُ مَا كُتِبَ فِيهِ مِنَ الأُمُورِ الغَيْبِيَّةِ وَ أَمَّا أَنَا فَرَجُلٌ مُذْنِبٌ نَجِسٌ بَعِيدٌ عَنِ اللَّهِ مُحْتَاجٌ إِلَى مَنْ يَأْخُذُ بِيَدِي وَ يُوصِلُنِي إِلَى اللَّهِ، وَ اللَّهُ مَلِكٌ عَظِيمٌ بَلْ هُوَ مَلِكُ المُلُوكِ، وَ مِنَ المُعْلُومِ عَادَةً أَنَّهُ إِذَا أَرَادَ أَحَدٌ مِنَ العَامَّةِ أَنْ يَدْخُلَ عَلَى المَلِكِ أَوْ يَتَقَرَّبَ إِلَيْهِ أَوْ يَسْأَلَهُ حَاجَةً فَعَلَيْهِ أَنْ يَسْتَعِينَ بِحَاشِيَةِ المَلِكِ فَهَؤُلاَءِ المَشَائِخُ أَقْرَبُ النَّاسِ إِلَى اللَّهِ وَ نَحْنُ نَسْتَعِينُ بِهِمْ! إِنَّا لِلَّهِ وَ إِنَّا إِلَيْهِ رَاجِعُونَ! 

إِنَّ اللَّهَ تَعَالَى جَعَلَ الرُّسُلَ عَلَيْهِمْ الصَّلاَةُ وَ السَّلاَمُ وَ سَائِطَ بَيْنَهُ وَ بَيْنَ سَائِرِ النَّاسِ فِي تَعْلِيمِ الدِّينِ وَ تَبْلِيغِ الشَّرَائِعِ وَ لَمْ يَجْعَلْ أَحَدًا لاَ مَلَكًا مُقَرَّبًا وَ لاَ نَبِيًّا مُرْسَلاً وَاسِطَةً بَيْنَهُ وَ بَيْنَ النَّاسِ فِي الدُّعَاءِ وَ العِبَادَةِ. فَاتِّخَاذُ العَبْدِ وَسَائِطَ بَيْنَهُ وَ بَيْنَ اللَّهِ فِي الدُّعَاءِ وَ العِبَادَةِ مِنَ الشِّرْكِ الأَعْظَمِ الَّذِي لاَ يَغْفِرُهُ اللَّهُ عَزَّ وَ جَلَّ. أَعَاذَنَا اللَّهُ مِنَ الشِّرْكِ الأَعْظَمِ.

ــــــــــــــــــــــــ

أجب على الأسئلة:

-هل يجوز لنا أن نقبل ما يخالف الكتاب و السنة؟

-هل يجب علينا أن نقبل ما ليس عليه دليل من الكتاب و السنة؟

-إن قال أحد من المريدين اتبع قول أستاذي و إن خالف الكتاب و السنة فما حكمه؟

-ما هي عادة الناس إذا أراد أحد من العامة أن يطلب حاجة من الملك؟ 

-هل تجري هذه العادة بين العبد و بين ربّه؟

-هل جعل اللّه وسائط بينه و بين الناس؟ 

-ما الشيء الذي جعل الله بينه و بين الناس وسائط فيه؟ 

-و ما الشيء الذي لم يجعل الله بينه و بين الناس وسائط فيه؟

\subsection{-١٢٣-}
لَيْسَ كَمِثْلِهِ شَيْءٌ

رَجَا (و) رَجَاءٌ. تَوَكَّلَ عَلَى... إِنَّهُ لَيَقُولُ. عَابِدٌ، عُبَّادٌ. فِي القَدِيمِ. وَثَنٌ، أَوْثَانٌ. عُبَّادُ الأَوْثَانِ. سَوَاءً بِسَوَاءٍ. شَفِيعٌ، شُفَعَاءُ. زُلْفَى. شَبَّهَ بِـ... ضَرَبَ مَثَلاً. مَا أَبْعَدَهُ عَنِ الحَقِّ! أَبْلَغَ. رَعِيَّةٌ، رَعَايَا. مَحْجُوبٌ عَنْ... قَصْرٌ، قُصُورٌ. يُجِيبُ دَعْوَةَ الدَّاعِ. حَبْلُ الوَرِيدِ. تَأْيِيدٌ. غَنِيٌّ عن... تَخَلَّى. تَعَالَى اللَّهُ عَنْ ذَلِكَ عُلُوًّا كَبِيرًا. كُنْ عَلَى حَذَرٍ. اِنْخَدَعَ. اِهْتَدَى. 

ــــــــــــــــــــــــ

إِنَّ أُوْلَئِكَ الَّذِينَ يَتَّخِذُونَ مَشَايِخَ وَ يَرْجُونَهُمْ وَ يَخَافُونَهُمْ وَ يَسْأَلُونَهُمُ الشَّفَاعَةَ عِنْدَ اللَّهِ وَ يَتَوَكَّلُونَ عَلَيْهِمْ وَ يَتَقَرَّبُونَ بِهِمْ إِلَى اللَّهِ وَ يَجْعَلُونَهُمْ وَسَائِطَ بَيْنَهُمْ وَ بَيْنَ اللَّهِ إِنَّهُمْ لَيَقُولُونَ كَمَا قَالَ عُبَّادُ الأَوْثَانِ فِي القَدِيمِ سَوَاءً بِسَوَاءٍ إِذْ قَالُوا (هَؤُلاَءِ شُفَعَاؤُنَا عِنْدَ اللَّهِ) يونس-١٨ وَ قَالُوا أَيْضًا (مَا نَعْبُدُهُمْ إِلاَّ لِيُقَرِّبُونَا إِلَى اللَّهِ زُلْفَى) الزمر-٣ لَقَدْ شَبَّهُوا اللَّهَ سُبْحَانَهُ وَ تَعَالَى بِمَلِكٍ فِي الأَرْضِ مِنْ بَنِي البَشَرِ وَ اللَّهُ يَقُولُ: (لَيْسَ كَمِثْلِهِ شَيْءٌ وَ هُوَ السَّمِيعُ البَصِيرُ) الشورى-١١ وَ يَقُولُ أَيْضًا (فَلاَ تَضْرِبُوا لِلَّهِ الأَمْثَالَ إِنَّ اللَّهَ يَعْلَمُ وَ أَنْتُمْ لاَ تَعْلَمُونَ) النمل-٧٤ فَمَا أَبْعَدَ هَؤُلاَءِ عَنِ الحَقِّ! إِنْ كَانَتْ مُلُوكُ الأَرْضِ لاَ تَعْلَمُ إِلاَّ مَا أُبْلِغُوهُ فَاللَّهُ عَالِمٌ بِكُلِّ شَيْءٍ غَيْرُ مُحْتَاجٍ إِلَى مَنْ يُبْلِغُهُ. وَ إِنْ كَانَتْ مُلُوكُ الأَرْضِ بَعِيدِينَ عَنْ رَعَايَاهُمْ مَحْجُوبِينَ فِي قُصُورِهِمْ فَاللَّهُ لَيْسَ كَذَلِكَ بَلْ هُوَ قَرِيبٌ يُجِيبُ دَعْوَةَ الدَّاعِ إِذَا دَعَاهُ وَ اللَّهُ يَقُولُ (وَ نَحْنُ أَقْرَبُ إِلَيْهِ مِنْ حَبْلِ الوَرِيدِ) ق-١٦ وَ يَقُولُ (وَ هُوَ مَعَكُمْ أَيْنَمَا كُنْتُمْ) الحديد-٤ إَنْ كَانَتْ مُلُوكُ الأَرْضِ مُحْتَاجِينَ إِلَى حَاشِيَتِهِمْ لِتَأْيِيدِهِمْ فَاللَّهُ غَنِيٌّ عَنِ العَالَمِينَ لاَ يَحْتَاجُ إِلَى تَأْيِيدِ أَحَدٍ. فَانْظُرُوا أَيُّهَا الطَّلَبَةُ إِلَى هَؤُلاَءِ كَيْفَ شَبَّهُوا اللَّهَ العَظِيمَ بِرَجُلٍ فِي الأَرْضِ يَأْكُلُ وَ يَشْرَبُ وَ يَتَخَلَّى تَعَالَى اللَّهُ عَنْ ذَلِكَ عُلُوًّا كَبِيرًا. فَكُونُوا أَيُّهَا الطَّلَبَةُ عَلَى حَذَرٍ وَ لاَ تَنْخَدِعُوا بِأَقْوَالِهِمْ، وَ اعْمَلُوا بِالكِتَابِ وَ السُّنَّةِ تَهْتَدُوا.

ــــــــــــــــــــــــ

أجب على الأسئلة:

-هل يجوز تشبيه اللّه بملك في الأرض من بني الإنسان؟

-و لماذا لا يجوز تشبيه الله بملك في الأرض؟

-هل في القرآن دليل على ذلك؟

-ما الفرق بين الله و بين ملوك البشر؟

-هل يجوز تشبيه الله بشيء من المخلوقات؟

-ماذا كان يقول عباد الأوثان في القديم بشأن عبادة الأوثان؟

-الله قريب من الإنسان أم بعيد عنه؟

-ما الدليل على أن الله قريب من الإنسان؟

-هل يحتاج الله إلى تأييد أحد كما تحتاج الملوك إلى تأييد حاشيتهم؟

-هل لله حاشية كالملوك؟ 

\subsection{-١٢٤-}
القُبُورِيُّونَ

قُبُورِيٌّ (ون). مُعْظَمُ... صَرَفَ مُعْظَمَ هَمِّهِ. مَيِّتٌ، أَمْوَاتٌ، مَوْتَى. ظَنًّا مِنْهُ. أَنْفَقَ. أَمْوَالٌ طَائِلَةٌ. اِعْتَقَدَ. كُلَّمَا كَانَ أَكْثَرَ كَانَ أَفْضَلَ. تَبَرَّكَ بِـ... صَاحِبُ القَبْرِ. أَلْصَقَ بِـ... تَمَسَّحَ بِـ... خِرْقَةٌ، خِرَقٌ. تُرْبَةٌ. نَفِيسٌ. يَا لَهُمْ مِنْ جُهَّالٍ! قَلَّ (ي) قِلَّةٌ. فَشَا (و) فُشُوٌّ. بَعُدَ (و) بُعْدٌ عَنْ... عَلَيْكَ بِـ... تِبْيَانٌ. (يَا) حَبَّذَا لَوْ... نَشَرَ (و) نَشْرٌ. أَنْشَأَ. كَسَا (و) كَسْوٌ. عَارٍ، عُرَاةٌ. بَعَثَ (ا) بَعْثٌ. بَيْتُ اللَّهِ الحَرَامُ. عَلَى سَبِيلِ العِبَادَةِ. ضَرِيحٌ، أَضْرِحَةٌ. نَعُوذُ بِاللَّهِ مِنْهُ. فِينَا مَنْ إِذَا دَخَلَ لَمْ يُسَلِّمْ.

ــــــــــــــــــــــــ

وَ مِنَ النَّاسِ مَنْ يَصْرِفُ مُعْظَمَ هَمِّهِ لِزِيَارَةِ قُبُورِ بَعْضِ المَوْتَى ظَنًّا مِنْهُمْ أَنَّهَا عِبَادَةٌ عَظِيمَةٌ يُتَقَرَّبُ بِهَا إِلَى اللَّهِ تَعَالَى، وَ يُنْفِقُونَ أَمْوَالاً طَائِلَةً فِي ذَلِكَ وَ يُسَافِرُونَ إِلَى آلاَفِ الكِيلُومِتْرَاتِ، وَ يَعْتَقِدُونَ أَنَّهُ كُلَّمَا كَانَتِ المَسَافَةُ أَبْعَدَ كَانَ الأَجْرُ أَعْظَمَ فَيَأْتُونَ القُبُورَ فَيُصَلُّونَ عِنْدَهَا وَ يَتَبَرَّكُونَ بِهَا وَ يَدْعُونَ أَصْحَابَهَا وَ يَسْأَلُونَهُمُ الشَّفَاعَةَ عِنْدَ اللَّهِ وَ كَثِيرًا مَا لاَ يَدْرُونَ مَنْ أَصْحَابُهَا وَ يُلْصِقُونَ خَدَّهُمْ بِالقَبْرِ وَ يَتَمَسَّحُونَ بِهِ، وَ قَدْ يَأْخُذُونَ تُرْبَةً أَوْ خِرْقَةً مِنْ قَبْرِهِ فَيَحْفَظُونَهَا فِي بَيْتِهِمْ كَشَيْءٍ نَفِيسٍ، فَيَا لَهُمْ مِنْ جُهَّالٍ! إِذَا قَلَّ العِلْمُ فَشَا الجَهْلُ فَظَهَرَتْ فِي النَّاسِ الأَبَاطِيلُ. هَذِهِ هِيَ نَتِيجَةُ بُعْدِ النَّاسِ عَنِ الكِتَابِ وَ السُّنَّةِ. فَعَلَيْكُمْ أَيُّهَا الصِّغَارُ بِتَعَلُّمِ الكِتَابِ وَ السُّنَّةِ فَفِيهِمَا عُلُومُ الدِّينِ وَ الدُّنْيَا، وَ فِيهِمَا الْحَلَالُ وَ الْحَرَامُ، وَ فِيهِمَا تِبْيَانُ كُلِّ شَيْئٍ. فَيَا حَبَّذَا لَوْ أَنَّ هَؤُلاءِ الْقُبُورِيُونَ أَنْفَقُوا أَمْوَالَهُمْ فِي تَعْلِيمِ الصِّغَارِ وَ نَشْرِ كُتُبِ الْإِسْلامِ، وَ إِنْشَاءِ الْمَدَارِسِ وَ بِنَاءِ الْمَسَاجِدِ، أَوْ أَطْعَمُوا الْجِيَاعَ أَوْ كَسَوُا الْعُرَاةَ، أَوْ بَعَثُوا بِهَا لِتَأْيِيدِ الْمُجَاهِدِينَ فِي سَبِيلِ اللهِ أَيْنَمَا كَانُوا سَوَاءَ فِي أَفْغَنِسْتَانَ أَوْ فَلَسْطِينَ أَوْ فِي أَرِتْرِيَا أَوْ فِي فِيلِبِّينَ أَوْ غَيْرِهَا. وَ كَثِيرٌ مِنَ النَّاسِ مَنْ إِذَا زَارَ الْقَبْرَ طَافَ بِهِ كَمَا يَطُوفُ الْحُجَّاجُ بِبَيْتِ اللهِ الْحَرَامِ، وَ الطَّوَافُ عَلَى سَبِيلِ الْعِبَادَةِ لَا يَجُوزٌ إِلَّا بِالْكَعْبَةِ، فَمَنْ طَافَ بِقَبْرِ وَلِيٍّ أَوْ ضَرِيحِ رَجُلٍ عَظِيمٍ عَلَى سَبِيلِ الْعِبَادَةِ فَقَدْ أَشْرَكَ، نَعُوذُ بِاللهِ.

\_\_\_\_\_\_\_\_\_\_\_\_\_\_\_\_\_\_\_\_\_

أجب على الأسئلة:

- هل يجوز إنفاق أموال طائلة في زيارة تلقبور و السفر إليها؟

- إذا زرنا قبر عالم أو رجل صالح فهل يجوز لنا أن ندعوه أو نسأله الشفاعة عند الله؟

- متى يفشو الجهل و تظهر الأباطيل في الناس؟

- ماذا يفشو الجهل و شظهر الأباطيل في الناس؟

- ماذا يجب علينا أن نعمل لكيلا تظهر الأباطيل في الناس؟

- ما هي نتيجة بعد الناس عن الكتاب و السنة؟

- أين ينبغي أن تنفق الأموال؟

\subsection{-١٢٥-}
زِيَارَةُ الْقُبُورِ مَشْرُوعَةٌ

مَشْرُوعٌ، بَشَرِيَّةٌ، أَبُو الْبَشَرِيَّةِ، أَخْفَى، مَوْقِعٌ، مَوَاقِعُ، ...كَذَا، فِي يَوْمِ كَذَا، أَفْضَلُ مِنْ... قُبَّةٌ، قِبَابٌ، قُبَّةُ الضَّرِيحِ، بَخِلَ (ا) بُخْلٌ عَلَى... بُخْلًا عَلَيْكَ، حَاشَ لِلَّهِ! لَعَنَ (ا) لَعْنٌ، يَهُودِيٌّ، يَهُودٌ، نَصْرَانِيٌّ، نَصَارَى، صَلُّوا عَلَيَّ، فِي بَعْضِ الأَحْيَانِ، دَعَا لَهُ، أُدْعُ لِى، شِرْكِيَّاتٌ، وَرَدَ فِي الْكِتَابِ، لَمْ يَرِدْ فِي الْحَدِيثِ، فِي... مَا تَقُولُ فِي هَذَا؟ أَذِنَ (ا) إِذْنٌ، أَذِنَ لِي، ذَكَّرَ.

\_\_\_\_\_\_\_\_\_\_\_\_\_

نَحْنُ نَقُولُ لِهَؤُلَاءِ الْقُبُورِيِّينَ: أَيْنَ قَبْرُ آدَمَ أَبِي الْبَشَرِيَّةِ كُلِّهَا؟ أَيْنَ قَبْرُ نُوحٍ وَ قَبْرُ إِبْرَاهِيمَ وَ قَبْرُ إِسْمَاعِيلَ وَ قَبْرُ إَسْحَاقَ؟ أَيْنَ قَبْرُ مُوسَى وَ قَبْرُ هَارُونَ؟ وَ أَيْنَ قُبُورُ غَيْرِهِمْ مِنْ كِبَارِ الرُّسُلِ وَ الْأَنْبُيَاءِ عَلَيْهِمْ الصَّلَوَاتُ وَ السَّلامُ. هَلْ زَارَ النَّبِيُّ صَلَّى اللهُ عَلَيْهِ وَ سَلَّمَ قُبُورَ هَؤُلاءِ؟ أَمْ جَعَلَ زِيَارَتَهَا عِبَادَةً عَظِيمَةً؟ أَمْ أَمَرَنَا فِي حَدِيثٍ أَنْ نَزُورَهَا؟ وَ لِمَاذَا أَخْفَى الله عَنَّا مَوَاقِعَ قُبُورِ هَؤُلاءِ وَ لَمْ يُخْبِرْنَا عَنْهَا تَفْصِيلًا؟ لِمَاذَا لَمْ يَقُلْ لَنَا نَبِيُّنَا قَبْرُ نَبِيِّ كَذَا فِي مَوْضِعِ كَذَا فَزُورُوهُ؟ أَمْ إِنَّ قُبُورَ مَشَايِخِهِمْ أَفْضَلُ مِنْ قُبُورِ الْأَنْبِيَاءِ وَ الْمُرْسَلِينَ؟ وَ لِمَاذَا لَمْ يَقُلْ لَنَا نَبِيُّنَا لِنَدْعُوَ عِنْدَ قَبْرِهِ أَوْ نَبْنِيَ عَلَيْهِ قُبَّةً أَوْ مَسْجِدًا؟ أَنَسِيَ النَّبِيُّ صَلَّى اللهُ عَلَيْهِ وَ سَلَّمَ أَمْ أَخْفَاهُ عَنَّا بُخْلًا عَلَيْنَا؟ حَاشَ لِلَّهِ! بَلْ قَالَ النَّبِيُّ صَلَّى اللهُ عَلَيْهِ وَ سَلَّمَ "لَعَنَ اللهُ الْيَهُودَ وَ النَّصَارَى اتَّخَذُوا قُبُورَ أَنْبِيَائِهِمْ مَسَاجِدَ" وَ قَالَ أَيْضًا: "لا تَجْعَلُوا بُيُوتَكُمْ قُبُورًا وَ لا تَجْعَلُوا قَبْرِي مَسْجِدًا وَ صَلُّوا عَلَيَّ فَإِنَّ صَلَاتَكُمْ تَبْلُغُنِي حَيْثُ كُنْتُمْ" إِنَّ زِيَارَةَ الْقُبُورِ أَمْرٌ مَشْرُوعٌ إِذَا قُمْنَا عَلَى الْوَجْهِ الَّذِي كَانَ يَقُومُ بِهَا النَّبِيُّ صَلَّى اللهُ عَلَيْهِ وَ سَلَّمَ فَإِنَّهُ كَانَ يَزُورُ الْقُبُورَ فِي بَعْضِ الْأَحْيَانِ فَيُسَلِّمُ عَلَى الْأَمْوَاتِ فِي الْقُبُورِ وَ يَدْعُو لَهُمْ وَ يَسْتَغْفِرُ لَهُمْ ثُمَّ يَعُودُ وَ لَا يَعْمَلُ شَيْئًا مِنَ الشِّرْكِيَّاتِ. حَتَّى لَمْ يَرِدْ أَنَّهُ قَرَأَ شَيْئًا مِنَ الْقُرْآنِ عَلَى الْقَبْرِ. وَ فِي زِيَارَةِ الْقُبُورِ قَالَ رَسُولُ اللهِ صَلَّى اللهُ عَلَيْهِ وَ سَلَّمَ :"كُنْتُ نَهَيْتُكُمْ عَنْ زِيَارَةِ الْقُبُورِ فَقَدْ أُذِنَ لِمُحَمَّدٍ فِي زِيَارِةِ قَبْرِ أُمِّهِ فَزُورُوهَا فَإِنَّهَا تُذَكِّرُكُمُ الآخِرَةَ"

\_\_\_\_\_\_\_\_\_\_\_\_

أجب على الأسئلة الآتية:

- ما حكم زيارة القبور في الإسلام

- كيف كانت زيارة الرسول صلى الله عليه و سلم للقبور؟ 

= صف لي زيارة الرسول للقبور؟

- هل كان الرسول صلى الله عليه و سلم يعمل الشركيات عند زيارة القبور؟

- ماذا قان الرسول صلى الله عليه و سلم في زيارة القبور؟

- ماذا ينبغي لنا أن نعمل عند زيارة القبور؟

- كم حديثا جاء في الدرس؟

- أيمكن أن يكتم عنا النبي شيئا من العبادات؟ أو أن ينسى أن يخبرنا بها؟

\subsection{-١٢٦-}
مِنْ أَيِّ مَصْدَرٍ نَأْخُذُ الدِّينَ

مَصْ\hyperlink{a0122outline}{-١٢٦-}دَرٌ، مَصَادِرُ، إِنْ قِيلَ لَكَ، تَمَسَّكَ بِ... مَا... مَا تَمَسَّكَ بِالْقُرْآنِ، ضَلَّ (ى) ضَلاَلٌ، ضَلاَلَةٌ، كُلُّ كِتَابٍ، كُلُّ الْكِتَابِ، تَدَبَّرِ، لا تَلْتَفِتْ إِلَى قَوْلِهِ، سَيِّدُ الْمُرْسَلِينَ، أَنَّى؟ أَنَّى لَنَا أَنْ نَفْهَمَهُ؟ كَيْدٌ، مَكْرٌ، مِنْ قِبَلِ... قَائِدٌ، قَادَةٌ، قَادَةُ الْكُفْرِ، صِهْيُونِيٌّ، صَهَايِنَةٌ، صَرَفَ عَنْ... سَهُلَ (و) سُهُولَةٌ عَلَى... اِسْتَوْلَى عَلَى... عَقْلٌ، عُقُولٌ، انْخَدَعَ، مَرِيضُ الْقَلْبِ، مَرْضَى الْقُلُوبِ، جَاهِلٌ بِالدِّينِ، مَكِيدَةٌ، مَكَايِدُ، مَكَايِدُ الشَّيْطَانِ، صَعُبَ (و) صُعُوبَةٌ عَلَى... اِحْتَرَزَ عَنْ... الدَّعْوَةُ إِلَى اللَّهِ، طَاغُوتٌ، طَوَاغِيتُ، وَ هُوَ لاَ يَدْرِي.

\_\_\_\_\_\_\_\_\_\_\_\_\_\_\_\_\_

إِنْ قِيلَ لَكَ: مِنْ أَيْنَ نَعْرِفُ مَا كَانَ يَفْعَلُهُ النَّبِيُّ صَلَّى اللهُ عَلَيْهِ وَ سَلَّمَ فَاقْرَأْ لَهُ الْحَدِيثَ الآتِي "تَرَكْتُ فِيكُمْ شَيْئَيْنِ مَا تَمَسَّكْتُمْ بِهِمَا فَلَنْ تَضْلُّوا كِتَابَ اللهِ وَ سُنَّةَ نَبِيِّهِ" وَ قُلْ لَهُ: إِنَّ فِي الْكِتَابِ وَ السُّنَّةِ كُلَّ الدِّينِ، فَإِذَا أَرَدْتَ أَنْ تَعْرِفَ الدَّينَ الصَّحِيحَ وَ الْعِبَادَةَ الصَّحِيحَةَ فَاقْرَأْهُمَا وَ ادْرُسَْهُمَا وَ تَدَبَّرْهُمَا وَ لَا تَلْتَفِتْ إِلَى فَوْلِ مَنْ قَالَ: "إِنَّنَا لَا نَفْهَمُ الْقُرْآنَ وَ لَا نَفْهَمُ الْحَدِيثَ لِأَنَّ الْقُرْآنَ كَلَامُ رَبِّ الْعَالَمْينَ وَ الْحَدِيثَ كَلَامُ سَيِّدِ الْمُرْسَلِينَ وَ نَحْنُ رِجَالٌ صِغَارٌ ضِعَافٌ جُهَّالٌ، فَأَنَّى لَنَا أَنْ نَفْهَمَهُمَا" فَهَذَا الْقَوْلُ كَيْدٌ عَظِيمٌ وِ مَكْرٌ شَدِيدٌ مِنْ قِبَلِ قَادَةِ الْكُفْرِ وِ أَئِمَّتِهِمْ لَا سِيَمَا الصَّهَايِنَةِ لِصَرْفِ جُهَّالِ الْمُسْلِمِينَ عَنْ كِتَابِ اللهِ وَ سُنَّةِ

\ رَسُولِهِ لِيَسْهُلَ عَلَيْهِمُ الاِسْتِبلاَءُ عَلَى عُقُولِ المُسْلِمِينَ فَانْخَدَعَ بِهِ مَرْضَى القُلُوبِ مِنَ المُسْلِمِينَ ، ِلأَنَّ المُسْلِمَ إِذَا كَانَ بَعِيدًا عَنْ كِتَابِ اللَّهِ وَ سُنَّةِ رَسُولِهِ لاَ يَقْرَؤُهُمَا إِلاَّ عَلَى الأَمْوَاتِ وَ القُبُورِ وَ لاَ يَتَدَبَّرُ مَعَانِيَهُمَا وَ لاَ يَعْمَلُ بِمَا فِيهِمَا كَانَ جَاهِلاً بِالدِّينِ وَ جَاهِلاً بِمَكَايِدِ الشَّيْطَانِ وَ جَاهِلاً بِأَنْوَاعِ الشِّرْكِ فَيَصْعُبُ عَلَيْهِ الاِحْتِرَازُ عَنِ الشَّيْطَانِ وَ مَكْرِهِ وَ عَنِ الشِّرْكِ فَيَقَعُ فِيهِ وَ هُوَ لاَ يَدْرِي كَمَا وَقَعَ فِيهِ القُبُورِيُّونَ. فَعَلَيْكُمْ أَيُّهَا الطَّلَبَةُ بِالكِتَابِ وَ السُّنَّةِ وَ عَلَيْكُمْ بِتَدَبُّرِ مَعَانِيهِمَا وَ العَمَلِ بِمَا فِيهِمَا. وَ مِنْ أَعْظَمِ مَا جَاءَ فِيهِمَا نَشْرُ الإِسْلاَمِ وَ دَعْوَةُ الخَلْقِ إِلَيْهِ وَ الجِهَادُ فِي سَبِيلِ اللَّهِ حَتَّى يَقُومَ حُكْمُ اللَّهِ فِي الأَرْضِ كُلِّهَا وَ يَذْهَبَ حُكْمُ الطَّاغُوتِ. 

ـــــــــــــــــــــ

أجب على الأسئلة الآتية:

-من أي مصدر يجب على المسلمين أن يعرفوا الدّين الصَحيح و العبَادة الصحيحة؟

-هل على حق من يقول نحن لا نفهم القرآن و الحديث؟

-ممن جاء هذا القول ؟ و لماذا جاء؟

-متى يسهل على الكافر الاستلاء على المسلم؟

-متى يصعب على المسلم الاحتراز عن الشيطان و عن الشّرك؟

-ماذا يجب على الطلبة حتى لا ينخدعوا بمثل تلك الأقوال؟

-كم شيئا ترك فينا النبي - صلى الله عليه و سلم - حتى لا نضلّ ؟ و ما هي؟

-هل حفظت الحديث الذي جاء في الدّرس؟

-إلى متى يجب علينا أن نجاهد؟

-متى ينتهى الجهاد في الأرض؟

-ماذا تفهم بحكم الطاغوت؟

Религиозные тексты

الإِيمَانُ

اِعْلَمْ أَنَّ أُصُولَ الدِّينِ ثَلاَثُ خِصَالٍ: الإِيمَانُ وَ الإِسْلاَمُ وَ الإِحْسَانُ . مَا هُوَ الإِيمَانُ؟

الإِيمَانُ هُوَ تَصْدِيقُ النَّبِيِّ- صلى الله عليه و سلم- فِي جَمِيعِ مَا أَخْبَرَ بِهِ عَنِ اللَّهِ تَعَالَى. وَ أَرْكَانُهُ أَيْ أَسَاسُهُ وَ قَوَاعِدُهُ سِتَّةٌ، وَ هِيَ: أَنْ تُؤْمِنَ بِاللَّهِ، وَ مَلاَئِكَتِهِ، وَ كُتُبِهِ، وَ رُسُلِهِ، وَ اليَوْمِ الآخِرِ، وَ بِالْقَدَرِ خَيْرِهِ وَ شَرِّهِ. 

الإِيمَانُ بِاللَّهِ

الأَوَّلُ مِنْ أَرْكَانِ الإِيمَانِ الإِيمَانُ بِاللَّهِ.

وَ مَعْنَى الإِيمَانِ بِاللَّهِ هُوَ أَنْ تَعْتَقِدَ أَنَّ اللَّهَ تَعَالَى مَوْجُودٌ، وَ أَنَّهُ وَاحِدٌ لاَ شَرِيكَ لَهُ، قَدِيمٌ لاَ أَوَّلَ لَهُ، بَاقٍ لاَ آخِرَ لَهُ، حَيٌّ لاَ يَمُوتُ، لَمْ يَزَلْ وَ لاَ يَزَالُ، مُخَالِفٌ لِلْحَوَادِثِ كُلِّهَا فِي ذَاتِهِ وَ صِفَاتِهِ وَ أَفْعَالِهِ، لاَ شَيْءَ يُمَاثِلُهُ، قَائِمٌ بِنَفْسِهِ، لاَ يَحْتَاجُ إِلَى مَحَلٍّ وَ لاَ مُوجِدٍ وَ لاَ إِلَى أَيِّ شَيْءٍ آخَرَ بَلْ كُلُّ شَيْءٍ مُحْتَاجٌ إِلَيْهِ تَعَالَى، سَمِيعٌ، بَصِيرٌ، عَالِمٌ لِلَأَشْيَاءِ كُلِّهَا مُتَكَلِّمٌ، مَرْئِيٌّ فِي الآخِرَةِ لِلْمُؤْمِنِينَ بِلاَ جِهَةٍ وَ لاَ كَيْفٍ، قَادِرٌ عَلَى كُلِّ شَيْءٍ، مُرِيدُ الخَيْرِ وَ الشَّرِّ وَ لَكِنْ لاَ يَرْضَى بِالشَّرِّ وَ لاَ يُحِبُّهُ. وَ تَعْتَقِدَ أَنَّ صِفَاتِ اللَّهِ جَمِيعَهَا قَدِيمَةٌ لاَ تُشْبِهُ صِفَاتِ المَخْلُوقِينَ فَعِلْمُهُ لَيْسَ كَعِلْمِنَا وَ سَمْعُهُ لَيْسَ كَسَمْعِنَا وَ بَصَرُهُ لَيْسَ كَبَصَرِنَا وَ حُبُّهُ وَ رِضَاهُ وَ غَضَبُهُ لَيْسَتْ كَحُبِّنَا وَ رِضَانَا وَ غَضَبِنَا. وَ تَعْتَقِدَ أَنَّهُ لَيْسَ لَهُ مَكَانٌ وَ لاَ جِهَةٌ، لاَ يُغَيِّرُهُ أَزْمَانٌ وَ لاَ يَتَغَيَّرُ عَلَيْهِ الزَّمَانُ.

وَ خُلاَصَةُ القَوْلِ إِنَّ اللَّهَ مَوْصُوفٌ بِجَمِيعِ صِفَاتِ الكَمَالِ وَ مُنَزَّهٌ عَنْ جَمِيعِ صِفَاتِ النُّقْصَانِ.

ــــــــــــــــــــ

 \includegraphics[width=2.5in,height=1.5728in]{images/MuhammadBagauddinprettified-img339.png}   \includegraphics[width=2.4791in,height=1.552in]{images/MuhammadBagauddinprettified-img340.png} 

1) 2)

 \includegraphics[width=2.5311in,height=1.5311in]{images/MuhammadBagauddinprettified-img341.png}   \includegraphics[width=2.3646in,height=1.6354in]{images/MuhammadBagauddinprettified-img342.png} 

\ 3) 4)

 \includegraphics[width=2.3752in,height=1.5626in]{images/MuhammadBagauddinprettified-img343.png}   \includegraphics[width=2.5in,height=1.5209in]{images/MuhammadBagauddinprettified-img344.png} 

\ 5) 6)

هَذِهِ سِتُّ صُورٍ يَظْهَرُ فِي جَمِيعِهَا وَلَدٌ وَاحِدٌ . وَ هِيَ مَوْضُوعَةٌ حَسَبَ تَرْتِيبِ أَعْمَالِهِ اِبْتِدَاءً مِنْ نَوْمِهِ حَتَّى ذَهَابِهِ إِلَى المَدْرَسَةِ. وَ كُلُّ صُورَةٍ مِنْهَا تُشِيرُ إِلَى عَمَلٍ مِنْ أَعْمَالِهِ. مِنْ وَاجِبِكَ أَنْ تَنْظُرَ إِلَى كُلِّ صُورَةٍ وَ تَذْكُرَ عَمَلَ الوَلَدِ فِيهَا عَلَى التَّرْتِيبِ وَ تَذْكُرَ أَسْمَاءَ الأَشْيَاءِ المَوْجُودَةِ فِي يَدِهِ وَ أَمَامَهُ وَ عَلَى يَمِينِهِ وَ عَلَى يَسَارِهِ وَ مِنْ وَرَائِهِ. وَ اذْكُرْ أَسْمَاءَ الغُرَفِ فِي الصُّورَةِ الأُولَى وَ الثَّالِثَةِ وَ الرَّابِعَةِ.

ـــــــــــــــــــــ

أسئلة للمناقشة:

١- كم هي أصول الدين؟

٢- ماذا تفهم بالإيمان؟

٣- كم أركان الإيمان؟ و ما هي؟ 

٤- و ما هو أول ركن من أركان الإيمان؟

٥- ما معنى الإيمان بالله؟

٦- هل يحتاج الله إلى شيء؟

٧- اذكر لي بعض صفات الله؟

٨- صفات الله حادثة ام قديمة؟ 

٩- متى يرى المؤمنون الله ؟ و كيف؟

١٠- هل الله مريد للشرّ؟

١١- هل يرضى الله بالشرّ؟

١٢- هل لله مكان و جهة؟

Иман (вера)

Знай, что основами религии являются три свойства: иман (вера, уверование), ислам (покорность, предание себя Богу), ихсан (совершение деяний превосходно, наилучшим образом).

Что такое иман?

Иман — это признание Пророка, — Да благословит Бог его и приветствует, — во всём том, что он сообщил от Бога Всевышнего. Иман имеет шесть столпов, т.е. основоположений. Это: вера в Бога, Его ангелов, Его книги, Его Посланников, Судный день, предопреде­ление как в доброе, так и в злое.

Вера в Бога

Первый из столпов имана — это вера в Бога. Смысл веры в Бога заключается в том, что ты убеждён, что Всевышний Бог существует, и что Он один без сотоварищей, Предвечный без начала, Вечный без конца, Живой и Бессмертный, Он не переставал и не перестанет быть. По своей сути, качествам и действиям отличающийся от всех сотворнёных вещей. Нет ничего подобного Ему. Сущий сам по себе. Он не нуждается ни в месте, ни в создателе и ни в чём ином, напротив, всё нуждается в Нём, Всевышнем. Он Всеслышащий, Всевидящий, Всезнающий, Говорящий, Видимый для правоверных в загробной жизни без стороны и без образа. Он Всемогущий, Имеющий волю над добром и злом, но не Довольный злом и не Любящий его. И что ты убеждён, что все свойства Бога предвечны и не подобны свойствам сотворнёных вещей. Его знание не такое, как наше знание, Его слух не такой, как наш слух, Его зрение не такое, как наше зрение, Его любовь, Его довольство и Его гнев не такие, как наша любовь, наше довольство и наш гнев. И что ты убеждён, что у Него нет ни места, ни стороны. Его не меняют времена, и время для Него не меняется.

Сказать вкратце: Богу свойственны все качества совершенства и чужды все отрицательные качества и недостатки.

الإِيمَانُ بِالمَلاَئِكَةِ

الثَّانِي مِنْ أَرْكَانِ الإِيمَانِ الإِيمَانُ بِالمَلاَئِكَةِ.

وَ مَعْنَى الإِيمَانِ بِالمَلاَئِكَةِ هُوَ أَنْ تَعْتَقِدَ أَنَّ المَلاَئِكَةَ مَوْجُودُونَ وَ أَنَّهُمْ أَجْسَامٌ لَطِيفَةٌ مَخْلُوقَةٌ مِنْ نُورٍ، لاَ يَأْكُلُونَ وَ لاَ يَشْرَبُونَ، وَ لاَ يُوصَفُونَ بِالذُّكُورَةِ وَ لاَ الأُنُوثَةِ، وَ هُمْ عِبَادُ اللَّهِ المُكْرَمُونَ، يَعْبُدُونَهُ وَ لاَ يَعْصُونَهُ لَحْظَةً، وَ هُمْ مَعْصُومُونَ مِنَ الذُّنُوبِ، دَأْبُهُمْ طَاعَةُ اللَّهِ، وَ لاَ يَعْلَمُ عَدَدَهُمْ إِلاَّ اللَّهُ، وَ هُمْ يَمُوتُونَ وَ يُبْعَثُونَ يَوْمَ القِيَامَةِ. مِنْهُمْ جَبْرَائِيلُ الَّذِي يَنْزِلُ بِالوَحِي عَلَى أَنْبِيَاءِ اللَّهِ وَ رُسُلِهِ. وَ مَلَكُ المُوْتِ المُوَكَّلُ بِقَبْضِ الأَرْوَاحِ. وَ مِيكَائِيلُ المُوَكَّلُ بِالمَطَرِ. وَ إِسْرَافِيلُ المُعَدُّ لِلنَّفْخِ فِي الصُّورِ. وَ رِضْوَانُ خَازِنُ الجَنَّةِ. وَ مَالِكٌ خَازِنُ النَّارِ. وَ مِنْهُمْ حَمَلَةُ العَرْشِ وَ هُمْ فِي أَيَّامِ الدُّنْيَا أَرْبَعَةٌ وَ يَوْمَ القِيَامَةِ ثَمَانِيَةٌ. وَ مِنْهُمْ مَنْ لَهُ وَظَائِفُ أُخْرَى. 

الإِيمَانُ بِالكُتُبِ

الثَّالِثُ مِنْ أَرْكَانِ الإِيمَانِ الإِيمَانُ بِالكُتُبِ.

وَ مَعْنَى الإِيمَانِ بِالكُتُبِ هُوَ أَنْ تَعْتَقِدَ أَنَّ لِلَّهِ تَعَالَى كُتُبًا أَنْزَلَهَا عَلَى رُسُلِهِ وَ بَيَّنَ فِيهَا أَمْرَهُ وَ نَهْيَهُ، وَ هِيَ كَلاَمُ اللَّهِ تَعَالَى حَقِيقَةً غَيْرُ مُخُلُوقٍ وَ أَنْزَلَهَا وَحْيًا. مِنْ تِلْكَ الكُتُبِ التَّوْراَةُ المُنَزَّلَةُ عَلَى مُوسَى، وَ الزَّبُورُ المُنَزَّلُ عَلَى دَاوُدَ، وَ الإِنْجِيلُ المُنَزَّلُ عَلَى عِيسَى، وَ القُرْآنُ المُنَزَّلُ عَلَى مُحَمَّدٍ صَلَوَاتُ اللَّهِ وَ سَلاَمُهُ عَلَيْهِمْ أَجْمَعِينَ. وَ مِنْهَا الصُّحُفُ المُنَزَّلَةُ قَبْلَ هَذِهِ الأَرْبَعَةِ. وَ القُرْآنُ أَعْظَمُهَا وَ أَشْرَفُهَا وَ نَاسِخٌ لِجَمِيعِ مَا قَبْلَهُ. وَ حُكْمُهُ بَاقٍ إِلَى يَوْمِ القِيَامَةِ.

ـــــــــــــــــــــ

أسئلة للمناقشة:

١ - ما هو الركن الثاني من أركان الإيمان ؟ و الثالث؟

٢ - كيف نُؤمنُ بالملائكةِ؟

٣ - من أيِّ شيء خلقتِ الملائكةُ؟

٤ - هل تأكل الملائكة أو تشرب؟

٥ - هل تُوصف الملائكة بالذكورة أو الأنوثةِ؟

٦- ما هو دأب الملائكة؟

٧- كم عددُ الملائكةِ؟

٨- متى يموتون و متى يُبعثون؟

٩- اذكر لي أسماء كبار الملائكة؟

١٠- اذكر لي وظائِف بعض الملائكة؟

١١- كيف نؤمن بكتب الله؟

١٢- كيف أنزل الله كتبه على رسله؟

١٣- على من نزلت التوراة ؟ الزبور؟

١٤- على من نزل الإنجيل؟

١٥- على من أنزل الله القرآن؟

١٦ - أيّ هذه الكتب أعظم؟

١٧- أيّ هذه الكتب آخرها نزولا؟

١٨- بأيّ كتاب نحكم؟

Вера в ангелов

\ Вторым столпом имана является вера в ангелов.

Смысл веры в ангелов заключается в том, что ты убеждён, что ангелы существуют, и что они бесплотные тела, созданные из света. Они не едят и не пьют, и не характеризуемы ни мужским, ни женским полом. И они — почётные рабы Бога. Они поклоняются Ему и не ослушиваются Его ни на миг. Они застрахованы от грехов. Их де­ло — повиноваться Богу. Их числа не знает никто, кроме Бога. И они умирают и возрождаются в День Воскресения.

Среди них — Джабраиль, который спускается с откровением к пророкам и Посланникам Бога. И Ангел смерти, имеющий поручение забирать души умерших. И Микаиль, ответственный за дождь. И Исрафиль, предназначенный для дутья в трубу. И Ризван, хранитель рая. И Малик, хранитель ада. Есть среди них и носители Престола. В мирской жизни их четверо, а в Судный день их восемь. Есть среди них и такие, которые имеют другие функции.

Вера в книги

Третий столп имана — это вера в книги.

Смысл веры в книги заключается в том, что ты убеждён, что Богу Всевышнему принадлежат книги, которые Он ниспослал Своим Посланникам, и в которых Он разъяснил Свои повеления и запреты. Книги эти — действительно слово Бога Всевышнего, несотворнёное, которое Бог ниспослал в виде откровения. К этим книгам относится Таврат (Тора), ниспосланный Мусе (Моисею), Инджиль (Евангелие), ниспосланный Исе (Иисусу), Забур (Псалтирь), ниспосланный Дауду, и Коран, ниспосланный Мухаммаду — Благословение Бога и благополучие всем им. К этим книгам относятся и свитки, ниспосланные раньше этих четырёх книг. Коран является самой великой и самой почётной из них, и отменяющим всё, что было до него. Действие Корана остаётся вплоть до Судного дня.

الإِيمَانُ بِالرُّسُلِ

الرَّابِعُ مِنْ أَرْكَانِ الإِيمَانِ الإِيمَانُ بِالرُّسُلِ.

وَ مَعْنَى الإِيمَانِ بِالرُّسُلِ هُوَ أَنْ تَعْتَقِدَ أَنَّ لِلَّهِ تَعَالَى رُسُلاً رِجَالاً أَرْسَلَهُمْ إِلَى الخَلْقِ مُبَشِّرِينَ لِلْمُحْسِنِ بِالثَّوَابِ وَ مُنْذِرِينَ لِلْمُسِيءِ بِالعِقَابِ وَ مُبَيِّنِينَ لَـهُمْ مَا يَحْتَاجُونَ إِلَيْهِ مِنْ أُمُورِ الدُّنْيَا وَ الدِّينِ. وَ هُمْ دُعَاةُ الخَلْقِ إِلَى الحَقِّ. وَ هُمْ خَيْرُ البَشَرِ. مِنْهُمْ خَمْسَةٌ وَ عِشْرُونَ مَذْكُورُونَ فِي القُرْآنِ العَظِيمِ، وَ هُمْ: آدَمُ، إِدْرِيسُ، نُوحٌ، هُودٌ، صَالِحٌ، إِبْرَاهِيمُ، لُوطٌ، إِسْمَاعِيلُ، إِسْحَاقُ، يَعْقُوبُ، يُوسُفُ، أَيُّوبُ، شُعَيْبٌ، مُوسَى، هَارُونُ، ذُو الكِفْلِ، دَاوُدُ، سُلَيْمَانُ، إِلْيَاسُ، أَلْيَسَعُ، يُونُسُ، زَكَرِيَّا، يَحْيَى، عِيسَى، مُحَمَّدٌ، عَلَيْهِمْ جَمِيعًا صَلَوَاتُ اللَّهِ تَعَالَى وَ سَلاَمُهُ.

وَ يَجِبُ الإِيمَانُ بِبَقِيَّةِ الأَنْبِيَاءِ وَ الرُّسُلِ إِجْمَالاً مِنْ غَيْرِ حَصْرٍ، بِأَنْ يَقُولَ: آمَنْتُ بِأَنْبِيَاءِ اللَّهِ تَعَالَى وَ رُسُلِهِ.

وَ مِمَّا يَجِبُ اِعْتِقَادُهُ أَنَّ نَبِيَّنَا مُحَمَّدًا- صلى الله عليه و سلم- خَاتَمُ النَّبِيِّينَ أَيْ لاَ نَبِيَّ بَعْدَهُ. وَ رِسَالَتُهُ عَامَةٌ لِكَافَّةِ الخَلْقِ مِنْ عَرَبٍ وَ تُرْكٍ وَ صِينِيِّينَ وَ رُوسٍ وَ أَمْرِيكَانَ وَ غَيْرِهِمْ. لاَ كَسَائِرِ الرُّسُلِ مِنْ قَبْلِهِ الَّذِينَ بُعِثُوا إِلَى أَقْوَامِهِمْ وَ أُمَمِهِمْ خَاصَّةً.

وَ خَيْرُ النَّاسِ بَعْدَ الأَنْبِيَاءِ أَصْحَابُ نَبِيِّنَا مُحَمَّدٍ- صلى الله عليه و سلم- أَفْضَلُهُمْ أَبُو بَكْرٍ فَعُمَرُ فَعُثْمَانُ فَعَلِيٌّ رِضْوَانُ اللَّهِ تَعَالَى عَلَيْهِمْ أَجْمَعِينَ.

ـــــــــــــــــــــ

أسئلة للمناقشة:

١- ما معنى الإيمان بالرّسل؟

٢- إِلاَمَ يدعو الرّسل الخلق؟

٣- من هم خير البشر؟

٤- كم نبيا ذكر اسمه في القرآن؟

٥- هل حفظت أسماء الرّسل المذكورين في القرآن؟

٦- من هو آخر الرّسل؟

٧- أي فرق بين الرّسول محمّد و بين سائر الرّسل؟

٨- في أي قوم بُعث محمد- صلى الله عليه و سلم؟

٩- أبُعِث محمد-صلى الله عليه و سلم- إلى العرب فقط أم إلى الناس جميعًا؟

١٠- من هم خير الناس بعد الأنبياء؟

١١- بعض الشعوب يقول: أن محمّدا بعث إلى العرب فقط، فهل هم على حقّ فيما يقولون؟

١٢- أيكون الإنسان مسلما مؤمنا إن لم يؤمن بمحمد؟

١٣- هل بعث بعد النبي محمد نبيّ آخر؟

Вера в Посланников

Четвёртый столп имана — это вера в Посланников.

Смысл веры в Посланников заключается в твоей убеждённости в существовании у Бога Всевышнего Посланников — мужчин, которых Он послал к людям, чтобы передать благочестивому радостную весть о вознаграждении, предупредить нечестивца о наказании и разъяснить людям, что нужно им из мирских и религиозных дел. Посланники — глашатаи истины среди людей. Это лучшие представители человечества. Имена 25 из них упомянуты в великом Коране. Это — Адам, Идрис, Нух, Худ, Салих, Ибрахим, Лут, Исмаиль, Исхак, Яакуб, Юсуф, Аййуб, Шуайб, Муса, Харун, Зуль-Кифль, Дауд, Сулайман, Ильяс, Аль-Ясаа, Юнус, Закариййа, Яхья, Иса, Мухаммад — Да ниспошлются им всем благословения Бога Всевышнего и Его салям.

Необходимо верить и в остальных Пророков и Посланников в целом без ограничения, сказав: „Я уверовал в Пророков Бога Всевышнего и Его Посланников".

Необходимо быть убеждённым и в том, что наш Пророк Мухам­мад — Да пошлёт Бог ему салят и салям — печать Пророков, т.е. нет после него ни одного Пророка, и его миссия универсальна для всех народов — арабов, турок, китайцев, русских, американцев и других, а не как у остальных Посланников, бывших до него, которые были посланы исключительно к своим народам и нациям.

Лучшие люди после Пророков — это сподвижники нашего Пророка Мухаммада — Да пошлёт Бог ему салят и салям — Наилучший из них — Абу Бакр, потом Умар, потом Усман, потом Али — Да будет Бог Всевышний доволен всеми ими.

الإِيمَانُ بِاليَوْمِ الآخِرِ

الخَامِسُ مِنْ أَرْكَانِ الإِيمَانِ الإِيمَانُ بِاليَوْمِ الآخِرِ.

وَ اليَوْمُ الآخِرُ هُوَ يَوْمٌ فَظِيعٌ عَظِيمٌ الأَهْوَالِ تَشِيبُ فِيهِ الأَطْفَالُ. وَ مَعْنَى الإِيمَانِ بِهِ أَنْ تَعْتَقِدَ أَنَّ اللَّهَ تَعَالَى يَبْعَثُ الخَلْقَ مِنْ قُبُورِهِمْ يَوْمَ القِيَامَةِ وَ يَحْشُرُهُمْ إِلَى المَوْقِفِ لِلْحِسَابِ، وَ يَضَعُ المِيزَانَ لِوَزْنِ أَعْمَالِ العِبَادِ، وَ يُحَاسِبُ الخَلْقَ ثُمَّ يُعْطِيهِمْ كُتُبَ الأَعْمَالِ إِمَّا بِالْيَمِينِ وَ إِمَّا بِالشِّمَالِ، وَ يَنْصِبُ الصِّرَاطَ وَ هُوَ جِسْرٌ مَمْدُودٌ عَلَى مَتْنِ جَهَنَّمَ أَدَقُّ مِنَ الشَّعَرِ وَ أَحَدُّ مِنَ السَّيْفِ لِيَمُرَّ النَّاسُ. فَبَعْضُهُمْ يُدْخِلُهُمْ الجَنَّةَ- وَ هِيَ دَارُ النَّعِيمِ المُقِيمِ- بِفَضْلِهِ، وَ بَعْضُهُمْ يُدْخِلُهُمْ النَّارَ- وَ هِيَ دَارُ العَذَابِ المُقِيمِ- بِعَدْلِهِ. وَ يَجِبُ الاِعْتِقَادُ بِأَنَّ سُؤَالَ مُنْكَرٍ وَ نَكِيرٍ فِي القَبْرِ حَقٌّ، وَ عَذَابَ القَبْرِ وَ نَعِيمَهُ حَقٌّ، وَ حَشْرَ الأَجْسَادِ حَقٌّ، وَ الحَوْضَ وَ الشَّفَاعَةَ حَقٌّ.

ـــــــــــــــــــــ

الإِيمَانُ بِالقَدَرِ

السَّادِسُ مِنْ أَرْكَانِ الإِيمَانِ الإِيمَانُ بِالقَدَرِ.

وَ مَعْنَى الإِيمَانِ بِالقَدَرِ هُوَ أَنْ تَعْتَقِدَ أَنَّ جَمِيعَ مَا يَجْرِي فِي العَالَمِ خَيْرًا كَانَ أَوْ شَرًّا، وَ كَذَلِكَ جَمِيعُ أَفْعَالِ العِبَادِ، بِإِرَادَةِ اللَّهِ تَعَالَى وَ تَقْدِيرِهِ وَ خَلْقِهِ. فَالخَلْقُ وَ التَّقْدِيرُ مِنَ اللَّهِ تعالى وَ الفِعْلُ مِنَ العِبَادِ بِاخْتِيَارِهِمْ وَ هُمَا يَجْرِيَانِ مَعًا. فَالعِبَادُ مُخْتَارُونَ لاَ مَجْبُورُونَ. وَ لَوْ لاَ ذَلِكَ لَكَانَ تَكْلِيفُهُمْ وَ بِعْثَةُ الأَنْبِيَاءِ إِلَيْهِمْ وَ إِنْزَالُ الكُتُبِ عَلَيْهِمْ عَبَثًا، وَ مَا كَانَ هُنَاكَ اسْتِحْقَاقٌ لِلْثَّوَابِ أَوْ العِقَابِ.

ـــــــــــــــــــــ

أسئلة للمناقشة:

١- ما هو اليوم الآخر و ما معنى الإيمان؟

٢- لماذا يحشر الله الخلق إلى الموقف؟

٣- لماذا يضع الله الميزان؟

٤- ما هو الصراط؟ و لماذا ينصب؟

٥- ما هي الجنة؟ و من يدخل الله الجنة؟

٦- ما هي النار؟ و من يدخل الله النار؟

٧- هل في القبر عذاب؟ أو نعيم؟

٨- هل تحشر أجسادنا يوم القيامة؟

٩- متى يوم القيامة؟

١٠- كيف الإيمان بالقدر؟

١١- هل العبد مجبور على عمله أم هو مختار؟

١٢- ما هو آخر ركن من أركان الإيمان؟

Вера в Судный день

Пятый из столпов имана — вера в Судный день.

Судный день — это страшный день, преисполненный великих ужасов, в который седеют дети. Смысл веры в него заключается в твоей убеждённости, что в День Воскресения Всевышний Бог поднимет людей из могил, соберёт их на стоянку для расчёта, установит весы для взвешивания поступков рабов Своих и произведёт расчёт с людьми. Затем Он вручит им книги, где записаны их дела, либо в правую руку, либо в левую руку. Затем Он проложит Сират (путь) — это представляет собой мост, перекинутый над адом, тоньше волоса и острее лезвия меча, чтобы по нему проходили люди. Одних из них Бог введёт в рай — обитель вечного блаженства, по Своей милости, а других введёт в огонь — обитель постоянного мучения, по Своей справедливости.

Необходима убеждённость, что допрос Мункара и Накира (два анге­ла, которые допрашивают мёртвых в могилах сразу же после захоро­нения) в могиле — истина, что муки могилы и блаженство её — истина, что телесное оживление — истина, что Хауз (бассейн, где напьются правоверные до входа в рай) и шафаат (заступничество) — истина.

Вера в предопределение

Шестой из столпов имана — вера в предопределение.

Смысл веры в предопределение заключается в твоей убеждённости в том, что всё происходящее в мире, будь то доброе или злое, а также все поступки рабов божьих обусловлено волей Бога Всевышнего, Его предопределением и творением. Творение и предоп­ределение от Бога Всевышнего, а действие от рабов Бога по их доброй воле." Эти два процесса происходят синхронно. Рабы Бога имеют свободу выбора, а не принуждаются. Если бы это было не так, то их обложение, миссия Пророков у них и ниспослание им Книг, было бы напрасным, и не было бы тогда заслуженности вознаграждения или наказания.

الإِسْلاَمُ

مَا هُوَ الإِسْلاَمُ؟

الإِسْلاَمُ هُوَ الاِنْقِيَادُ ِلأَوَامِرِ اللَّهِ وَ نَوَاهِيهِ. وَ أَرْكَانُهُ خَمْسَةٌ، وَ هِيَ: شَهَادَةُ أَنْ لاَ إِلَهَ إِلاَّ اللَّهُ وَ أَنَّ مُحَمَّدًا رَسُولُ اللَّهِ، وَ إِقَامُ الصَّلاَةِ، وَ إِيتَاءُ الزَّكَاةِ، وَ صَوْمُ رَمَضَانَ، وَ حَجُّ البَيْتِ مَنِ اسْتَطَاعَ إِلَيْهِ سَبِيلاً.

١) فَمَعْنَى الشَّهَادَتَيْنِ هُوَ أَنْ يُقِرَّ المُسْلِمُ بِاللِّسَانِ وَ يُصَدِّقَ بِالقَلْبِ بِأَنَّ الإِلَهَ الحَقَّ الَّذِي يَسْتَحِقُّ العِبَادَةَ هُوَ اللَّهُ وَحْدَهُ، وَ أَنَّ مُحَمَّدًا نَبِيٌّ أَرْسَلَهُ اللَّهُ لِيَهْدِيَ النَّاسَ طَرِيقَ الحَقِّ وَ يُخْرِجَهُمْ مِنَ الظُّلُمَاتِ إِلَى النُّورِ.

٢) وَ مَعْنَى إِقَامِ الصَّلاَةِ أَنْ يُؤَدِّيَ المُسْلِمُ الصَّلَوَاتِ الخَمْسَ المَفْرُوضَةَ كُلَّ يَوْمٍ وَ لَيْلَةٍ فِي أَوْقَاتِهَا بِحُقُوقِهَا.

٣) وَ مَعْنَى إِيتَاءِ الزَّكَاةِ هُوَ أَنْ يُعْطِيَ المُسْلِمُ الفَقِيرَ حِصَّةً مِنْ مَالِهِ كُلَّ سَنَةٍ كَيْ يَعِيشَ الفَقِيرُ وَ يَحْفَظَ اللَّهُ مَالَ الغَنِيِّ.

٤) وَ مَعْنَى صَوْمِ رَمَضَانَ هُوَ أَنْ يَصُومَ المُسْلِمُ فِي السَّنَةِ شَهْرًا كَامِلاً هُوَ شَهْرُ رَمَضَانَ يَمْتَنِعُ فِيهِ عَنِ الأَكْلِ وَ الشُّرْبِ وَ سَائِرِ المُفَطِّرَاتِ مِنْ طُلُوعِ الفَجْرِ إِلَى غُرُوبِ الشَّمْسِ.

٥) وَ مَعْنَى حَجِّ البَيْتِ أَنْ يَذْهَبَ مُسْلِمٌ قَادِرٌ مَرَّةً فِي عُمْرِهِ إِلَى مَكَّةَ المُكَرَّمَةِ وَ يَطُوفَ حَوْلَ الكَعْبَةِ وَ يَصْعَدَ إِلَى عَرَفَاتٍ مُحْرِمًا. وَ هُنَاكَ يَجْتَمِعُ بِإِخْوَانِهِ المُؤْمِنِينَ وَ يَتَعَرَّفُ إِلَيْهِمْ.

ـــــــــــــــــــــ

أسئلة للمناقشة:

١- ما هو الإسلام؟

٢- كم أركان الإسلام؟

٣- ما هو أول ركن و أخر ركن للإسلام؟

٤- ما معنى الشهادتين؟ ما معنى إقام الصّلاة؟ ما معنى إيتاء الزكاة؟

٥- ما معنى صوم رمضان؟ ما معنى حج البيت؟

٦- هل يجب حج البيت على كل مسلم و مسلمة؟

٧- أتجب الصلاة على غير المسلم؟

٨- ما هو شهر الصوم؟

٩- عم يمتنع الصّائم؟

١٠- كم صلاة يؤدي المسلم كل يوم و ليلة؟

١١- كم مرّة يجب على المسلم حج البيت؟

١٢- إذا حج المسلم مرة فهل يجب عليه أن يحج ثانية؟

١٣- ما فائدة الزكاة؟

Что такое Ислам?

Ислам — это повиновение повелениям Бога и Его запретам, [слам имеет пять столпов, это — приведение свидетельства, что нет никакого божества, кроме Бога, и Мухаммад — Посланник Бога; ыстаивание намаза; выдача закята; соблюдение уразы; совершение хаджа к Каабе для тех, кто имеет на то возможность.

1. Смысл обеих частей свидетельства заключается в том, что [усульманин признает языком и верит сердцем, что истинное божество, остойное поклонения — только один Бог, и что Мухаммад — Пророк, посланный Богом, чтобы наставить людей на истинный путь и вывести их из мрака к свету.

2. Смысл выстаивания намаза состоит в том, что мусульманин овершает пять обязательных намазов каждые сутки в указанные сроки

соответствующим ритуалом.

3. Смысл выдачи закята состоит в том, что мусульманин ежегодно »тдаёт бедному часть своего имущества, чтобы и бедный проживал, и югатому Бог сохранял имущество.

4. Смысл соблюдения уразы в месяц рамадан заключается в том, что мусульманин постится в году в течение полного месяца рамадана, во время которого он воздерживается от еды, питья и других вещей, нарушающих уразу, от восхода зари и до захода солнца.

5. Смысл хаджа (паломничества) к Каабе заключается в том, что имеющий возможность мусульманин раз в жизни отправляется в Высокочтимую Мекку, совершает таваф (обход) вокруг Каабы, восходит на гору Арафат, вступив в святое паломничество. Там он встречается со своими правоверными братьями и знакомится с ними.

الإِحْسَانُ

مَا هُوَ الإِحْسَانُ؟

الإِحْسَانُ هُوَ أَنْ تَعْبُدَ اللَّهَ كَأَنَّكَ تَرَاهُ فَإِنْ لَمْ تَكُنْ تَرَاهُ فَإِنَّهُ يَرَاكَ. وَ الإِحْسَانُ هُوَ المَقَامُ الَّذِي يَنْبَغِي لِكُلِّ وَاحِدٍ أَنْ يَسْعَى إِلَيْهِ حَتَّى يَبْلُغَهُ فَإِذَا بَلَغَهُ أَدَّى كُلَّ عَمَلٍ مِنْ أَعْمَالِهِ عَلَى أَحْسَنِ وَجْهٍ يُرْضِي اللَّهَ مُسْتَشْعِرًا رُؤْيَةَ اللَّهِ لَهُ إِذْ لاَ شَكَّ فِي رُؤْيَتِهِ لَهُ. وَ كَمَالُ الإِيمَانِ إِقْرَارٌ بِاللِّسَانِ وَ تَصْدِيقٌ بِالجَنَانِ وَ عَمَلٌ بِالأَعْضَاءِ كَالصَّلَوَاتِ الخَمْسِ وَ نَحْوِهَا وَ اِتِّبَاعُ السُّنَّةِ. وَ السُّنَّةُ هِيَ طَرِيقَةُ النَّبِيِّ- صلى الله عليه و سلم- فِي أَقْوَالِهِ وَ أَفْعَالِهِ وَ أَحْوَالِهِ.

فَمَنْ تَرَكَ الإِقْرَارَ بِاللِّسَانِ بِأَنِ امْتَنَحَ عَنِ الشَّهَادَتَيْنِ فَهُوَ كَافِرٌ. وَ مَنْ تَرَكَ التَّصْدِيقَ بِالجَنَانِ بِأَنْ أَقَرَّ بِاللِّسَانِ وَ لَمْ يُصَدِّقْ بِقَلْبِهِ فَهُوَ مُنَافِقٌ يُخَلِّدُهُمَا اللَّهُ فِي النَّارِ أَبَدًا. وَ مَنْ تَرَكَ العَمَلَ بِالأَعْضَاءِ فَهُوَ فَاسِقٌ. وَ مَنْ تَرَكَ اِتِّبَاعَ السُّنَّةِ فَهُوَ مُبْتَدِعٌ ضَالٌّ ِلأَنَّهُ تَرَكَ سُنَّةَ النَّبِيِّ وَ ابْتَدَعَ طَرِيقًا لَمْ يَكُنْ عَلَيْهِ النَّبِيُّ-صلى الله عليه و سلم- وَ لاَ الصَّحَابَةُ يَجِبُ عَلَيْهِمَا التَّوْبَةُ فَوْرًا.

ـــــــــــــــــــــ

أسئلة للمناقشة:

١- ما هو الإحسان؟

٢- بماذا يكون كمال الإيمان؟

٣- ما هي السنة؟

٤- ما يقال شرعا لمن ترك الإقرار باللّسان؟

٥- ما يقال شرعا لمن ترك التصديق بالجنان؟

٦- من هو الكافر؟

٧- أين يكون الكافر يوم القيامة؟

٨- من هو المنافق؟

٩- هل يكون المنافق مؤمنا؟

١٠- من هو المبتدع؟

١١- من هو الفاسق؟

١٢- ماذا يجب على المبتدع و الفاسق؟

١٣- هل فينا مبتدعون؟

Ихсан

Что такое ихсан?

Ихсан — это поклонение Богу, как будто ты видишь Его, поскольку, хотя ты Его не видишь, Он видит тебя. Ихсан — это положение, к достижению которого должен стремиться каждый, пока не достигнет его. Когда же он достигнет его, тогда всякое действие он будет выполнять наилучшим образом, удовлетворяющим Бога, чувствуя на себе взгляд Бога, так как несомненно, что Он видит его.

Полнота имана состоит: 1. в устном признании; 2. в уверовании душой; 3. в действии органами, как, например, совершение пятикрат­ного намаза и т.д.; 4. в следовании сунне. Сунна же — жизненный путь Пророка Мухаммада — Да благословит Бог его и привет­ствует — в его высказываниях, поступках и различных ситуациях. Тот, кто оставил устное признание, отказавшись от приведения обеих частей свидетельства, тот — кафир (неверный). А тот, кто перестал верить душой, т.е., признав на словах, не поверил душой, тот — мунафик (лицемер, двуличник). Их обоих вечно заставит Бог пребывать в адском огне. Тот же, кто перестал действовать органами, тот — фасик (нечестивый, ослушавшийся). А тот, кто перестал следовать сунне, тот — заблудший еретик, поскольку он оставил сунну Пророка и выдумал другой путь, по которому не шли ни Пророк, — Салят ему и салям — ни его последователи. Этим обоим следует немедленно покаяться.

الوُضُوءُ

١) اِجْلِسْ لِلْوُضُوءِ فِي مَكَانٍ طَاهِرٍ مُسْتَقْبِلَ القِبْلَةِ وَ انْوِ بِقَلْبِكَ الوُضُوءَ، ثُمَّ سَمِّ اللَّهِ (بِسْمِ اللَّهِ الرَّحْمَانِ الرَّحِيمِ)، وَ غْسِلْ يَدَيْكَ إِلَى الرُّسْغَيْنِ ثَلاَثَ مَرَّاتٍ وَ خَلِّلْ أَصَابِعَكَ.

٢) خُذْ بِيَدِكَ المَاءَ وَضَعْهُ فِي فَمِكَ وَ تَمَضْمَضْ ثَلاَثَ مَرَّاتٍ.

٣) اِسْتَنْشِقِ المَاءَ ثَلاَثًا وَ نَظِّفْ أَنْفَكَ مِمَّا فِيهِ مِنْ مُخَاطٍ وَ وَسَخٍ.

٤) اِغْسِلْ وَجْهَكَ ثَلاَثًا وَ تَحَقَّقْ مِنْ وُصُولِ المَاءِ إِلَى وَجْهِكَ كُلِّهِ. 

٥) اِغْسِلْ يَدَكَ اليُمْنَى إِلَى المِرْفَقِ ثَلاَثًا.

٦) اِغْسِلْ يَدَكَ اليُسْرَى إِلَى المِرْفَقِ ثَلاَثًا.

٧) خُذِ المَاءَ بِيَدَيْكَ وَ امْسَحْ بِهِ رَأْسَكَ مِنْ مُقَدَّمِهِ إِلَى مُؤَخَّرِهِ كُلَّهُ أَوْ بَعْضَهُ.

٨) اِمْسَحْ أُذُنَيْكَ ظَاهِرَهُمَا وَ بَاطِنَهُمَا بِإِبْهَامِكَ وَ سَبَّابَتِكَ.

٩) اِغْسِلْ رِجْلَكَ اليُمْنَى إِلَى الكَعْبَيْنِ ثَلاَثًا.

١٠) اِغْسِلْ رِجْلَكَ اليُسْرَى إِلَى الكَعْبَيْنِ ثَلاَثًا. وَ رَاعِ هَذَا التَّرْتِيبَ.

١١) وَ إِذَا فَرَغْتَ مِنْ وُضُوءِكَ اِقْرَأْ هَذَا الدُّعَاءَ المَأْثُورَ: (أَشْهَدُ أَنْ لاَ إِلَهَ إِلاَّ اللَّهُ وَحْدَهُ لاَ شَرِيكَ لَهُ، وَ أَشْهَدُ أَنَّ مُحَمَّدًا عَبْدُهُ وَ رَسُولُهُ. اللَّهُمَّ اجْعَلْنِي مِنَ التَّوَّابِينَ وَ اجْعَلْنِي مِنَ المُتَطَهِّرِينَ).

ـــــــــــــــــــــ

أسئلة للمناقشة:

١- لماذا يتوضأ الرجل؟

٢- أين ينبغي أن يجلس من يريد التوضؤ؟ و كيف يجلس؟

٣- اذكر لي أعمال الوضوء من أولها إلى آخرها على الترتيب في الدرس.

٤- ماذا تقول في أول الوضوء؟

٥- أي دعاء تقرأ بعد فراغك من الوضوء؟

٦- ما معنى الدعاء المأثور؟

٧- كم مرة تغسل الأعضاء في الوضوء؟

Омовение

1. Для омовения присядь в чистом месте, обратившись лицом к Кибле, и вознамерься душой совершить омовение. Затем произнеси имя Бога — „Во имя Бога Милостивого, Милосердного". Помой руки до запястий три раза, промывая между пальцами сцеплением кистей.

2. Зачерпни рукой воды, набери её в рот и прополощи его три раза.

3. Носом набери воды три раза и прочисть его от содержащихся в нём слизи и грязи.

4. Умой лицо трижды и убедись в том, что вода дошла до всех частей лица.

5. Помой правую руку до локтевого сустава трижды.

6. Помой левую руку до локтевого сустава трижды.

7. Зачерпни воды двумя руками и протри ею голову, начиная с передней части к затылку всю голову или частично.

8. Протри оба уха снаружи и внутри большим и указательным пальцами.

9. Помой правую ступню до щиколоток трижды.

10. Помой левую ступню до щиколоток трижды. И соблюдай данную последовательность.

11. Закончив омовение, прочитай эту молитву Пророка: „Свиде­тельствую, что нет никакого божества, кроме Бога одного, у которого нет сотоварища. Еще свидетельствую, что Мухаммад Его раб и Его Посланник. О Бог, сделай меня из числа много кающихся и сделай меня из числа очищающихся".

الصَّلاَةُ

قَبْلَ الشُّرُوعِ فِي الصَّلاَةِ عَلَيْكَ أَنْ تَسْتَكْمِلَ الشُّرُوطَ الوَاجِبَةَ لَهَا، وَ هِيَ: الوُضُوءُ، وَ طَهَارَةُ البَدَنِ وَ الثَّوْبِ وَ المَكَانِ، وَ سَتْرُ العَوْرَةِ، وَ دُخُولُ وَقْتِ الصَّلاَةِ، وَ اسْتِقْبَالُ القِبْلَةِ. وَ بَعْدَ أَنِ اسْتَقْبَلْتَ القِبْلَةَ وَ تَوَحَّهْتَ بِقَلْبِكَ إِلَى اللَّهِ تَفْعَلُ مَا يَأْتِي:

\begin{center}
\includegraphics[width=1.3299in,height=2.4717in]{images/MuhammadBagauddinprettified-img345.png}
\end{center}
١) اِفْتَحِ الصَّلاَةَ بِالتَّكْبِيرِ رَافِعًا يَدَيْكَ إِلَى الكَتِفَيْنِ نَاوِيًا الصَّلاَةَ قَائِلاً (اللَّهُ أَكْبَرُ). اُنْظُرْ 

(شكل ١).

\ وَ تُسَمَّى هَذِهِ تَكْبِيرَةَ الإِحْرَامِ. 

٢) ضَعْ يَدَكَ اليُمْنَى عَلَى اليُسْرَى عَلَى صَدْرِكَ وَ أَنْتَ قَائِمٌ. انظر (شكل ٢)،

وَ اقْرَأْ دُعَاءَ الاِسْتِفْتَاحِ: (سُبْحَانَكَ اللَّهُمَّ وَ بِحَمْدِكَ وَ تَبَارَكَ اسْمُكَ وَ تَعَالَى جَدُّكَ وَ لاَ إِلَهَ غَيْرُكَ). وَ بَعْدَ ذَلِكَ اسْتَعِذْ (أَعُوذُ بِاللَّهِ مِنَ الشَّيْطَانِ الرَّجِيمِ) ثُمَّ اقْرَأِ البَسْمَلَةَ: (بِسْمِ اللَّهِ الرَّحْمَنِ الرَّحِيمِ) ثم اقرَأْ سُورَةَ الفَاتِحَةِ: (الحَمْدُ لِلَّهِ رَبِّ العَالَمِينَ الرَّحْمَنِ الرَّحِيمِ مَالِكِ يَوْمِ الدِّينِ إِيَّاكَ نَعْبُدُ وَ إِيَّاكَ نَسْتَعِينُ اِهْدِنَا الصِّرَاطَ المُسْتَقِيمَ صِرَاطَ الَّذِينَ أَنْعَمْتَ عَلَيْهِمْ غَيْرِ المَغْضُوبِ عَلَيْهِمْ وَ لاَ الضَّالِّينَ) آمين. وَ بَعْدَ قِرَاءَةِ الفَاتِحَةِ اِقْرَأْ شَيْئًا مِمَّا تَحْفَظُ مِنْ سُورِ القُرْآنِ وَ آيَاتِهِ وَ ذَلِكَ فِي الرَّكْعَتَيْنِ الأُولَيَيْنِ فَقَطْ مِنْ كُلِّ صَلاَةٍ. تَقْرَأُ مَثَلاً: (بِسْمِ اللَّهِ الرَّحْمَنِ الرَّحِيمِ. قُلْ هُوَ اللَّهُ أَحَدٌ اللَّهُ الصَّمَدُ لَمْ يَلِدْ وَ لَمْ يُولَدْ وَ لَمْ يَكُنْ لَهُ كُفُوًا أَحَدٌ) أُوْ تَقْرَأُ أَيَّةَ سُورَةٍ أَوْ آيَةٍ أُخْرَى مِنَ القُرْآنِ.

ـــــــــــــــــــــ

أسئلة للمناقشة:

١- ماذا يجب عليك قبل الشروع في الصلاة؟

٢- كم هي شروط الصلاة؟ اذكرها.

٣- بماذا تُفتتح الصلاة؟

٤- ما تسمى التكبيرة التي تفتتح بها الصلاة؟

٥- في أية ركعة من ركعات الصلاة تقرأ سورة من القرآن بعد الفاتحة؟

ـــــــــــــــــــــ

Намаз

Перед началом намаза ты должен выполнить обязательные для него условия: омовение; чистота тела, одежды и места; прикрытие аврат (стыдных мест); наступление времени намаза; обращение лицом к Кибле. После того, как ты обратился лицом к Кибле и устремился сердцем к Богу, сделай следующее:

1. Открой намаз такбиром (произнесение „Богу акбар"), подняв обе руки к плечам, намереваясь исполнить намаз и произнося „Богу акбар" („Бог превыше всего") (см. рис. 1). Это славословие называется ,,Такбиратуль — Ихрам" (такбир для вступления в святость).

2. Стоя, положи правую руку на левую на груди (см. рис. 2) и прочти Дуааль-Истифтах (молитва вступления): „Свят Ты, о Бог! И хвала Тебе! Имя Твоё благословилось и величие Твоё возросло, и нет другого божества, кроме Тебя". После этого прибегай к Богу: „Я прибегаю к Богу от изгнанного сатаны". Затем читай бисмилла: „Во имя Бога Милостивого, Милосердного". Затем прочти суру „Аль-фатиха" (первую, открывающую суру Корана): „Вся хвала подлежит Богу, Владыке всех миров, Милостивому, Милосердному, Властителю Судного дня. Тебе одному мы поклоняемся и к Тебе одному мы взываем о помощи. Наставь нас на путь правый, путь тех, которых Ты одарил Своими благами, тех, кто не находится под гневом, и тех, кто не впал в заблуждение. Аминь" (Да будет). После прочтения „Аль-фатиха" прочти что-нибудь из того, что ты знаешь наизусть из сур и аятов Корана. И это только в двух первых ракатах каждого намаза. Можно прочесть, например: „Во имя Бога' Милостивого, Милосердного". Скажи: „Он есть Бог, Единый Бог, Независимый и всеми Искомый. Он не родил и не был рождён. И не был равным Ему ни один". Или же прочтёшь любую другую суру или аят из Корана.

٣) بَعْدَ قِرَاءَةِ الفَاتِحَةِ وَ سُورَةٍ قَصِيرَةٍ كَبِّرْ (اللَّهُ أَكْبَرْ) ثُمَّ ارْكَعْ وَضَعْ يَدَيْكَ عَلَى رُكْبَتَيْكَ، انظر (شكل ٣)

\begin{center}
\includegraphics[width=1.7075in,height=1.5283in]{images/MuhammadBagauddinprettified-img346.png}
\end{center}
وَ قُلْ: (سُبْحَانَ رَبِّيَ العَظِيمِ) ثَلاَثَ مَرَّاتٍ.

٤) ثُمَّ اعْتَدِلْ قَائِمًا قَائِلاً: (سَمِعَ اللَّهُ لِمَنْ حَمِدَهُ). فَإِذَا اعْتَدَلْتَ تَمَامًا فَقُلْ: (رَبَّنَا وَ لَكَ الحَمْدُ) انظر (شكل ٤).

٥) ثُمَّ اهْوِ سَاجِدًا إِلَى الأَرْضِ قَائِلاً: (اللَّهُ أَكْبَرْ)، وَاضِعًا رُكْبَتَيْكَ قَبْلَ يَدَيْكَ عَلَى الأَرْضِ ثُمَّ جَبْهَتَكَ وَ أَنْفَكَ وَ قُلْ: (سُبْحَانَ رَبِّيَ الأَعْلَى) ثَلاَثَ مَرَّاتٍ. انظر (شكل ٥).

\begin{center}
\includegraphics[width=2.1228in,height=1.3866in]{images/MuhammadBagauddinprettified-img347.png}
\end{center}
٦) اِرْفَعْ رَأْسَكَ مُكَبِّرًا وَ اجْلِسْ عَلَى رِجْلِكَ اليُسْرَى وَ انْصِبِ اليُمْنَى وَ قُلْ فِي جُلُوسِكَ: (رَبِّ اغْفِرْلِي، رَبِّ اغْفِرْلِي). انظر (شكل ٦).

\begin{center}
\includegraphics[width=1.8772in,height=1.9055in]{images/MuhammadBagauddinprettified-img348.png}
\end{center}
٧) ثُمَّ اسْجُدْ السَّجْدَةَ الثَّانِيَةَ تَمَامًا كَمَا سَجَدْتَ السَّجْدَةَ الأُولَى مُكَبِّرًا وَ قُلْ فِي سُجُودِكَ: (سُبْحَانَ رَبِّيَ الأَعْلَى) ثَلاَثًا (انظر شكل ٥).

٨) بَعْدَ الاِنْتِهَاءِ مِنَ السَّجْدَةِ الثَّانِيَةِ انْهَضْ قَائِمًا وَ مُكَبِّرًا. وَ بِهَذَا تَتِمُّ الرَّكْعَةُ الأُولَى.

ـــــ

الرَّكْعَةُ الثَّانِيَةُ

٩) وَ فِي وُقُوفِكَ لِلرَّكْبَةِ الثَّانِيَةِ اِقْرَأِ الفَاتِحَةَ ثُمَّ اِقْرَأْ قَلِيلاً مِنَ القُرْآنِ ثُمَّ ارْكَعْ ثُمَّ اسْجُدْ سَجْدَتَيْنِ كَمَا فَعَلْتَ فِي الرَّكْعَةِ الأُولَى تَمَامًا.

ـــــــــــــــــــــ

١- ماذا تقول عند الركوع؟

٢- ماذا تقول في الركوع؟

٣- كم سجدة تسجد في كل ركعة؟

٤- ما تقول عند الرفع عن الركوع؟

٥- ما تقول في اعتدالك بعد الركوع؟

٦- ماذا تقول في سجودك؟

٧- ما تقول في جلوسك بين السجدتين؟

ـــــــــــــــــــــ

3. После того, как ты прочёл „Аль-фатиха" и короткую другую суру, произнеси такбир „Бог превыше всего". Затем наклонись, положи руки на колени (см. рис. 3) и скажи: „Свят мой Великий Господь" три раза.

4. После этого выпрямись во весь рост со словами: „Слышал Бог того, кто восхвалил Его". Когда ты выпрямишься в полный рост, скажи: „Господь наш, и Тебе хвала" (см. рис. 4).

5. Затем пади ниц на землю со словами: „Бог превелик", кладя оба колена раньше рук на землю, а затем лоб и нос, и скажи: „Свят Господь мой Всевышний" три раза (см. рис. 5).

6. Подними голову, произнося такбир, и сядь на левую ногу, вертикально ставь правую ногу, и в положении сидя произнеси: „Господь, прости меня, Господь, прости меня" (см. рис. 6).

7. Затем пади ниц второй раз, точно таким же образом, как и в первый, произнося такбир, и в положении суджуда (падение ниц) скажи: „Свят Господь мой Всевышний" три раза (см. рис. 5).

8. После окончания второго суджуда встань в полный рост, произнося такбир. И этим заканчивается первый ракат.

Второй ракат:

9. В положении стоя для второго раката прочти ,,Аль-фатиха", затем прочти немного из Корана. Потом наклонись на руку (горизонтальный наклон вперёд), затем пади ниц дважды, точно так же, как ты делал в первом ракате.

١٠) عِنْدَمَا تَرْفَعُ رَأْسَكَ مِنَ السَّجْدَةِ الثَّانِيَةِ عَلَيْكَ أَنْ تَجْلِسَ جُلُوسَكَ بَيْنَ السَّجْدَتَيْنِ غَيْرَ أَنَّكَ تَضُمُّ هُنَا أَصَابِعَ يَدِكَ اليُمْنَى وَ تُرْسِلُ السَّبَّابَةَ. انظر (شكل ٧).

\begin{center}
\includegraphics[width=1.8583in,height=2.1417in]{images/MuhammadBagauddinprettified-img349.png}
\end{center}
١١) الآنَ وَ أَنْتَ جَالِسٌ اِقْرَأْ "التَّحِيَّاتُ" مَعَ "الصَّلاَةِ الإِبْرَاهِيمِيَّةِ" إِنْ كَانَتِ الصَّلاَةُ رَكْعَتَيْنِ مِثْلَ صَلاَةِ الصَّبْحِ مَثَلاً. أَمَّا إِذَا كَانَتِ الصَّلاَةُ أَرْبَعَ رَكَعَاتٍ أَوْ ثَلاَثًا فَتَقِفُ دُونَ أَنْ تَقْرَأَ الصَّلاَةَ الإِبْرَاهِيمِيَّةَ. 

\begin{center}
\includegraphics[width=1.802in,height=2.1319in]{images/MuhammadBagauddinprettified-img350.png}
\end{center}
١٢) عِنْدَ جُلُوسِكَ وَ قِرَاءَتِكَ "التَّحِيَّاتُ" وَ وُصُولِكَ لِلتَّشَهُّدِ اِرْفَعْ سَبَّابَتَكَ اليُمْنَى. انظر (شكل ٨)

ـــــ

التَّحِيَّاتُ:

(التَّحِيَّاتُ لِلَّهِ وَ الصَّلَوَاتُ وَ الطَّيِّبَاتُ. السَّلاَمُ عَلَيْكَ أَيُّهَا النَّبِيُّ وَ رَحْمَةُ اللَّهِ وَ بَرَكَاتُهُ. السَّلاَمُ عَلَيْنَا وَ عَلَى عِبَادِ اللَّهِ الصَّالِحِينَ أَشْهَدُ أَنْ لاَ إِلَهَ إِلاَّ اللَّهُ وَ أَشْهَدُ أَنَّ مُحَمَّدًا عَبْدُهُ وَ رَسُولُهُ) وَ هُنَا عِنْدَ قَوْلِكَ "إِلاَّ اللَّهُ" تَرْفَعُ السَّبَّابَةَ مِنَ اليَدِ اليُمْنَى.

ـــــ

الصَّلاَةُ الإبْرَاهِيمِيَّةُ:

(اللَّهُمَّ صَلِّ عَلَى مُحَمَّدٍ وَ عَلَى آلِ مُحَمَّدٍ كَمَا صَلَّيْتَ عَلَى إِبْرَاهِيمَ وَ عَلَى آلِ إِبْرَاهِيمَ. وَ بَارِكْ عَلَى مُحَمَّدٍ وَ عَلَى آلِ مُحَمَّدٍ كَمَا بَارَكْتَ عَلَى إِبْرَاهِيمَ وَ عَلَى آلِ إِبْرَاهِيمَ فِي العَالَمِينَ، إِنَّكَ حَمِيدٌ مَجِيدٌ).

١٣) بَعْدَ الاِنْتِهَاءِ مِنْ قِرَاءَةِ الصَّلاَةِ الإِبْرَاهِيمِيَّةِ اِلْتَفِتْ يَمِينًا وَ قُلْ: (السَّلاَمُ عَلَيْكُمْ وَ رَحْمَةُ اللَّهِ وَ بَرَكَاتَهُ) ثُمَّ الْتَفِتْ يَسَارًا وَ سَلِّمْ نَفْسَ السَّلاَمِ وَ بِهَذَا تَتِمُّ صَلاَتُكَ. انظر الشكلين (٩، ١٠)

\begin{center}
\includegraphics[width=1.698in,height=1.9902in]{images/MuhammadBagauddinprettified-img351.png}
\end{center}
\begin{center}
\includegraphics[width=1.7165in,height=1.7925in]{images/MuhammadBagauddinprettified-img352.png}
\end{center}
ـــــــــــــــــــــ 

أسئلة للمناقشة:

١- ما هو أول أعمال الصلاة؟

٢- ما هو آخر أعمال الصلاة؟

٣- بأي شيء تتم صلاتك؟

٤- إلى أية جهة يلتفت المصلى عند التسليمة الأولى؟ عند التسليمة الثانية؟

٥- يلتفت المصلى أولا إلى اليمين أو إلى اليسار؟

٦- هل حفظت "التحيات"؟

٧- هل حفظت الصلاة الإبراهيمية؟

٨- أتقرأ الصلاة الإبراهيمية في كل صلاة؟

٩- كيف تسلم عند ختام الصلاة؟ و ما صيغتها؟

١٠- إذا وصلت بعد انتهاء الدروس إلى منزلك فانظر أيحفظ أخوك الصغير أو أختك الصغيرة "الفاتحة" و "التحيات"

10. Когда ты поднимаешь свою голову после второго суджуда, ты должен сесть, как между двумя суджудами, однако, тут сожмёшь пальцы правой руки и вытянешь указательный палец (см. рис. 7).

11. А теперь, находясь в положении сидя, прочти „Аттахиййат" с "салятом Ибрахима", если намаз был двухракатный, как, например, утренний намаз. А если же намаз четырёх- или трёхракатный, то встанешь, не прочитав салят Ибрахима.

12. Во время сидения и чтения „Аттахиййат", дойдя до ташаххуда (приведение свидетельства), подними свой указательный палец (см. рис. 8).

Аттахиййат:

„Почести принадлежат Богу, и молитвы, и хорошие слова, и дела. Мир и здравие тебе, о Пророк, и милость Бога и Его благодать. Мир и здравие нам и благочестивым рабам Бога. Свидетельствую, что нет божества, кроме Бога, и свидетельствую, что Мухаммад, Его раб и Его Посланник". Вот тут при словах „Илля-ллах" поднимешь указательный палец правой руки.

Салят Ибрахима:

„О Бог, ниспошли салят Мухаммаду и семье Мухаммада так же, как ниспослал Ибрахиму и семье Ибрахима, и ниспошли благодать Мухаммаду и семье Мухаммада так же, как ниспослал Ибрахиму и семье Ибрахима во всех мирах. Поистине, Ты достин хвалы и чести".

13. После окончания чтения салята Ибрахима поверни лицо направо и скажи: „Мир вам, милость Бога и Его благодать". Затем поверни лицо налево и приветствуй тем же приветствием. На этом твой намаз завершается (см. рис. 9, 10).

كَيْفِيَّةُ الصَّلَوَاتِنَا المَفْرُوضَةِ

١) صَلاَةُ الصُّبْحِ رَكْعَتَانِ.

كَيْفِيَّتُهَا: اِقْرَأْ فِي الرَّكْعَتَيْنِ الفَاتِحَةَ وَ شَيئًا مِنَ القُرْآنِ ثُمَّ اقْرَأْ "التحيّات" مَعَ "الصَّلاَةِ الإِبْرَاهِيمِيَّةِ" و سَلِّمْ.

٢) صَلاَةُ الظُهْرِ رَبْعُ رَكَعَاتٍ.

كَيْفِيَّتُهَا: اِقْرَأْ فِي الرَّكْعَتَيْنِ الأُولَيَيْنِ الفَاتِحَةَ وَ سُورَةً مِنَ القُرْآنِ، وَ بَعْدَهَا تَجْلِسُ وَ تَقْرَأُ "التحيّات" إِلَى آخِرِ التَّشَهُّدِ. ثُمَّ تَنْهَضُ وَ تُصَلِّي رَكْعَتَيْنِ أُخَرَيَيْنَ دُونَ أَنْ تَقْرَأََ بَعْدَ الفَاتِحَةِ سُورَةً. ثُمَّ تَجْلِسُ وَ تَقْرَأُ "التحيَّات" مَعَ الصَّلاَةِ الإِبْرَاهِيمِيَّةِ. وَ تَنْتَهِي بِالتَّسْلِيمِ.

٣) صَلاَةُ العَصْرِ أَرْبَعُ رَكَعَاتٍ:

كَيْفِيَّتُهَا: وَ كَيْفِيَّتُهَا مِثْلُ صَلاَةِ الظُّهْرِ سَوَاءً بِسَوَاءٍ.

٤) صَلاَةُ المَغْرِبِ ثَلاَثُ رَكَعَاتٍ:

كَيْفِيَّتُهَا: تَقْرَأُ فِي الرَّكْعَتَيْنِ الأُولَيَيْنِ الفَاتِحَةَ وَ سُورَةً مِنَ القُرْآنِ. وَ بَعْدَهَا تَجْلِسُ وَ تَقْرَأُ "التحيّات" إِلَى آخِرِ التَّشَهُّدِ، وَ بَعْدَهَا تَنْهَضُ وَ تَكْمِلُ رَكْعَةً وَاحِدَةً ثُمَّ تَجْلِسُ وَ تَقْرَاُ "التحيّات" كَامِلَةً وَ تَنْتَهِي بِالتَّسْلِيمِ.

٥) صَلاَةُ العِشَاءِ أَرْبَعُ رَكَعَاتٍ.

كَيْفِيَّتُهَا: مِثْلُ صَلاَةِ الظُّهْرِ سَوَاءً بِسَوَاءٍ.

تَنْبِيهٌ:

تَجْهَرُ بِقِرَاءَةِ الفَاتِحَةِ وَ سُورَةٍ بَعْدَهَا فِي الرَّكْعَتَيْنِ الأُولَيَيْنِ مِنْ صَلاَةِ المَغْرِبِ وَ صَلاَةِ العِشَاءِ وَ فِي صَلاَةِ الصُّبْحِ إِمَامًا وَ مُنْفَرِدًا وَ تَخْفُتُ بِهِمَا فِي غَيْرِهَا.

ـــــــــــــــــــــ

١- كم ركعة صلاة الصبح؟

٢- كم ركعة صلاة الظهر؟ العصر؟ العشاء؟

٣- كم ركعة صلاة المغرب؟

٤- ما هي أولى صلوات اليوم و اللّيلة؟

٥- اذكر لي كيفية صلاة الصبح؟

٦- اذكر لي كيفية صلاة الظهر؟

٧- اذكر لي كيفية صلاة المغرب؟

٨- متى تجهر بقراءة الفاتحة في صلواتك و متى تخفت بها؟

٩- في أيهما الركعات أكثر في صلاة الظهر أم في صلاة المغرب؟

١٠- ما ترتيب الصلوات المفروضة؟

Способ выполнения обязательных намазов:

1. Утренний намаз — два раката. Способ его совершения: в двух ракатах прочти ,,Аль-фатиха" и что-нибудь из Корана, затем прочти „Аттахиййат" вместе с салятом Ибрахима и дай салям.

2. Полуденный намаз — четыре раката. Как его совершать: в двух первых ракатах прочти ,,Аль-фатиха" и какую-нибудь суру из Корана После этого сядешь и прочтёшь „Аттахиййат" до конца ташаххуда Затем встанешь и выполнишь два других раката, не прочитав после „Аль-фатиха" никакой суры. Затем сядешь и прочтёшь "Аттахиййат' вместе с салятом Ибрахима. И закончишь салямом.

3. Предвечерний намаз — четыре раката. Как его совершать: способ его совершения подобен способу совершения полуденного намаза совершенно одинаково.

4. Вечерний намаз — три раката. Способ его совершения: в первых двух ракатах прочтешь „Аль-фатиха" и какую-нибудь суру из Корана После этого сядешь и прочтёшь „Аттахиййат" до конца ташаххуда После этого встанешь и завершишь один ракат, затем сядешь и прочтёшь „Аттахиййат" полностью, и завершишь салямом.

5. Ночной намаз — четыре раката. Способ его совершения, как полуденный намаз, совершенно одинаков.

Важное примечание:

В первых двух ракатах вечернего и ночного намазов и в утреннем намазе „Аль-фатиха" и следующую за ней суру читаешь вслух, как будучи имамом, так и будучи одиночным, а в остальных случаях читаешь их тихо.

الأَذْكَارُ وَ الأَدْعِيَةُ المَأْثُورَةُ

١) عِنْدَ الاِنْتِهَاءِ عَنِ الصَّلاَةِ: (أَسْتَغْفِرُ اللَّهَ (ثَلاَثًا) أَللَّهُمَّ أَنْتَ السَّلاَمُ وَ مِنْكَ السَّلاَمُ تَبَارَكْتَ وَ تَعَالَيْتَ يَا ذَا الجَلاَلِ وَ الإِكْرَامِ. اللَّهُمَّ أَعِنِّي عَلَى ذِكْرِكَ وَ شُكْرِكَ وَ حُسْنِ عِبَادَتِكَ).

٢) عِنْدَ النَّوْمِ: اِضْطَجِعْ عَلَى جَنْبِكَ الأَيْمَنِ وَضَعْ كَفَّكَ اليُمْنَى تَحْتَ خَدِّكَ الأَيْمَنِ وَ قُلْ (بِاسْمِكَ رَبِّي وَضَعْتُ جَنْبِي وَ بِكَ أَرْفَعُهُ إِنْ أَمْسَكْتَ نَفْسِي فَارْحَمْهَا وَ إِنْ أَرْسَلْتَهَا فَاحْفَظْهَا بِمَا تَحْفَظُ بِهِ عِبَادَكَ الصَّالِحِينَ).

٣) عِنْدَ الاِسْتِيقَاظِ مِنَ النَّوْمِ: (الحَمْدُ لِلَّهِ الَّذِي أَحْيَانَا بَعْدَمَا أَمَاتَنَا وَ إِلَيْهِ النُّشُورُ) أَوْ (الحَمْدُ لِلَّهِ الَّذِي رَدَّ عَلَيَّ رُوحِي وَ عَافَانِي فِي جَسَدِي وَ أَذِنَ لِي بِذِكْرِهِ).

Ритуальные азкары и молитвы Пророка

1. При завершении намаза: „Прошу прощения у Бога" (трижды). „О Бог! Ты есть салям (мир, здравие), и от Тебя исходит салям. Умножилась Твоя благодать и возвысился Ты, о Обладатель величия и почёта. О Бог! Помоги мне поминать Тебя, благодарить тебя и хорошо поклоняться Тебе".

2. Перед сном: ляг на правый бок, положи правую руку под правую щёку и скажи: „Именем Твоим. Господь мой, я положил свой бок и именем Твоим я подниму его. Если Ты удержишь мою душу, будь милостив к ней, а если Ты отошлёшь её, охрани её, так же, как Ты хранишь Своих благочестивых рабов".

3. Просыпаясь: „Хвала Богу, который оживил нас, после того, как умертвил нас, и к Нему возвращаемся, воскреснув", или: „Хвала Богу, Который вернул мне мою душу, здоровье моему телу дал и поминать Себя позволил".

٤) عِنْدَ تَنَاوُلِ طَعَامٍ أَوْ شَرَابٍ: (اللَّهُمَّ بَارِكْ لَنَا فِيمَا رَزَقْتَنَا وَ قِنَا عَذَابَ النَّارِ. بِسْمِ اللَّهِ) فَإِنْ نَسِيتَهُ فِي أَوَّلِ الطَّعَامِ فَقُلْ فِي أَثْنَائِهِ (بِسْمِ اللَّهِ أَوَّلَهُ وَ آخِرَهُ).

٥) عِنْدَ الاِنْتِهَاءِ عَنِ الأَكْلِ: (الحَمْدُ لِلَّهِ الَّذِي أَطْعَمَنِي هَذَا وَ رَزَقَنِيهِ مِنْ غَيْرِ حَوْلٍ مِنِّي وَ لاَ قُوَّةٍ) أَوْ (الحَمْدُ لِلَّهِ الَّّذِي أَطْعَمَنَا وَ سَقَانَا وَ جَعَلَنَا مُسْلِمِينَ).

٦) عِنْدَ الخُرُوجِ مِنَ المَنْزِلِ: (بِسْمِ اللَّهِ تَوَكَّلْتُ عَلَى اللَّهِ، لاَ حَوْلَ وَ لاَ قُوَّةَ إِلاَّ بِاللَّهِ) أَوْ (بِسْمِ اللَّهِ تَوَكَّلْتُ عَلَى اللَّهِ، اللَّهُمَّ إِنِّي أَعُوذُ بِكَ أَنْ أَضِلَّ أَوْ أُضَلَّ أَوْ أَزِلَّ أَوْ أُزَلَّ أَوْ أَظْلِمَ أَوْ أُظْلَمَ أَوْ أَجْهَلَ أَوْ يُجْهَلَ عَلَيَّ).

4. Во время еды или питья: ,,О Бог, благослови нас в том, чем Ты наделил нас, и убереги нас от мучений ада, во имя Бога". Если ты забыл это в начале еды, то скажи в ходе еды: ,,Во имя Бога,

и в начале, и в конце".

5. Закончив еду: ,,Хвала Богу, Который накормил меня этим v наделил меня этим без никакой мощи и силы с моей стороны", или „Хвала Богу, Который накормил нас, напоил нас и сделал на< мусульманами (покорными Богу).

6. При выходе из дома: ,,Во имя Бога (выхожу из дома), ; уповаю на Бога; нет мощи и силы, кроме как от Бога", или „Во имя Бога (выхожу из дома), я уповаю на Бога. О Бог я прибегаю к Тебе, чтобы не заблудиться и не быть введённым заблуждение, чтобы не ошибиться и не быть ввергнутым в ошибку чтобы не притеснять и не быть притесннёным, чтобы не быть дерзким и не быть подвергнутым дерзости".

٧) عِنْدَ دُخُولِ المَنْزِلِ: (اللَّهُمَّ إِنِّي أَسْأَلُكَ خَيْرَ المَوْلِجِ وَ خَيْرَ المَخْرَجِ بِسْمِ اللَّهِ وَلَجْنَا وَ بِسْمِ اللَّهِ خَرَجْنَا وَ عَلَى اللَّهِ رَبِّنَا تَوَكَّلْتُ).

٨) عِنْدَ دُخُولِ السُّوقِ: (لاَ إِلَهَ إِلاَّ اللَّهُ وَحْدَهُ لاَ شَرِيكَ لَهُ لَهُ المُلْكُ وَ لَهُ الحَمْدُ يُحْيِي وَ يُمِيتُ وَ هُوَ حَيٌّ لاَ يَمُوتُ بِيَدِهِ الخَيْرُ وَ هُوَ عَلَى كُلِّ شَيْءٍ قَدِيرٌ).

٩) قَبْلَ دُخُولِ الخَلاَءِ: (بِسْمِ اللَّهِ، اللَّهُمَّ إِنِّي أَعُوذُ بِكَ مِنَ الخُبُثِ وَ الخَبَائِثِ) ثُمَّ تَدْخُلُ مُقَدِّمًا رِجْلَكَ اليُسْرَى.

١٠) بَعْدَ الخُرُوجِ مِنَ الخَلاَءِ: تَخْرُجُ مُقَدِّمًا رِجْلَكَ اليُمْنَى ثُمَّ تَقُولُ: (غُفْرَانَكَ) أَوْ (الحَمْدُ لِلَّهِ الَّذِي أَذْهَبُ عَنِّي الأَذَى وَ عَافَانِي).

7. При входе в дом: „О Бог, прошу у Тебя добра входа и добра выхода. Во имя Бога, мы входим, во имя Бога, мы выходим и на Бога, Господа нашего, уповаем".

8. При входе на рынок: „Нет божества, кроме Бога Единого, не Имеющего сотоварища. Ему принадлежит царство, И Ему принадлежит хвала. Он оживляет и умерщвляет, и Он Живой, не умирает. В Его руках добро, и Он Всемогущ".

9. Перед входом в туалет: „Во имя Бога (иду в туалет). О Бог, прошу твоей защиты от коварных бесов мужского и женского пола". Затем входишь, вначале ступая левой ногой.

10. После выхода из туалета: выходишь, вначале ступая правой ногой, затем говоришь: „Прощения Твоего (прошу)", или: „Хвала Богу, Который отвёл от меня неприятное и сохранил моё здоровье".

١١) عِنْدَ رُكُوبِ طَائِرَةٍ أَوْ سَيَّارَةٍ أَوْ سَفِينَةٍ أَوْ دَابَّةٍ أَوْ غَيْرِهَا: (بِسْمِ اللَّهِ) وَ بَعْدَ الرُّكُوبِ (الحَمْدُ لِلَّهِ ، اللَّهُ أَكْبَرُ اللَّهُ أَكْبَرُ اللَّهُ أَكْبَرُ . سُبْحَانَ الَّذِي سَخَّرَ لَنَا هَذَا وَ مَا كُنَّا لَهُ مُقْرِنِينَ وَ إِنَّا إِلَى رَبِّنَا لَمُنْقَلِبُونَ).

١٢) عِنْدَ السَّفَرِ: (اللَّهُمَّ إِنَّا نَسْأَلُكَ فِي سَفْرِنَا هَذَا البِرَّ وَ التَّقْوَى وَ مِنَ العَمَلِ مَا تَرْضَى اللَّهُمَّ هَوِّنْ عَلَيْنَا سَفَرَنَا هَذَا وَ اطْوِ عَنَّا بُعْدَهُ، اللَّهُمَّ أَنْتَ الصَّاحِبُ فِي السَّفَرِ وَ الخَلِيفَةُ فِي الأَهْلِ، اللَّهُمَّ إِنَّا نَعُوذُ بِكَ مِنْ وَعْثَاءِ السَّفَرِ وَ كَآبَةِ المَنْظَرِ وَ سُوءِ المُنْقَلَبِ فِي المَالِ وَ الأَهْلِ وَ الوَلَدِ).

١٣) عِنْدَ لُبْسِ الثَّوْبِ: (الحَمْدُ لِلَّهِ الَّذِي كَسَانِي هَذَا وَ رَزَقَنِيهِ مِنْ غَيْرِ حَوْلٍ مِنِّي وَ لاَ قُوَّةٍ).

11. При посадке на самолёт, машину, корабль, на верховое животное и прочее: ,,Во имя Бога", и после посадки: ,,Хвала Богу, Бог превелик, Бог превелик, Бог превелик. Свят Тот, кто подчинил нам это, на что мы не были бы способны. Истинно, к I осподу нашему мы возвращаемся".

12. При выходе в дорогу: ,,0 Бог, мы просим у Тебя в этом нашем пути праведности и богобоязненности, и из поступков то. что угодно Тебе. О Бог, облегчи нам этот наш путь и сократи для нас его дальность. О Бог, Ты спутник в пути и преемник в семье. О Бог, истинно, мы прибегаем к Тебе от трудностей пути, печальною вида и плохого возвращения в имуществе, семье и потомстве".

13. Когда одеваются: „Хвала Богу, Который одел меня в это без никакой мощи и силы с моей стороны".

١٤) عِنْدَ لُبْسِ ثَوْبٍ جَدِيدٍ: (اللَّهُمَّ لَكَ الحَمْدُ أَنْتَ كَسَوْتَنِيهِ أَسْأَلُكَ خَيْرَهُ وَ خَيْرَ مَا صُنِعَ لَهُ وَ أَعُوذُ بِكَ مِنْ شَرِّهِ شَرِّ مَا صُنِعَ لَهُ).

١٥) عِنْدَ دُخُولِ المَسْجِدِ: تُقَدِّمُ رِجْلَكَ اليُمْنَى عَلَى عَكْسِ الخَلاَءِ وَ تَقُولُ: (بِسْمِ اللَّهِ وَ السَّلاَمُ عَلَى رَسُولِ اللَّهِ، اللَّهُمَّ افْتَحْ لِي أَبْوَابَ رَحْمَتِكَ).

١٦) عِنْدَ الخُرُوجِ مِنَ المَسْجِدِ: تُقَدِّمُ رِجْلَكَ اليُسْرَى وَ تَقُولُ: (بِسْمِ اللَّهِ وَ السَّلاَمُ عَلَى رَسُولِ اللَّهِ، اللَّهُمَّ إِنِّي أَسْأَلُكَ مِنْ فَضْلِكَ).

١٧) عِنْدَ الذَّبْحِ: (بِسْمِ اللَّهِ، اللَّهُ أَكْبَرُ).

14. Надевая на себя новую одежду: „О Бог, Тебе хвала Ты одел меня этим. Я прошу у Тебя добра его и добра того, для чего оно сделано. И прибегаю к Тебе от зла его и зла того, для чего оно сделано".

15. При входе в мечеть: вначале ступаешь правой ногой, в отличие от входа в туалет, и говоришь: „Во имя Бога, и салям Посланнику Бога. О Бог, открой для меня врата милости Твоей".

16. При выходе из мечети: вначале ступаешь левой ногой и говоришь: „Во имя Бога (выхожу), и салям Посланнику Бога; О Бог, прошу блага Твоего".

17. Когда режете скот: „Во имя Бога (режу), Бог превелик".

١٨) عِنْدَ العُطَاسِ: يَقُولُ العَاطِسُ: (الحَمْدُ لِلَّهِ) فَيُجِيبُهُ السَّامِعُ: (يَرْحَمُكَ اللَّهُ) فَيُرَدُّ عَلَيْهِ العَاطِسُ: (يَهْدِيكُمُ اللَّهُ وَ يُصْلِحُ بَالَكُمْ).

١٩) عِنْدَ خِتَامِ المَجْلِسِ لِيَكُونَ كَفَّارَةً لِلْمَجْلِسِ: (سُبْحَانَكَ اللَّهُمَّ وَ بِحَمْدِكَ أَشْهَدُ أَنْ لاَ إِلَهَ إِلاَّ أَنْتَ أَسْتَغْفِرُكَ وَ أَتُوبُ إِلَيْكَ).

٢٠) عِنْدَ الإِفْطَارِ: (اللَّهُمَّ لَكَ صُمْتُ وَ عَلَى رِزْقِكَ أَفْطَرْتُ) أَوْ (ذَهَبَ الظَّمَأُ وَ ابْتَلَّتِ العُرُوقُ وَ ثَبَتَ الأَجْرُ إِنْ شَاءَ اللَّهُ تَعَالَى، بِسْمِ اللَّهِ).

٢١) عِنْدَ هُبُوبِ الرِّيحِ: (اللَّهُمَّ إِنِّي أَسْأَلُكَ خَيْرَهَا وَ خَيْرَ مَا فِيهَا وَ خَيْرَ مَا أُرْسِلَتْ بِهِ، وَ أُعُوذُ بِكَ مِنْ شَرِّهَا وَ شَرِّ مَا فِيهَا وَ شَرِّ مَا أُرْسِلَتْ بِهِ).

18. При чихании: чихающий говорит „Хвала Богу"; слышащий чихающего отвечает: ,,Да помилует тебя Бог"; чихнувший в ответ на это говорит; „Да поведёт вас Бог по правильному пути и исправит Он ваше обстоятельство".

19. При завершении маджлиса (заседания, собрания), чтобы было искуплением данного маджлиса: „Свят Ты, о Бог, и с хвалой Тебе. Свидетельствую, что нет божества, кроме Тебя. Прошу Твоего прощения и каюсь перед Тобой".

20. При розговеньи: „О Бог, Тебе я постился и наделенным Тобой я разговелся", или „Прошла жажда, и наполнились влагой жилы, и утвердилось вознаграждение, если угодно Богу Всевышнему. Во имя Бога (разговляюсь).

21. При дуновении ветра: „О Бог, прошу у Тебя добра его и добра того, что в нём, и добра того, для чего он (ветер) был послан, и убереги меня от зла его и зла того, что в нём, и зла того, с чем он послан".

٢٢) عِنْدَ قَصْفِ الرَّعْدِ: (سُبْحَانَ الَّذِي يُسَبِّحُ الرَّعْدُ بِحَمْدِهِ وَ المَلاَئِكَةُ مِنْ خِيفَتِهِ) (ثَلاَثًا).

٢٣) عِنْدَ الأَرَقِ فِي اللَّيْلِ: (اللَّهُمَّ غَارَتِ النُّجُومُ وَ هَدَأَتِ العُيُونُ وَ أَنْتَ حَيٌّ قَيُّومٌ لاَ تَأْخُذُكَ سِنَةٌ وَ لاَ نَوْمٌ، يَا حَيُّ يَا قَيُّومُ اهْدَأْ لَيْلِي وَ أَنِمْ عَيْنِي).

٢٣) عِنْدَ وَدَاعِ المُسَافِرِ: (أَسْتَوْدِعُ اللَّهَ دِينَكَ وَ أَمَانَتَكَ وَ خَوَاتِيمَ عَمَلِكَ).

٢٥) عِنْدَ انْتِهَاءِ الأَذَانِ: (اللَّهُمَّ صَلِّ عَلَى مُحَمَّدٍ. اللَّهُمَّ رَبَّ هَذِهِ الدَّعْوَةِ التَّامَّةِ وَ الصَّلاَةِ القَائِمَةِ آتِ مُحَمَّدًا الوَسِيلَةَ وَ الفَضِيلَةَ، وَ ابْعَثْهُ مَقَامًا مَحْمُودًا الَّذِي وَعَدْتَهُ).

22. Когда гремит гром; „Свят Тот, Кого с хвалой восславляют гром и ангелы от страха перед Ним" (трижды).

23. При бессонице ночью: „О Бог, закатились звёзды, успокоились глаза, и только Ты .Живой и Независимосущий, не одолевают Тебя ни дремота, ни сон. О Живой, о Независимый, сделай тихой мою ночь и усыпи мои глаза".

24. Провожая отъезжающего: „Вверяю Богу твою религию, твою верность и завершение твоих действий".

25. При завершении азана: „О Бог, ниспошли салят Мухаммаду. О Бог, Господь этого совершенного призыва и выстаиваемого намаза, дай Мухаммаду Василя (наивысшую ступень в раю) и превосходное качество, и воскрешай его на похвальном месте, которое Ты обещал ему" (т.е. место, где в Судный день Пророк станет для оказания заступничества за мусульман

٢٦) عِنْدَ زِيَارَةِ المَرِيضِ: (أَذْهِبِ البَأْسَ رَبَّ النَّاسِ اِشْفِ أَنْتَ الشَّافِي لاَ شِفَاءَ إِلاَّ شِفَاؤُكَ شِفَاءً لاَ يُغَادِرُ سُقْمًا).

٢٧) عِنْدَ الإِصَابَةِ بِمَكْرُوهٍ: (إِنَّا لِلَّهِ وَ إِنَّا إِلَيْهِ رَاجِعُونَ).

٢٨) عِنْدَ دُخُولِ المَقْبَرَةِ: ( السَّلاَمُ عَلَيْكُمْ أَهْلَ الدِّيَارِ مِنَ المُؤْمِنِينَ وَ المُسْلِمِينَ وَ إِنَّا إِنْ شَاءَ اللَّهُ بِكُمْ لاَحِقُونَ نَسْأَلُ اللَّهَ لَنَا وَ لَكُمْ العَافِيَةَ).

٢٩) عِنْدَ المُبَارَكَةِ بِالزَّوَاجِ: (بَارَكَ اللَّهُ لَكَ وَ بَارَكَ عَلَيْكَ وَ جَمَعَ بَيْنَكُمَا فِي خَيْرٍ).

٣٠) عِنْدَ نُزُولِ المَطَرِ: (اللَّهُمَّ صَيِّبًا هَنِيئًا).

26. При посещении больного: „Удали вред, Господь, людей излечи, Ты ведь Лечащий; нет излечения, кроме как от Тебя, лечением, не оставляющим никакой болезни".

27. Когда постигла неприятность: „Истинно, мы принадлежим Богу, истинно, к Нему и возвращаемся".

28. При входе на кладбище: „Мир вам, обитатели данной земли, из правоверных и мусульман, и мы по воле Бога догоним вас. Просим у Бога себе и вам благополучия".

29. При благословении брака: „Да благословит Бог тебя и пошлёт благо тебе, и да сведёт вас обоих в добре".

30. Когда идёт дождь: „О Бог, приятного дождя".

٣١) عِنْدَ لِقَاءِ العَدُوِّ: (اللَّهُمَّ إِنَّا نَجْعَلُكَ فِي نُحُورِهِمْ وَ نَعُوذُ بِكَ مِنْ شُرُورِهِمْ).

٣٢) عِنْدَ رُؤْيَةِ الهِلاَلِ: (اللَّهُ أَكْبَرُ، أَللَّهُمَّ أَهِلَّهُ عَلَيْنَا بِالأَمْنِ وَ الإِيمَانِ وَ السَّلاَمَةِ وَ الإِسْلاَمِ رَبُّنَا وَ رَبُّكَ اللَّهُ. هِلاَلُ رُشْدٍ وَ خَيْرٍ).

٣٣) عِنْدَ النُّزُولِ فِي مَكَانٍ مَا: (أَعُوذُ بِكَلِمَاتِ اللَّهِ التَّامَّاتِ مِنْ شَرِّ مَا خَلَقَ).

٣٤) عِنْدَ الكَرْبِ وَ الحَزَنِ: (اللَّهُمَّ إِنَّا نَعُوذُ بِكَ مِنَ الهَمِّ وَ الحَزَنِ، وَ نَعُوذُ بِكَ مِنَ العَجْزِ وَ الكَسَلِ وَ نَعُوذُ بِكَ مِنَ الجُبْنِ وَ البُخْلِ وَ نَعُوذُ بِكَ مِنْ غَلَبَةِ الدَّيْنِ وَ قَهْرِ الرِّجَالِ).

31. При встрече с врагом: „О Бог, истинно мы направляем Тебя в их груди и прибегаем к тебе от их зла".

32. Увидев новолуние: „Бог превелик! О Бог, сделай его появившимся над нами с безопасностью, с иманом, с благополучием и с Исламом. Наш Господь и твой Господь — Бог. Новолуние благоразумия и добра".

33. Когда останавливаешься где-либо: „Прибегаю к совершеннейшим словам Бога от зла того, что Он создал".

34. При горести и печали: „О Бог, к Тебе, поистине, прибегаем от заботы и печали, и к Тебе прибегаем от немощи и лености, и к Тебе прибегаем от трусости и скупости, и к Тебе прибегаем от засилия долга и человеческого насилия".

آيَةُ الكُرْسِيِّ

أَفْضَلُ آيَةٍ فِي القُرْآنِ آيَةُ الكُرْسِيِّ فَاقْرَأْهَا دُبُرَ كُلِّ صَلاَةٍ مُفْرُوضَةٍ. (اللَّهُ لاَ إِلَهَ إِلاَّ هُوَ الحَيُّ القَيُّومُ لاَ تَأْخُذُهُ سِنَةٌ وَ لاَ نَوْمٌ لَهُ مَا فِي السَّمَاوَاتِ وَ مَا فِي الأَرْضِ مَنْ ذَا الَّذِي يَشْفَعُ عِنْدَهُ إِلاَّ بِإِذْنِهِ يَعْلَمُ مَا بَيْنَ أَيْدِيهِمْ وَ مَا خَلْفَهُمْ وَ لاَ يُحِيطُونَ بِشَيْءٍ مِنْ عِلْمِهِ إِلاَّ بِمَا شَاءَ وَسِعَ كُرْسِيُّهُ السَّمَاوَاتِ وَ الأَرْضَ وَ لاَ يَئُودُهُ حِفْظُهُمَا وَ هُوَ العَلِيُّ العَظِيمُ) (سورة البقرة- ٢٥٥)

سَيِّدُ الاِسْتِغْفَارِ

(اللَّهُمَّ أَنْتَ رَبِّي لاَ إِلَهَ إِلاَّ أَنْتَ خَلَقْتَنِي وَ أَنَا عَبْدُكَ وَ أَنَا عَلَى عَهْدِكَ وَ وَعْدِكَ مَا اسْتَطَعْتُ أَعُوذُ بِكَ مِنْ شَرِّ مَا صَنَعْتَ، أَبُوءُ لَكَ بِنِعْمَتِكَ عَلَيَّ وَ أَبُوءُ بِذَنْبِي فَاغْفِرْ لِي فَإِنَّهُ لاَ يَغْفِرُ الذُّّنُوبَ إِلاَّ أَنْتَ).

Аятуль-Курси (аят Трона)

Самый предпочтительный аят в Коране — это Аятуль-Курси. Читай же его после каждого обязательного намаза.

„Бог, нет божества кроме Него, Живого, Самосущего, не овладевают Им ни дремота, ни сон. Ему принадлежит всё, что есть в небесах, и всё, что на земле. Кто может заступиться перед Ним, кроме как с позволения Его? Он знает, что впереди них, и что позади них; и ничего не могут охватить они из ведения Его, кроме того, что Он захочет. Его Престол объемлет небеса и Землю, и не отягощает Его охрана их. И Он — Всевышний, Великий".

Главная молитва о прощении грехов

,,О Бог, Ты мой Господь. Нет бога, кроме Тебя. Ты создал меня, и я Твой раб. Я верен обету и слову, данному Тебе по мере моих сил. Прибегаю я к Тебе от зла того, что я совершил. Признаюсь Тебе в Твоей милости ко мне, и признаюсь в своем грехе. Прости же меня, ведь, истинно, никто не прощает грехов, кроме Тебя".

الحمد لله على تمام معصم شرح معانى مفردات "الدروس الاولية" و كان الفراغ منه 4 يناير عام 1992 الموافق لآخر يوم من جمادى الاخرى لعام 1412 هـ و انا بهاء الدين محمد النـﭭرى الداغستانى

\clearpage{\centering
بِسْـــمِ اللهِ الرَّحْمَنِ الرَّحِيمِ
\par}

{\centering
\textbf{الدُّرُوسُ الاَوَّلِيَّةُ}
\par}

\section{ловарь‎}
\subsection[Урок 1‎]{\textstyleDropCaps{Урок 1‎}}
\textstyleCaptioncharacters{هُوَ }\textstyleDropCaps{он‎}

\textstyleCaptioncharacters{هُمْ }\textstyleDropCaps{они‎}

\textstyleCaptioncharacters{اَنْتَ }\textstyleDropCaps{ты ‎}

\textstyleCaptioncharacters{اَنْتُمْ }\textstyleDropCaps{вы ‎}

\textstyleCaptioncharacters{اَنَا }\textstyleDropCaps{я‎}

\textstyleCaptioncharacters{نَحْنُ }\textstyleDropCaps{мы‎}

\textstyleCaptioncharacters{كَبِيرٌ }\textstyleDropCaps{большой‎}

\textstyleCaptioncharacters{صَغِيرٌ \ }\textstyleDropCaps{маленький‎}

\subsection[Урок 2‎]{\textstyleDropCaps{Урок 2‎}}
\textstyleCaptioncharacters{هِىَ }\textstyleDropCaps{она‎}

\textstyleCaptioncharacters{هُنَّ }\textstyleDropCaps{они (ж.р.)‎}

\textstyleCaptioncharacters{اَنْتِ }\textstyleDropCaps{ты (ж.р.)‎}

\textstyleCaptioncharacters{اَنْتُنَّ }\textstyleDropCaps{вы (ж.р.)‎}

\textstyleCaptioncharacters{اَنَا }\textstyleDropCaps{я‎}

\textstyleCaptioncharacters{نَحْنُ }\textstyleDropCaps{мы‎}

\textstyleCaptioncharacters{كَبِيرَةٌ }\textstyleDropCaps{большая‎}

\textstyleCaptioncharacters{صَغِيرَةٌ }\textstyleDropCaps{маленькая‎}

\subsection[Урок 3‎]{\textstyleDropCaps{Урок 3‎}}
\textstyleCaptioncharacters{هَذَا }\textstyleDropCaps{этот, это‎}

\textstyleCaptioncharacters{هَذِهِ }\textstyleDropCaps{эта, это‎}

\textstyleCaptioncharacters{هَؤُلاَءِ }\textstyleDropCaps{эти, это‎}

\textstyleCaptioncharacters{رَجُلٌ }\textstyleDropCaps{мужчина‎}

\textstyleCaptioncharacters{اِمْرَأَةٌ }\textstyleDropCaps{женщина‎}

\textstyleCaptioncharacters{طَوِيلٌ }\textstyleDropCaps{длинный; высо­кий, высокого роста‎}

\textstyleCaptioncharacters{قَصِيرٌ }\textstyleDropCaps{короткий; низкий, низкого роста‎}

\subsection[Урок 4‎]{\textstyleDropCaps{Урок 4‎}}
\textstyleCaptioncharacters{مُعَلِّمٌ }\textstyleDropCaps{учитель‎}

\textstyleCaptioncharacters{مُعَلِّمَةٌ }\textstyleDropCaps{учительница‎}

\textstyleCaptioncharacters{تِلْمِيذٌ }\textstyleDropCaps{ученик‎}

\textstyleCaptioncharacters{تِلْمِيذَةٌ }\textstyleDropCaps{ученица‎}

\textstyleCaptioncharacters{مَنْ؟ }\textstyleDropCaps{кто?‎}

\subsection[Урок 5‎]{\textstyleDropCaps{Урок 5‎}}
\textstyleCaptioncharacters{كِتَابٌ }\textstyleDropCaps{книга‎}

\textstyleCaptioncharacters{دَفْتَرٌ }\textstyleDropCaps{тетрадь‎}

\textstyleCaptioncharacters{مِحْفَظَةٌ }\textstyleDropCaps{портфель‎}

\textstyleCaptioncharacters{قَلَمٌ }\textstyleDropCaps{ручка; карандаш‎}

\textstyleCaptioncharacters{مَا؟ }\textstyleDropCaps{что?‎}

\textstyleCaptioncharacters{مَا هَذَا؟ }\textstyleDropCaps{что это?‎}

\textstyleCaptioncharacters{لِمَنْ؟ }\textstyleDropCaps{чей?‎}

\textstyleCaptioncharacters{لِلْمُعَلِّمِ }\textstyleDropCaps{учителя‎}

\subsection[Урок 6‎]{\textstyleDropCaps{Урок 6‎}}
\textstyleCaptioncharacters{بَيْتٌ }\textstyleDropCaps{дом‎}

\textstyleCaptioncharacters{حُجْرَةٌ }\textstyleDropCaps{комната‎}

\textstyleCaptioncharacters{مَدْرَسَةٌ }\textstyleDropCaps{школа‎}

\textstyleCaptioncharacters{فَصْلٌ }\textstyleDropCaps{класс‎}

\textstyleCaptioncharacters{ذَاكٌ }\textstyleDropCaps{(вон) тот, (вон) то‎}

\textstyleCaptioncharacters{تِلْكَ }\textstyleDropCaps{та‎}

\textstyleCaptioncharacters{أُولَئِكَ }\textstyleDropCaps{те‎}

\textstyleCaptioncharacters{اَيْنَ؟ }\textstyleDropCaps{где?‎}

\textstyleCaptioncharacters{هُنَا }\textstyleDropCaps{здесь‎}

\textstyleCaptioncharacters{هُنَاكَ }\textstyleDropCaps{там‎}

\textstyleCaptioncharacters{فِى...ِ }\textstyleDropCaps{в (предлог)‎}

\textstyleCaptioncharacters{فِى الْمَدْرَسَةِ }\textstyleDropCaps{в школе‎}

\subsection[Урок 7‎]{\textstyleDropCaps{Урок 7‎}}
\textstyleCaptioncharacters{جَرِيدَةٌ }\textstyleDropCaps{газета‎}

\textstyleCaptioncharacters{مَجَلَّةٌ }\textstyleDropCaps{журнал‎}

\textstyleCaptioncharacters{عَرَبِيٌّ }\textstyleDropCaps{арабский, араб‎}

\textstyleCaptioncharacters{رُوسِيٌّ }\textstyleDropCaps{русский‎}

\textstyleCaptioncharacters{جَمِيلٌ }\textstyleDropCaps{красивый‎}

\textstyleCaptioncharacters{وَ }\textstyleDropCaps{и (союз)‎}

\textstyleCaptioncharacters{كِتَابٌ وَ دَفْتَرٌ }\textstyleDropCaps{книга и тетрадь‎}

\textstyleCaptioncharacters{لَهُ }\textstyleDropCaps{у него, его‎}

\textstyleCaptioncharacters{لَهَا }\textstyleDropCaps{у нее, её‎}

\textstyleCaptioncharacters{لَكَ }\textstyleDropCaps{у тебя, твой‎}

\textstyleCaptioncharacters{لَكِ }\textstyleDropCaps{у тебя, твой (ж.р.)‎}

\textstyleCaptioncharacters{لِى }\textstyleDropCaps{у меня, мой‎}

\subsection[Урок 8‎]{\textstyleDropCaps{Урок 8‎}}
\textstyleCaptioncharacters{كَلْبٌ }\textstyleDropCaps{собака‎}

\textstyleCaptioncharacters{دِيكٌ }\textstyleDropCaps{петух‎}

\textstyleCaptioncharacters{دَجَاجَةٌ }\textstyleDropCaps{курица‎}

\textstyleCaptioncharacters{مَكْتَبَةٌ }\textstyleDropCaps{библиотека; книж­ный магазин‎}

\textstyleCaptioncharacters{هَلْ }\textstyleDropCaps{ли (вопросительная ча­стица)‎}

\textstyleCaptioncharacters{نَعَمْ }\textstyleDropCaps{да‎}

\textstyleCaptioncharacters{لاَ }\textstyleDropCaps{нет‎}

\textstyleCaptioncharacters{لَيْسَ }\textstyleDropCaps{он не (есть)‎}

\textstyleCaptioncharacters{لَيْسَتْ }\textstyleDropCaps{она не (есть)‎}

\subsection[Урок 9‎]{\textstyleDropCaps{Урок 9‎}}
\textstyleCaptioncharacters{دَرْسٌ }\textstyleDropCaps{урок‎}

\textstyleCaptioncharacters{اِقْرَأْ }\textstyleDropCaps{(ты) читай‎}

\textstyleCaptioncharacters{اِقْرَئِى }\textstyleDropCaps{(ты) читай (ж.р.)‎}

\textstyleCaptioncharacters{اُدْخُلْ }\textstyleDropCaps{войди‎}

\textstyleCaptioncharacters{اُخْرُجْ }\textstyleDropCaps{выйди‎}

\textstyleCaptioncharacters{اُكْتُبْ }\textstyleDropCaps{пиши‎}

\textstyleCaptioncharacters{يَا }\textstyleDropCaps{о! эй! (частица обраще­ния)‎}

\textstyleCaptioncharacters{يَا مُحَمَّدُ }\textstyleDropCaps{о Мухаммад!‎}

\textstyleCaptioncharacters{يَا فَاطِمَةُ }\textstyleDropCaps{о Фатима!‎}

\textstyleCaptioncharacters{يَا مُحَمَّدُ اِقْرَأْ }\textstyleDropCaps{Мухам­мад, читай‎}

\textstyleCaptioncharacters{يَا فَاطِمَةُ اِقْرَئِى \ }\textstyleDropCaps{Фатим­а, читай‎}

\textstyleCaptioncharacters{مِنْ...ِ }\textstyleDropCaps{из, от (предлог)‎}

\textstyleCaptioncharacters{مِنَ الْبَيْتِ }\textstyleDropCaps{из дома‎}

\subsection[Урок 10‎]{\textstyleDropCaps{Урок 10‎}}
\textstyleCaptioncharacters{مَقْعَدُ التِّلْمِيذِ }\textstyleDropCaps{парта‎}

\textstyleCaptioncharacters{يَقْرَأُ }\textstyleDropCaps{читает‎}

\textstyleCaptioncharacters{يَكْتُبُ }\textstyleDropCaps{пишет‎}

\textstyleCaptioncharacters{خُذْ }\textstyleDropCaps{на, бери‎}

\textstyleCaptioncharacters{خُذِى }\textstyleDropCaps{на, бери (ж.р.)‎}

\textstyleCaptioncharacters{هَاتِ }\textstyleDropCaps{дай‎}

\textstyleCaptioncharacters{هَاتِى }\textstyleDropCaps{дай (ж.р.)‎}

\textstyleCaptioncharacters{كُرْسِيٌّ }\textstyleDropCaps{стул‎}

\subsection[Урок 11‎]{\textstyleDropCaps{Урок 11‎}}
\textstyleCaptioncharacters{مَدِينَةٌ }\textstyleDropCaps{город‎}

\textstyleCaptioncharacters{مَكْتَبٌ }\textstyleDropCaps{письменный стол‎}

\textstyleCaptioncharacters{مَكْتَبُ الْمُعَلِّمِ }\textstyleDropCaps{учитель­ский стол‎}

\textstyleCaptioncharacters{كَاتِبٌ }\textstyleDropCaps{писатель‎}

\textstyleCaptioncharacters{كَاتِبَةٌ }\textstyleDropCaps{писательница‎}

\textstyleCaptioncharacters{عَلَى }\textstyleDropCaps{на (предлог)‎}

\textstyleCaptioncharacters{عَلَى مَقْعَدِ التِّلْمِيذِ }\textstyleDropCaps{на парте‎}

\textstyleCaptioncharacters{عَلَيْهِ }\textstyleDropCaps{на нём‎}

\textstyleCaptioncharacters{بَلْ }\textstyleDropCaps{наоборот, а‎}

\subsection[Урок 12‎]{\textstyleDropCaps{Урок 12‎}}
\textstyleCaptioncharacters{وَلَدٌ }\textstyleDropCaps{мальчик‎}

\textstyleCaptioncharacters{اَوْلاَدٌ }\textstyleDropCaps{дети‎}

\textstyleCaptioncharacters{خُبْزٌ }\textstyleDropCaps{хлеб‎}

\textstyleCaptioncharacters{كُوبٌ }\textstyleDropCaps{стакан‎}

\textstyleCaptioncharacters{لَبَنٌ }\textstyleDropCaps{молоко‎}

\textstyleCaptioncharacters{مَاءٌ }\textstyleDropCaps{вода‎}

\textstyleCaptioncharacters{بَارِدٌ }\textstyleDropCaps{холодный‎}

\textstyleCaptioncharacters{سَخِينٌ }\textstyleDropCaps{горячий‎}

\textstyleCaptioncharacters{لَذِيذٌ }\textstyleDropCaps{вкусный‎}

\textstyleCaptioncharacters{كُلْ }\textstyleDropCaps{ешь, кушай‎}

\textstyleCaptioncharacters{اِشْرَبْ }\textstyleDropCaps{пей‎}

\textstyleCaptioncharacters{فِيهِ }\textstyleDropCaps{в нём, там‎}

\textstyleCaptioncharacters{يَا اَيُّهَا الْوَلَدُ }\textstyleDropCaps{о мальчик!‎}

\textstyleCaptioncharacters{اِقْرَأْهُ }\textstyleDropCaps{читай его‎}

\textstyleCaptioncharacters{اُكْتُبْهُ }\textstyleDropCaps{пиши его‎}

\subsection[Урок 13‎]{\textstyleDropCaps{Урок 13‎}}
\textstyleCaptioncharacters{بَابٌ }\textstyleDropCaps{дверь‎}

\textstyleCaptioncharacters{شُبَّاكٌ }\textstyleDropCaps{окно‎}

\textstyleCaptioncharacters{طَاوِلَةٌ }\textstyleDropCaps{стол‎}

\textstyleCaptioncharacters{بَابُ الْبَيْتِ }\textstyleDropCaps{дверь дома‎}

\textstyleCaptioncharacters{شُبَّاكُ الْحُجْرَةِ }\textstyleDropCaps{окно ком­наты‎}

\textstyleCaptioncharacters{مَفْتُوحٌ }\textstyleDropCaps{открытый‎}

\textstyleCaptioncharacters{مُغْلَقٌ }\textstyleDropCaps{закрытый‎}

\textstyleCaptioncharacters{بَابٌ مَفْتُوحٌ }\textstyleDropCaps{открытая дверь‎}

\textstyleCaptioncharacters{مِقْلَمَةٌ }\textstyleDropCaps{пенал‎}

\textstyleCaptioncharacters{تَعَالَ }\textstyleDropCaps{иди (сюда)‎}

\textstyleCaptioncharacters{تَعَالَىْ }\textstyleDropCaps{иди (сюда) (ж.р)‎}

\subsection[Урок 14‎]{\textstyleDropCaps{Урок 14‎}}
\textstyleCaptioncharacters{فِنْجَانٌ }\textstyleDropCaps{чашка‎}

\textstyleCaptioncharacters{سُكَّرٌ }\textstyleDropCaps{сахар‎}

\textstyleCaptioncharacters{سُكَّرِيَّةٌ }\textstyleDropCaps{сахарница‎}

\textstyleCaptioncharacters{قَهْوَةٌ }\textstyleDropCaps{кофе‎}

\textstyleCaptioncharacters{شَاىٌ }\textstyleDropCaps{чай‎}

\textstyleCaptioncharacters{بِالسُّكَّرِ }\textstyleDropCaps{с сахаром‎}

\textstyleCaptioncharacters{مَاذَا }\textstyleDropCaps{что?‎}

\textstyleCaptioncharacters{مَاذَا تَفْعَلُ }\textstyleDropCaps{что ты дела­ешь?‎}

\textstyleCaptioncharacters{لاَ يَفْعَلُ }\textstyleDropCaps{не делает‎}

\textstyleCaptioncharacters{هَلْ تَفْعَلُ }\textstyleDropCaps{ты делаешь? ты будешь делать?‎}

\textstyleCaptioncharacters{هُوَ يَفْعَلُ }\textstyleDropCaps{он делает‎}

\textstyleCaptioncharacters{هِىَ تَفْعَلُ }\textstyleDropCaps{она делает‎}

\textstyleCaptioncharacters{اَنْتَ تَفْعَلُ }\textstyleDropCaps{ты делаешь‎}

\textstyleCaptioncharacters{اَنْتِ تَفْعَلِينَ }\textstyleDropCaps{ты (ж.р.) де­лаешь‎}

\textstyleCaptioncharacters{اَنَا اَفْعَلُ }\textstyleDropCaps{я делаю‎}

\subsection[Урок 15‎]{\textstyleDropCaps{Урок 15‎}}
\textstyleCaptioncharacters{اَبٌ }\textstyleDropCaps{отец‎}

\textstyleCaptioncharacters{اُمٌّ }\textstyleDropCaps{мать‎}

\textstyleCaptioncharacters{اَخٌ }\textstyleDropCaps{брат‎}

\textstyleCaptioncharacters{اُخْتٌ }\textstyleDropCaps{сестра‎}

\textstyleCaptioncharacters{اِبْنٌ }\textstyleDropCaps{сын‎}

\textstyleCaptioncharacters{بِنْتٌ }\textstyleDropCaps{дочь‎}

\textstyleCaptioncharacters{اَبُوهُ }\textstyleDropCaps{его отец‎}

\textstyleCaptioncharacters{اَبُوهَا }\textstyleDropCaps{её отец‎}

\textstyleCaptioncharacters{اَبُوكَ }\textstyleDropCaps{твой отец‎}

\textstyleCaptioncharacters{اَبُوكِ }\textstyleDropCaps{твой (ж.р.) отец‎}

\textstyleCaptioncharacters{اَبِى }\textstyleDropCaps{мой отец‎}

\textstyleCaptioncharacters{ـهُ... }\textstyleDropCaps{его‎}

\textstyleCaptioncharacters{ـهَا... }\textstyleDropCaps{её‎}

\textstyleCaptioncharacters{ـكَ... }\textstyleDropCaps{твой‎}

\textstyleCaptioncharacters{ـكِ... }\textstyleDropCaps{твой (ж.р)‎}

\textstyleCaptioncharacters{ـى... }\textstyleDropCaps{мой‎}

\subsection[Урок 16‎]{\textstyleDropCaps{Урок 16‎}}
\textstyleCaptioncharacters{مِمْحَاةٌ }\textstyleDropCaps{резинка‎}

\textstyleCaptioncharacters{رِيشَةٌ }\textstyleDropCaps{перо‎}

\textstyleCaptioncharacters{دَارٌ }\textstyleDropCaps{дом‎}

\textstyleCaptioncharacters{سَاحَةٌ }\textstyleDropCaps{двор‎}

\textstyleCaptioncharacters{يَرْكُضُ }\textstyleDropCaps{бегает‎}

\textstyleCaptioncharacters{يَلْعَبُ }\textstyleDropCaps{играет‎}

\textstyleCaptioncharacters{اُرْكُضْ }\textstyleDropCaps{бегай‎}

\textstyleCaptioncharacters{اِلْعَبْ }\textstyleDropCaps{играй‎}

\textstyleCaptioncharacters{فِى سَاحَةِ الدَّارِ }\textstyleDropCaps{во дво­ре дома‎}

\textstyleCaptioncharacters{فِى سَاحَةِ الْمَدْرَسَةِ }\textstyleDropCaps{на школьном дворе‎}

\subsection[Урок 17‎]{\textstyleDropCaps{Урок 17‎}}
\textstyleCaptioncharacters{هُوَ قَرَأَ }\textstyleDropCaps{он читал‎}

\textstyleCaptioncharacters{هِىَ قَرَأَتْ }\textstyleDropCaps{она читала‎}

\textstyleCaptioncharacters{اَنْتَ قَرَأْتَ }\textstyleDropCaps{ты читал‎}

\textstyleCaptioncharacters{اَنْتِ قَرَأْتِ }\textstyleDropCaps{ты читала‎}

\textstyleCaptioncharacters{اَنَا قَرَأْتُ }\textstyleDropCaps{я читал‎}

\textstyleCaptioncharacters{كَتَبَ }\textstyleDropCaps{написал‎}

\textstyleCaptioncharacters{اَخَذَ }\textstyleDropCaps{взял‎}

\textstyleCaptioncharacters{اَكَلَ }\textstyleDropCaps{покушал, съел‎}

\textstyleCaptioncharacters{شَرِبَ }\textstyleDropCaps{выпил‎}

\textstyleCaptioncharacters{دَخَلَ }\textstyleDropCaps{вошёл‎}

\textstyleCaptioncharacters{خَرَجَ }\textstyleDropCaps{вышел‎}

\textstyleCaptioncharacters{هَوَ مَا قَرَأَ }\textstyleDropCaps{он не прочи­тал‎}

\textstyleCaptioncharacters{ثُمَّ }\textstyleDropCaps{потом, затем‎}

\textstyleCaptioncharacters{مَا قَرَأْتُ بَعْدُ }\textstyleDropCaps{я ещё не читал‎}

\textstyleCaptioncharacters{هَلْ قَرَأْتَ؟ }\textstyleDropCaps{ты читал?‎}

\textstyleCaptioncharacters{مَاذَا فَعَلْتَ؟ }\textstyleDropCaps{что ты де­лал?‎}

\subsection[Урок 18‎]{\textstyleDropCaps{Урок 18‎}}
\textstyleCaptioncharacters{كَلِمَةٌ }\textstyleDropCaps{слово‎}

\textstyleCaptioncharacters{قَاعَةٌ }\textstyleDropCaps{зал‎}

\textstyleCaptioncharacters{أَيْضًا }\textstyleDropCaps{тоже, также‎}

\textstyleCaptioncharacters{بَعْضٌ }\textstyleDropCaps{некоторые, одни‎}

\textstyleCaptioncharacters{اَمْسِ }\textstyleDropCaps{вчера‎}

\textstyleCaptioncharacters{اَلْيَوْمَ }\textstyleDropCaps{сегодня‎}

\textstyleCaptioncharacters{غَدًا }\textstyleDropCaps{завтра‎}

\textstyleCaptioncharacters{مَقْعَدٌ }\textstyleDropCaps{скамейка, седенье‎}

\textstyleCaptioncharacters{أَقَرَأْتَ؟ }\textstyleDropCaps{ты читал?‎}

\textstyleCaptioncharacters{اِلَى...ِ }\textstyleDropCaps{в, к, до (обозначает направление)‎}

\textstyleCaptioncharacters{اِلَى الْبَيْتِ }\textstyleDropCaps{к дому, домой‎}

\textstyleCaptioncharacters{اِنْ شَاءَ اللَّهُ }\textstyleDropCaps{если Богу будет угодно‎}

\textstyleCaptioncharacters{فَهِمَ }\textstyleDropCaps{понял‎}

\textstyleCaptioncharacters{جَلَسَ }\textstyleDropCaps{сел‎}

\textstyleCaptioncharacters{ذَهَبَ }\textstyleDropCaps{ушёл, пошёл, уехал‎}

\subsection[Урок 19‎]{\textstyleDropCaps{Урок 19‎}}
\textstyleCaptioncharacters{لَحْمٌ }\textstyleDropCaps{мясо‎}

\textstyleCaptioncharacters{زَمِيلٌ }\textstyleDropCaps{товарищ‎}

\textstyleCaptioncharacters{مَعَ...ِ }\textstyleDropCaps{с, вместе с‎}

\textstyleCaptioncharacters{قَلِيلٌ مِنْ...ِ }\textstyleDropCaps{немного, не­множко‎}

\textstyleCaptioncharacters{قَلِيلٌ مِنَ الْخُبْزِ }\textstyleDropCaps{немного хлеба‎}

\textstyleCaptioncharacters{كُرَةٌ }\textstyleDropCaps{мяч‎}

\textstyleCaptioncharacters{لاَ تَلْعَبْ }\textstyleDropCaps{не играй‎}

\textstyleCaptioncharacters{لاَ تَلْعَبِى }\textstyleDropCaps{не играй (ж.р.)‎}

\subsection[Урок 20‎]{\textstyleDropCaps{Урок 20‎}}
\textstyleCaptioncharacters{اَيْنَ كُنْتَ؟ }\textstyleDropCaps{где ты был?‎}

\textstyleCaptioncharacters{كُنْتُ }\textstyleDropCaps{я был‎}

\textstyleCaptioncharacters{مَتَى؟ }\textstyleDropCaps{когда?‎}

\textstyleCaptioncharacters{صَبَاحًا }\textstyleDropCaps{утром‎}

\textstyleCaptioncharacters{مَسَاءً }\textstyleDropCaps{вечером‎}

\textstyleCaptioncharacters{اَلآنَ }\textstyleDropCaps{сейчас, теперь‎}

\textstyleCaptioncharacters{قَبْلَ...ِ }\textstyleDropCaps{перед, до‎}

\textstyleCaptioncharacters{بَعْدَ...ِ }\textstyleDropCaps{после‎}

\textstyleCaptioncharacters{قَامَ }\textstyleDropCaps{встал‎}

\textstyleCaptioncharacters{قَامَ فَقَرَأَ دَرْسَهُ }\textstyleDropCaps{встал и прочитал свой урок‎}

\textstyleCaptioncharacters{قُمْ }\textstyleDropCaps{вставай, встань‎}

\textstyleCaptioncharacters{لا تَقُمْ }\textstyleDropCaps{не вставай‎}

\textstyleCaptioncharacters{قُمْ وَ اقْرَإِ الدَّرْسَ }\textstyleDropCaps{встань и читай урок‎}

\subsection[Урок 21‎]{\textstyleDropCaps{Урок 21‎}}
\textstyleCaptioncharacters{مُهَنْدِسٌ }\textstyleDropCaps{инженер‎}

\textstyleCaptioncharacters{مُدَرِّسٌ }\textstyleDropCaps{преподаватель‎}

\textstyleCaptioncharacters{مَاهِرٌ }\textstyleDropCaps{умелый, искусный, квалифицированный‎}

\textstyleCaptioncharacters{جَدِيدٌ }\textstyleDropCaps{новый‎}

\textstyleCaptioncharacters{دَافِئٌ }\textstyleDropCaps{тёплый‎}

\textstyleCaptioncharacters{سَيَّارَةٌ }\textstyleDropCaps{автомобиль, маши­на‎}

\textstyleCaptioncharacters{رَكِبَ السَّيّارَةَ }\textstyleDropCaps{сел в (на) машину‎}

\textstyleCaptioncharacters{قَادَ السَّيّارَةَ \ }\textstyleDropCaps{водил авто­мобиль‎}

\textstyleCaptioncharacters{عِنْدَ...ِ }\textstyleDropCaps{у‎}

\textstyleCaptioncharacters{لَسْتَ }\textstyleDropCaps{ты не (есть)‎}

\textstyleCaptioncharacters{لَسْتِ }\textstyleDropCaps{ты не (есть) (ж.р)‎}

\textstyleCaptioncharacters{لَسْتُ }\textstyleDropCaps{я не (есть)‎}

\textstyleCaptioncharacters{رَكِبَ }\textstyleDropCaps{сел (верхом), в ма­шину и т.п‎}

\textstyleCaptioncharacters{قَادَ }\textstyleDropCaps{водил, управлял‎}

\textstyleCaptioncharacters{رَجَعَ }\textstyleDropCaps{вернулся‎}

\subsection[Урок 22‎]{\textstyleDropCaps{Урок 22‎}}
\textstyleCaptioncharacters{بُسْتَانٌ }\textstyleDropCaps{сад‎}

\textstyleCaptioncharacters{قَرْيَةٌ }\textstyleDropCaps{деревня, селение‎}

\textstyleCaptioncharacters{شَجَرٌ }\textstyleDropCaps{дерево, деревья‎}

\textstyleCaptioncharacters{دِيوَانٌ }\textstyleDropCaps{диван‎}

\textstyleCaptioncharacters{نَبَاتٌ }\textstyleDropCaps{растение‎}

\textstyleCaptioncharacters{نَافِذَةٌ }\textstyleDropCaps{окно‎}

\textstyleCaptioncharacters{مَنْزِلٌ }\textstyleDropCaps{дом, жилище‎}

\textstyleCaptioncharacters{نَادِرٌ }\textstyleDropCaps{редкий‎}

\textstyleCaptioncharacters{مُخْتَلِفٌ }\textstyleDropCaps{разный, различ­ный‎}

\textstyleCaptioncharacters{كَثِيرٌ }\textstyleDropCaps{многочисленный, много‎}

\textstyleCaptioncharacters{وَثِيرٌ }\textstyleDropCaps{мягкий‎}

\textstyleCaptioncharacters{قَلِيلاً }\textstyleDropCaps{немного, немножко‎}

\textstyleCaptioncharacters{جِدًّا }\textstyleDropCaps{очень‎}

\textstyleCaptioncharacters{قُرْبَ...ِ }\textstyleDropCaps{близ, около‎}

\textstyleCaptioncharacters{ذَلِكَ }\textstyleDropCaps{тот, то‎}

\textstyleCaptioncharacters{مَاذَا تُرِيدُ؟ }\textstyleDropCaps{что ты хо­чешь?‎}

\textstyleCaptioncharacters{اُرِيدُ }\textstyleDropCaps{я хочу‎}

\textstyleCaptioncharacters{اَتُرِيدُ؟ }\textstyleDropCaps{ты хочешь?‎}

\textstyleCaptioncharacters{لا اُرِيدُ }\textstyleDropCaps{не хочу‎}

\textstyleCaptioncharacters{تَعَالَ هُنَا نَقْرَإِ الدَّرْسَ }\textstyleDropCaps{иди сюда, прочитаем урок‎}

\subsection[Урок 23‎]{\textstyleDropCaps{Урок 23‎}}
\textstyleCaptioncharacters{مَسْجِدٌ }\textstyleDropCaps{мечеть‎}

\textstyleCaptioncharacters{شِقَّةٌ }\textstyleDropCaps{квартира‎}

\textstyleCaptioncharacters{اَلَيْسَ؟ }\textstyleDropCaps{нел ли? не (есть) ли?‎}

\textstyleCaptioncharacters{عَنْ...ِ }\textstyleDropCaps{от (предлог)‎}

\textstyleCaptioncharacters{قَرِيبٌ }\textstyleDropCaps{близкий‎}

\textstyleCaptioncharacters{بَعِيدٌ }\textstyleDropCaps{делёкий‎}

\textstyleCaptioncharacters{قَدِيمٌ }\textstyleDropCaps{старый, древний‎}

\textstyleCaptioncharacters{هُمْ... }\textstyleDropCaps{их‎}

\textstyleCaptioncharacters{هُنَّ... }\textstyleDropCaps{их (ж.р.)‎}

\textstyleCaptioncharacters{كُمْ... }\textstyleDropCaps{ваш‎}

\textstyleCaptioncharacters{كُنَّ... }\textstyleDropCaps{ваш (ж.р.)‎}

\textstyleCaptioncharacters{نَا... }\textstyleDropCaps{наш‎}

\textstyleCaptioncharacters{بَيْتُهُمْ }\textstyleDropCaps{их дом‎}

\textstyleCaptioncharacters{بَيْتُهُنَّ }\textstyleDropCaps{их (ж.р.) дом‎}

\textstyleCaptioncharacters{بَيْتُكُمْ }\textstyleDropCaps{ваш дом‎}

\textstyleCaptioncharacters{بَيْتُكُنَّ }\textstyleDropCaps{ваш (ж.р.) дом‎}

\textstyleCaptioncharacters{بَيْتُنَا }\textstyleDropCaps{наш дом‎}

\textstyleCaptioncharacters{سَكَنَ }\textstyleDropCaps{жил‎}

\textstyleCaptioncharacters{نَظَرَ }\textstyleDropCaps{посмотрел‎}

\textstyleCaptioncharacters{رَحَلَ }\textstyleDropCaps{уехал (откуда), переехал‎}

\textstyleCaptioncharacters{هُمْ سَكَنُوا }\textstyleDropCaps{они жили‎}

\textstyleCaptioncharacters{هُنَّ سَكَنَّ }\textstyleDropCaps{они (ж.р.) жили‎}

\textstyleCaptioncharacters{اَنْتُمْ سَكَنْتُمْ }\textstyleDropCaps{вы жили‎}

\textstyleCaptioncharacters{اَنْتُنَّ سَكَنْتُنَّ }\textstyleDropCaps{вы (ж.р.) жили‎}

\textstyleCaptioncharacters{نَحْنُ سَكَنَّا }\textstyleDropCaps{мы жили‎}

\subsection[Урок 24‎]{\textstyleDropCaps{Урок 24‎}}
\textstyleCaptioncharacters{هِرٌّ }\textstyleDropCaps{кот, кошка‎}

\textstyleCaptioncharacters{فَأْرٌ }\textstyleDropCaps{мышь‎}

\textstyleCaptioncharacters{فَتًى }\textstyleDropCaps{юноша, парень‎}

\textstyleCaptioncharacters{فَتَاةٌ }\textstyleDropCaps{девушка‎}

\textstyleCaptioncharacters{شَابٌّ }\textstyleDropCaps{молодой‎}

\textstyleCaptioncharacters{اُسْتَاذٌ }\textstyleDropCaps{профессор‎}

\textstyleCaptioncharacters{فَلاَّحٌ }\textstyleDropCaps{крестьянин‎}

\textstyleCaptioncharacters{فَلاَّحَةٌ }\textstyleDropCaps{крестьянка‎}

\textstyleCaptioncharacters{غَنِيٌّ }\textstyleDropCaps{богатый‎}

\textstyleCaptioncharacters{فَقِيرٌ }\textstyleDropCaps{бедный‎}

\textstyleCaptioncharacters{مَشْهُورٌ }\textstyleDropCaps{известный, знамен­итый‎}

\textstyleCaptioncharacters{رَفٌّ }\textstyleDropCaps{полка‎}

\textstyleCaptioncharacters{شَيْخٌ }\textstyleDropCaps{старик‎}

\textstyleCaptioncharacters{عَجُوزٌ }\textstyleDropCaps{старуха‎}

\textstyleCaptioncharacters{مُجْتَهِدٌ }\textstyleDropCaps{прилежный, ста­рательный‎}

\textstyleCaptioncharacters{مِصْرٌ }\textstyleDropCaps{Египет‎}

\textstyleCaptioncharacters{مِصْرِىٌّ }\textstyleDropCaps{египетский, егип­тянин‎}

\subsection[Урок 25‎]{\textstyleDropCaps{Урок 25‎}}
\textstyleCaptioncharacters{ثَوْرٌ }\textstyleDropCaps{бык‎}

\textstyleCaptioncharacters{بَقَرَةٌ }\textstyleDropCaps{корова‎}

\textstyleCaptioncharacters{فَرَسٌ }\textstyleDropCaps{лошадь‎}

\textstyleCaptioncharacters{حِمَارٌ }\textstyleDropCaps{осёл, ишак‎}

\textstyleCaptioncharacters{حَقْلٌ }\textstyleDropCaps{поле‎}

\textstyleCaptioncharacters{بَلَدٌ }\textstyleDropCaps{страна‎}

\textstyleCaptioncharacters{كُلُّهُمْ }\textstyleDropCaps{они все‎}

\textstyleCaptioncharacters{كُلُّكُمْ }\textstyleDropCaps{вы все‎}

\textstyleCaptioncharacters{كُلُّنَا }\textstyleDropCaps{мы все‎}

\textstyleCaptioncharacters{اَنْ يَفْعَلَ }\textstyleDropCaps{делать‎}

\textstyleCaptioncharacters{لِيَفْعَلَ }\textstyleDropCaps{чтобы делать, де­лать‎}

\textstyleCaptioncharacters{هُمْ يَقْرَؤُونَ }\textstyleDropCaps{они читают‎}

\textstyleCaptioncharacters{هُنَّ يَقْرَأْنَ }\textstyleDropCaps{они (ж.р.) чи­тают‎}

\textstyleCaptioncharacters{اَنْتُمْ تَقْرَؤُونَ }\textstyleDropCaps{вы читае­те‎}

\textstyleCaptioncharacters{اَنْتُنَّ تَقْرَأْنَ }\textstyleDropCaps{вы (ж.р.) чи­таете‎}

\textstyleCaptioncharacters{نَحْنُ نَقْرَأُ }\textstyleDropCaps{мы читаем‎}

\subsection[Урок 26‎]{\textstyleDropCaps{Урок 26‎}}
\textstyleCaptioncharacters{حَلِيبٌ }\textstyleDropCaps{молоко‎}

\textstyleCaptioncharacters{مَائِدَةٌ }\textstyleDropCaps{обеденный стол‎}

\textstyleCaptioncharacters{لَوْحٌ }\textstyleDropCaps{доска‎}

\textstyleCaptioncharacters{طَبَاشِيرُ }\textstyleDropCaps{мел‎}

\textstyleCaptioncharacters{مِحْبَرَةٌ }\textstyleDropCaps{чернильница‎}

\textstyleCaptioncharacters{مِلْحٌ }\textstyleDropCaps{соль‎}

\textstyleCaptioncharacters{مِمْلَحَةٌ }\textstyleDropCaps{солонка‎}

\textstyleCaptioncharacters{مِمْسَحَةٌ }\textstyleDropCaps{тряпка‎}

\textstyleCaptioncharacters{حِبْرٌ }\textstyleDropCaps{чернила‎}

\textstyleCaptioncharacters{فِلْمٌ }\textstyleDropCaps{фильм‎}

\textstyleCaptioncharacters{حَارٌّ }\textstyleDropCaps{горячий, жаркий‎}

\textstyleCaptioncharacters{جَيِّدٌ }\textstyleDropCaps{хороший, отличный‎}

\textstyleCaptioncharacters{يُوجَدُ }\textstyleDropCaps{есть, имеется‎}

\textstyleCaptioncharacters{اَجَلْ }\textstyleDropCaps{да‎}

\textstyleCaptioncharacters{فَقَطْ }\textstyleDropCaps{только‎}

\textstyleCaptioncharacters{وَاحِدٌ }\textstyleDropCaps{один‎}

\subsection[Урок 27‎]{\textstyleDropCaps{Урок 27‎}}
\textstyleCaptioncharacters{وَرْدَةٌ }\textstyleDropCaps{роза‎}

\textstyleCaptioncharacters{زَوْجٌ }\textstyleDropCaps{муж‎}

\textstyleCaptioncharacters{زَوْجَةٌ }\textstyleDropCaps{жена‎}

\textstyleCaptioncharacters{مَسْرَحٌ }\textstyleDropCaps{театр‎}

\textstyleCaptioncharacters{مَيْدَانٌ }\textstyleDropCaps{площадь‎}

\textstyleCaptioncharacters{حَدِيقَةٌ }\textstyleDropCaps{парк‎}

\textstyleCaptioncharacters{اَجْنَبِىٌّ }\textstyleDropCaps{иностранный, ино­странец‎}

\textstyleCaptioncharacters{مَنْ هُوَ هَذَا الرَّجُلُ }\textstyleDropCaps{кто этот мужчина, что это за мужчина?‎}

\textstyleCaptioncharacters{كَثِيرٌ مِنْ...ِ }\textstyleDropCaps{много‎}

\textstyleCaptioncharacters{كَثِيرٌ مِنَ الْكُتُبِ }\textstyleDropCaps{много книг‎}

\textstyleCaptioncharacters{اَمَامَ...ِ }\textstyleDropCaps{перед‎}

\textstyleCaptioncharacters{وَرَاءَ...ِ }\textstyleDropCaps{за, сзади‎}

\textstyleCaptioncharacters{يَمِينَ...ِ }\textstyleDropCaps{справа, направо‎}

\textstyleCaptioncharacters{يَسَار...ِ }\textstyleDropCaps{слева, налево‎}

\textstyleCaptioncharacters{فَوْقَ...ِ }\textstyleDropCaps{над‎}

\textstyleCaptioncharacters{تَحْتَ...ِ }\textstyleDropCaps{под‎}

\subsection[Урок 28‎]{\textstyleDropCaps{Урок 28‎}}
\textstyleCaptioncharacters{جَامِعَةٌ }\textstyleDropCaps{университет‎}

\textstyleCaptioncharacters{شَارَةٌ }\textstyleDropCaps{значок‎}

\textstyleCaptioncharacters{بَدْلَةٌ }\textstyleDropCaps{костюм‎}

\textstyleCaptioncharacters{عَامِلٌ }\textstyleDropCaps{рабочий‎}

\textstyleCaptioncharacters{عَامِلاَةٌ }\textstyleDropCaps{работница, рабо­чая‎}

\textstyleCaptioncharacters{مَعْمَلٌ }\textstyleDropCaps{фабрика‎}

\textstyleCaptioncharacters{حَيَاةٌ }\textstyleDropCaps{жизнь‎}

\textstyleCaptioncharacters{عَمَّاذَا؟ }\textstyleDropCaps{о чем?‎}

\textstyleCaptioncharacters{مِمَّنْ؟ }\textstyleDropCaps{от кого‎}

\textstyleCaptioncharacters{كَانَ }\textstyleDropCaps{(он) был‎}

\textstyleCaptioncharacters{تَسَلَّمَ }\textstyleDropCaps{получил‎}

\textstyleCaptioncharacters{عِنْدَ مَا خَرَجَ }\textstyleDropCaps{когда вы­шел‎}

\textstyleCaptioncharacters{عَنْ... }\textstyleDropCaps{о, об (предлог)‎}

\textstyleCaptioncharacters{رِسَالَةٌ }\textstyleDropCaps{письмо, послание‎}

\textstyleCaptioncharacters{عَرَفَ }\textstyleDropCaps{знал, узнал‎}

\textstyleCaptioncharacters{دَرَسَ }\textstyleDropCaps{учился; изучил‎}

\textstyleCaptioncharacters{عَمِلَ }\textstyleDropCaps{работал‎}

\subsection[Урок 29‎]{\textstyleDropCaps{Урок 29‎}}
\textstyleCaptioncharacters{اِقْرَأْ }\textstyleDropCaps{читай ‎}

\textstyleCaptioncharacters{اِقْرَؤُوا }\textstyleDropCaps{читайте‎}

\textstyleCaptioncharacters{اِقْرَئِى }\textstyleDropCaps{читай (ж.р.)‎}

\textstyleCaptioncharacters{اِقْرَأْنَ }\textstyleDropCaps{читайте (ж.р.)‎}

\textstyleCaptioncharacters{لاَ تَقْرَأْ }\textstyleDropCaps{не читай‎}

\textstyleCaptioncharacters{لاَ تَقْرَئِى }\textstyleDropCaps{не читай (ж.р.)‎}

\textstyleCaptioncharacters{لاَ تَقْرَؤُوا }\textstyleDropCaps{не читайте‎}

\textstyleCaptioncharacters{لاَ تَقْرَأْنَ }\textstyleDropCaps{не читайте (ж.р.)‎}

\textstyleCaptioncharacters{نَزَلَ }\textstyleDropCaps{сошёл, слез‎}

\textstyleCaptioncharacters{لَبِسَ }\textstyleDropCaps{надел‎}

\textstyleCaptioncharacters{فَتَحَ }\textstyleDropCaps{открыл‎}

\textstyleCaptioncharacters{شَرَحَ }\textstyleDropCaps{объяснил‎}

\textstyleCaptioncharacters{مَسَحَ }\textstyleDropCaps{вытер, стёр‎}

\textstyleCaptioncharacters{مَعْنًى }\textstyleDropCaps{смысл, значение‎}

\textstyleCaptioncharacters{آخَرُ }\textstyleDropCaps{другой‎}

\textstyleCaptioncharacters{عِنْدَ ذَلِكَ }\textstyleDropCaps{тогда, при этом‎}

\textstyleCaptioncharacters{اَلدُّنْيَا بَرْدٌ }\textstyleDropCaps{на улице хо­лодно‎}

\subsection[Урок 30‎]{\textstyleDropCaps{Урок 30‎}}
\textstyleCaptioncharacters{تَمْرٌ }\textstyleDropCaps{финики‎}

\textstyleCaptioncharacters{تِينٌ }\textstyleDropCaps{инжир‎}

\textstyleCaptioncharacters{زَيْتُونٌ }\textstyleDropCaps{оливы, маслины‎}

\textstyleCaptioncharacters{سَمَاءٌ }\textstyleDropCaps{небо‎}

\textstyleCaptioncharacters{مَطَرٌ }\textstyleDropCaps{дождь‎}

\textstyleCaptioncharacters{مَاءُ الْمَطَرِ }\textstyleDropCaps{дождевая вода‎}

\textstyleCaptioncharacters{وَرْدٌ }\textstyleDropCaps{розы‎}

\textstyleCaptioncharacters{ثَلْجٌ }\textstyleDropCaps{снег‎}

\textstyleCaptioncharacters{زَرْعٌ }\textstyleDropCaps{посев‎}

\textstyleCaptioncharacters{مُفِيدٌ }\textstyleDropCaps{полезный‎}

\textstyleCaptioncharacters{نَافِعٌ }\textstyleDropCaps{полезный, выгодный‎}

\textstyleCaptioncharacters{سَأَلَ }\textstyleDropCaps{спросил‎}

\textstyleCaptioncharacters{رَسَمَ }\textstyleDropCaps{нарисовал‎}

\textstyleCaptioncharacters{قَالَ }\textstyleDropCaps{сказал‎}

\textstyleCaptioncharacters{سَقَى }\textstyleDropCaps{поливал‎}

\textstyleCaptioncharacters{لِمَاذَا؟ }\textstyleDropCaps{почему? зачем?‎}

\textstyleCaptioncharacters{قِيلَ لَهُ }\textstyleDropCaps{ему сказали‎}

\textstyleCaptioncharacters{لا اَدْرِى }\textstyleDropCaps{не знаю‎}

\textstyleCaptioncharacters{نَزَلَ الْمَطَرُ }\textstyleDropCaps{пошёл дождь‎}

\subsection[Урок 31‎]{\textstyleDropCaps{Урок 31‎}}
\textstyleCaptioncharacters{عَلَمٌ }\textstyleDropCaps{знамя, флаг‎}

\textstyleCaptioncharacters{عَلَمُ الإِسلاَمِ }\textstyleDropCaps{знамя Исла­ма‎}

\textstyleCaptioncharacters{جُبْنٌ }\textstyleDropCaps{сыр‎}

\textstyleCaptioncharacters{طَبْلٌ }\textstyleDropCaps{барабан‎}

\textstyleCaptioncharacters{زَرَّاعٌ }\textstyleDropCaps{земледелец‎}

\textstyleCaptioncharacters{بَيَّاعٌ }\textstyleDropCaps{продавец‎}

\textstyleCaptioncharacters{رَبِيعٌ }\textstyleDropCaps{весна‎}

\textstyleCaptioncharacters{مَرْفُوعٌ }\textstyleDropCaps{поднятый‎}

\textstyleCaptioncharacters{حَسَنًا }\textstyleDropCaps{хорошо‎}

\textstyleCaptioncharacters{اَبَدًا }\textstyleDropCaps{никогда‎}

\textstyleCaptioncharacters{قَفَزَ }\textstyleDropCaps{прыгнул‎}

\textstyleCaptioncharacters{زَرَعَ }\textstyleDropCaps{посеял‎}

\textstyleCaptioncharacters{رَفَعَ }\textstyleDropCaps{поднял‎}

\textstyleCaptioncharacters{كَسَرَ }\textstyleDropCaps{поломал, сломал‎}

\textstyleCaptioncharacters{نَقَرَ }\textstyleDropCaps{бил, стучал‎}

\textstyleCaptioncharacters{بَاعَ }\textstyleDropCaps{продал‎}

\textstyleCaptioncharacters{رَقَصَ }\textstyleDropCaps{танцевал‎}

\textstyleCaptioncharacters{نَقَرَ عَلَى الطَّبْلِ }\textstyleDropCaps{барабан­ил, бил в барабан‎}

\textstyleCaptioncharacters{رَفْرَفَ }\textstyleDropCaps{развевался‎}

\subsection[Урок 32‎]{\textstyleDropCaps{Урок 32‎}}
\textstyleCaptioncharacters{فَرَاشَةٌ }\textstyleDropCaps{бабочка‎}

\textstyleCaptioncharacters{عُشٌّ }\textstyleDropCaps{гнездо‎}

\textstyleCaptioncharacters{طَيْرٌ }\textstyleDropCaps{птица‎}

\textstyleCaptioncharacters{رِيشٌ }\textstyleDropCaps{перо, перья‎}

\textstyleCaptioncharacters{غُصْنٌ }\textstyleDropCaps{ветка‎}

\textstyleCaptioncharacters{شُرْطِىُّ الْمُرُورِ }\textstyleDropCaps{регули­ровщик движения‎}

\textstyleCaptioncharacters{رَادِيُو }\textstyleDropCaps{радио‎}

\textstyleCaptioncharacters{مُسْلِمٌ }\textstyleDropCaps{мусульманин‎}

\textstyleCaptioncharacters{نَشِيطٌ }\textstyleDropCaps{активный, энергичн­ый‎}

\textstyleCaptioncharacters{اَلْحَمْدُ لِلَّهِ }\textstyleDropCaps{слава Богу‎}

\textstyleCaptioncharacters{شُرْطِىٌّ }\textstyleDropCaps{полицейский‎}

\textstyleCaptioncharacters{قَشٌّ }\textstyleDropCaps{солома‎}

\textstyleCaptioncharacters{اِنْسَانٌ }\textstyleDropCaps{человек‎}

\textstyleCaptioncharacters{شَجَرَةٌ }\textstyleDropCaps{(одно) дерево‎}

\textstyleCaptioncharacters{شَارِعٌ }\textstyleDropCaps{улица‎}

\textstyleCaptioncharacters{اِذَاعَةٌ }\textstyleDropCaps{передача‎}

\textstyleCaptioncharacters{اِذَاعَةُ رَادِيُو }\textstyleDropCaps{радиопере­дача‎}

\textstyleCaptioncharacters{عُبُورٌ }\textstyleDropCaps{переход‎}

\textstyleCaptioncharacters{مُرُورٌ }\textstyleDropCaps{движение‎}

\textstyleCaptioncharacters{جَمِيعًا }\textstyleDropCaps{все‎}

\textstyleCaptioncharacters{عَالَمٌ }\textstyleDropCaps{мир‎}

\textstyleCaptioncharacters{فِى الْعَالَمِ }\textstyleDropCaps{в мире‎}

\textstyleCaptioncharacters{سَمِعَ }\textstyleDropCaps{слышал‎}

\textstyleCaptioncharacters{وَقَفَ }\textstyleDropCaps{встал, стоял‎}

\textstyleCaptioncharacters{صَنَعَ }\textstyleDropCaps{делал, изготовил‎}

\textstyleCaptioncharacters{جَمَعَ }\textstyleDropCaps{собрал‎}

\textstyleCaptioncharacters{طَارَ }\textstyleDropCaps{летал, улетел‎}

\textstyleCaptioncharacters{سَاعَدَ }\textstyleDropCaps{помог‎}

\subsection[Урок 33‎]{\textstyleDropCaps{Урок 33‎}}
\textstyleCaptioncharacters{يَدٌ }\textstyleDropCaps{рука‎}

\textstyleCaptioncharacters{خِزَانَةٌ }\textstyleDropCaps{шкаф‎}

\textstyleCaptioncharacters{خِزَانَةُ كُتُبٍ }\textstyleDropCaps{книжный шкаф‎}

\textstyleCaptioncharacters{خَرِيفٌ }\textstyleDropCaps{осень‎}

\textstyleCaptioncharacters{مَعْهَدٌ }\textstyleDropCaps{институт‎}

\textstyleCaptioncharacters{اَىٌّ؟ }\textstyleDropCaps{какой?‎}

\textstyleCaptioncharacters{اَيَّةٌ؟ }\textstyleDropCaps{какая?‎}

\textstyleCaptioncharacters{اَىٌّ كِتَابٍ }\textstyleDropCaps{какая книга?‎}

\textstyleCaptioncharacters{جَيِّدًا }\textstyleDropCaps{хорошо, отлично‎}

\textstyleCaptioncharacters{عَادَ }\textstyleDropCaps{вернулся‎}

\textstyleCaptioncharacters{نَامَ }\textstyleDropCaps{спал‎}

\textstyleCaptioncharacters{اَطَاعَ }\textstyleDropCaps{подчинился, слушал­ся‎}

\textstyleCaptioncharacters{تَغَدَّى }\textstyleDropCaps{обедал‎}

\textstyleCaptioncharacters{تَعَشَّى }\textstyleDropCaps{ужинал‎}

\textstyleCaptioncharacters{لُغَةٌ }\textstyleDropCaps{язык, речь‎}

\textstyleCaptioncharacters{غُرْفَةٌ }\textstyleDropCaps{комната‎}

\textstyleCaptioncharacters{مَخْزَنٌ }\textstyleDropCaps{магазин‎}

\textstyleCaptioncharacters{اِنَاءٌ }\textstyleDropCaps{сосуд‎}

\textstyleCaptioncharacters{دَرَسَ اللُّغَةٌ الْعَرَبِيَّةَ }\textstyleDropCaps{изучил арабский ‎}

\textstyleCaptioncharacters{بِاللُّغَةِ الْعَرَبِيَّةِ }\textstyleDropCaps{язык на арабском языке‎}

\textstyleCaptioncharacters{نِمْتُ }\textstyleDropCaps{я спал‎}

\textstyleCaptioncharacters{نِمْنَا }\textstyleDropCaps{мы спали‎}

\textstyleCaptioncharacters{تَغَدَّيْتُ }\textstyleDropCaps{я обедал‎}

\subsection[Урок 34‎]{\textstyleDropCaps{Урок 34‎}}
\textstyleCaptioncharacters{سَمَكٌ }\textstyleDropCaps{рыба‎}

\textstyleCaptioncharacters{مِنَشَّةٌ }\textstyleDropCaps{хлопушка‎}

\textstyleCaptioncharacters{سَمَّاكٌ }\textstyleDropCaps{рыбак‎}

\textstyleCaptioncharacters{نَهْرٌ }\textstyleDropCaps{река‎}

\textstyleCaptioncharacters{بَحْرٌ }\textstyleDropCaps{море‎}

\textstyleCaptioncharacters{بُحَيْرَةٌ }\textstyleDropCaps{озеро‎}

\textstyleCaptioncharacters{سُوقٌ }\textstyleDropCaps{рынок, базар‎}

\textstyleCaptioncharacters{ذُبَابٌ }\textstyleDropCaps{мухи‎}

\textstyleCaptioncharacters{شُكْرًا لَكَ }\textstyleDropCaps{спасибо тебе‎}

\textstyleCaptioncharacters{خَبَرٌ }\textstyleDropCaps{известие, новость‎}

\textstyleCaptioncharacters{بِحَمْدِ اللَّهِ }\textstyleDropCaps{слава Богу‎}

\textstyleCaptioncharacters{قِرَاءَةٌ }\textstyleDropCaps{читать, чтение‎}

\textstyleCaptioncharacters{كِتَابَةٌ }\textstyleDropCaps{писать, письмо‎}

\textstyleCaptioncharacters{عَلَّمَ }\textstyleDropCaps{учил‎}

\textstyleCaptioncharacters{كُلٌّ }\textstyleDropCaps{каждый, всякий‎}

\textstyleCaptioncharacters{كُلَّ يَوْمٍ }\textstyleDropCaps{каждый день‎}

\textstyleCaptioncharacters{عَادَةً }\textstyleDropCaps{обычно‎}

\textstyleCaptioncharacters{أَوَّلَ اَمْسِ }\textstyleDropCaps{позавчера‎}

\textstyleCaptioncharacters{بَعْدَ غَدٍ }\textstyleDropCaps{послезавтра‎}

\textstyleCaptioncharacters{عَصًا }\textstyleDropCaps{палка‎}

\textstyleCaptioncharacters{عَاشَ }\textstyleDropCaps{жил‎}

\textstyleCaptioncharacters{صَادَ }\textstyleDropCaps{ловил, охотился‎}

\textstyleCaptioncharacters{طَرَدَ }\textstyleDropCaps{прогнал, выгнал‎}

\textstyleCaptioncharacters{سَبَحَ }\textstyleDropCaps{плавал, плыл‎}

\subsection[Урок 35‎]{\textstyleDropCaps{Урок 35‎}}
\textstyleCaptioncharacters{مَلْعَبٌ }\textstyleDropCaps{стадион‎}

\textstyleCaptioncharacters{كُرَةُ الْقَدَمِ }\textstyleDropCaps{футбол‎}

\textstyleCaptioncharacters{دِينٌ }\textstyleDropCaps{религия, вера‎}

\textstyleCaptioncharacters{يَوْمٌ }\textstyleDropCaps{день‎}

\textstyleCaptioncharacters{يَوْمُ الْعِيدِ }\textstyleDropCaps{праздничный день‎}

\textstyleCaptioncharacters{خَشَبِىٌّ }\textstyleDropCaps{деревянный‎}

\textstyleCaptioncharacters{جِعَادٌ }\textstyleDropCaps{джихад, борьба за веру‎}

\textstyleCaptioncharacters{اَحَبَّ }\textstyleDropCaps{любил, хотел‎}

\textstyleCaptioncharacters{طَالِبٌ }\textstyleDropCaps{студент‎}

\textstyleCaptioncharacters{طَالِبَةٌ }\textstyleDropCaps{студентка‎}

\textstyleCaptioncharacters{اِنْتِهَاءٌ }\textstyleDropCaps{окончание, конец‎}

\textstyleCaptioncharacters{اِخْوَانٌ }\textstyleDropCaps{братья‎}

\textstyleCaptioncharacters{صَلاَةٌ }\textstyleDropCaps{намаз, молитва‎}

\textstyleCaptioncharacters{عِيدٌ }\textstyleDropCaps{праздник‎}

\textstyleCaptioncharacters{عِيدُكَ مُبَارَكٌ!ِ }\textstyleDropCaps{с празд­ником!‎}

\textstyleCaptioncharacters{وَ عِيدُكَ اَنْتَ!ِ }\textstyleDropCaps{и вас тоже с праздником!‎}

\textstyleCaptioncharacters{دَارَ }\textstyleDropCaps{обошл, походил (во­круг)‎}

\textstyleCaptioncharacters{تَهْنِئَةٌ }\textstyleDropCaps{поздравление‎}

\textstyleCaptioncharacters{صَلاَةُ الْعِيدِ }\textstyleDropCaps{празднич­ный намаз‎}

\textstyleCaptioncharacters{كَثِيرًا }\textstyleDropCaps{много‎}

\textstyleCaptioncharacters{اَللَّهُ يَرْعَاكَ }\textstyleDropCaps{да сохранит тебя Бог‎}

\subsection[Урок 36‎]{\textstyleDropCaps{Урок 36‎}}
\textstyleCaptioncharacters{سَاعَةٌ }\textstyleDropCaps{часы‎}

\textstyleCaptioncharacters{طَبِيبٌ }\textstyleDropCaps{врач, доктор‎}

\textstyleCaptioncharacters{مَرِيضٌ }\textstyleDropCaps{больной‎}

\textstyleCaptioncharacters{سَطْلٌ }\textstyleDropCaps{ведро‎}

\textstyleCaptioncharacters{حِكَايَةٌ }\textstyleDropCaps{рассказ, история‎}

\textstyleCaptioncharacters{عَمٌّ }\textstyleDropCaps{дядя (по отцу)‎}

\textstyleCaptioncharacters{هَدِيَّةٌ }\textstyleDropCaps{подарок‎}

\textstyleCaptioncharacters{سَنَةٌ }\textstyleDropCaps{год‎}

\textstyleCaptioncharacters{وَقْتٌ }\textstyleDropCaps{время‎}

\textstyleCaptioncharacters{ضَعِيفٌ }\textstyleDropCaps{слабый, бессиль­ный‎}

\textstyleCaptioncharacters{صَبَاحِيٌّ }\textstyleDropCaps{утренний‎}

\textstyleCaptioncharacters{مَاضٍ }\textstyleDropCaps{прошлый‎}

\textstyleCaptioncharacters{وَ لِذَلِكَ }\textstyleDropCaps{и поэтому‎}

\textstyleCaptioncharacters{اَلسَّنَةُ الْمَاضِيَةُ }\textstyleDropCaps{прош­лый год‎}

\textstyleCaptioncharacters{نَهَارُكَ سَعِيدٌ!ِ }\textstyleDropCaps{добрый день!‎}

\textstyleCaptioncharacters{اَهْلاً وَ سَهْلاً }\textstyleDropCaps{добро по­жаловать‎}

\textstyleCaptioncharacters{دَخَلَ الْجَامِعَةَ }\textstyleDropCaps{посту­пил в университет‎}

\textstyleCaptioncharacters{اِنَّ }\textstyleDropCaps{поистине, же‎}

\textstyleCaptioncharacters{اِنَّهُ }\textstyleDropCaps{поистине он, он же‎}

\textstyleCaptioncharacters{اَمَا قَرَأَ؟ }\textstyleDropCaps{он не читал?‎}

\textstyleCaptioncharacters{اَمَا قُلْتُ لَكَ؟ }\textstyleDropCaps{я тебе не говорил? я же тебе говорил?‎}

\textstyleCaptioncharacters{زَارَ }\textstyleDropCaps{посетил‎}

\subsection[Урок 37‎]{\textstyleDropCaps{Урок 37‎}}
\textstyleCaptioncharacters{الْكُرَةُ الطَّائِرَةُ }\textstyleDropCaps{волейбол‎}

\textstyleCaptioncharacters{مِقَصٌّ }\textstyleDropCaps{ножницы‎}

\textstyleCaptioncharacters{زَهْرِيَّةٌ }\textstyleDropCaps{ваза (цветочная)‎}

\textstyleCaptioncharacters{رِيَاضَةٌ }\textstyleDropCaps{спорт‎}

\textstyleCaptioncharacters{رِيَاضِىٌّ }\textstyleDropCaps{спортсмен‎}

\textstyleCaptioncharacters{فُرْصَةٌ }\textstyleDropCaps{перемена‎}

\textstyleCaptioncharacters{طَيَّارَةٌ }\textstyleDropCaps{самолёт‎}

\textstyleCaptioncharacters{اَرْضِيَّةٌ }\textstyleDropCaps{пол‎}

\textstyleCaptioncharacters{حَائِطٌ }\textstyleDropCaps{стена‎}

\textstyleCaptioncharacters{صَيَّادُ السَّمَكِ }\textstyleDropCaps{рыболов‎}

\textstyleCaptioncharacters{شِصٌّ }\textstyleDropCaps{удочка‎}

\textstyleCaptioncharacters{بِسَاطٌ }\textstyleDropCaps{ковёр‎}

\textstyleCaptioncharacters{خَرِيطَةٌ }\textstyleDropCaps{карта‎}

\textstyleCaptioncharacters{زَاوِيَةٌ }\textstyleDropCaps{угол‎}

\textstyleCaptioncharacters{جُغْرَافِىٌّ }\textstyleDropCaps{географический‎}

\textstyleCaptioncharacters{خَرِيطَةٌ جُغْرَافِيَّةٌ \ }\textstyleDropCaps{гео­графическая карта‎}

\textstyleCaptioncharacters{عَرِيضٌ }\textstyleDropCaps{широкий‎}

\textstyleCaptioncharacters{مُدَوَّرٌ }\textstyleDropCaps{круглый‎}

\textstyleCaptioncharacters{فِى الْغَدِ }\textstyleDropCaps{на следующий день‎}

\textstyleCaptioncharacters{قَصَّ }\textstyleDropCaps{стриг, срезал‎}

\textstyleCaptioncharacters{رَأَى }\textstyleDropCaps{видел‎}

\textstyleCaptioncharacters{عَمِلَ }\textstyleDropCaps{делал‎}

\subsection[Урок 38‎]{\textstyleDropCaps{Урок 38‎}}
\textstyleCaptioncharacters{صُنْدُوقٌ }\textstyleDropCaps{ящик, сундук‎}

\textstyleCaptioncharacters{قُفْلٌ }\textstyleDropCaps{замок‎}

\textstyleCaptioncharacters{مِفْتَاحٌ }\textstyleDropCaps{ключ‎}

\textstyleCaptioncharacters{حِنْطَةٌ }\textstyleDropCaps{пшеница‎}

\textstyleCaptioncharacters{طَمَاطِمُ }\textstyleDropCaps{помидоры‎}

\textstyleCaptioncharacters{بَطَاطِسُ }\textstyleDropCaps{картошка‎}

\textstyleCaptioncharacters{رَبٌّ }\textstyleDropCaps{господь; хозяин‎}

\textstyleCaptioncharacters{اِلَهٌ }\textstyleDropCaps{Бог‎}

\textstyleCaptioncharacters{نَبِىٌّ }\textstyleDropCaps{Пророк‎}

\textstyleCaptioncharacters{مُؤْمِنٌ }\textstyleDropCaps{верующий, право­верный‎}

\textstyleCaptioncharacters{لُعْبَةٌ }\textstyleDropCaps{игрушка‎}

\textstyleCaptioncharacters{نُقُودٌ }\textstyleDropCaps{деньги‎}

\textstyleCaptioncharacters{خُضْرَوَاتٌ }\textstyleDropCaps{овощи‎}

\textstyleCaptioncharacters{وَ غَيْرُ هَا مِنَ الْخُضْرَوَاتِ }\textstyleDropCaps{и другие овощи‎}

\textstyleCaptioncharacters{مَكَّةُ }\textstyleDropCaps{Мекка‎}

\textstyleCaptioncharacters{اَلْمَدِينَةٌ }\textstyleDropCaps{Медина‎}

\textstyleCaptioncharacters{اِسْمٌ }\textstyleDropCaps{имя, название‎}

\textstyleCaptioncharacters{مَا اسْمُكَ؟ }\textstyleDropCaps{как тебя зо­вут?‎}

\textstyleCaptioncharacters{وُلِدَ }\textstyleDropCaps{родился (его родили)‎}

\textstyleCaptioncharacters{دُفِنَ }\textstyleDropCaps{его похоронили‎}

\textstyleCaptioncharacters{فَرِحَ }\textstyleDropCaps{обрадовался‎}

\textstyleCaptioncharacters{مَاتَ }\textstyleDropCaps{умер‎}

\subsection[Урок 39‎]{\textstyleDropCaps{Урок 39‎}}
\textstyleCaptioncharacters{عُلْبَةٌ }\textstyleDropCaps{коробка, пачка‎}

\textstyleCaptioncharacters{حَبْلٌ }\textstyleDropCaps{верёвка, канат‎}

\textstyleCaptioncharacters{كِيسٌ }\textstyleDropCaps{мешок‎}

\textstyleCaptioncharacters{قَمْحٌ }\textstyleDropCaps{пшеница‎}

\textstyleCaptioncharacters{شَعِيرٌ }\textstyleDropCaps{ячмень‎}

\textstyleCaptioncharacters{صُورَةٌ }\textstyleDropCaps{снимок, рисунок, картина‎}

\textstyleCaptioncharacters{حَوْضٌ }\textstyleDropCaps{бассейн‎}

\textstyleCaptioncharacters{حَوْضُ السِّبَاحَةِ }\textstyleDropCaps{плава­тельный бассейн‎}

\textstyleCaptioncharacters{حَشِيشٌ }\textstyleDropCaps{сено‎}

\textstyleCaptioncharacters{صِبْغٌ }\textstyleDropCaps{краска‎}

\textstyleCaptioncharacters{عُلْبَةٌ اَصْبَاغٍ }\textstyleDropCaps{пачка кра­сок‎}

\textstyleCaptioncharacters{حَيْوَانٌ }\textstyleDropCaps{животное‎}

\textstyleCaptioncharacters{حِصَانٌ }\textstyleDropCaps{конь, жеребец‎}

\textstyleCaptioncharacters{طَعَامٌ }\textstyleDropCaps{пища, еда‎}

\textstyleCaptioncharacters{لَوْنٌ }\textstyleDropCaps{цвет‎}

\textstyleCaptioncharacters{فُطُورٌ }\textstyleDropCaps{завтрак‎}

\textstyleCaptioncharacters{صَمْغٌ }\textstyleDropCaps{клей‎}

\textstyleCaptioncharacters{صِغَرٌ }\textstyleDropCaps{детство, малолетство‎}

\textstyleCaptioncharacters{فِى الصِّغَرِ }\textstyleDropCaps{в детстве‎}

\textstyleCaptioncharacters{سِبَاحَةٌ }\textstyleDropCaps{плавание‎}

\textstyleCaptioncharacters{وَاسِعٌ }\textstyleDropCaps{просторный, широ­кий‎}

\textstyleCaptioncharacters{عَمِيقٌ }\textstyleDropCaps{глубокий‎}

\textstyleCaptioncharacters{وَقْتَ الْفَرَاغِ }\textstyleDropCaps{на досуге, в свободное время‎}

\textstyleCaptioncharacters{قَدَّمَ }\textstyleDropCaps{подал, дал‎}

\textstyleCaptioncharacters{اَلْصَقَ }\textstyleDropCaps{клеил‎}

\textstyleCaptioncharacters{تَعَلَّمَ }\textstyleDropCaps{учился, научился‎}

\textstyleCaptioncharacters{تَعَلَّمْ }\textstyleDropCaps{учись, научись‎}

\textstyleCaptioncharacters{حَمَلَ }\textstyleDropCaps{носил, тащил, возил‎}

\textstyleCaptioncharacters{رَبَطَ }\textstyleDropCaps{привязал, связал‎}

\textstyleCaptioncharacters{صَبَغَ }\textstyleDropCaps{красил‎}

\subsection[Урок 40‎]{\textstyleDropCaps{Урок 40‎}}
\textstyleCaptioncharacters{اَسَدٌ }\textstyleDropCaps{лев‎}

\textstyleCaptioncharacters{نَمِرٌ }\textstyleDropCaps{тигр‎}

\textstyleCaptioncharacters{قِرْدٌ }\textstyleDropCaps{обезьяна‎}

\textstyleCaptioncharacters{دُبٌّ }\textstyleDropCaps{медведь‎}

\textstyleCaptioncharacters{ذِئْبٌ }\textstyleDropCaps{волк‎}

\textstyleCaptioncharacters{غَزَالٌ }\textstyleDropCaps{газель‎}

\textstyleCaptioncharacters{ذُبَّابَةٌ }\textstyleDropCaps{муха‎}

\textstyleCaptioncharacters{عَنْكَبُوتٌ }\textstyleDropCaps{паук‎}

\textstyleCaptioncharacters{نَسِيجُ الْعَنْكَبُوتِ }\textstyleDropCaps{пау­тина‎}

\textstyleCaptioncharacters{شَبَكَةٌ }\textstyleDropCaps{сетка‎}

\textstyleCaptioncharacters{رَوْضَةٌ }\textstyleDropCaps{садик‎}

\textstyleCaptioncharacters{خَيْطٌ }\textstyleDropCaps{нитка‎}

\textstyleCaptioncharacters{نَسَّاجٌ }\textstyleDropCaps{ткач‎}

\textstyleCaptioncharacters{حَشَرَةٌ }\textstyleDropCaps{насекомое‎}

\textstyleCaptioncharacters{وَحْشِىٌّ }\textstyleDropCaps{дикий‎}

\textstyleCaptioncharacters{دَقِيقٌ }\textstyleDropCaps{тонкий‎}

\textstyleCaptioncharacters{لَطِيفٌ }\textstyleDropCaps{интересный, забав­ный‎}

\textstyleCaptioncharacters{حِكَايَةٌ لَطِيفَةٌ }\textstyleDropCaps{интерес­ный рассказ‎}

\textstyleCaptioncharacters{طِفْلٌ }\textstyleDropCaps{ребёнок, дитя‎}

\textstyleCaptioncharacters{رَوْضَةُ الأَطْفَالِ \ }\textstyleDropCaps{дет­ский сад‎}

\textstyleCaptioncharacters{حَدِيقَةُ الْحَيْوَانَاتِ }\textstyleDropCaps{зоо­парк‎}

\textstyleCaptioncharacters{بَيْتُ الْعَنْكَبُوتِ }\textstyleDropCaps{дом пау­ка, паутина‎}

\textstyleCaptioncharacters{نَسَجَ }\textstyleDropCaps{ткал‎}

\textstyleCaptioncharacters{وَقَعَ }\textstyleDropCaps{попал‎}

\textstyleCaptioncharacters{عَيْنٌ }\textstyleDropCaps{глаз‎}

\textstyleCaptioncharacters{ضَحِكَ }\textstyleDropCaps{смеялся‎}

\textstyleCaptioncharacters{وَثَبَ }\textstyleDropCaps{бросился, набросил­ся‎}

\subsection[Урок 41‎]{\textstyleDropCaps{Урок 41‎}}
\textstyleCaptioncharacters{جِهَازُ التِّلِفُونِ }\textstyleDropCaps{телефон, телефонный аппарат‎}

\textstyleCaptioncharacters{تِلِفِزْيُونٌ }\textstyleDropCaps{телевизор‎}

\textstyleCaptioncharacters{جِهَازُ الرَّادِيُو \ }\textstyleDropCaps{ра­диоприёмник‎}

\textstyleCaptioncharacters{مَطْبَخٌ }\textstyleDropCaps{кухня‎}

\textstyleCaptioncharacters{مَصْعَدٌ }\textstyleDropCaps{лифт‎}

\textstyleCaptioncharacters{اَثَاثٌ }\textstyleDropCaps{мебель‎}

\textstyleCaptioncharacters{وَسَطٌ }\textstyleDropCaps{середина, центр‎}

\textstyleCaptioncharacters{فِى وَسَطِ الْمَدِينَةِ }\textstyleDropCaps{в центре города‎}

\textstyleCaptioncharacters{زَاوِيَةٌ }\textstyleDropCaps{угол‎}

\textstyleCaptioncharacters{نَظِيفٌ }\textstyleDropCaps{чистый‎}

\textstyleCaptioncharacters{مُرَبَّعٌ }\textstyleDropCaps{четырёхугольный‎}

\textstyleCaptioncharacters{لِمَ؟ }\textstyleDropCaps{почему? зачем?‎}

\textstyleCaptioncharacters{لِأَنَّهُ }\textstyleDropCaps{потому что он‎}

\textstyleCaptioncharacters{تَقْرِيبًا }\textstyleDropCaps{почти, примерно, приблизительно‎}

\textstyleCaptioncharacters{لَكِنَّ...ِ }\textstyleDropCaps{однако, но‎}

\textstyleCaptioncharacters{لَكِنَّهُ }\textstyleDropCaps{но он‎}

\textstyleCaptioncharacters{هَكَذَا }\textstyleDropCaps{так‎}

\textstyleCaptioncharacters{لَمْ يَقْرَأْ }\textstyleDropCaps{он не читал‎}

\textstyleCaptioncharacters{اَتَدْرِى؟ }\textstyleDropCaps{ты знаешь?‎}

\textstyleCaptioncharacters{لَوْ كُنْتُ مَكَانَكَ }\textstyleDropCaps{если бы я был на твоём месте‎}

\textstyleCaptioncharacters{بَكَى }\textstyleDropCaps{плакал‎}

\textstyleCaptioncharacters{ضَرَبَ }\textstyleDropCaps{бил, ударил‎}

\subsection[Урок 42‎]{\textstyleDropCaps{Урок 42‎}}
\textstyleCaptioncharacters{جَبَلٌ }\textstyleDropCaps{гора‎}

\textstyleCaptioncharacters{كُوخٌ }\textstyleDropCaps{шалаш, хибара, хи­жина‎}

\textstyleCaptioncharacters{مَوْقِدُ الْغَازِ }\textstyleDropCaps{газовая плита‎}

\textstyleCaptioncharacters{فَرْخٌ }\textstyleDropCaps{птенец, цыплёнок‎}

\textstyleCaptioncharacters{بَيْضَةٌ }\textstyleDropCaps{яйцо‎}

\textstyleCaptioncharacters{جَوْزٌ }\textstyleDropCaps{орехи‎}

\textstyleCaptioncharacters{مَصْيَفٌ }\textstyleDropCaps{курорт, дача‎}

\textstyleCaptioncharacters{صَدِيقٌ }\textstyleDropCaps{друг‎}

\textstyleCaptioncharacters{غَازٌ }\textstyleDropCaps{газ‎}

\textstyleCaptioncharacters{سُؤَالٌ }\textstyleDropCaps{вопрос‎}

\textstyleCaptioncharacters{مَسْئَلَةٌ }\textstyleDropCaps{задача‎}

\textstyleCaptioncharacters{مَامَا }\textstyleDropCaps{мама‎}

\textstyleCaptioncharacters{الأَوَّلُ }\textstyleDropCaps{первый‎}

\textstyleCaptioncharacters{اَلثَّانِى }\textstyleDropCaps{второй‎}

\textstyleCaptioncharacters{اَاثَّالِثُ }\textstyleDropCaps{третий‎}

\textstyleCaptioncharacters{عِدَّةٌ }\textstyleDropCaps{несколько‎}

\textstyleCaptioncharacters{صَفٌّ }\textstyleDropCaps{курс; класс‎}

\textstyleCaptioncharacters{عَالٍ }\textstyleDropCaps{высокий‎}

\textstyleCaptioncharacters{مَسْرُورٌ }\textstyleDropCaps{радостный, обра­дованный‎}

\textstyleCaptioncharacters{مَوْقِدٌ }\textstyleDropCaps{печь, плита‎}

\textstyleCaptioncharacters{سَهْلٌ }\textstyleDropCaps{лёгкий‎}

\textstyleCaptioncharacters{صَعْبٌ }\textstyleDropCaps{трудный‎}

\textstyleCaptioncharacters{لَمَّا }\textstyleDropCaps{когда‎}

\textstyleCaptioncharacters{لَمَّا خَرَجَ }\textstyleDropCaps{когда вышел‎}

\textstyleCaptioncharacters{حَقًّا }\textstyleDropCaps{действительно, на самом деле, поистине‎}

\textstyleCaptioncharacters{مِنْهُمْ }\textstyleDropCaps{из них‎}

\textstyleCaptioncharacters{اَهْلٌ }\textstyleDropCaps{семья; родня‎}

\textstyleCaptioncharacters{صَلَّى }\textstyleDropCaps{совершил намаз, мо­лился‎}

\textstyleCaptioncharacters{طَبَخَ }\textstyleDropCaps{варил; готовил‎}

\textstyleCaptioncharacters{وَجَدَ }\textstyleDropCaps{нашёл; застал‎}

\textstyleCaptioncharacters{ضَرَبَ }\textstyleDropCaps{он бил‎}

\textstyleCaptioncharacters{ضُرِبَ }\textstyleDropCaps{его били‎}

\textstyleCaptioncharacters{أُخِذَ }\textstyleDropCaps{его взяли‎}

\textstyleCaptioncharacters{كُتِبَ }\textstyleDropCaps{его написали, он был написан‎}

\subsection[Урок 43‎]{\textstyleDropCaps{Урок 43‎}}
\textstyleCaptioncharacters{قِطَارٌ }\textstyleDropCaps{поезд‎}

\textstyleCaptioncharacters{حَقِيبَةٌ }\textstyleDropCaps{чемодан‎}

\textstyleCaptioncharacters{دُمْيَةٌ }\textstyleDropCaps{кукла‎}

\textstyleCaptioncharacters{تُفَّاحٌ }\textstyleDropCaps{яблоко‎}

\textstyleCaptioncharacters{كُمَّثْرَى }\textstyleDropCaps{груша‎}

\textstyleCaptioncharacters{خَوْخٌ }\textstyleDropCaps{персики‎}

\textstyleCaptioncharacters{عِنَبٌ }\textstyleDropCaps{виноград‎}

\textstyleCaptioncharacters{نَخْلٌ }\textstyleDropCaps{пальмы‎}

\textstyleCaptioncharacters{خَيْمَةٌ }\textstyleDropCaps{палатка, шатёр‎}

\textstyleCaptioncharacters{سَفَرٌ }\textstyleDropCaps{путешествие, поезд­ка‎}

\textstyleCaptioncharacters{ثَوْبٌ }\textstyleDropCaps{одежда; форма‎}

\textstyleCaptioncharacters{ثِيَابُ الْمَدْرَسَةِ }\textstyleDropCaps{школь­ная форма‎}

\textstyleCaptioncharacters{مِشْمِسٌ }\textstyleDropCaps{абрикосы‎}

\textstyleCaptioncharacters{كُرَّاسَةٌ }\textstyleDropCaps{тетрадь‎}

\textstyleCaptioncharacters{صَاحِبٌ }\textstyleDropCaps{друг; товарищ‎}

\textstyleCaptioncharacters{بَعْدَ سَنَةٍ }\textstyleDropCaps{через год‎}

\textstyleCaptioncharacters{تَنَاوَلَ }\textstyleDropCaps{принимал; брал‎}

\textstyleCaptioncharacters{تَنَاوَلَ الطَّعَامَ }\textstyleDropCaps{принял пищу, кушал‎}

\textstyleCaptioncharacters{بَابَا }\textstyleDropCaps{папа‎}

\textstyleCaptioncharacters{حَوْلَ...ِ }\textstyleDropCaps{вокруг‎}

\textstyleCaptioncharacters{مَايَزَالُ }\textstyleDropCaps{он ещё, он всё ещё‎}

\textstyleCaptioncharacters{مَايَزَالُ صَغِيرًا }\textstyleDropCaps{он ещё маленький‎}

\textstyleCaptioncharacters{قَابَلَ }\textstyleDropCaps{встретил‎}

\textstyleCaptioncharacters{سَلَّمَ عَلَى }\textstyleDropCaps{приветствовал‎}

\textstyleCaptioncharacters{بَيْنَ }\textstyleDropCaps{среди‎}

\textstyleCaptioncharacters{بَيْنَنَا }\textstyleDropCaps{среди нас‎}

\textstyleCaptioncharacters{نَصَبَ }\textstyleDropCaps{установил, ставил‎}

\textstyleCaptioncharacters{نَصَبَ الْخَيْمَةَ }\textstyleDropCaps{разбил по­латку‎}

\textstyleCaptioncharacters{حَانَ }\textstyleDropCaps{настало (время)‎}

\textstyleCaptioncharacters{حَانَ وَقْتُ الصَّلاَةِ }\textstyleDropCaps{на­стало время намаза‎}

\textstyleCaptioncharacters{كَبُرَ }\textstyleDropCaps{вырос, подрос‎}

\subsection[Урок 44‎]{\textstyleDropCaps{Урок 44‎}}
\textstyleCaptioncharacters{جُنْدِىٌّ }\textstyleDropCaps{солдат‎}

\textstyleCaptioncharacters{جَرَسٌ }\textstyleDropCaps{звонок, колокол‎}

\textstyleCaptioncharacters{جُنْدٌ }\textstyleDropCaps{войско‎}

\textstyleCaptioncharacters{وَطَنٌ }\textstyleDropCaps{родина‎}

\textstyleCaptioncharacters{شَعْبٌ }\textstyleDropCaps{народ‎}

\textstyleCaptioncharacters{جَيْشٌ }\textstyleDropCaps{армия‎}

\textstyleCaptioncharacters{بَطَلٌ }\textstyleDropCaps{герой‎}

\textstyleCaptioncharacters{الْجَيْشُ الْبَطَلِ }\textstyleDropCaps{герои­ческая армия‎}

\textstyleCaptioncharacters{صَفٌّ }\textstyleDropCaps{ряд‎}

\textstyleCaptioncharacters{حَوْشٌ }\textstyleDropCaps{двор‎}

\textstyleCaptioncharacters{اِسْلاَمِىٌّ }\textstyleDropCaps{исламский‎}

\textstyleCaptioncharacters{رَعَاكَ اللَّهُ }\textstyleDropCaps{да сохранит тебя Бог‎}

\textstyleCaptioncharacters{دَافَعَ عَنْ...ِ }\textstyleDropCaps{защищал‎}

\textstyleCaptioncharacters{دَافَعَ عَنْ دِينِهِ }\textstyleDropCaps{защи­тил свою религию‎}

\textstyleCaptioncharacters{وَقَفَ }\textstyleDropCaps{становился‎}

\textstyleCaptioncharacters{وَقَفَ فِى صَفٍّ }\textstyleDropCaps{стано­вился в ряд‎}

\textstyleCaptioncharacters{دَقَّ }\textstyleDropCaps{стучал, колотил‎}

\textstyleCaptioncharacters{دَقَّ الْجَرَسُ }\textstyleDropCaps{прозвенел звонок‎}

\textstyleCaptioncharacters{يَحْيَى!ِ }\textstyleDropCaps{Да здравствует...!‎}

\textstyleCaptioncharacters{يَحْيَى الإِسْلاَمُ!ِ }\textstyleDropCaps{Да здравствует Ислам!‎}

\textstyleCaptioncharacters{دَوْلَةٌ }\textstyleDropCaps{государство‎}

\textstyleCaptioncharacters{تَحْيَى دَوْلَةُ الإِسْلاَمِ }\textstyleDropCaps{Да здравствует Исламское государство!‎}

\textstyleCaptioncharacters{عَلَيْنَا اَنْ نَذْهَبَ }\textstyleDropCaps{мы должны идти, нам надо идти‎}

\textstyleCaptioncharacters{عَلَيْنَا اَنْ...ِ }\textstyleDropCaps{мы должны, нам надо‎}

\subsection[Урок 45‎]{\textstyleDropCaps{Урок 45‎}}
\textstyleCaptioncharacters{دَبَّابَةٌ }\textstyleDropCaps{танк‎}

\textstyleCaptioncharacters{طَائِرَةٌ }\textstyleDropCaps{самолет‎}

\textstyleCaptioncharacters{سَلاَّةٌ }\textstyleDropCaps{корзина‎}

\textstyleCaptioncharacters{مُهْمَلاَتٌ }\textstyleDropCaps{отбросы, не­нужное‎}

\textstyleCaptioncharacters{سَلَّةُ الْمُهْمَلاَتِ }\textstyleDropCaps{корзи­на для бумаг, отбросов‎}

\textstyleCaptioncharacters{مَصْنَعٌ }\textstyleDropCaps{завод‎}

\textstyleCaptioncharacters{وَرَقَةٌ }\textstyleDropCaps{бумага‎}

\textstyleCaptioncharacters{مَصْنَعُ السَّيَّارَاتِ }\textstyleDropCaps{авто­мобильный завод‎}

\textstyleCaptioncharacters{مَصْنَعُ الدَّبَّابَاتِ }\textstyleDropCaps{тан­ковый завод‎}

\textstyleCaptioncharacters{مَصْنَعُ الطَّائِرَاتِ }\textstyleDropCaps{авиа­ционный завод‎}

\textstyleCaptioncharacters{يَنْبَغِى }\textstyleDropCaps{следует, надо‎}

\textstyleCaptioncharacters{يَنْبَغِى اَنْ تَعْلَمَ }\textstyleDropCaps{тебе следует знать‎}

\textstyleCaptioncharacters{اَرْضٌ }\textstyleDropCaps{земля; пол‎}

\textstyleCaptioncharacters{عَلَى الأَرْضِ }\textstyleDropCaps{на полу‎}

\textstyleCaptioncharacters{رِحْلَةٌ }\textstyleDropCaps{поездка; путеше­ствие; экскурсия‎}

\textstyleCaptioncharacters{مَشَى }\textstyleDropCaps{ходил, шел‎}

\textstyleCaptioncharacters{رَمَى }\textstyleDropCaps{бросил‎}

\textstyleCaptioncharacters{وَضَعَ }\textstyleDropCaps{положил, поставил‎}

\textstyleCaptioncharacters{جَرَى }\textstyleDropCaps{побежал‎}

\textstyleCaptioncharacters{قَامَ بِ }\textstyleDropCaps{совершил, выпол­нил, делал‎}

\textstyleCaptioncharacters{قَامَ بِرِحْلَةٍ }\textstyleDropCaps{совершил поездку‎}

\textstyleCaptioncharacters{سَلاَمٌ }\textstyleDropCaps{приветствие‎}

\textstyleCaptioncharacters{رَدَّ السَّلاَمَ عَلَى...ِ }\textstyleDropCaps{от­ветил на приветствие‎}

\textstyleCaptioncharacters{خَلْفَ...ِ }\textstyleDropCaps{за...‎}

\textstyleCaptioncharacters{جَرَى خَلْفَهُ }\textstyleDropCaps{побежал за ним‎}

\subsection[Урок 46‎]{\textstyleDropCaps{Урок 46‎}}
\textstyleCaptioncharacters{سَبُّورَةٌ }\textstyleDropCaps{классная доска‎}

\textstyleCaptioncharacters{قَلَمُ رَصَاصٍ }\textstyleDropCaps{простой ка­рандаш‎}

\textstyleCaptioncharacters{قَلَمُ حِبْرٍ }\textstyleDropCaps{ручка, авторуч­ка‎}

\textstyleCaptioncharacters{فَكَّرَ }\textstyleDropCaps{подумал‎}

\textstyleCaptioncharacters{فَكَّرَ فِيمَا يَفْعَلُ }\textstyleDropCaps{поду­мал, что делать‎}

\textstyleCaptioncharacters{آتٍ }\textstyleDropCaps{следующий‎}

\textstyleCaptioncharacters{عِبَارَةٌ }\textstyleDropCaps{выражение, фраза‎}

\textstyleCaptioncharacters{اُسْبُوعٌ }\textstyleDropCaps{неделя‎}

\textstyleCaptioncharacters{خَرَخَ فِى رِحْلَةٍ }\textstyleDropCaps{вышел на экскурсию‎}

\textstyleCaptioncharacters{وَهَكَذَا }\textstyleDropCaps{итак, таким об­разом‎}

\textstyleCaptioncharacters{فَاِذَا }\textstyleDropCaps{и вдруг‎}

\textstyleCaptioncharacters{جَنْبَ...ِ }\textstyleDropCaps{рядом с...‎}

\textstyleCaptioncharacters{قَادِمٌ }\textstyleDropCaps{будущий, наступаю­щий‎}

\textstyleCaptioncharacters{غَابَةٌ }\textstyleDropCaps{лес‎}

\textstyleCaptioncharacters{سَأَلَ عَنْ...ِ }\textstyleDropCaps{спросил о чем-либо...‎}

\textstyleCaptioncharacters{اِمَّا... اَوْ...ِ }\textstyleDropCaps{или... или‎}

\textstyleCaptioncharacters{اِذًا }\textstyleDropCaps{тогда, если так, в таком случае‎}

\textstyleCaptioncharacters{يَكُونُ }\textstyleDropCaps{бывает‎}

\textstyleCaptioncharacters{سَيَكُونُ }\textstyleDropCaps{будет‎}

\textstyleCaptioncharacters{يَقْرَأُ }\textstyleDropCaps{читает‎}

\textstyleCaptioncharacters{سَيَقْرَأُ }\textstyleDropCaps{будет читать,про­читает‎}

\textstyleCaptioncharacters{كَمَا قَالَ }\textstyleDropCaps{как сказал‎}

\textstyleCaptioncharacters{مَا لَوْنُهُ؟ }\textstyleDropCaps{какого он цвета?‎}

\textstyleCaptioncharacters{اَبْيَضُ }\textstyleDropCaps{белый‎}

\textstyleCaptioncharacters{اَسْوَدُ }\textstyleDropCaps{чёрный‎}

\textstyleCaptioncharacters{اَحْمَرُ }\textstyleDropCaps{красный‎}

\textstyleCaptioncharacters{شَكَرَ }\textstyleDropCaps{благодарил‎}

\subsection[Урок 47‎]{\textstyleDropCaps{Урок 47‎}}
\textstyleCaptioncharacters{طَابِعٌ }\textstyleDropCaps{марка‎}

\textstyleCaptioncharacters{ظَرْفٌ }\textstyleDropCaps{конверт‎}

\textstyleCaptioncharacters{صُنْدُوقُ الْبَرِيدِ }\textstyleDropCaps{почто­вый ящик‎}

\textstyleCaptioncharacters{خِطَابٌ }\textstyleDropCaps{письмо‎}

\textstyleCaptioncharacters{طَرِيقٌ }\textstyleDropCaps{дорога‎}

\textstyleCaptioncharacters{حَجَرٌ }\textstyleDropCaps{камень‎}

\textstyleCaptioncharacters{بَرِيدٌ }\textstyleDropCaps{почта‎}

\textstyleCaptioncharacters{سَاعِى الْبَرِيدِ }\textstyleDropCaps{почтальон‎}

\textstyleCaptioncharacters{مَكْتَبُ الْبَرِيدِ }\textstyleDropCaps{почтовое отделение‎}

\textstyleCaptioncharacters{اِشْتَرَى }\textstyleDropCaps{купил‎}

\textstyleCaptioncharacters{هَذَا هُوَ }\textstyleDropCaps{вот это‎}

\textstyleCaptioncharacters{هَيَّا }\textstyleDropCaps{давай, пошли, скорее‎}

\textstyleCaptioncharacters{هَيَّا نَجْلِسْ }\textstyleDropCaps{давайте ся­дем‎}

\textstyleCaptioncharacters{لاَبُدَّ مِنْ...ِ }\textstyleDropCaps{необходимо, надо‎}

\textstyleCaptioncharacters{ثَقِيلٌ }\textstyleDropCaps{тяжелый‎}

\textstyleCaptioncharacters{وَ مَعَ ذَلِكِ }\textstyleDropCaps{и все же, не­смотря на то‎}

\textstyleCaptioncharacters{بَعِيدًا }\textstyleDropCaps{далеко‎}

\textstyleCaptioncharacters{غَيْرَ بَعِيدٍ }\textstyleDropCaps{недалеко‎}

\textstyleCaptioncharacters{كَيْفَ؟ }\textstyleDropCaps{как?‎}

\textstyleCaptioncharacters{تَرَكَ }\textstyleDropCaps{оставил‎}

\textstyleCaptioncharacters{دَعَا }\textstyleDropCaps{позвал‎}

\textstyleCaptioncharacters{سَدَّ }\textstyleDropCaps{закрыл, преградил‎}

\subsection[Урок 48‎]{\textstyleDropCaps{Урок 48‎}}
\textstyleCaptioncharacters{مِحْرَاثٌ }\textstyleDropCaps{плуг‎}

\textstyleCaptioncharacters{جَرَّارَةٌ }\textstyleDropCaps{трактор‎}

\textstyleCaptioncharacters{فِرْجَارٌ }\textstyleDropCaps{циркуль‎}

\textstyleCaptioncharacters{جَارٌ }\textstyleDropCaps{сосед‎}

\textstyleCaptioncharacters{عَمَلٌ }\textstyleDropCaps{работа, труд‎}

\textstyleCaptioncharacters{نَشَّافَةٌ }\textstyleDropCaps{промакашка‎}

\textstyleCaptioncharacters{نَظَافَةٌ }\textstyleDropCaps{чистота‎}

\textstyleCaptioncharacters{بِسُرْعَةٍ }\textstyleDropCaps{быстро‎}

\textstyleCaptioncharacters{غَيْرُ كَبِيرٍ }\textstyleDropCaps{небольшой‎}

\textstyleCaptioncharacters{غَيْرُ نَظِيفٍ }\textstyleDropCaps{нечистый‎}

\textstyleCaptioncharacters{فِكْرَةٌ }\textstyleDropCaps{мысль, идея‎}

\textstyleCaptioncharacters{كُلُّ وَاحِدٍ }\textstyleDropCaps{каждый, каж­дый человек‎}

\textstyleCaptioncharacters{مِسْطَرَةٌ }\textstyleDropCaps{линейка‎}

\textstyleCaptioncharacters{غَيْطٌ }\textstyleDropCaps{поле; сад‎}

\textstyleCaptioncharacters{وَالِدٌ }\textstyleDropCaps{отец‎}

\textstyleCaptioncharacters{مُهِمٌّ }\textstyleDropCaps{важный‎}

\textstyleCaptioncharacters{آخِرٌ }\textstyleDropCaps{конец‎}

\textstyleCaptioncharacters{وَصَلَ }\textstyleDropCaps{дошёл; прибыл, прие­хал‎}

\textstyleCaptioncharacters{مَرَّ }\textstyleDropCaps{прошел‎}

\textstyleCaptioncharacters{جَرَّ }\textstyleDropCaps{тащил, тянул‎}

\textstyleCaptioncharacters{حَرَثَ }\textstyleDropCaps{пахал‎}

\textstyleCaptioncharacters{حَدَثَ }\textstyleDropCaps{сслучился, произо­шел‎}

\textstyleCaptioncharacters{هَلْ حَدَثَ مِنْ شَئْءٍ؟ }\textstyleDropCaps{что-нибдуь случилось?‎}

\textstyleCaptioncharacters{مَا حَدَثَ شَئْءٌ }\textstyleDropCaps{ничего не случилось‎}

\textstyleCaptioncharacters{دَارَ اِلَى الْيَمِينَ }\textstyleDropCaps{повер­нул направо‎}

\textstyleCaptioncharacters{دَارَ اِلَى الْوَرَاءِ }\textstyleDropCaps{развер­нулся, повернулся назад‎}

\textstyleCaptioncharacters{كَعَادَتِهِ }\textstyleDropCaps{по своему обык­новению‎}

\textstyleCaptioncharacters{كَنَّاسٌ }\textstyleDropCaps{подметальщик, дворник‎}

\textstyleCaptioncharacters{نَظَّفَ }\textstyleDropCaps{чистил‎}

\subsection[Урок 49‎]{\textstyleDropCaps{Урок 49‎}}
\textstyleCaptioncharacters{عُصْفورٌ }\textstyleDropCaps{воробей‎}

\textstyleCaptioncharacters{قَفَصٌ }\textstyleDropCaps{клетка‎}

\textstyleCaptioncharacters{قَوْلٌ }\textstyleDropCaps{слова, речь, сказан­ное‎}

\textstyleCaptioncharacters{شَىْءٌ }\textstyleDropCaps{вещь, предмет‎}

\textstyleCaptioncharacters{هَوَاءٌ }\textstyleDropCaps{воздух‎}

\textstyleCaptioncharacters{حُرِّيَّةٌ }\textstyleDropCaps{свобода‎}

\textstyleCaptioncharacters{بُسْتَانِىٌّ }\textstyleDropCaps{садовник‎}

\textstyleCaptioncharacters{وَقَعَ الطَّيْرُ عَلَى غُصْنٍ }\textstyleDropCaps{птица села на ветку‎}

\textstyleCaptioncharacters{هُوَ لا يَفْهَمُ شَيْئًا }\textstyleDropCaps{он ничего не понимает‎}

\textstyleCaptioncharacters{أَمْسَكَ }\textstyleDropCaps{держал, хватал‎}

\textstyleCaptioncharacters{اَطْلَقَ }\textstyleDropCaps{отпустил, освобо­дил‎}

\textstyleCaptioncharacters{غَرَّدَ }\textstyleDropCaps{чирикал‎}

\textstyleCaptioncharacters{أَخْبَرَ }\textstyleDropCaps{рассказал‎}

\textstyleCaptioncharacters{أَجَابَ }\textstyleDropCaps{ответил‎}

\textstyleCaptioncharacters{وَحْشٌ }\textstyleDropCaps{зверь‎}

\textstyleCaptioncharacters{سَقَطَ }\textstyleDropCaps{упал‎}

\textstyleCaptioncharacters{صَاحَ }\textstyleDropCaps{закричал‎}

\textstyleCaptioncharacters{حَبَسَ }\textstyleDropCaps{заключал, сажал в тюрьму‎}

\textstyleCaptioncharacters{جَاءَ }\textstyleDropCaps{пришёл‎}

\textstyleCaptioncharacters{أَحَدٌ }\textstyleDropCaps{кто-нибудь; никто‎}

\textstyleCaptioncharacters{هَلْ جَاءَ اَحَدٌ }\textstyleDropCaps{кто-нибудь пришёл?‎}

\textstyleCaptioncharacters{لَمْ يَجِئْ اَحَدٌ }\textstyleDropCaps{никто не пришёл‎}

\textstyleCaptioncharacters{كُتُبِىٌّ }\textstyleDropCaps{продавец книг‎}

\textstyleCaptioncharacters{يَوْمًا }\textstyleDropCaps{однажды‎}

\subsection[Урок 50‎]{\textstyleDropCaps{Урок 50‎}}
\textstyleCaptioncharacters{رُمَّانٌ }\textstyleDropCaps{гранаты (плоды)‎}

\textstyleCaptioncharacters{حَمَامٌ }\textstyleDropCaps{голуби‎}

\textstyleCaptioncharacters{فُولٌ }\textstyleDropCaps{бобы‎}

\textstyleCaptioncharacters{أُسْرَةٌ }\textstyleDropCaps{семья‎}

\textstyleCaptioncharacters{خَالٌ }\textstyleDropCaps{дядя (по матери)‎}

\textstyleCaptioncharacters{مَرْعًى }\textstyleDropCaps{пастбище‎}

\textstyleCaptioncharacters{رِيفٌ }\textstyleDropCaps{деревня, село‎}

\textstyleCaptioncharacters{وِعَاءٌ }\textstyleDropCaps{сосуд‎}

\textstyleCaptioncharacters{زُبْدٌ }\textstyleDropCaps{масло‎}

\textstyleCaptioncharacters{شَحْمٌ }\textstyleDropCaps{сало, жир‎}

\textstyleCaptioncharacters{خُضَرٌ }\textstyleDropCaps{зелень‎}

\textstyleCaptioncharacters{بُرْتُقَالٌ }\textstyleDropCaps{апельсин‎}

\textstyleCaptioncharacters{مَغْلِىٌّ }\textstyleDropCaps{кипяченый‎}

\textstyleCaptioncharacters{مَاْكُولاَتٌ }\textstyleDropCaps{кушанья, съестное, продукты‎}

\textstyleCaptioncharacters{جَزَّارٌ }\textstyleDropCaps{мясник‎}

\textstyleCaptioncharacters{اَخْضَرُ }\textstyleDropCaps{зеленый‎}

\textstyleCaptioncharacters{يُبَاعُ }\textstyleDropCaps{продается‎}

\textstyleCaptioncharacters{حَصَلَ عَلَى...ِ }\textstyleDropCaps{получил‎}

\textstyleCaptioncharacters{حَلَبَ }\textstyleDropCaps{доил‎}

\textstyleCaptioncharacters{رَعَى }\textstyleDropCaps{пасся‎}

\textstyleCaptioncharacters{فُولٌ اَخْضَرُ }\textstyleDropCaps{зелёные бобы‎}

\textstyleCaptioncharacters{تَفَرَّجَ }\textstyleDropCaps{смотрел, осматри­вал, любовался‎}

\textstyleCaptioncharacters{فَرْدٌ }\textstyleDropCaps{один человек, член‎}

\textstyleCaptioncharacters{اَفْرَادُ اْلأُسْرَةِ }\textstyleDropCaps{члены се­мьи‎}

\subsection[Урок 51‎]{\textstyleDropCaps{Урок 51‎}}
\textstyleCaptioncharacters{يَوْمُ الْجُمُعَةِ }\textstyleDropCaps{пятница‎}

\textstyleCaptioncharacters{صَلاَةُ الْجُمُعَةِ }\textstyleDropCaps{джума-намаз, пятничный намаз‎}

\textstyleCaptioncharacters{صَلَّى الْجُمُعَةَ }\textstyleDropCaps{совер­шил джума-намаз‎}

\textstyleCaptioncharacters{جَمَاعَةٌ }\textstyleDropCaps{джамаат, группа‎}

\textstyleCaptioncharacters{صَلاَةُ الْجَمَاعَةِ }\textstyleDropCaps{джама­ат намаз, коллективный намаз‎}

\textstyleCaptioncharacters{صَلَّى مَعَ الْجَمَاعَةِ }\textstyleDropCaps{мо­лился с джамаатом, в группе‎}

\textstyleCaptioncharacters{صَلَّى بِالنَّاسِ }\textstyleDropCaps{совершил намаз перед людьми‎}

\textstyleCaptioncharacters{رَتَّبَ }\textstyleDropCaps{убрал, навел порядок‎}

\textstyleCaptioncharacters{اِمَامٌ }\textstyleDropCaps{имам‎}

\textstyleCaptioncharacters{فَرَّاشٌ }\textstyleDropCaps{уборщик‎}

\textstyleCaptioncharacters{مُؤَذِّنٌ }\textstyleDropCaps{муэдзин‎}

\textstyleCaptioncharacters{مِكْنَسَةٌ }\textstyleDropCaps{метла, веник‎}

\textstyleCaptioncharacters{شَفَّاطَةُ الْغُبَارِ }\textstyleDropCaps{пылесос‎}

\textstyleCaptioncharacters{تَمَّ }\textstyleDropCaps{закончился, завершился‎}

\textstyleCaptioncharacters{تَمَّتِ الصَّلاَةُ }\textstyleDropCaps{закончился намаз‎}

\textstyleCaptioncharacters{كَنَسَ }\textstyleDropCaps{подметал‎}

\textstyleCaptioncharacters{حَضَرَ }\textstyleDropCaps{присутствовал‎}

\textstyleCaptioncharacters{جَامِعٌ }\textstyleDropCaps{соборная мечеть‎}

\textstyleCaptioncharacters{حَضَرَ الصَّلاَةَ }\textstyleDropCaps{присут­ствовал на намазе‎}

\textstyleCaptioncharacters{حَضَرَ الدَّرْسَ }\textstyleDropCaps{присут­ствовал на уроке‎}

\textstyleCaptioncharacters{قَدْ }\textstyleDropCaps{уже‎}

\textstyleCaptioncharacters{قَدْ ذَهَبَ }\textstyleDropCaps{уже ушёл‎}

\textstyleCaptioncharacters{قَدْ جَاءَ }\textstyleDropCaps{уже пришёл‎}

\subsection[Урок 52‎]{\textstyleDropCaps{Урок 52‎}}
\textstyleCaptioncharacters{خَرُوفٌ }\textstyleDropCaps{барашек‎}

\textstyleCaptioncharacters{مَكِنَةٌ }\textstyleDropCaps{машина, станок‎}

\textstyleCaptioncharacters{قَنَاةٌ }\textstyleDropCaps{канал‎}

\textstyleCaptioncharacters{فَرَاشٌ }\textstyleDropCaps{бабочки‎}

\textstyleCaptioncharacters{شَكْلٌ }\textstyleDropCaps{форма, вид‎}

\textstyleCaptioncharacters{صُوفٌ }\textstyleDropCaps{шерсть‎}

\textstyleCaptioncharacters{صَوْتٌ }\textstyleDropCaps{голос; звук‎}

\textstyleCaptioncharacters{اَوَّلٌ }\textstyleDropCaps{начало‎}

\textstyleCaptioncharacters{اَلإِثْنَانِ }\textstyleDropCaps{оба, двое‎}

\textstyleCaptioncharacters{عَجِيبٌ }\textstyleDropCaps{удивительный‎}

\textstyleCaptioncharacters{سَمِينٌ }\textstyleDropCaps{жирный‎}

\textstyleCaptioncharacters{بِرْسِيمٌ }\textstyleDropCaps{люцерна‎}

\textstyleCaptioncharacters{طُولَ الْيَوْمِ }\textstyleDropCaps{целый день‎}

\textstyleCaptioncharacters{دَارَتِ الْمَكِنَةُ }\textstyleDropCaps{крутилась машина‎}

\textstyleCaptioncharacters{جَرَى الْمَاءُ }\textstyleDropCaps{текла вода‎}

\textstyleCaptioncharacters{رَوَى }\textstyleDropCaps{орошал, поливал‎}

\textstyleCaptioncharacters{مَكِنَةُ الرَّىِّ }\textstyleDropCaps{ороситель­ная машина‎}

\textstyleCaptioncharacters{سَارَ }\textstyleDropCaps{ходил, шёл‎}

\subsection[Урок 53‎]{\textstyleDropCaps{Урок 53‎}}
\textstyleCaptioncharacters{عُلْبَةٌ }\textstyleDropCaps{банка‎}

\textstyleCaptioncharacters{فَاكِهَةٌ }\textstyleDropCaps{фрукт‎}

\textstyleCaptioncharacters{مُرَبًّى }\textstyleDropCaps{варенье, джем‎}

\textstyleCaptioncharacters{مَحْفُوظَاتٌ }\textstyleDropCaps{консервы‎}

\textstyleCaptioncharacters{عُلَبُ مَحْفُوظَاتٍ }\textstyleDropCaps{кон­сервные банки‎}

\textstyleCaptioncharacters{خُشَافٌ }\textstyleDropCaps{компот‎}

\textstyleCaptioncharacters{عَدَدٌ كَبِيرٌ مِنْ...ِ }\textstyleDropCaps{много, большое число‎}

\textstyleCaptioncharacters{عَدَدٌ كَبِيرٌ مِنَ اْلأَشْجَارِ }\textstyleDropCaps{много деревьев‎}

\textstyleCaptioncharacters{حَفِظَ }\textstyleDropCaps{консервировал‎}

\textstyleCaptioncharacters{حَفِظَ السَّمَكَ }\textstyleDropCaps{консер­вировал рыбу‎}

\textstyleCaptioncharacters{خَدَمَ }\textstyleDropCaps{служил‎}

\textstyleCaptioncharacters{نَفَعَ }\textstyleDropCaps{дал, принес пользу‎}

\textstyleCaptioncharacters{فَتَحَ عُلْبَةَ الْمَحْفُوظَاتِ }\textstyleDropCaps{открыл консервную банку‎}

\textstyleCaptioncharacters{قِسْمٌ }\textstyleDropCaps{отдел, отсек‎}

\textstyleCaptioncharacters{شَهْرٌ }\textstyleDropCaps{месяц‎}

\textstyleCaptioncharacters{فَرَحٌ }\textstyleDropCaps{радость‎}

\textstyleCaptioncharacters{دَهْشَةٌ }\textstyleDropCaps{удивление, изум­ление‎}

\textstyleCaptioncharacters{كَانَ فِى فَرَحٍ }\textstyleDropCaps{был в ра­дости‎}

\subsection[Урок 54‎]{\textstyleDropCaps{Урок 54‎}}
\textstyleCaptioncharacters{ثَعْلَبٌ }\textstyleDropCaps{лиса‎}

\textstyleCaptioncharacters{حَظِيرَةٌ }\textstyleDropCaps{загон‎}

\textstyleCaptioncharacters{ذَاتَ مَرَّةٍ }\textstyleDropCaps{однажды‎}

\textstyleCaptioncharacters{سَعِيدٌ }\textstyleDropCaps{счастливый‎}

\textstyleCaptioncharacters{ذَيْلٌ }\textstyleDropCaps{хвост‎}

\textstyleCaptioncharacters{حِيلَةٌ }\textstyleDropCaps{хитрость, уловка‎}

\textstyleCaptioncharacters{كَلاَمٌ }\textstyleDropCaps{разговор, речь‎}

\textstyleCaptioncharacters{مِنْ فَضْلِكَ }\textstyleDropCaps{пожалуйста‎}

\textstyleCaptioncharacters{مَكَّارٌ }\textstyleDropCaps{хитрый‎}

\textstyleCaptioncharacters{عَالَجَ }\textstyleDropCaps{лечил‎}

\textstyleCaptioncharacters{عِلاَجٌ }\textstyleDropCaps{лечение‎}

\textstyleCaptioncharacters{مَرِضَ }\textstyleDropCaps{заболел‎}

\textstyleCaptioncharacters{بَحَثَ عَنْ...ِ }\textstyleDropCaps{искал‎}

\textstyleCaptioncharacters{أَغْلاَقَ }\textstyleDropCaps{закрыл‎}

\textstyleCaptioncharacters{هَا }\textstyleDropCaps{вот‎}

\textstyleCaptioncharacters{هَا هُوَ الْبَيْتُ }\textstyleDropCaps{вот он дом‎}

\textstyleCaptioncharacters{مَنْ كَانَ يُرِيدُ }\textstyleDropCaps{тот, кто хочет‎}

\textstyleCaptioncharacters{لِيَكْتُبْ }\textstyleDropCaps{пусть пишет‎}

\textstyleCaptioncharacters{لِيَقْرَأْ }\textstyleDropCaps{пусть читает‎}

\textstyleCaptioncharacters{عَاشَ }\textstyleDropCaps{жил‎}

\textstyleCaptioncharacters{شَكَا }\textstyleDropCaps{жаловался‎}

\textstyleCaptioncharacters{أَنَّ }\textstyleDropCaps{стонал‎}

\textstyleCaptioncharacters{أَشَارَ اِلَى...ِ }\textstyleDropCaps{показал, ука­зал на что-л.‎}

\subsection[Урок 55‎]{\textstyleDropCaps{Урок 55‎}}
\textstyleCaptioncharacters{عُطْلَةٌ }\textstyleDropCaps{каникулы‎}

\textstyleCaptioncharacters{عُطْلَةُ نِصْفِ السَّنَةِ }\textstyleDropCaps{ка­никулы полугодия‎}

\textstyleCaptioncharacters{نِصْفٌ }\textstyleDropCaps{половина‎}

\textstyleCaptioncharacters{بَعْدَ وَقْتٍ قَصِيرٍ }\textstyleDropCaps{спустя немного, скоро, через некоторое время‎}

\textstyleCaptioncharacters{نُورٌ }\textstyleDropCaps{свет‎}

\textstyleCaptioncharacters{كَهْرَبَاءٌ }\textstyleDropCaps{электричество‎}

\textstyleCaptioncharacters{نُورُ الْكَهْرَبَاءِ }\textstyleDropCaps{электриче­ский свет‎}

\textstyleCaptioncharacters{مَحَطَّةٌ }\textstyleDropCaps{станция‎}

\textstyleCaptioncharacters{مَحَطَّةُ الْكَهْرَبَاءِ }\textstyleDropCaps{элек­тростанция‎}

\textstyleCaptioncharacters{طَاهِرٌ }\textstyleDropCaps{чистый‎}

\textstyleCaptioncharacters{صَافٍ }\textstyleDropCaps{прозрачный‎}

\textstyleCaptioncharacters{مَاءٌ صَافٍ }\textstyleDropCaps{прозрачная вода‎}

\textstyleCaptioncharacters{كُلُّ عَامٍ وَ اَنْتُمْ بِخَيْرٍ }\textstyleDropCaps{С Новым годом!‎}

\textstyleCaptioncharacters{وَ اَنْتُمْ بِالْخَيْرِ وَ السَّعَادَةِ }\textstyleDropCaps{и вам желаем добра и счастья‎}

\textstyleCaptioncharacters{حَكَى }\textstyleDropCaps{рассказал‎}

\textstyleCaptioncharacters{اِنْتَهَى }\textstyleDropCaps{закончился ‎}

\subsection[Урок 56‎]{\textstyleDropCaps{Урок 56‎}}
\textstyleCaptioncharacters{مِطْرَقَةٌ }\textstyleDropCaps{молоток‎}

\textstyleCaptioncharacters{فَأْسٌ }\textstyleDropCaps{топор‎}

\textstyleCaptioncharacters{مِنْشَارٌ }\textstyleDropCaps{пила‎}

\textstyleCaptioncharacters{شَوْكَةٌ }\textstyleDropCaps{вилка‎}

\textstyleCaptioncharacters{صَحْنٌ }\textstyleDropCaps{тарелка, блюдо‎}

\textstyleCaptioncharacters{مِلْعَقَةٌ }\textstyleDropCaps{ложка‎}

\textstyleCaptioncharacters{جَدٌّ }\textstyleDropCaps{дед, дедушка‎}

\textstyleCaptioncharacters{جَدَّةٌ }\textstyleDropCaps{бабушка‎}

\textstyleCaptioncharacters{اَبَوَانِ }\textstyleDropCaps{родители‎}

\textstyleCaptioncharacters{اَقَارِبِ }\textstyleDropCaps{родственники, род­ные‎}

\textstyleCaptioncharacters{مُطِيعٌ }\textstyleDropCaps{послушный‎}

\textstyleCaptioncharacters{عَاصٍ }\textstyleDropCaps{непослушный, непок­орный‎}

\textstyleCaptioncharacters{مُؤَدَّبٌ }\textstyleDropCaps{воспитанный‎}

\textstyleCaptioncharacters{غَيْرُ مُؤَدَّبٍ }\textstyleDropCaps{невоспитан­ный‎}

\textstyleCaptioncharacters{كُنْ }\textstyleDropCaps{будь‎}

\textstyleCaptioncharacters{لاَ تَكُنْ }\textstyleDropCaps{не будь‎}

\textstyleCaptioncharacters{شُورْبَةٌ }\textstyleDropCaps{суп, шурпа‎}

\textstyleCaptioncharacters{صَيْفٌ }\textstyleDropCaps{лето‎}

\textstyleCaptioncharacters{خَشَبٌ }\textstyleDropCaps{лес, лесоматериа­лы, деревяшка‎}

\textstyleCaptioncharacters{مِسْمَارٌ }\textstyleDropCaps{гвоздь‎}

\textstyleCaptioncharacters{اِلْتَفَتَ }\textstyleDropCaps{оборачивался, по­ворачивался‎}

\textstyleCaptioncharacters{اِحْتَرَمَ }\textstyleDropCaps{уважал‎}

\textstyleCaptioncharacters{دَائِمًا }\textstyleDropCaps{всегда‎}

\textstyleCaptioncharacters{اَثْنَاءَ...ِ }\textstyleDropCaps{во время‎}

\textstyleCaptioncharacters{اَثْنَاءَ الدَّرْسِ }\textstyleDropCaps{во время урока, на уроке‎}

\textstyleCaptioncharacters{نَجَّارٌ }\textstyleDropCaps{плотник, столяр‎}

\textstyleCaptioncharacters{آلَةٌ }\textstyleDropCaps{инструмент, орудие‎}

\textstyleCaptioncharacters{غَسَلَ }\textstyleDropCaps{мыл; стирал‎}

\textstyleCaptioncharacters{قَطَعَ }\textstyleDropCaps{резал‎}

\textstyleCaptioncharacters{نَشَرَ }\textstyleDropCaps{пилил‎}

\subsection[Урок 57‎]{\textstyleDropCaps{Урок 57‎}}
\textstyleCaptioncharacters{كُوبِيكٌ }\textstyleDropCaps{копейка‎}

\textstyleCaptioncharacters{رُوبِلٌ }\textstyleDropCaps{рубль‎}

\textstyleCaptioncharacters{بَاصٌ }\textstyleDropCaps{автобус‎}

\textstyleCaptioncharacters{وَاحِدٌ }\textstyleDropCaps{один‎}

\textstyleCaptioncharacters{اِثْنَانِ }\textstyleDropCaps{два‎}

\textstyleCaptioncharacters{ثَلاَثَةٌ }\textstyleDropCaps{три‎}

\textstyleCaptioncharacters{اَرْبَعَةٌ }\textstyleDropCaps{четыре‎}

\textstyleCaptioncharacters{خَمْسَةٌ }\textstyleDropCaps{пять‎}

\textstyleCaptioncharacters{سِتَّةٌ }\textstyleDropCaps{шесть‎}

\textstyleCaptioncharacters{سَبْعَةٌ }\textstyleDropCaps{семь‎}

\textstyleCaptioncharacters{ثَمَانِيَةٌ }\textstyleDropCaps{восемь‎}

\textstyleCaptioncharacters{تِسْعَةٌ }\textstyleDropCaps{девять‎}

\textstyleCaptioncharacters{عَشَرَةٌ }\textstyleDropCaps{десять‎}

\textstyleCaptioncharacters{سِكِّينٌ }\textstyleDropCaps{нож‎}

\textstyleCaptioncharacters{صَحْنُ الْمَسْجِدِ }\textstyleDropCaps{двор мечети‎}

\textstyleCaptioncharacters{ضَيْفٌ }\textstyleDropCaps{гость‎}

\textstyleCaptioncharacters{غَالٍ }\textstyleDropCaps{дорогой (по цене)‎}

\textstyleCaptioncharacters{رَخِيصٌ }\textstyleDropCaps{дешевый‎}

\textstyleCaptioncharacters{قِيمَةٌ }\textstyleDropCaps{цена, стоимость‎}

\textstyleCaptioncharacters{جَيْبٌ }\textstyleDropCaps{карман‎}

\textstyleCaptioncharacters{بَاكِرًا }\textstyleDropCaps{рано‎}

\textstyleCaptioncharacters{مُتَأَخِّرًا }\textstyleDropCaps{поздно, с опозда­нием‎}

\textstyleCaptioncharacters{أَقَامَ }\textstyleDropCaps{прожил; оставался‎}

\textstyleCaptioncharacters{تَخَلَّفَ عَنْ...ِ }\textstyleDropCaps{отсутство­вал на...‎}

\textstyleCaptioncharacters{كَمْ؟ }\textstyleDropCaps{сколько?‎}

\textstyleCaptioncharacters{بِكَمْ؟ }\textstyleDropCaps{почём? за сколько?‎}

\textstyleCaptioncharacters{يُسَاوِى }\textstyleDropCaps{стоит‎}

\textstyleCaptioncharacters{كَمْ يُسَاوِى؟ }\textstyleDropCaps{сколько стоит?‎}

\textstyleCaptioncharacters{نَهَضَ }\textstyleDropCaps{встал, поднялся‎}

\textstyleCaptioncharacters{نَهَضَ مِنْ نَّوْمِهِ }\textstyleDropCaps{встал с постели‎}

\textstyleCaptioncharacters{نَمْ }\textstyleDropCaps{спи‎}

\textstyleCaptioncharacters{لاَ تَنَمْ }\textstyleDropCaps{не спи‎}

\textstyleCaptioncharacters{رَجُلٌ }\textstyleDropCaps{(один) мужчина‎}

\textstyleCaptioncharacters{رَجُلاَنِ }\textstyleDropCaps{двое мужчин‎}

\textstyleCaptioncharacters{بَيْتٌ }\textstyleDropCaps{(один) дом‎}

\textstyleCaptioncharacters{بَيْتَانِ }\textstyleDropCaps{два дома‎}

\subsection[Урок 58‎]{\textstyleDropCaps{Урок 58‎}}
\textstyleCaptioncharacters{دِهْلِيزٌ }\textstyleDropCaps{коридор‎}

\textstyleCaptioncharacters{سِينَمَا }\textstyleDropCaps{кино‎}

\textstyleCaptioncharacters{مِنْ زَمَانٍ }\textstyleDropCaps{давно, давно уже‎}

\textstyleCaptioncharacters{نَادٍ }\textstyleDropCaps{клуб‎}

\textstyleCaptioncharacters{كَيْفَ حَالُكُمْ؟ }\textstyleDropCaps{как ваши дела? как поживаете?‎}

\textstyleCaptioncharacters{طَيِّبٌ }\textstyleDropCaps{хорошо‎}

\textstyleCaptioncharacters{اَشْكُرُكُمْ }\textstyleDropCaps{благодарю вас‎}

\textstyleCaptioncharacters{اِلَى اللِّقَاءِ }\textstyleDropCaps{до свидания!‎}

\textstyleCaptioncharacters{مَعَ السَّلاَمَةِ }\textstyleDropCaps{всего хоро­шего!‎}

\textstyleCaptioncharacters{حَاضِرٌ }\textstyleDropCaps{присутствующий‎}

\textstyleCaptioncharacters{غَائِبٌ }\textstyleDropCaps{отсутствующий‎}

\textstyleCaptioncharacters{اَمَامِيٌّ }\textstyleDropCaps{передний‎}

\textstyleCaptioncharacters{بِجَانِبِ...ِ }\textstyleDropCaps{рядом с...‎}

\textstyleCaptioncharacters{بَعْدَ قَلِيلٍ }\textstyleDropCaps{вскоре, через некоторое время‎}

\textstyleCaptioncharacters{مَعًا }\textstyleDropCaps{вместе‎}

\textstyleCaptioncharacters{تَذَاكَرَ }\textstyleDropCaps{учили, повторяли совместно‎}

\textstyleCaptioncharacters{اِكْرَامٌ }\textstyleDropCaps{почтение, уважение‎}

\textstyleCaptioncharacters{اِكْرَامًا لَهُ }\textstyleDropCaps{в честь его, в знак уважения к нему‎}

\textstyleCaptioncharacters{عَلَى الْفَوْرِ }\textstyleDropCaps{сразу, тотчас‎}

\textstyleCaptioncharacters{غُرْفَةُ الدَّرْسِ }\textstyleDropCaps{аудитория‎}

\textstyleCaptioncharacters{مَحَلَّةٌ }\textstyleDropCaps{квартал‎}

\textstyleCaptioncharacters{مَسْجِدُ الْمَحَلَّةِ }\textstyleDropCaps{квар­тальная мечеть‎}

\textstyleCaptioncharacters{تَفَرَّقَ }\textstyleDropCaps{разошёлся‎}

\textstyleCaptioncharacters{يُعْرَضُ }\textstyleDropCaps{идет, показывают‎}

\textstyleCaptioncharacters{ذَهَبَا }\textstyleDropCaps{оба ушли‎}

\textstyleCaptioncharacters{يَذْهَبَانِ }\textstyleDropCaps{оба идут‎}

\subsection[Урок 59‎]{\textstyleDropCaps{Урок 59‎}}
\textstyleCaptioncharacters{مِصْبَاحٌ }\textstyleDropCaps{лампа‎}

\textstyleCaptioncharacters{مِصْبَاحٌ كَهْرَبَائِىٌّ }\textstyleDropCaps{лам­почка‎}

\textstyleCaptioncharacters{كَهْرَبَائِىٌّ }\textstyleDropCaps{электрический‎}

\textstyleCaptioncharacters{مِنْضَدَةٌ }\textstyleDropCaps{стол‎}

\textstyleCaptioncharacters{مِنْضَدَةُ الْكُتُبِ }\textstyleDropCaps{эта­жерка‎}

\textstyleCaptioncharacters{سِيَاسِىٌّ }\textstyleDropCaps{политический‎}

\textstyleCaptioncharacters{أَدَبِىٌّ }\textstyleDropCaps{литературный‎}

\textstyleCaptioncharacters{وَطَنِىٌّ }\textstyleDropCaps{национальный, отечественный‎}

\textstyleCaptioncharacters{ضَرُورِىٌّ }\textstyleDropCaps{необходимый‎}

\textstyleCaptioncharacters{وَدِّىٌّ }\textstyleDropCaps{дружеский, друже­ственный‎}

\textstyleCaptioncharacters{عَلاَقَةٌ }\textstyleDropCaps{отношение, связь‎}

\textstyleCaptioncharacters{عَلاَقَاتٌ وُدِّيَّةٌ }\textstyleDropCaps{друже­ские отношение‎}

\textstyleCaptioncharacters{صَحِيفَةٌ }\textstyleDropCaps{газета‎}

\textstyleCaptioncharacters{مَقَالَةٌ }\textstyleDropCaps{статья‎}

\textstyleCaptioncharacters{شَبَابٌ }\textstyleDropCaps{молодёжь‎}

\textstyleCaptioncharacters{صَدَاقَةٌ }\textstyleDropCaps{дружба‎}

\textstyleCaptioncharacters{سَلاَمٌ }\textstyleDropCaps{мир‎}

\textstyleCaptioncharacters{بِالْقُرْبِ مِنْ...ِ }\textstyleDropCaps{около, близ, возле...‎}

\textstyleCaptioncharacters{لاَسِيَّمَا }\textstyleDropCaps{особенно‎}

\textstyleCaptioncharacters{كَذَلِكَ }\textstyleDropCaps{также‎}

\textstyleCaptioncharacters{لِكَىْ...ِ }\textstyleDropCaps{чтобы, для того, чтобы...‎}

\textstyleCaptioncharacters{لِكَىْ يَعِيشَ }\textstyleDropCaps{чтобы жить‎}

\textstyleCaptioncharacters{طَوِيلاً }\textstyleDropCaps{долго‎}

\textstyleCaptioncharacters{لا َيَكُونُ اِلاَّ... }\textstyleDropCaps{будет только, не будет...кроме‎}

\textstyleCaptioncharacters{لاَ يَفْهَمُهُ اِلاَّ هُوَ }\textstyleDropCaps{это по­нимает только он‎}

\textstyleCaptioncharacters{غَيْرَأَنَّ...ِ }\textstyleDropCaps{однако, но‎}

\textstyleCaptioncharacters{مَطْبُوعٌ }\textstyleDropCaps{печатный, напечат­анный‎}

\textstyleCaptioncharacters{مَخْطُوطٌ }\textstyleDropCaps{рукописный‎}

\textstyleCaptioncharacters{مُطَالَعَةٌ }\textstyleDropCaps{чтение‎}

\textstyleCaptioncharacters{قَاعَةُ الْمُطَالَعَةِ }\textstyleDropCaps{читаль­ный зал‎}

\textstyleCaptioncharacters{نَيِّرٌ }\textstyleDropCaps{светлый‎}

\textstyleCaptioncharacters{مَشْغُولٌ }\textstyleDropCaps{занятый‎}

\textstyleCaptioncharacters{اَمَّا...فَ...ِ }\textstyleDropCaps{что касается..., то...‎}

\textstyleCaptioncharacters{بَيْنَ...ِ }\textstyleDropCaps{среди‎}

\textstyleCaptioncharacters{بَيْنَهُمْ }\textstyleDropCaps{среди них‎}

\textstyleCaptioncharacters{بَيْنَكُمْ }\textstyleDropCaps{среди вас‎}

\textstyleCaptioncharacters{بَيْنَنَا }\textstyleDropCaps{среди нас‎}

\textstyleCaptioncharacters{مَكَثَ }\textstyleDropCaps{оставался, находил­ся, пребывал‎}

\subsection[Урок 60‎]{\textstyleDropCaps{Урок 60‎}}
\textstyleCaptioncharacters{عِشْرُونَ }\textstyleDropCaps{двадцать‎}

\textstyleCaptioncharacters{ثَلاَثُونَ }\textstyleDropCaps{тридцать‎}

\textstyleCaptioncharacters{أَرْبَعُونَ }\textstyleDropCaps{сорок‎}

\textstyleCaptioncharacters{خَمْسُونَ }\textstyleDropCaps{пятьдесят‎}

\textstyleCaptioncharacters{سِتُّونَ }\textstyleDropCaps{шестьдесят‎}

\textstyleCaptioncharacters{سَبْعُونَ }\textstyleDropCaps{семьдесят‎}

\textstyleCaptioncharacters{ثَمَانُونَ }\textstyleDropCaps{восемьдесят‎}

\textstyleCaptioncharacters{تِسْعُونَ }\textstyleDropCaps{девяносто‎}

\textstyleCaptioncharacters{مِائَةٌ }\textstyleDropCaps{сто‎}

\textstyleCaptioncharacters{مِائَتَانِ }\textstyleDropCaps{двести‎}

\textstyleCaptioncharacters{ثَلاَثُمِائَةٍ }\textstyleDropCaps{триста‎}

\textstyleCaptioncharacters{أَلْفٌ }\textstyleDropCaps{тысяча‎}

\textstyleCaptioncharacters{اَلْفَانِ }\textstyleDropCaps{две тысячи‎}

\textstyleCaptioncharacters{ثَلاَثَةُ آلاَفٍ }\textstyleDropCaps{три тысячи‎}

\textstyleCaptioncharacters{مِلْيُونٌ }\textstyleDropCaps{миллион‎}

\textstyleCaptioncharacters{أَحَدَ عَشَرَ }\textstyleDropCaps{одиннадцать‎}

\textstyleCaptioncharacters{اِثْنَا عَشَرَ }\textstyleDropCaps{двенадцать‎}

\textstyleCaptioncharacters{ثَلاَثَةَ عَشَرَ }\textstyleDropCaps{тринадцать‎}

\textstyleCaptioncharacters{مِائَةٌ وَ اَحَدَ عَشَرَ }\textstyleDropCaps{сто одиннадцать‎}

\textstyleCaptioncharacters{رَبِيعٌ }\textstyleDropCaps{весна ‎}

\textstyleCaptioncharacters{صَيْفٌ }\textstyleDropCaps{лето‎}

\textstyleCaptioncharacters{خَرِيفٌ }\textstyleDropCaps{осень‎}

\textstyleCaptioncharacters{شِتَاءٌ }\textstyleDropCaps{зима‎}

\textstyleCaptioncharacters{سَاعَةٌ }\textstyleDropCaps{час‎}

\textstyleCaptioncharacters{دَقِيقَةٌ }\textstyleDropCaps{минута‎}

\textstyleCaptioncharacters{ثَانِيَةٌ }\textstyleDropCaps{секунда‎}

\textstyleCaptioncharacters{فَصْلٌ }\textstyleDropCaps{время года, квартал‎}

\textstyleCaptioncharacters{فُصُولُ السَّنَةِ }\textstyleDropCaps{времена года‎}

\textstyleCaptioncharacters{كُتَّابٌ }\textstyleDropCaps{начальная школа‎}

\subsection[Урок 61‎]{\textstyleDropCaps{Урок 61‎}}
\textstyleCaptioncharacters{خَيَّاطٌ }\textstyleDropCaps{портной‎}

\textstyleCaptioncharacters{خَيَّاطَةٌ }\textstyleDropCaps{портниха‎}

\textstyleCaptioncharacters{دَوَاءٌ }\textstyleDropCaps{лекарство‎}

\textstyleCaptioncharacters{كُرَةُ السَّلَّةِ }\textstyleDropCaps{баскетбол‎}

\textstyleCaptioncharacters{اَلَّّذِى }\textstyleDropCaps{который‎}

\textstyleCaptioncharacters{اَلَّذِينَ }\textstyleDropCaps{которые‎}

\textstyleCaptioncharacters{اَلَّتِى }\textstyleDropCaps{которая‎}

\textstyleCaptioncharacters{اَللاَّتِى }\textstyleDropCaps{которые (ж.р.)‎}

\textstyleCaptioncharacters{يُسَمَّى }\textstyleDropCaps{называется, его на­зывают‎}

\textstyleCaptioncharacters{عَزَبٌ }\textstyleDropCaps{холостой‎}

\textstyleCaptioncharacters{عَزَبَةٌ }\textstyleDropCaps{незамужняя‎}

\textstyleCaptioncharacters{مَكْتُوبٌ }\textstyleDropCaps{письмо‎}

\textstyleCaptioncharacters{جُنَيْنَةٌ }\textstyleDropCaps{сад, садик‎}

\textstyleCaptioncharacters{اُسْتَاذَةٌ }\textstyleDropCaps{профессор (ж.р.)‎}

\textstyleCaptioncharacters{فَرِيقٌ }\textstyleDropCaps{команда‎}

\textstyleCaptioncharacters{لِبَاسٌ }\textstyleDropCaps{одежда‎}

\textstyleCaptioncharacters{نَهَارًا }\textstyleDropCaps{днем‎}

\textstyleCaptioncharacters{آنِفًا }\textstyleDropCaps{недавно, только что‎}

\textstyleCaptioncharacters{مُنْذُ اُسْبُوعٍ }\textstyleDropCaps{неделя тому назад, уже неделя‎}

\textstyleCaptioncharacters{رَاجِعٌ }\textstyleDropCaps{возвращающийся‎}

\textstyleCaptioncharacters{خَاطَ }\textstyleDropCaps{шил‎}

\textstyleCaptioncharacters{لَقِىَ }\textstyleDropCaps{встретил‎}

\textstyleCaptioncharacters{فَازَ }\textstyleDropCaps{выиграл, победил‎}

\textstyleCaptioncharacters{اِسْتَعْمَلَ }\textstyleDropCaps{использовал, употреблял, применял‎}

\textstyleCaptioncharacters{أَعْطَى }\textstyleDropCaps{дал‎}

\textstyleCaptioncharacters{شُفِىَ }\textstyleDropCaps{выздоровел‎}

\textstyleCaptioncharacters{اِذَا لَمْ تَفْعَلْ }\textstyleDropCaps{если не сде­лаешь‎}

\subsection[Урок 62‎]{\textstyleDropCaps{Урок 62‎}}
\textstyleCaptioncharacters{ذُرَةٌ }\textstyleDropCaps{кукуруза‎}

\textstyleCaptioncharacters{طَاحُونٌ }\textstyleDropCaps{мельница‎}

\textstyleCaptioncharacters{مِحَشٌّ }\textstyleDropCaps{коса‎}

\textstyleCaptioncharacters{مِنْجَلٌ }\textstyleDropCaps{серп‎}

\textstyleCaptioncharacters{عُشْبٌ }\textstyleDropCaps{трава‎}

\textstyleCaptioncharacters{مَاشِيَةٌ }\textstyleDropCaps{скот‎}

\textstyleCaptioncharacters{شُونَةٌ }\textstyleDropCaps{сарай‎}

\textstyleCaptioncharacters{جَارَةٌ }\textstyleDropCaps{соседка‎}

\textstyleCaptioncharacters{شَمْسٌ }\textstyleDropCaps{солнце‎}

\textstyleCaptioncharacters{حَادٌّ }\textstyleDropCaps{острый‎}

\textstyleCaptioncharacters{كَهَامٌ }\textstyleDropCaps{тупой‎}

\textstyleCaptioncharacters{حُبُوبٌ }\textstyleDropCaps{зерно‎}

\textstyleCaptioncharacters{دَقِيقٌ }\textstyleDropCaps{мука‎}

\textstyleCaptioncharacters{نُخَالَةٌ }\textstyleDropCaps{отруби‎}

\textstyleCaptioncharacters{اِشْتَغَلَ }\textstyleDropCaps{занимался, рабо­тал‎}

\textstyleCaptioncharacters{اِشْتَغَلَ فِى الْحَقْلِ }\textstyleDropCaps{ра­ботал на поле‎}

\textstyleCaptioncharacters{اِشْتَغَلَ بِالْكُتُبِ }\textstyleDropCaps{зани­мался с книгами‎}

\textstyleCaptioncharacters{اِسْتَرَاحَ }\textstyleDropCaps{отдыхал‎}

\textstyleCaptioncharacters{كَدَّسَ }\textstyleDropCaps{собрал, свалил в кучу‎}

\textstyleCaptioncharacters{حَشَّ }\textstyleDropCaps{косил‎}

\textstyleCaptioncharacters{حَصَدَ }\textstyleDropCaps{жал‎}

\textstyleCaptioncharacters{طَحَنَ }\textstyleDropCaps{молол‎}

\textstyleCaptioncharacters{خَزَنَ }\textstyleDropCaps{хранил‎}

\textstyleCaptioncharacters{نَقَلَ }\textstyleDropCaps{переносил, перевозил‎}

\textstyleCaptioncharacters{يَبِسَ }\textstyleDropCaps{высох‎}

\textstyleCaptioncharacters{اَحَدَّ }\textstyleDropCaps{точил‎}

\textstyleCaptioncharacters{يَحْصُلُ الدَّقِيقُ }\textstyleDropCaps{мука по­лучается‎}

\textstyleCaptioncharacters{اِذَا جَاءَ }\textstyleDropCaps{когда придет, когда наступит‎}

\textstyleCaptioncharacters{حَتَّى يَجِئَ }\textstyleDropCaps{пока не при­дет ‎}

\textstyleCaptioncharacters{رَدَّ }\textstyleDropCaps{вернул, отдал обратно‎}

\subsection[Урок 63‎]{\textstyleDropCaps{Урок 63‎}}
\textstyleCaptioncharacters{اِبْرِيقٌ }\textstyleDropCaps{кувшин‎}

\textstyleCaptioncharacters{صَابُونٌ }\textstyleDropCaps{мыло‎}

\textstyleCaptioncharacters{حَنَفِيَّةٌ }\textstyleDropCaps{кран‎}

\textstyleCaptioncharacters{طَسْتٌ }\textstyleDropCaps{таз‎}

\textstyleCaptioncharacters{مِنْشَفَةٌ }\textstyleDropCaps{полотенце‎}

\textstyleCaptioncharacters{فُرْشَةُ اْلأَسْنَانِ }\textstyleDropCaps{зубная щетка‎}

\textstyleCaptioncharacters{سِنٌّ }\textstyleDropCaps{зуб‎}

\textstyleCaptioncharacters{مِشْجَبَةٌ }\textstyleDropCaps{вешалка‎}

\textstyleCaptioncharacters{سَرِيرٌ }\textstyleDropCaps{кровать, койка‎}

\textstyleCaptioncharacters{وَجْهٌ }\textstyleDropCaps{лицо‎}

\textstyleCaptioncharacters{رَكْعَةٌ }\textstyleDropCaps{ракат‎}

\textstyleCaptioncharacters{وُضُوءٌ }\textstyleDropCaps{омовение‎}

\textstyleCaptioncharacters{تَوَضَّأَ }\textstyleDropCaps{совершил омовение‎}

\textstyleCaptioncharacters{أَذَانٌ }\textstyleDropCaps{азан, призыв к нама­зу‎}

\textstyleCaptioncharacters{فَجْرٌ }\textstyleDropCaps{заря, раннее утро‎}

\textstyleCaptioncharacters{اَذَانُ الْفَجْرِ }\textstyleDropCaps{утренний азан‎}

\textstyleCaptioncharacters{نَشَّفَ }\textstyleDropCaps{вытер, высушил‎}

\textstyleCaptioncharacters{اِسْتَيْقَظَ }\textstyleDropCaps{проснулся‎}

\textstyleCaptioncharacters{أَخَّرَ }\textstyleDropCaps{отложил‎}

\textstyleCaptioncharacters{مُعَلَّقٌ }\textstyleDropCaps{висящий‎}

\textstyleCaptioncharacters{صُبْحٌ }\textstyleDropCaps{утро‎}

\textstyleCaptioncharacters{ظُهْرٌ }\textstyleDropCaps{полдень‎}

\textstyleCaptioncharacters{عَصْرٌ }\textstyleDropCaps{предвечернее время‎}

\textstyleCaptioncharacters{مَغْرِبٌ }\textstyleDropCaps{вечер‎}

\textstyleCaptioncharacters{عِشَاءٌ }\textstyleDropCaps{поздний вечер, ночь‎}

\subsection[Урок 64‎]{\textstyleDropCaps{Урок 64‎}}
\textstyleCaptioncharacters{مِبْرَاةٌ }\textstyleDropCaps{перочинный нож‎}

\textstyleCaptioncharacters{مِحْفَظَةُ نُقُودٍ }\textstyleDropCaps{кошелёк‎}

\textstyleCaptioncharacters{دِينَارٌ }\textstyleDropCaps{динар (золотой)‎}

\textstyleCaptioncharacters{دِرْهَمٌ }\textstyleDropCaps{дирхам (серебря­ный)‎}

\textstyleCaptioncharacters{مَعْرِضٌ }\textstyleDropCaps{выставка‎}

\textstyleCaptioncharacters{يَا أَبَتِ }\textstyleDropCaps{батюшка! о отец!‎}

\textstyleCaptioncharacters{أَرْجُوكَ }\textstyleDropCaps{прошу тебя‎}

\textstyleCaptioncharacters{لاَ ضَرُورَةَ لِذَلِكَ }\textstyleDropCaps{нет необходимости в том‎}

\textstyleCaptioncharacters{بَدَلاً مِنْهُ }\textstyleDropCaps{вместо него‎}

\textstyleCaptioncharacters{لَسْتُ بِحَاجَةٍ اِلَى...ِ }\textstyleDropCaps{мне не нужно, я не нуждаюсь‎}

\textstyleCaptioncharacters{مُتَضَايِقٌ }\textstyleDropCaps{сердитый, не в настроении, не в духе‎}

\textstyleCaptioncharacters{يَكْفِينِى }\textstyleDropCaps{мне хватит‎}

\textstyleCaptioncharacters{جُمْلَةٌ }\textstyleDropCaps{предложение‎}

\textstyleCaptioncharacters{قَضِيَّةٌ }\textstyleDropCaps{вопрос, дело‎}

\textstyleCaptioncharacters{أَمِينٌ }\textstyleDropCaps{верный, честный, на­дежный‎}

\textstyleCaptioncharacters{أَمَانَةٌ }\textstyleDropCaps{верность, честность‎}

\textstyleCaptioncharacters{شَدِيدٌ }\textstyleDropCaps{сильный‎}

\textstyleCaptioncharacters{عَظِيمٌ }\textstyleDropCaps{великий, большой‎}

\textstyleCaptioncharacters{أَحْضَرَ }\textstyleDropCaps{принес, пришёл , доставил‎}

\textstyleCaptioncharacters{أَعْلَنَ }\textstyleDropCaps{объявил‎}

\textstyleCaptioncharacters{اِعْلاَنٌ }\textstyleDropCaps{объявление‎}

\textstyleCaptioncharacters{رَاجَعَ }\textstyleDropCaps{обратился‎}

\textstyleCaptioncharacters{اِسْتَحَقَّ }\textstyleDropCaps{заслужил‎}

\textstyleCaptioncharacters{نَسِىَ }\textstyleDropCaps{забыл‎}

\textstyleCaptioncharacters{فَقَدَ }\textstyleDropCaps{потерял‎}

\textstyleCaptioncharacters{حَزِنَ }\textstyleDropCaps{опечалился, огор­чился‎}

\textstyleCaptioncharacters{عَثَرَ عَلَى... }\textstyleDropCaps{нашёл, нат­кнулся‎}

\textstyleCaptioncharacters{سُرَّ }\textstyleDropCaps{обрадовался‎}

\textstyleCaptioncharacters{سُرُورٌ }\textstyleDropCaps{радость‎}

\textstyleCaptioncharacters{مُكَافَأَةٌ }\textstyleDropCaps{награда, вознагра­ждение‎}

\textstyleCaptioncharacters{سَلَّمَ }\textstyleDropCaps{сдал, отдал‎}

\textstyleCaptioncharacters{ثَوَابٌ }\textstyleDropCaps{воздаяние, возна­граждение‎}

\subsection[Урок 65‎]{\textstyleDropCaps{Урок 65‎}}
\textstyleCaptioncharacters{ضَابِطٌ }\textstyleDropCaps{офицер‎}

\textstyleCaptioncharacters{قَائِدٌ }\textstyleDropCaps{водитель‎}

\textstyleCaptioncharacters{قَائِدُ دَبَّابَةٍ }\textstyleDropCaps{танкист, во­дитель танка‎}

\textstyleCaptioncharacters{مَا لَكَ؟ }\textstyleDropCaps{что тебе? что с то­бой?‎}

\textstyleCaptioncharacters{وَلاَحَاجَةَ }\textstyleDropCaps{ничего (ответ)‎}

\textstyleCaptioncharacters{حَالٌ }\textstyleDropCaps{положение, состоя­ние‎}

\textstyleCaptioncharacters{صِحَّةٌ }\textstyleDropCaps{здоровье‎}

\textstyleCaptioncharacters{لاَبَأْسَبِهِ }\textstyleDropCaps{неплохо, ничего, нормально‎}

\textstyleCaptioncharacters{كُلُّ شَيْءٍ عَلَى مَا يُرَامُ }\textstyleDropCaps{все в порядке‎}

\textstyleCaptioncharacters{نَوْبَتْجِىٌّ }\textstyleDropCaps{дежурный‎}

\textstyleCaptioncharacters{مِنْ جَدِيدٍ }\textstyleDropCaps{снова‎}

\textstyleCaptioncharacters{تَخَرَّجَ مِنْ ...ِ }\textstyleDropCaps{окончил (учебное заведение)‎}

\textstyleCaptioncharacters{حَرْبِىٌّ }\textstyleDropCaps{военный‎}

\textstyleCaptioncharacters{مَدْرَسَةٌ حَرْبِيَّةٌ }\textstyleDropCaps{военное училище‎}

\textstyleCaptioncharacters{مُلاَزِمٌ }\textstyleDropCaps{лейтенант‎}

\textstyleCaptioncharacters{رُتْبَةٌ رُتَبٌ }\textstyleDropCaps{звание‎}

\textstyleCaptioncharacters{صَبَاحَ الْخَيْرِ }\textstyleDropCaps{доброе утро!‎}

\textstyleCaptioncharacters{صَبَاحَ النُّورِ }\textstyleDropCaps{доброе утро (ответ)!‎}

\textstyleCaptioncharacters{فِرْقَةٌ }\textstyleDropCaps{дивизия‎}

\textstyleCaptioncharacters{الفِرْقَتُ الْمُدَرَّعَةُ }\textstyleDropCaps{бро­нетанковая дивизия‎}

\textstyleCaptioncharacters{اَكَادِيمِيَّةُ الْمُدَرَّعَاتِ }\textstyleDropCaps{бронетанковая академия‎}

\textstyleCaptioncharacters{مُحَارِبٌ }\textstyleDropCaps{воин‎}

\textstyleCaptioncharacters{فِى سَبِيلِ اللَّهِ }\textstyleDropCaps{на пути Бога, во имя Бога‎}

\textstyleCaptioncharacters{عَسْكَرِىٌّ }\textstyleDropCaps{военный‎}

\textstyleCaptioncharacters{اِنْ وَفَّقَنِىَ اللَّهُ }\textstyleDropCaps{если Бог мне поможет‎}

\textstyleCaptioncharacters{حَتَّى اَكُونَ مُحَارِبًا }\textstyleDropCaps{чтобы я был воином‎}

\textstyleCaptioncharacters{خِلاَلَ شَهْرٍ }\textstyleDropCaps{в течении месяца‎}

\textstyleCaptioncharacters{تَدْرِيبٌ }\textstyleDropCaps{учение, практика‎}

\textstyleCaptioncharacters{تَدْرِيبَاتٌ عَسْكَرِيَّةٌ }\textstyleDropCaps{во­енные учения‎}

\textstyleCaptioncharacters{بَسِيطٌ }\textstyleDropCaps{простой‎}

\textstyleCaptioncharacters{مَبْسُوطٌ بِ... }\textstyleDropCaps{доволь­ный‎}

\textstyleCaptioncharacters{كَمْ رَأَيْتُ!ِ }\textstyleDropCaps{сколько же я видел!‎}

\textstyleCaptioncharacters{حَدَّثَ }\textstyleDropCaps{рассказал‎}

\textstyleCaptioncharacters{خَطَفَ }\textstyleDropCaps{схватил‎}

\textstyleCaptioncharacters{تَخَرَّجَ مِنَ الْمَدْرَسَةِ الْحَرْبِيَّةِ }\textstyleDropCaps{окончил военное училище‎}

\textstyleCaptioncharacters{هَرَبَ }\textstyleDropCaps{убежал, сбежал‎}

\subsection[Урок 66‎]{\textstyleDropCaps{Урок 66‎}}
\textstyleCaptioncharacters{وَاصَلَ }\textstyleDropCaps{продолжал‎}

\textstyleCaptioncharacters{حَاوَلَ }\textstyleDropCaps{попытался, попро­бовал‎}

\textstyleCaptioncharacters{اَيْقَظَ }\textstyleDropCaps{разбудил‎}

\textstyleCaptioncharacters{اَعَادَ }\textstyleDropCaps{повторил‎}

\textstyleCaptioncharacters{اِسْتَيْقَظَ مَعَ الْفَجْرِ }\textstyleDropCaps{проснулся на заре, с зарей‎}

\textstyleCaptioncharacters{غَلَطٌ }\textstyleDropCaps{ошибка‎}

\textstyleCaptioncharacters{بِلاَأَغْلاَطٍ }\textstyleDropCaps{без ошибок‎}

\textstyleCaptioncharacters{نَصٌّ }\textstyleDropCaps{текст‎}

\textstyleCaptioncharacters{مَرَّةً }\textstyleDropCaps{раз, один раз‎}

\textstyleCaptioncharacters{كَمْ مَرَّةً؟ }\textstyleDropCaps{сколько раз‎}

\textstyleCaptioncharacters{اَحْصَى }\textstyleDropCaps{считал, исчислял‎}

\textstyleCaptioncharacters{تَرْجَمَ }\textstyleDropCaps{переводил‎}

\textstyleCaptioncharacters{اَغْلَقَ الْكِتَابَ }\textstyleDropCaps{закрыл книгу‎}

\textstyleCaptioncharacters{خَافِتٌ }\textstyleDropCaps{тихий‎}

\textstyleCaptioncharacters{بِصَوْتٍ خَافِتٍ }\textstyleDropCaps{тихим го­лосом‎}

\textstyleCaptioncharacters{يَعْنِى }\textstyleDropCaps{значит, означает‎}

\textstyleCaptioncharacters{مَاذَا يَعْنِى؟ }\textstyleDropCaps{что значит?‎}

\textstyleCaptioncharacters{زَادَ }\textstyleDropCaps{прибавил, добавил, дал ещё‎}

\textstyleCaptioncharacters{مَزِيدٌ }\textstyleDropCaps{больше, побольше‎}

\textstyleCaptioncharacters{هَلْ اَزِيدُكَ؟ }\textstyleDropCaps{добавить тебе? дать тебе ещё?‎}

\textstyleCaptioncharacters{ظَلاَّسَةٌ }\textstyleDropCaps{тряпка‎}

\textstyleCaptioncharacters{كِفَايَه! }\textstyleDropCaps{достаточно! хва­тит!‎}

\textstyleCaptioncharacters{لاحَاجَةَ اِلَى...ِ }\textstyleDropCaps{не надо, не нужно‎}

\textstyleCaptioncharacters{نَزَلَ بِهِ }\textstyleDropCaps{остановился у него (гостем)‎}

\textstyleCaptioncharacters{هَلْ لَّكَ؟ }\textstyleDropCaps{мог бы ты? хо­тел бы ты?‎}

\textstyleCaptioncharacters{وَاجِبٌ }\textstyleDropCaps{обязанность, зада­ние‎}

\textstyleCaptioncharacters{وَاجِبُ الْمَنْزِلِ }\textstyleDropCaps{домаш­нее задание‎}

\textstyleCaptioncharacters{تَمْرِينَاتٌ رِيَاضِيَّةٌ }\textstyleDropCaps{физ­культура, физзарядка‎}

\textstyleCaptioncharacters{بِانْتِظَامٍ }\textstyleDropCaps{регулярно‎}

\textstyleCaptioncharacters{قَطَعَ }\textstyleDropCaps{прервал, прекратил‎}

\textstyleCaptioncharacters{طَلَبَ }\textstyleDropCaps{попросил, потребо­вал‎}

\textstyleCaptioncharacters{اَمْلَى }\textstyleDropCaps{диктовал‎}

\textstyleCaptioncharacters{اِمْلاَءٌ }\textstyleDropCaps{диктант‎}

\subsection[Урок 67‎]{\textstyleDropCaps{Урок 67‎}}
\textstyleCaptioncharacters{زَعَمُوا }\textstyleDropCaps{говорят‎}

\textstyleCaptioncharacters{وَصَفَ }\textstyleDropCaps{описал‎}

\textstyleCaptioncharacters{صِنَاعَةٌ }\textstyleDropCaps{промышленность‎}

\textstyleCaptioncharacters{زِرَاعَة }\textstyleDropCaps{земледелие; сель­ское хозяйство‎}

\textstyleCaptioncharacters{مَنَاخٌ }\textstyleDropCaps{климат‎}

\textstyleCaptioncharacters{لَوْلاَ }\textstyleDropCaps{если бы не‎}

\textstyleCaptioncharacters{نََصِيحَةٌ }\textstyleDropCaps{совет‎}

\textstyleCaptioncharacters{اِمْتِحَانٌ }\textstyleDropCaps{экзамен‎}

\textstyleCaptioncharacters{اَدَّى اْلإِمْتِحَانَ }\textstyleDropCaps{сдал эк­замен‎}

\textstyleCaptioncharacters{رَسَبَ فِى اْلإِمْتِحَانِ }\textstyleDropCaps{провалился на экзамене‎}

\textstyleCaptioncharacters{خَسِرَ }\textstyleDropCaps{потерял, лишился‎}

\textstyleCaptioncharacters{كَامِلاً }\textstyleDropCaps{полностью‎}

\textstyleCaptioncharacters{اِنْ لَمْ...ِ }\textstyleDropCaps{если не...‎}

\textstyleCaptioncharacters{اِنْ لَمْ تَفْعَلْ مَا آمُرُكَ }\textstyleDropCaps{если не сделаешь то, что я приказываю‎}

\textstyleCaptioncharacters{حَدِيثٌ }\textstyleDropCaps{разговор, беседа‎}

\textstyleCaptioncharacters{اِسْتَمَعَ اِلَى...ِ }\textstyleDropCaps{слушал‎}

\textstyleCaptioncharacters{مُتَّسَعٌ مِنَ الْوَقْتِ }\textstyleDropCaps{до­статочно времени‎}

\textstyleCaptioncharacters{رَغْبَةٌ عَظِيمَةٌ فِى...ِ }\textstyleDropCaps{большое желание‎}

\textstyleCaptioncharacters{كُنْتُ عَلَى رَغْبَةٍ عَظِيمَةٍ فِى...ِ }\textstyleDropCaps{у меня было большое желание (что-то сделать)‎}

\textstyleCaptioncharacters{حَالَ دُونَ...ِ }\textstyleDropCaps{помешал‎}

\textstyleCaptioncharacters{مَكَانٌ }\textstyleDropCaps{место‎}

\textstyleCaptioncharacters{مَكَانُ الْعَمَلِ }\textstyleDropCaps{рабочее ме­сто‎}

\textstyleCaptioncharacters{عَالِمٌ }\textstyleDropCaps{учёный‎}

\textstyleCaptioncharacters{جَاهِلٌ }\textstyleDropCaps{неграмотный, неуч, невежда‎}

\textstyleCaptioncharacters{مُمَثِّلٌ }\textstyleDropCaps{артист, актёр‎}

\textstyleCaptioncharacters{قَادِرٌ }\textstyleDropCaps{способный; могучий‎}

\textstyleCaptioncharacters{قَادِرٌ عَلَى كُلِّ شَيْءٍ }\textstyleDropCaps{всемогущий‎}

\textstyleCaptioncharacters{كَمَا تَعْرِفُ }\textstyleDropCaps{как тебе из­вестно‎}

\textstyleCaptioncharacters{أَدَاءٌ }\textstyleDropCaps{выполнение, исполне­ние‎}

\textstyleCaptioncharacters{خَطِيرٌ }\textstyleDropCaps{серьезный‎}

\textstyleCaptioncharacters{اِسْتَعَدَّ }\textstyleDropCaps{приготовился, го­товился‎}

\textstyleCaptioncharacters{مِنَ الآنَ }\textstyleDropCaps{отныне‎}

\textstyleCaptioncharacters{دُونَ...ِ }\textstyleDropCaps{а не...‎}

\textstyleCaptioncharacters{صَحِيحٌ }\textstyleDropCaps{правильный, вер­ный‎}

\textstyleCaptioncharacters{هَلْ هَذَا صَحِيحٌ؟ }\textstyleDropCaps{это правда? это верно?‎}

\textstyleCaptioncharacters{تَرَكَ عَمَلَهُ اِلَى وَقْتٍ آخَرِ }\textstyleDropCaps{отложил свою работу на другое время‎}

\subsection[Урок 68‎]{\textstyleDropCaps{Урок 68‎}}
\textstyleCaptioncharacters{رَسُولٌ }\textstyleDropCaps{посланник‎}

\textstyleCaptioncharacters{صَلَّى اللَّهُ عَلَيْهِ وَ سَلَّمَ }\textstyleDropCaps{да благословит его Бог и приветствует‎}

\textstyleCaptioncharacters{يَتِيمٌ }\textstyleDropCaps{сирота‎}

\textstyleCaptioncharacters{عُمْرٌ }\textstyleDropCaps{возраст, жизнь‎}

\textstyleCaptioncharacters{اَرْسَلَ }\textstyleDropCaps{послал‎}

\textstyleCaptioncharacters{اَنْزَلَ }\textstyleDropCaps{ниспослал‎}

\textstyleCaptioncharacters{نَشَأَ }\textstyleDropCaps{вырос‎}

\textstyleCaptioncharacters{بَلَغَ }\textstyleDropCaps{достиг‎}

\textstyleCaptioncharacters{اِبْنُ الْعَمِّ }\textstyleDropCaps{двоюродный брат‎}

\textstyleCaptioncharacters{بَشَرٌ }\textstyleDropCaps{человек; люди‎}

\textstyleCaptioncharacters{شُرْفَةٌ }\textstyleDropCaps{балкон‎}

\textstyleCaptioncharacters{ذَاكَرَ }\textstyleDropCaps{учил уроки‎}

\textstyleCaptioncharacters{قَاعِدَةٌ }\textstyleDropCaps{правило‎}

\textstyleCaptioncharacters{حَفِظَ }\textstyleDropCaps{выучил наизусть‎}

\textstyleCaptioncharacters{حَفِظَ الْقَاعِدَةَ }\textstyleDropCaps{выучил правило наизусть‎}

\textstyleCaptioncharacters{اَلنَّحْوُ }\textstyleDropCaps{грамматика; син­таксис‎}

\textstyleCaptioncharacters{اَلصَّرْفُ }\textstyleDropCaps{морфология‎}

\textstyleCaptioncharacters{اِهْتِمَامٌ }\textstyleDropCaps{внимание, ин­терес‎}

\textstyleCaptioncharacters{لَفَتَ اْلإِهْتِمَامَ اِلَى...ِ }\textstyleDropCaps{обратил внимание‎}

\textstyleCaptioncharacters{تَلْحِينٌ }\textstyleDropCaps{интонация‎}

\textstyleCaptioncharacters{نَطَقَ }\textstyleDropCaps{произносил‎}

\textstyleCaptioncharacters{بَلِيَّةٌ }\textstyleDropCaps{беда, бедствие‎}

\textstyleCaptioncharacters{رَصِيفُ النَّهْرِ }\textstyleDropCaps{набереж­ная реки‎}

\textstyleCaptioncharacters{هُوَ الآخَرُ }\textstyleDropCaps{от тоже‎}

\textstyleCaptioncharacters{ذَاهِبٌ }\textstyleDropCaps{идущий‎}

\textstyleCaptioncharacters{مَا يَزَالُ هُنَا }\textstyleDropCaps{он еще здесь‎}

\textstyleCaptioncharacters{غَادَرَ }\textstyleDropCaps{покинул; уехал‎}

\textstyleCaptioncharacters{مَضَى }\textstyleDropCaps{прошел‎}

\textstyleCaptioncharacters{حَدِيثٌ }\textstyleDropCaps{хадис, предание‎}

\textstyleCaptioncharacters{مَا هُوَ؟ }\textstyleDropCaps{какой?‎}

\textstyleCaptioncharacters{مُنْذُ اَنْ... }\textstyleDropCaps{с тех пор, как...; с того времени, как...‎}

\subsection[Урок 69‎]{\textstyleDropCaps{Урок 69‎}}
\textstyleCaptioncharacters{سَرَّ }\textstyleDropCaps{обрадовал‎}

\textstyleCaptioncharacters{تَمَنَّى }\textstyleDropCaps{пожелал‎}

\textstyleCaptioncharacters{يَا لَهُ مِنْ }\textstyleDropCaps{ну и какой! о, какой он!‎}

\textstyleCaptioncharacters{يَا لَهُ مِنْ رَجُلٍ!ِ }\textstyleDropCaps{о, ка­кой он человек!‎}

\textstyleCaptioncharacters{مُتَوَقَّعٌ }\textstyleDropCaps{ожидаемый‎}

\textstyleCaptioncharacters{غَيْرُ مُتَوَقَّعٍ }\textstyleDropCaps{неожидан­ный‎}

\textstyleCaptioncharacters{شَخْصٌ }\textstyleDropCaps{кто-то, человек‎}

\textstyleCaptioncharacters{ذَكِيٌّ }\textstyleDropCaps{способный, сообра­зительный‎}

\textstyleCaptioncharacters{مُشْمِسٌ }\textstyleDropCaps{солнечный‎}

\textstyleCaptioncharacters{بَعْدَ رَنِّ الْجَرَسِ }\textstyleDropCaps{после звонка‎}

\textstyleCaptioncharacters{بِدُونِ اَنْ...ِ }\textstyleDropCaps{без того, что­бы‎}

\textstyleCaptioncharacters{اِسْتِرَاحَةٌ }\textstyleDropCaps{отдых; перерыв‎}

\textstyleCaptioncharacters{اِسْمَحْ لِى }\textstyleDropCaps{разрешите, можно?‎}

\textstyleCaptioncharacters{أَلَيْسَ كَذَلِكَ؟ }\textstyleDropCaps{не так ли? разве не так? не правда ли?‎}

\textstyleCaptioncharacters{بَلَى }\textstyleDropCaps{да, конечно‎}

\textstyleCaptioncharacters{فُرْسَةٌ سَعِيدَةٌ!ِ }\textstyleDropCaps{рад вас видеть‎}

\textstyleCaptioncharacters{اَنَا اَسْعَدُ }\textstyleDropCaps{очень приятно (ответ)‎}

\textstyleCaptioncharacters{يَا صَاحِ!ِ }\textstyleDropCaps{о друг! дружи­ще!‎}

\textstyleCaptioncharacters{يَوْمًامَّا }\textstyleDropCaps{когда-нибудь, как-то раз‎}

\textstyleCaptioncharacters{تُصْبِحُ عَلَى خَيْرٍ }\textstyleDropCaps{спо­койной ночи‎}

\textstyleCaptioncharacters{وَ اَنْتَ بِخَيْرٍ }\textstyleDropCaps{всего хоро­шего (ответ)‎}

\textstyleCaptioncharacters{سَلِّمْ لِى عَلَى...ِ }\textstyleDropCaps{пере­дай от меня...‎}

\textstyleCaptioncharacters{سَيِّدَةٌ }\textstyleDropCaps{дама, госпожа‎}

\textstyleCaptioncharacters{اَبْلَغَ }\textstyleDropCaps{сообщил, уведомил‎}

\textstyleCaptioncharacters{مُشْتَاقٌ اِلَى...ِ }\textstyleDropCaps{сильно же­лающий, жаждущий, тоскующий‎}

\textstyleCaptioncharacters{كَمِ السَّاعَةُ؟ }\textstyleDropCaps{который час? сколько времени?‎}

\textstyleCaptioncharacters{اَلسَّاعَةُ تُشِيرُ اِلَى... }\textstyleDropCaps{часы показывают...‎}

\textstyleCaptioncharacters{مُنْتَصَفُ اللَّيْلِ }\textstyleDropCaps{полночь‎}

\textstyleCaptioncharacters{بَعْدَ مُنْتَصَفِ اللَّيْلِ }\textstyleDropCaps{по­сле полуночи‎}

\textstyleCaptioncharacters{اِبْنُ الْخَالِ }\textstyleDropCaps{двоюродный брат‎}

\textstyleCaptioncharacters{سَاعَةُ الْمِعْصَمِ }\textstyleDropCaps{наруч­ные часы‎}

\textstyleCaptioncharacters{سَاعَةُ الْجَيْبِ }\textstyleDropCaps{карман­ные часы‎}

\textstyleCaptioncharacters{سَاعَةُ الْجِدَارِ }\textstyleDropCaps{настен­ные часы‎}

\textstyleCaptioncharacters{سَاعَةٌْ مُنَبِّهَةٌ }\textstyleDropCaps{часы-будильник‎}

\textstyleCaptioncharacters{نَفْسُهُ }\textstyleDropCaps{он сам‎}

\textstyleCaptioncharacters{لِنَفْسِهِ }\textstyleDropCaps{себе‎}

\textstyleCaptioncharacters{هُرِعَ }\textstyleDropCaps{поспешил‎}

\textstyleCaptioncharacters{وَدَّ لَوْ...ِ }\textstyleDropCaps{желал, хотел‎}

\textstyleCaptioncharacters{وَدِدْتُ لَوْ... }\textstyleDropCaps{я хотел бы‎}

\textstyleCaptioncharacters{قَصَدَ }\textstyleDropCaps{направился‎}

\textstyleCaptioncharacters{شَدَّمَا... }\textstyleDropCaps{как сильно!‎}

\textstyleCaptioncharacters{اِشْتَاقَ اِلَى... }\textstyleDropCaps{тосковал, скучал по кому-л.‎}

\subsection[Урок 70‎]{\textstyleDropCaps{Урок 70‎}}
\textstyleCaptioncharacters{سِكَّةٌ حَدِيدِيَّةٌ }\textstyleDropCaps{же­лезная дорога‎}

\textstyleCaptioncharacters{مَحَطَّةُ السِّكَّةِ الْحَدِيدِيَّةِ }\textstyleDropCaps{железнодорожная станция‎}

\textstyleCaptioncharacters{قُطْنٌ }\textstyleDropCaps{хлопок‎}

\textstyleCaptioncharacters{قِطَافٌ }\textstyleDropCaps{сбор (урожая)‎}

\textstyleCaptioncharacters{مَاكِنَةُ قِطَافِ الْقُطْنِ }\textstyleDropCaps{хлопкоуборочная машина‎}

\textstyleCaptioncharacters{نَجَاحٌ }\textstyleDropCaps{успех‎}

\textstyleCaptioncharacters{اَتَمَنَّى لَكُمُ النَّجَاحَ }\textstyleDropCaps{же­лаю вам успеха‎}

\textstyleCaptioncharacters{مَصَحٌّّ }\textstyleDropCaps{санаторий‎}

\textstyleCaptioncharacters{مَوْجُودٌ }\textstyleDropCaps{имеющийся, наход­ящийся‎}

\textstyleCaptioncharacters{حَالِيًّا }\textstyleDropCaps{в настоящее время, сейчас‎}

\textstyleCaptioncharacters{تَفْصِيلاً }\textstyleDropCaps{подробно‎}

\textstyleCaptioncharacters{اَخْبَرَ تَفْصِيلاً }\textstyleDropCaps{подробно рассказал‎}

\textstyleCaptioncharacters{كَلِمَةً كَلِمَةً }\textstyleDropCaps{по словам‎}

\textstyleCaptioncharacters{جُمْلَةً جُمْلَةً }\textstyleDropCaps{по предло­жениям‎}

\textstyleCaptioncharacters{شَاهَدَ }\textstyleDropCaps{увидел, смотрел‎}

\textstyleCaptioncharacters{بِكُلِّ اَرْتِيَاحٍ }\textstyleDropCaps{с удоволь­ствием‎}

\textstyleCaptioncharacters{بَاتَ }\textstyleDropCaps{переночевал‎}

\textstyleCaptioncharacters{تَوَجَّهَ }\textstyleDropCaps{направился‎}

\textstyleCaptioncharacters{مِيعَادٌ }\textstyleDropCaps{свидание‎}

\textstyleCaptioncharacters{تَاَخَّرَ عَنِ الْمِيعَادِ }\textstyleDropCaps{опоз­дал на свидание‎}

\textstyleCaptioncharacters{تَكَلَّمَ }\textstyleDropCaps{разговаривал, гово­рил‎}

\textstyleCaptioncharacters{كَفَى التَّكَلُّمُ }\textstyleDropCaps{хватит го­ворить!‎}

\textstyleCaptioncharacters{اِسْتَعْجَلَ }\textstyleDropCaps{торопился‎}

\textstyleCaptioncharacters{فَاتَهُ الْقِطَارُ }\textstyleDropCaps{не успел на поезд, опоздал на поезд‎}

\textstyleCaptioncharacters{حَفْلَةٌ }\textstyleDropCaps{торжество‎}

\textstyleCaptioncharacters{اَلثَّامِنَةُ تَمَامًا }\textstyleDropCaps{ровно во­семь‎}

\textstyleCaptioncharacters{تَوَقَّفَ }\textstyleDropCaps{остановился‎}

\textstyleCaptioncharacters{مُدَّةٌ }\textstyleDropCaps{период, промежуток (времени); срок‎}

\textstyleCaptioncharacters{مُدَّةُ تَوَقُّفِ الْقِطَارِ }\textstyleDropCaps{сто­янка поезда‎}

\textstyleCaptioncharacters{اِسْمَحْ لِى اَنْ اَشْكُرَكَ }\textstyleDropCaps{разрешите мне поблагодарить вас‎}

\textstyleCaptioncharacters{لا شُكْرَ عَلَى وَاجِبٍ }\textstyleDropCaps{не стоит благодарности, не за что‎}

\textstyleCaptioncharacters{دَعْوَةٌ }\textstyleDropCaps{приглашение‎}

\textstyleCaptioncharacters{لَبَّى الدَّعْوَةَ }\textstyleDropCaps{принял при­глашение‎}

\textstyleCaptioncharacters{وَصَلَ الْقِطَارُ }\textstyleDropCaps{поезд при­был‎}

\textstyleCaptioncharacters{لِئَلاَّ... }\textstyleDropCaps{чтобы не...‎}

\textstyleCaptioncharacters{لِكَيْلاَ... }\textstyleDropCaps{чтобы не...‎}

\textstyleCaptioncharacters{تَاَخَّرَ }\textstyleDropCaps{опоздал‎}

\textstyleCaptioncharacters{حَوَالَىْ... }\textstyleDropCaps{около, прибли­зительно, примерно‎}

\textstyleCaptioncharacters{سَيِّدٌ }\textstyleDropCaps{господин‎}

\textstyleCaptioncharacters{يَا سَيِّدِى }\textstyleDropCaps{о мой госпо­дин!‎}

\textstyleCaptioncharacters{كَهَذَا }\textstyleDropCaps{вот такой, наподо­бие этого‎}

\subsection[Урок 71‎]{\textstyleDropCaps{Урок 71‎}}
\textstyleCaptioncharacters{زُجَاجَةٌ }\textstyleDropCaps{бутылка‎}

\textstyleCaptioncharacters{كَأْسٌ }\textstyleDropCaps{кружка, бокал‎}

\textstyleCaptioncharacters{مُرَطِّبَاتٌ }\textstyleDropCaps{прохадитель­ные напитки‎}

\textstyleCaptioncharacters{زُجَاجَةُ الْمُرَطِّبَاتِ }\textstyleDropCaps{бу­тылка с прохладительным напитком‎}

\textstyleCaptioncharacters{عَطِشَ }\textstyleDropCaps{захотелось пить, жаждал‎}

\textstyleCaptioncharacters{قَطُّ }\textstyleDropCaps{никогда, ни разу‎}

\textstyleCaptioncharacters{شَعَرَبِ...ِ }\textstyleDropCaps{почувствовал‎}

\textstyleCaptioncharacters{مَاْكُولاَتٌ شَرْقِيَّةٌ }\textstyleDropCaps{вос­точные кушанья‎}

\textstyleCaptioncharacters{اَلْفُ شُكْرٍ }\textstyleDropCaps{весьма благо­дарен, тысяча благодарностей‎}

\textstyleCaptioncharacters{عَفْوًا }\textstyleDropCaps{пожалуйста (ответ); простите, извините‎}

\textstyleCaptioncharacters{لا اَعْرِفُ كَيْفَ اَشْكُرُكُمْ }\textstyleDropCaps{не знаю, как благодарить вас‎}

\textstyleCaptioncharacters{نَحْوَ اْلإِسْلاَمِ }\textstyleDropCaps{перед Исла­мом‎}

\textstyleCaptioncharacters{قَوَّمَ }\textstyleDropCaps{оценил‎}

\textstyleCaptioncharacters{كِتَابٌ لا يُقَوَّمُ بِثَمَنٍ }\textstyleDropCaps{бесценная, неоценимая книга‎}

\textstyleCaptioncharacters{اَلْقَى سُؤَالاً }\textstyleDropCaps{задал вопрос‎}

\textstyleCaptioncharacters{أَهْلَمَ }\textstyleDropCaps{невнимательно отно­сился, оставил без внимания, не заботился‎}

\textstyleCaptioncharacters{بِلاَجَوَابٍ }\textstyleDropCaps{без ответа‎}

\textstyleCaptioncharacters{أَزْعَجَ }\textstyleDropCaps{беспокоил‎}

\textstyleCaptioncharacters{تَرَكَهُ وَ شَأْنَهُ }\textstyleDropCaps{оставил его в покое, не трогал его‎}

\textstyleCaptioncharacters{تَمَنِّيَاتٌ طَيِّبَةُ }\textstyleDropCaps{хоро­шие пожелания‎}

\textstyleCaptioncharacters{نَصَرَ عَلَى...ِ }\textstyleDropCaps{дал победу‎}

\textstyleCaptioncharacters{نَصَرَهُ عَلَى أَعْدَائِهِ }\textstyleDropCaps{дал ему победу над его врагами‎}

\textstyleCaptioncharacters{عَدُوٌّ }\textstyleDropCaps{враг‎}

\textstyleCaptioncharacters{وَدَّعَ }\textstyleDropCaps{прощался‎}

\textstyleCaptioncharacters{نُوَدِّعُكُمْ }\textstyleDropCaps{прощаемся с вами‎}

\textstyleCaptioncharacters{عَلَى أَمَلِ اللِّقَاءِ }\textstyleDropCaps{в наде­жде на встречу‎}

\textstyleCaptioncharacters{عَنْ قَرِيبٍ }\textstyleDropCaps{скоро, вскоре‎}

\textstyleCaptioncharacters{نَسْتَوْدِعُكُمُ اللَّهَ }\textstyleDropCaps{до сви­дания, вверяем вас Богу‎}

\textstyleCaptioncharacters{رَحْمَةُ اللَّهِ }\textstyleDropCaps{милость Бога‎}

\textstyleCaptioncharacters{السَّلاَمُ عَلَيْكُمْ }\textstyleDropCaps{привет вам, мир вам‎}

\textstyleCaptioncharacters{وَ عَلَيْكُمُ السَّلاَمُ وَ رَحْمَةُ اللَّهِ }\textstyleDropCaps{и вам привет и милость Бога‎}

\textstyleCaptioncharacters{سَأَلَ اللَّهَ }\textstyleDropCaps{просил Бога‎}

\textstyleCaptioncharacters{مَلَأَ }\textstyleDropCaps{наполнил, заполнил‎}

\textstyleCaptioncharacters{مَلَأَ السَّطْلَ مَاءً }\textstyleDropCaps{напол­нил ведро водой‎}

\subsection[Урок 72‎]{\textstyleDropCaps{Урок 72‎}}
\textstyleCaptioncharacters{نُكْتَةٌ }\textstyleDropCaps{анектод; шутка‎}

\textstyleCaptioncharacters{قَصَّ }\textstyleDropCaps{рассказал‎}

\textstyleCaptioncharacters{قَصَّ نُكْتَةً }\textstyleDropCaps{рассказал анекдот‎}

\textstyleCaptioncharacters{نَعَمْ؟ }\textstyleDropCaps{что?‎}

\textstyleCaptioncharacters{مُزَاحٌ }\textstyleDropCaps{шутка‎}

\textstyleCaptioncharacters{جَامِدٌ }\textstyleDropCaps{застывший; нежи­вой‎}

\textstyleCaptioncharacters{مُتَحَجِّرٌ }\textstyleDropCaps{окаменевший; безразличный‎}

\textstyleCaptioncharacters{تَحَدَّثَ }\textstyleDropCaps{поговорил, побе­седовал‎}

\textstyleCaptioncharacters{اَمْرٌ }\textstyleDropCaps{дело, вопрос‎}

\textstyleCaptioncharacters{فِيمَا بَعْدُ }\textstyleDropCaps{после позже, в дальнейшем, потом‎}

\textstyleCaptioncharacters{سَنَتَحَدَّثُ فِى اْلأَمْرِ }\textstyleDropCaps{мы поговорим по этому вопросу‎}

\textstyleCaptioncharacters{حِصَّةٌ }\textstyleDropCaps{урок, учебный час‎}

\textstyleCaptioncharacters{مَا فَعَلْتَهُ }\textstyleDropCaps{то, что ты делал‎}

\textstyleCaptioncharacters{اِعْتَبَرَ }\textstyleDropCaps{считал‎}

\textstyleCaptioncharacters{اِنْتِهَاكٌ }\textstyleDropCaps{нарушение‎}

\textstyleCaptioncharacters{حُرْمَةٌ }\textstyleDropCaps{святость, неприкос­новенность‎}

\textstyleCaptioncharacters{اِنْتِهَاكٌ لِحُرْمَةِ }\textstyleDropCaps{неуваже­ние, оскорбление‎}

\textstyleCaptioncharacters{عِلْمٌ }\textstyleDropCaps{наука; знание‎}

\textstyleCaptioncharacters{تَسْلِيَةٌ }\textstyleDropCaps{развлечение; уте­шение‎}

\textstyleCaptioncharacters{قَصَدَ الْمُزَاحَ }\textstyleDropCaps{он хотел по­шутить‎}

\textstyleCaptioncharacters{مَحَلٌّ }\textstyleDropCaps{магазин‎}

\textstyleCaptioncharacters{يَانَصِيبٌ }\textstyleDropCaps{лотерея‎}

\textstyleCaptioncharacters{وَرَقُ اليَانَصِيبِ }\textstyleDropCaps{ло­терейный билет‎}

\textstyleCaptioncharacters{وَسْوَسَ }\textstyleDropCaps{внушил плохие мысли‎}

\textstyleCaptioncharacters{شَيْطَانٌ }\textstyleDropCaps{сатана, дьявол‎}

\textstyleCaptioncharacters{طَلَبَ }\textstyleDropCaps{попросил; потребо­вал‎}

\textstyleCaptioncharacters{اَخْرَجَ }\textstyleDropCaps{вытащил‎}

\textstyleCaptioncharacters{تَالٍ }\textstyleDropCaps{следующий‎}

\textstyleCaptioncharacters{قَلَقٌ }\textstyleDropCaps{беспокойство‎}

\textstyleCaptioncharacters{سِرُّ }\textstyleDropCaps{секрет, причина‎}

\textstyleCaptioncharacters{مَا سِرُّ ذَلِكَ؟ }\textstyleDropCaps{в чем се­крет этого?‎}

\textstyleCaptioncharacters{نَتِيجَةٌ }\textstyleDropCaps{результат‎}

\textstyleCaptioncharacters{ظَهَرَ }\textstyleDropCaps{появился‎}

\textstyleCaptioncharacters{مَاذَا فِى الأَمْرِ؟ }\textstyleDropCaps{что тут такое?‎}

\textstyleCaptioncharacters{حَرَامٌ }\textstyleDropCaps{запрещённый, запретный‎}

\textstyleCaptioncharacters{نَوْعٌ }\textstyleDropCaps{вид, разновидность‎}

\textstyleCaptioncharacters{قِمَارٌ }\textstyleDropCaps{азартная игра‎}

\textstyleCaptioncharacters{زُقَاقٌ }\textstyleDropCaps{улица, переулок‎}

\textstyleCaptioncharacters{بَيْعٌ }\textstyleDropCaps{продажа‎}

\textstyleCaptioncharacters{جَائِزٌ }\textstyleDropCaps{разрешённый, поз­воленный, который можно‎}

\textstyleCaptioncharacters{غَيْرُ جَائِزٍ }\textstyleDropCaps{непозволен­ный, запрещённый, который нельзя‎}

\subsection[Урок 73‎]{\textstyleDropCaps{Урок 73‎}}
\textstyleCaptioncharacters{نَجَحَ }\textstyleDropCaps{преуспел, имел успех‎}

\textstyleCaptioncharacters{حَتْمًا }\textstyleDropCaps{безусловно, обяза­тельно, бесповортотно‎}

\textstyleCaptioncharacters{مَغْرُورٌ }\textstyleDropCaps{обманутый, тще­славный, самодовольный‎}

\textstyleCaptioncharacters{غُرُورٌ }\textstyleDropCaps{тщеславие‎}

\textstyleCaptioncharacters{أَعْمَى }\textstyleDropCaps{ослепил‎}

\textstyleCaptioncharacters{بَصِيرَةٌ }\textstyleDropCaps{разум‎}

\textstyleCaptioncharacters{حَقِيقَةٌ }\textstyleDropCaps{истина, действи­тельность, сущность‎}

\textstyleCaptioncharacters{سَلَبَ }\textstyleDropCaps{отобрал, отнял‎}

\textstyleCaptioncharacters{جَعَلَ }\textstyleDropCaps{сделал, положил‎}

\textstyleCaptioncharacters{زُمْرَةٌ }\textstyleDropCaps{группа‎}

\textstyleCaptioncharacters{فَاشِلٌ }\textstyleDropCaps{провалившийся, по­терпевший неудачу, неудачный‎}

\textstyleCaptioncharacters{قَرِيبٌ }\textstyleDropCaps{скоро‎}

\textstyleCaptioncharacters{رَأَى بِأُمِّ عَيْنَيْهِ }\textstyleDropCaps{видел собственными глазами‎}

\textstyleCaptioncharacters{صِدْقٌ }\textstyleDropCaps{истинность, прав­дивость‎}

\textstyleCaptioncharacters{بَقِىَ }\textstyleDropCaps{остался‎}

\textstyleCaptioncharacters{طَرِيحُ الْفِرَاشِ }\textstyleDropCaps{прико­ванный к постели‎}

\textstyleCaptioncharacters{ضَاعَ }\textstyleDropCaps{пропал‎}

\textstyleCaptioncharacters{أَدْرَكَ }\textstyleDropCaps{понял, постиг‎}

\textstyleCaptioncharacters{خَطَأٌ }\textstyleDropCaps{ошибка, ошибоч­ность‎}

\textstyleCaptioncharacters{تَفْكِيرٌ }\textstyleDropCaps{мышление‎}

\textstyleCaptioncharacters{مَرَضٌ }\textstyleDropCaps{болезнь‎}

\textstyleCaptioncharacters{قَاتِلٌ }\textstyleDropCaps{убивающий, смер­тельный‎}

\textstyleCaptioncharacters{يَجِبُ عَلَى... }\textstyleDropCaps{должен, обязан‎}

\textstyleCaptioncharacters{تَخَلَّصَ مِنْ...ِ }\textstyleDropCaps{избавил­ся, освободился от чего-либо‎}

\textstyleCaptioncharacters{رَدَّدَ }\textstyleDropCaps{повторял‎}

\textstyleCaptioncharacters{يَا لَلْخَسَارَةِ }\textstyleDropCaps{какая поте­ря!‎}

\textstyleCaptioncharacters{يَا لَلأَسَفِ! }\textstyleDropCaps{как жаль!‎}

\textstyleCaptioncharacters{مُتَأَسِّفٌ! }\textstyleDropCaps{сожалеющий; из­вините‎}

\textstyleCaptioncharacters{اَلْعَفْوَ }\textstyleDropCaps{пожалуйста‎}

\textstyleCaptioncharacters{اَنَا مُتَأَسِّفٌ عَلَى اْلإِزْعَاجِ }\textstyleDropCaps{извините за беспокойсто‎}

\textstyleCaptioncharacters{مِنْ كُلِّ بُدٍّ }\textstyleDropCaps{обязательно, непременно‎}

\textstyleCaptioncharacters{مَا زَالَ يَقُولُ }\textstyleDropCaps{он все го­ворил‎}

\textstyleCaptioncharacters{فَاِنَّ...ِ }\textstyleDropCaps{потому что, ибо‎}

\subsection[Урок 74‎]{\textstyleDropCaps{Урок 74‎}}
\textstyleCaptioncharacters{سَكَتَ }\textstyleDropCaps{молчал‎}

\textstyleCaptioncharacters{مَادَامَ }\textstyleDropCaps{пока он‎}

\textstyleCaptioncharacters{اِنْصَرَفَ }\textstyleDropCaps{ушел‎}

\textstyleCaptioncharacters{فِى سَاعَةٍ مُتَأَخِّرَةٍ مِنَ اللَّيْلِ }\textstyleDropCaps{поздно ночью‎}

\textstyleCaptioncharacters{تَلْفَنَ }\textstyleDropCaps{звонил, говорил по телефону‎}

\textstyleCaptioncharacters{طَالَمَا }\textstyleDropCaps{долго‎}

\textstyleCaptioncharacters{غَرَّ }\textstyleDropCaps{обманул, обольстил‎}

\textstyleCaptioncharacters{شُيُوعِيَّةٌ }\textstyleDropCaps{коммунизм‎}

\textstyleCaptioncharacters{اِنْكَشَفَ }\textstyleDropCaps{открылся, обна­ружился‎}

\textstyleCaptioncharacters{أَسْوَأُ }\textstyleDropCaps{самый плохой, наи­худший‎}

\textstyleCaptioncharacters{نِظَامٌ }\textstyleDropCaps{строй, система‎}

\textstyleCaptioncharacters{اِخْتَرَعَ }\textstyleDropCaps{изобрел, придумал‎}

\textstyleCaptioncharacters{هَوًى }\textstyleDropCaps{прихоть, каприз‎}

\textstyleCaptioncharacters{بَشَرِيٌّ }\textstyleDropCaps{человеческий‎}

\textstyleCaptioncharacters{أَحْسَنُ }\textstyleDropCaps{лучший, наилуч­ший‎}

\textstyleCaptioncharacters{نَازِيَّةٌ }\textstyleDropCaps{нацизм‎}

\textstyleCaptioncharacters{فَاشِيَّةٌ }\textstyleDropCaps{фашизм‎}

\textstyleCaptioncharacters{اِصْلاَحٌ }\textstyleDropCaps{улучшение‎}

\textstyleCaptioncharacters{خَالَفَ }\textstyleDropCaps{противоречил‎}

\textstyleCaptioncharacters{تَصَرُّفٌ }\textstyleDropCaps{поведение, дей­ствие, распоряжение‎}

\textstyleCaptioncharacters{نَدِمَ عَلَى... }\textstyleDropCaps{пожалел о чем-либо‎}

\textstyleCaptioncharacters{وَبَالٌ }\textstyleDropCaps{беда, вред, тяжелые последствия‎}

\textstyleCaptioncharacters{كَانَ وَبَالاً لَيْهِمْ }\textstyleDropCaps{был па­губным для них‎}

\textstyleCaptioncharacters{فِى آخِرِ اْلأَمْرِ }\textstyleDropCaps{в конце концов‎}

\textstyleCaptioncharacters{رَئِيسٌ }\textstyleDropCaps{глава, главарь‎}

\textstyleCaptioncharacters{كُفْرٌ }\textstyleDropCaps{неверие, безбожие‎}

\textstyleCaptioncharacters{ذَمَّ }\textstyleDropCaps{порицал, хулил‎}

\textstyleCaptioncharacters{سَابِقٌ }\textstyleDropCaps{предыдущий, пред­шествующий‎}

\textstyleCaptioncharacters{لاَحِقٌ }\textstyleDropCaps{последующий‎}

\textstyleCaptioncharacters{قَلَبَ }\textstyleDropCaps{перевернул‎}

\textstyleCaptioncharacters{رَأْسًا عَلَى عَقِبٍ }\textstyleDropCaps{вверх дном‎}

\textstyleCaptioncharacters{مُسْتَعْرِبٌ }\textstyleDropCaps{арабист‎}

\textstyleCaptioncharacters{تَارِيخٌ }\textstyleDropCaps{история‎}

\textstyleCaptioncharacters{بُلْدَانٌ }\textstyleDropCaps{страны‎}

\textstyleCaptioncharacters{جُغْرَافِيَا }\textstyleDropCaps{география‎}

\textstyleCaptioncharacters{كَافِرٌ }\textstyleDropCaps{гяур, неверующий‎}

\textstyleCaptioncharacters{كَانَ قَدْ ذَهَبَ }\textstyleDropCaps{он уже ушел‎}

\textstyleCaptioncharacters{يَكُونُ قَدْ ذَهَبَ }\textstyleDropCaps{он уже уйдет‎}

\textstyleCaptioncharacters{لَيْسَ اَحَدٌ مِنْهُمْ اِلاَّ وَ يَعْرِفُ }\textstyleDropCaps{нет никого из них (ни один) который не знает‎}

\textstyleCaptioncharacters{لَيْسَ اَحَدٌ مِنْ بُلْدَانِ الْعَالَمِ اِلاَّ وَ فِيهِ الْمُسْلِمُونَ }\textstyleDropCaps{нет ни одной страны в мире, где бы не было мусульман‎}

\subsection[Урок 75‎]{\textstyleDropCaps{Урок 75‎}}
\textstyleCaptioncharacters{مِسَاحَةٌ }\textstyleDropCaps{площадь, терри­тория‎}

\textstyleCaptioncharacters{اَكْثَرُ }\textstyleDropCaps{больше, более‎}

\textstyleCaptioncharacters{اَكْثَرُ مِنْ مِائَةٍ }\textstyleDropCaps{более ста‎}

\textstyleCaptioncharacters{مُرَبَّعٌ }\textstyleDropCaps{квадратный‎}

\textstyleCaptioncharacters{اَلإِتِّحَادُ السُّوفْيَيْتِىُّ }\textstyleDropCaps{Советский Союз‎}

\textstyleCaptioncharacters{طُولٌ }\textstyleDropCaps{длина‎}

\textstyleCaptioncharacters{كِيلُومَتْرٌ }\textstyleDropCaps{километр‎}

\textstyleCaptioncharacters{مَسَافَةٌ }\textstyleDropCaps{расстояние‎}

\textstyleCaptioncharacters{سَدٌّ }\textstyleDropCaps{плотина‎}

\textstyleCaptioncharacters{سَدُّ أَسْوَانَ }\textstyleDropCaps{Асуанская плотина‎}

\textstyleCaptioncharacters{غَرْبِىٌّ }\textstyleDropCaps{западный‎}

\textstyleCaptioncharacters{حَدٌّ }\textstyleDropCaps{граница‎}

\textstyleCaptioncharacters{أَهَمُّ }\textstyleDropCaps{самый важный, важ­нейший‎}

\textstyleCaptioncharacters{فَحْمٌ حَجَرِىٌّ }\textstyleDropCaps{каменный уголь‎}

\textstyleCaptioncharacters{نِفْطٌ }\textstyleDropCaps{нефть‎}

\textstyleCaptioncharacters{طَبِيعِىٌّ }\textstyleDropCaps{природный‎}

\textstyleCaptioncharacters{غَازٌ طَبِيعِىٌّ }\textstyleDropCaps{природный газ‎}

\textstyleCaptioncharacters{ثَرْوَةٌ }\textstyleDropCaps{богатство‎}

\textstyleCaptioncharacters{ثَرَوَاتٌ طَبِيعِيَّةٌ }\textstyleDropCaps{природн­ое богатсво‎}

\textstyleCaptioncharacters{مَعْدِنٌ }\textstyleDropCaps{минерал; металл‎}

\textstyleCaptioncharacters{أَطْوَلُ }\textstyleDropCaps{длиннее; самый длинный‎}

\textstyleCaptioncharacters{نَهْرُ النِّيلِ }\textstyleDropCaps{река Нил‎}

\textstyleCaptioncharacters{نَهْرُ الْفُولْغَا }\textstyleDropCaps{река Волга‎}

\textstyleCaptioncharacters{يَقُومُ }\textstyleDropCaps{стоит‎}

\textstyleCaptioncharacters{مِنَ الْمُؤْسِفِ اَنَّ...ِ }\textstyleDropCaps{жаль, огорчительно, что...‎}

\textstyleCaptioncharacters{كَلاَمٌ سَلِيمٌ }\textstyleDropCaps{верно, со­вершенно верно‎}

\subsection[Урок 76‎]{\textstyleDropCaps{Урок 76‎}}
\textstyleCaptioncharacters{اِسْمَحْ بِالتِّلِفُونِ }\textstyleDropCaps{можно к телефону? разрешите позвонить по телефону?‎}

\textstyleCaptioncharacters{تَفَضَّلْ }\textstyleDropCaps{пожалуйста‎}

\textstyleCaptioncharacters{مَبْرُوكْ }\textstyleDropCaps{поздравляю!‎}

\textstyleCaptioncharacters{اَللَّهُ يُبَارِكُ فِيكَ }\textstyleDropCaps{да благословит тебя Бог!‎}

\textstyleCaptioncharacters{بَلَغَنِى }\textstyleDropCaps{дошло до меня, я получил известие‎}

\textstyleCaptioncharacters{سَلاَمَةٌ }\textstyleDropCaps{благополучие‎}

\textstyleCaptioncharacters{وُصُولٌ }\textstyleDropCaps{приезд, прибытие‎}

\textstyleCaptioncharacters{سَلاَمَةُ الْوُصُولِ }\textstyleDropCaps{благо­получный приезд‎}

\textstyleCaptioncharacters{أُهَنِّئُكَ عَلَى سَلاَمَةِ الْوُصُولِ }\textstyleDropCaps{поздравляю тебя с благополучным приездом‎}

\textstyleCaptioncharacters{يَا بُنَيَّ }\textstyleDropCaps{о сынок!‎}

\textstyleCaptioncharacters{طَقْسٌ }\textstyleDropCaps{погода; климат‎}

\textstyleCaptioncharacters{رَدِىءٌ }\textstyleDropCaps{плохой, нехоро­ший‎}

\textstyleCaptioncharacters{جَوْلَةٌ }\textstyleDropCaps{путешествие, поезд­ка, экскурсия‎}

\textstyleCaptioncharacters{عَامٌ }\textstyleDropCaps{год‎}

\textstyleCaptioncharacters{قَضَاءٌ }\textstyleDropCaps{проводить‎}

\textstyleCaptioncharacters{صَيْفِىٌّ }\textstyleDropCaps{летний‎}

\textstyleCaptioncharacters{عُطْلَةٌ صَيْفِيَّةٌ }\textstyleDropCaps{летние ка­никулы‎}

\textstyleCaptioncharacters{سَاحِلٌ }\textstyleDropCaps{берег, побережье‎}

\textstyleCaptioncharacters{حَيْثُ...ِ }\textstyleDropCaps{там, где...‎}

\textstyleCaptioncharacters{وَالِدَانِ }\textstyleDropCaps{родители‎}

\textstyleCaptioncharacters{قَارِئٌ }\textstyleDropCaps{чтец, читатель‎}

\textstyleCaptioncharacters{مُجِيدٌ }\textstyleDropCaps{умелый, хороший, хорошо знающий‎}

\textstyleCaptioncharacters{مُنْذُ وَقْتٍ قَرِيبٍ }\textstyleDropCaps{недавн­о‎}

\textstyleCaptioncharacters{تَرْتِيلٌ }\textstyleDropCaps{пение, чтение на распев‎}

\textstyleCaptioncharacters{رَتَّلَ الْقُرْآنَ }\textstyleDropCaps{читал Коран нараспев‎}

\textstyleCaptioncharacters{أَصْغَى اِلَى...ِ }\textstyleDropCaps{послушал‎}

\textstyleCaptioncharacters{تِلاَوَةٌ }\textstyleDropCaps{чтение‎}

\textstyleCaptioncharacters{طَبْعًا }\textstyleDropCaps{конечно, разумеется‎}

\textstyleCaptioncharacters{لِمَ لاَ؟ }\textstyleDropCaps{почему бы нет?‎}

\textstyleCaptioncharacters{عَالٍ!ِ }\textstyleDropCaps{отлично!‎}

\textstyleCaptioncharacters{وَفَّقَ }\textstyleDropCaps{помогал, содейство­вал‎}

\textstyleCaptioncharacters{وَفَّقَكَ اللَّهُ!ِ }\textstyleDropCaps{да поможет тебе Бог!‎}

\textstyleCaptioncharacters{حَتَّى اْلآنَ }\textstyleDropCaps{до сих пор‎}

\textstyleCaptioncharacters{لَوْ...ِ }\textstyleDropCaps{если бы...‎}

\textstyleCaptioncharacters{لَوْ عَرَفْتُ }\textstyleDropCaps{если бы я знал‎}

\textstyleCaptioncharacters{كَمَا قُلْتُ لَكَ }\textstyleDropCaps{как я тебе говорил‎}

\textstyleCaptioncharacters{رَجَانِى }\textstyleDropCaps{он меня просил‎}

\textstyleCaptioncharacters{أَرْجُوكَ }\textstyleDropCaps{я прошу тебя‎}

\textstyleCaptioncharacters{اللَّهُ يُسَلِّمُكَ }\textstyleDropCaps{да сохранит тебя Бог!‎}

\subsection[Урок 77‎]{\textstyleDropCaps{Урок 77‎}}
\textstyleCaptioncharacters{زَوْرَقٌ }\textstyleDropCaps{лодка‎}

\textstyleCaptioncharacters{سَاحَةُ التَّزَلُّجِ }\textstyleDropCaps{каток‎}

\textstyleCaptioncharacters{طَنْجَرَةٌ }\textstyleDropCaps{кастрюля‎}

\textstyleCaptioncharacters{سَاحَةٌ }\textstyleDropCaps{площадь‎}

\textstyleCaptioncharacters{هُتَافٌ }\textstyleDropCaps{крик, возглас, восклицание‎}

\textstyleCaptioncharacters{عَاشَ الإِسْلاَمُ }\textstyleDropCaps{Да здрав­ствует Ислам!‎}

\textstyleCaptioncharacters{لاَفِتَةٌ }\textstyleDropCaps{плакат, транспо­рант, вывеска‎}

\textstyleCaptioncharacters{قَامَتْ دَوْلَةُ الْمُسْلِمِينَ }\textstyleDropCaps{Да образуется, установится государство мусульман‎}

\textstyleCaptioncharacters{لِيَسْقُطْ }\textstyleDropCaps{долой! да подёт!‎}

\textstyleCaptioncharacters{اِسْتِعْمَارٌ }\textstyleDropCaps{колониализм‎}

\textstyleCaptioncharacters{اِمْبِرِيَالِيَّةٌ }\textstyleDropCaps{имериализм‎}

\textstyleCaptioncharacters{وَ عَلَى رَأْسِهَا }\textstyleDropCaps{во главе с...‎}

\textstyleCaptioncharacters{اُتْرُكْ هَذَا الأَمْرَ اِلَىَّ }\textstyleDropCaps{оставь это дело мне‎}

\textstyleCaptioncharacters{عَصِيدَةٌ }\textstyleDropCaps{каша‎}

\textstyleCaptioncharacters{عَصِيدَةٌ بِالَّبَنِ }\textstyleDropCaps{каша на молоке‎}

\textstyleCaptioncharacters{تَنَزَّهَ }\textstyleDropCaps{гулял; катался‎}

\textstyleCaptioncharacters{تَنَزَّهَ فِى الزَّوْرَقِ }\textstyleDropCaps{пока­тался на лодке‎}

\textstyleCaptioncharacters{قَسَّمَ }\textstyleDropCaps{делил, распределил‎}

\textstyleCaptioncharacters{فِئَةٌ }\textstyleDropCaps{группа‎}

\textstyleCaptioncharacters{حَسَبَ...ِ }\textstyleDropCaps{по..., согласно...‎}

\textstyleCaptioncharacters{عَلَى حِدَةٍ }\textstyleDropCaps{отдельно‎}

\textstyleCaptioncharacters{اِشْرَافٌ }\textstyleDropCaps{руководство, на­блюдение‎}

\textstyleCaptioncharacters{تَحْتَ اِشْرَافِ }\textstyleDropCaps{под руко­водством‎}

\textstyleCaptioncharacters{مُدَرِّبٌ }\textstyleDropCaps{тренер‎}

\textstyleCaptioncharacters{اِنْتَقَلَ }\textstyleDropCaps{перешёл, переехал‎}

\textstyleCaptioncharacters{حَىٌّ }\textstyleDropCaps{квартал, район‎}

\textstyleCaptioncharacters{مُجَاوِرٌ }\textstyleDropCaps{соседний‎}

\textstyleCaptioncharacters{مَا لاَ يَقِلُّ عَنْ...ِ }\textstyleDropCaps{не ме­нее...‎}

\subsection[Урок 78‎]{\textstyleDropCaps{Урок 78‎}}
\textstyleCaptioncharacters{كَفٌّ }\textstyleDropCaps{кисть руки‎}

\textstyleCaptioncharacters{اِصْبَعٌ }\textstyleDropCaps{палец‎}

\textstyleCaptioncharacters{ظُفْرٌ }\textstyleDropCaps{ноготь‎}

\textstyleCaptioncharacters{فَمٌ }\textstyleDropCaps{рот‎}

\textstyleCaptioncharacters{لِسَانٌ }\textstyleDropCaps{язык‎}

\textstyleCaptioncharacters{اُذُنٌ }\textstyleDropCaps{ухо‎}

\textstyleCaptioncharacters{كَذَبَ }\textstyleDropCaps{соврал, солгал‎}

\textstyleCaptioncharacters{حَفِظَ لِسَانَهُ }\textstyleDropCaps{берег свой язык‎}

\textstyleCaptioncharacters{تَكَلَّمَ }\textstyleDropCaps{говорил, разговари­вал‎}

\textstyleCaptioncharacters{مُسْتَحِيلٌ! }\textstyleDropCaps{не может быть! невозможно!‎}

\textstyleCaptioncharacters{تَغَيَّرَ }\textstyleDropCaps{изменился‎}

\textstyleCaptioncharacters{صَرَفَ }\textstyleDropCaps{тратил, расходовал‎}

\textstyleCaptioncharacters{صَرَفَ الْوَقْتَ }\textstyleDropCaps{тратил вре­мя‎}

\textstyleCaptioncharacters{مُعْظَمٌ }\textstyleDropCaps{большинство, большая часть‎}

\textstyleCaptioncharacters{لَهْوٌ }\textstyleDropCaps{развлечение, забава‎}

\textstyleCaptioncharacters{لَعِبٌ }\textstyleDropCaps{игра, забава, шутка‎}

\textstyleCaptioncharacters{تَمَامًا }\textstyleDropCaps{полностью‎}

\textstyleCaptioncharacters{عَلَى مَا يَبْدُو }\textstyleDropCaps{по-види­мому, как видно‎}

\textstyleCaptioncharacters{غَرِيبٌ!ِ }\textstyleDropCaps{странно!‎}

\textstyleCaptioncharacters{أَسَاءَ اِلَى...ِ }\textstyleDropCaps{плохо обра­тился, обидел‎}

\textstyleCaptioncharacters{تَوَقَّعَ }\textstyleDropCaps{ожидал‎}

\textstyleCaptioncharacters{فَضْلٌ }\textstyleDropCaps{милость, одолжение‎}

\textstyleCaptioncharacters{عَيْبٌ عَلَيْكَ }\textstyleDropCaps{стыдно тебе‎}

\textstyleCaptioncharacters{يُمْكِنُ }\textstyleDropCaps{можно, возможно‎}

\textstyleCaptioncharacters{كَيْفَ يُمْكِنُ؟ }\textstyleDropCaps{как мож­но?‎}

\textstyleCaptioncharacters{مِثْلٌ \ }\textstyleDropCaps{такой как, подоб­ный, наподобие‎}

\textstyleCaptioncharacters{مِثْلُ هَذَا }\textstyleDropCaps{такой, как этот; наподобие этого‎}

\textstyleCaptioncharacters{دَخَّنَ }\textstyleDropCaps{курил‎}

\textstyleCaptioncharacters{مَمْنُوعٌ }\textstyleDropCaps{запрещенный‎}

\textstyleCaptioncharacters{شَرَعَ فِى...ِ }\textstyleDropCaps{начал, при­ступил‎}

\textstyleCaptioncharacters{اَهْلٌ }\textstyleDropCaps{жители‎}

\textstyleCaptioncharacters{اَهْلُ الْقَرْيَةِ }\textstyleDropCaps{жители се­ления‎}

\textstyleCaptioncharacters{بِنَاءٌ }\textstyleDropCaps{постройка‎}

\textstyleCaptioncharacters{عَلَى اَثَرِ... }\textstyleDropCaps{вслед за..., сра­зу после...‎}

\textstyleCaptioncharacters{اِجْتِمَاعٌ }\textstyleDropCaps{собрание‎}

\textstyleCaptioncharacters{بِشَأْنِ...ِ }\textstyleDropCaps{по поводу..., по вопросу...‎}

\textstyleCaptioncharacters{عَقَدَ الإِجْتِمَاعَ }\textstyleDropCaps{созвал собрание‎}

\textstyleCaptioncharacters{كَسُولٌ }\textstyleDropCaps{ленивый‎}

\textstyleCaptioncharacters{فِيمَا اَعْلَمُ }\textstyleDropCaps{насколько я знаю‎}

\subsection[Урок 79‎]{\textstyleDropCaps{Урок 79‎}}
\textstyleCaptioncharacters{مَضَى عَلَ...ِ }\textstyleDropCaps{прошло с тех пор, как...‎}

\textstyleCaptioncharacters{مَضَى عَلَى مَا مَاتَ }\textstyleDropCaps{прошло с тех пор, как он умер‎}

\textstyleCaptioncharacters{مَا يَزِيدُ عَلَى مِائَةٍ }\textstyleDropCaps{бо­лее ста‎}

\textstyleCaptioncharacters{اِفْتَرَقَ }\textstyleDropCaps{расстался, разлу­чился‎}

\textstyleCaptioncharacters{مَا اَسْرَعَ مَا... }\textstyleDropCaps{как бы­стро‎}

\textstyleCaptioncharacters{صَالِحٌ }\textstyleDropCaps{добрый, благочестив­ый‎}

\textstyleCaptioncharacters{مُحْسِنٌ اِلَى... }\textstyleDropCaps{благоде­тельный, делающий добро‎}

\textstyleCaptioncharacters{يَغْفِرُ اللَّهُ لَهُ }\textstyleDropCaps{да простит его Бог‎}

\textstyleCaptioncharacters{هُوَ عَلَى قَيْدِ الْحَيَاةِ }\textstyleDropCaps{он еще жив‎}

\textstyleCaptioncharacters{دَرَّسَ }\textstyleDropCaps{преподавал, давал уроки‎}

\textstyleCaptioncharacters{اَكْثَرُهُمْ }\textstyleDropCaps{большинство, большая часть их‎}

\textstyleCaptioncharacters{بِمَا فِى ذَلِكَ }\textstyleDropCaps{в том числе‎}

\textstyleCaptioncharacters{يَا خَسَارَه!ِ }\textstyleDropCaps{какая потеря!‎}

\textstyleCaptioncharacters{يَا مُصِيبَه!ِ }\textstyleDropCaps{какая беда!что за напасть!‎}

\textstyleCaptioncharacters{عُنْفُوَانُ الشَّبَابِ }\textstyleDropCaps{рас­цвет молодости‎}

\textstyleCaptioncharacters{فَارَقَ الْحَيَاةَ }\textstyleDropCaps{умер, расстался с жизнью‎}

\textstyleCaptioncharacters{اِنْتَقَلَ اِلَى جِوَارِ رَبِّهِ }\textstyleDropCaps{умер (перешел на соседство к своему Господу)‎}

\textstyleCaptioncharacters{رَحِمَهُ اللَّهُ }\textstyleDropCaps{да смилуется над ним Бог‎}

\textstyleCaptioncharacters{مَوْتٌ }\textstyleDropCaps{смерть‎}

\textstyleCaptioncharacters{حَتْمٌ }\textstyleDropCaps{неизбежный, неми­нуемый‎}

\textstyleCaptioncharacters{صَبَرَ }\textstyleDropCaps{терпел‎}

\textstyleCaptioncharacters{شُدَّ حَيْلَكَ }\textstyleDropCaps{держись, кре­пись‎}

\textstyleCaptioncharacters{اِسْتَسْلَمَ }\textstyleDropCaps{сдался‎}

\textstyleCaptioncharacters{نَصْرٌ }\textstyleDropCaps{победа‎}

\textstyleCaptioncharacters{صَابِرٌ }\textstyleDropCaps{терпеливый‎}

\textstyleCaptioncharacters{أَجْزَاخَانَةٌ }\textstyleDropCaps{аптека‎}

\textstyleCaptioncharacters{مِنْ قَبْلُ }\textstyleDropCaps{раньше, прежде‎}

\textstyleCaptioncharacters{عَادَ الْمَرِيضَ }\textstyleDropCaps{посетил больного‎}

\textstyleCaptioncharacters{مُسْتَشْفًى }\textstyleDropCaps{больница‎}

\textstyleCaptioncharacters{اَلشِّفَاءَ الْعَاجِلَ }\textstyleDropCaps{ско­рейшего выздоровления!‎}

\textstyleCaptioncharacters{مُجَاهِدٌ }\textstyleDropCaps{муджахид, борец за Ислам‎}

\textstyleCaptioncharacters{حَادِثٌ }\textstyleDropCaps{происшествие, со­бытие‎}

\textstyleCaptioncharacters{حَادِثُ طَرِيقٍ }\textstyleDropCaps{дорож­ное происшествие‎}

\textstyleCaptioncharacters{فِى سَبِيلِ اللَّهِ }\textstyleDropCaps{на пути Бога, за дело Бога‎}

\subsection[Урок 80‎]{\textstyleDropCaps{Урок 80‎}}
\textstyleCaptioncharacters{اَثْبَتَ }\textstyleDropCaps{доказал, подтвердил‎}

\textstyleCaptioncharacters{زَمَنٌ }\textstyleDropCaps{время‎}

\textstyleCaptioncharacters{فِى الْعَالَمِ بِأَسْرِهِ }\textstyleDropCaps{во всем мире‎}

\textstyleCaptioncharacters{دُوَيْلَةٌ }\textstyleDropCaps{маленькое, не­большое государство‎}

\textstyleCaptioncharacters{اِتَّحَدَ }\textstyleDropCaps{объединил, соеди­нил‎}

\textstyleCaptioncharacters{كَوَّنَ }\textstyleDropCaps{создал, образовал‎}

\textstyleCaptioncharacters{دَوْلَةٌ اِسْلاَمِيَّةٌ عُظْمَى }\textstyleDropCaps{великое исламское государство‎}

\textstyleCaptioncharacters{مُنَظَّمَةٌ }\textstyleDropCaps{организация‎}

\textstyleCaptioncharacters{الإِخْوَانُ الْمُسْلِمُونَ }\textstyleDropCaps{"Братья-мусульмане"‎}

\textstyleCaptioncharacters{قِيَادَةٌ }\textstyleDropCaps{руководство‎}

\textstyleCaptioncharacters{تَحْتَ قِيَادَةِ...ِ }\textstyleDropCaps{под руко­водством, под предводительством‎}

\textstyleCaptioncharacters{اِمَامٌ }\textstyleDropCaps{имам, вождь‎}

\textstyleCaptioncharacters{ثَوْرَةٌ }\textstyleDropCaps{революция‎}

\textstyleCaptioncharacters{نَهْضَةٌ }\textstyleDropCaps{возрождение, подъ­ем‎}

\textstyleCaptioncharacters{نَهْضَةٌ اِسْلاَمِيَّةٌ }\textstyleDropCaps{ислам­ское возрождение‎}

\textstyleCaptioncharacters{ثَوْرَةٌ اِسْلاَمِيَّةٌ }\textstyleDropCaps{ислам­ская революция‎}

\textstyleCaptioncharacters{سَبَبٌ }\textstyleDropCaps{причина‎}

\textstyleCaptioncharacters{أَشْعَلَ }\textstyleDropCaps{зажег; включил‎}

\textstyleCaptioncharacters{عَصْرٌ }\textstyleDropCaps{век, эпоха, время‎}

\textstyleCaptioncharacters{فِى عَصْرِنَا هَذَا }\textstyleDropCaps{в на­шем веке‎}

\textstyleCaptioncharacters{ضَوْءٌ }\textstyleDropCaps{свет‎}

\textstyleCaptioncharacters{أَشْعَلَ الضَّوْءَ }\textstyleDropCaps{включил свет‎}

\textstyleCaptioncharacters{عَكَفَ عَلَى...ِ }\textstyleDropCaps{упорно за­нимался‎}

\textstyleCaptioncharacters{حَتَّى النَّوْمِ }\textstyleDropCaps{до сна‎}

\textstyleCaptioncharacters{قَبْلَ النَّوْمِ }\textstyleDropCaps{перед сном‎}

\textstyleCaptioncharacters{أَطْفَأَ }\textstyleDropCaps{потушил; выключил‎}

\textstyleCaptioncharacters{أَحْسَنَ }\textstyleDropCaps{хорошо умел, знал; хорошо делал‎}

\textstyleCaptioncharacters{أَحْسَنَ السِّبَاحَةَ }\textstyleDropCaps{хоро­шо плавал‎}

\textstyleCaptioncharacters{أَحْسَنَ اللُّغَةَ }\textstyleDropCaps{хорошо знал язык‎}

\textstyleCaptioncharacters{فِيمَا بَيْنَهُمْ }\textstyleDropCaps{между со­бой‎}

\textstyleCaptioncharacters{أَصْلاً }\textstyleDropCaps{совсем, вовсе‎}

\textstyleCaptioncharacters{جُبْنَةٌ }\textstyleDropCaps{кусок сыра‎}

\textstyleCaptioncharacters{خَوْفًا مِنْهُ }\textstyleDropCaps{побоявшись его, из страха перед ним‎}

\textstyleCaptioncharacters{تَسَلَّقَ }\textstyleDropCaps{забрался, вскораб­кался‎}

\textstyleCaptioncharacters{لَيْتَ }\textstyleDropCaps{о, если бы...‎}

\textstyleCaptioncharacters{صَحِيحٌ }\textstyleDropCaps{здоровый‎}

\textstyleCaptioncharacters{جَاهَدَ }\textstyleDropCaps{делал джихад, бо­ролся за Ислам‎}

\textstyleCaptioncharacters{قَوِىٌّ }\textstyleDropCaps{сильный‎}

\textstyleCaptioncharacters{صَحْوَةٌ }\textstyleDropCaps{пробуждение‎}

\textstyleCaptioncharacters{صَحْوَةُ الشَّبَابِ }\textstyleDropCaps{пробу­ждение молодежи‎}

\textstyleCaptioncharacters{يَلِيهِ }\textstyleDropCaps{следует за ним‎}

\textstyleCaptioncharacters{يَلِى هَذَا }\textstyleDropCaps{следует за этим‎}

\subsection[Урок 81‎]{\textstyleDropCaps{Урок 81‎}}
\textstyleCaptioncharacters{مُنْتَبِهٌ }\textstyleDropCaps{внимательный‎}

\textstyleCaptioncharacters{اَحْسَنَتَ!ِ }\textstyleDropCaps{браво! моло­дец!‎}

\textstyleCaptioncharacters{زُجَاجٌ }\textstyleDropCaps{стекло‎}

\textstyleCaptioncharacters{لَوْحُ الزُّجَاجِ }\textstyleDropCaps{лист стекла‎}

\textstyleCaptioncharacters{خَطَأً }\textstyleDropCaps{по ошибке, случай­но, нечаянно‎}

\textstyleCaptioncharacters{عَنَّفَ }\textstyleDropCaps{бранил, ругал‎}

\textstyleCaptioncharacters{بِبُطْءٍ }\textstyleDropCaps{медленно‎}

\textstyleCaptioncharacters{فِى طَرِيقِهِ اِلَى...ِ }\textstyleDropCaps{по до­роге, идя куда-л.‎}

\textstyleCaptioncharacters{اَهْلاً وَ سَهْلاً }\textstyleDropCaps{добро по­жаловать‎}

\textstyleCaptioncharacters{مَرْحَبًا بِكُمْ }\textstyleDropCaps{добро пожа­ловать, здравствуйте‎}

\textstyleCaptioncharacters{بَيْنَمَا }\textstyleDropCaps{в то время, как; меж­ду тем, как‎}

\textstyleCaptioncharacters{اِذْ }\textstyleDropCaps{вдруг‎}

\textstyleCaptioncharacters{غَرِيبٌ }\textstyleDropCaps{незнакомый‎}

\textstyleCaptioncharacters{جَانِبٌ }\textstyleDropCaps{сторона‎}

\textstyleCaptioncharacters{لَمْ اَرَهُ }\textstyleDropCaps{я его не видел‎}

\textstyleCaptioncharacters{عَلَى عَجَلٍ }\textstyleDropCaps{поспешно, то­ропливо, в спешке‎}

\textstyleCaptioncharacters{عَزِيزٌ }\textstyleDropCaps{дорогой‎}

\textstyleCaptioncharacters{صَادَفَ }\textstyleDropCaps{случайно встретил‎}

\textstyleCaptioncharacters{تَبَادَلَ }\textstyleDropCaps{обменялся‎}

\textstyleCaptioncharacters{تَحِيَّةٌ }\textstyleDropCaps{приветствие‎}

\textstyleCaptioncharacters{رَأْىٌ }\textstyleDropCaps{мнение, взгляд‎}

\textstyleCaptioncharacters{تَبَادَلْنَا الرَّأْىَ }\textstyleDropCaps{мы обме­нялись мнениями‎}

\textstyleCaptioncharacters{هَا قَدْ خَاءَ }\textstyleDropCaps{вот пришёл‎}

\textstyleCaptioncharacters{مَنْجَمٌ }\textstyleDropCaps{шахта, рудник‎}

\textstyleCaptioncharacters{عَامِلُ مَنْجَمٍ }\textstyleDropCaps{шахтёр‎}

\textstyleCaptioncharacters{لَعِبٌ }\textstyleDropCaps{игра‎}

\textstyleCaptioncharacters{رِيَاضِىٌّ }\textstyleDropCaps{спортивный‎}

\textstyleCaptioncharacters{اَلْعَابٌ رِيَاضِيَّةٌ }\textstyleDropCaps{спортив­ные игры‎}

\textstyleCaptioncharacters{فَتْرَةُ لإِسْتِرَاحَةِ }\textstyleDropCaps{пере­рыв на отдых, перемена‎}

\textstyleCaptioncharacters{تَمَازَحَ }\textstyleDropCaps{шутили друг с дру­гом‎}

\textstyleCaptioncharacters{شُغْلٌ }\textstyleDropCaps{занятие, работа, дело‎}

\textstyleCaptioncharacters{بِلاَ شُغْلٍ }\textstyleDropCaps{без дела‎}

\textstyleCaptioncharacters{فَرْضٌ }\textstyleDropCaps{задание, обязан­ность‎}

\textstyleCaptioncharacters{اَلْعَالَمُ الإِسْلاَمِىُّ }\textstyleDropCaps{ислам­ский мир‎}

\subsection[Урок 82‎]{\textstyleDropCaps{Урок 82‎}}
\textstyleCaptioncharacters{سَهِرَ }\textstyleDropCaps{бодрствовал, не спал‎}

\textstyleCaptioncharacters{اَلْبَارِحَةَ }\textstyleDropCaps{вчера ночью‎}

\textstyleCaptioncharacters{مِنَ الثَّابِتِ أَنَّ...ِ }\textstyleDropCaps{бесспорно, установлено, что...‎}

\textstyleCaptioncharacters{تبْدِيلٌ }\textstyleDropCaps{менять‎}

\textstyleCaptioncharacters{تَبْدِيلُ الْهَوَاءِ }\textstyleDropCaps{менять воз­дух‎}

\textstyleCaptioncharacters{مَسْكَنٌ }\textstyleDropCaps{жилье, жилище, дом‎}

\textstyleCaptioncharacters{حِينٌ }\textstyleDropCaps{время‎}

\textstyleCaptioncharacters{حِينًا بَعْدَ حِينٍ }\textstyleDropCaps{время от времени‎}

\textstyleCaptioncharacters{تَعَرَّضَ ل‍ِ...ِ }\textstyleDropCaps{подвергался‎}

\textstyleCaptioncharacters{مِظَلَّةٌ }\textstyleDropCaps{зонт, зонтик‎}

\textstyleCaptioncharacters{مَجْرَى الْهَوَاءِ }\textstyleDropCaps{сквозняк‎}

\textstyleCaptioncharacters{تَعَرَّضَ لِمَجْرَى الْهَوَاءِ }\textstyleDropCaps{подвергался сквозняку‎}

\textstyleCaptioncharacters{ضَيِّقٌ }\textstyleDropCaps{тесный, узкий‎}

\textstyleCaptioncharacters{مُلْتَوٍ }\textstyleDropCaps{кривой, извилистый‎}

\textstyleCaptioncharacters{مُسْتَقِيمٌ }\textstyleDropCaps{прямой‎}

\textstyleCaptioncharacters{اَصْبَحَ }\textstyleDropCaps{стал‎}

\textstyleCaptioncharacters{كِلاَ هُمَا }\textstyleDropCaps{они оба‎}

\textstyleCaptioncharacters{كِلْتَا هُمَا }\textstyleDropCaps{они обе‎}

\textstyleCaptioncharacters{فَرَغَ مِنْ...ِ }\textstyleDropCaps{закончил‎}

\textstyleCaptioncharacters{تَفَرَّغَ ل‍ِ...ِ }\textstyleDropCaps{имел досуг, на­шел свободное время‎}

\textstyleCaptioncharacters{كَانَ عَلَيْنَا اَنْ...ِ }\textstyleDropCaps{мы должны были, нам надо было‎}

\textstyleCaptioncharacters{هَمْسًا }\textstyleDropCaps{шёпотом‎}

\textstyleCaptioncharacters{فِى الْمَنَامِ }\textstyleDropCaps{во сне‎}

\textstyleCaptioncharacters{فِى اسْتِطَاعَتِى }\textstyleDropCaps{я могу‎}

\textstyleCaptioncharacters{لَيْسَ فِى اسْتِطَاعَتِى }\textstyleDropCaps{я не могу‎}

\textstyleCaptioncharacters{مُتَرْجِمٌ }\textstyleDropCaps{переводчик‎}

\textstyleCaptioncharacters{يَوْمٌ مُصْحٍ }\textstyleDropCaps{ясный день‎}

\textstyleCaptioncharacters{مَا كَادَ يَذْهَبُ حَتَّى رَجَعَ }\textstyleDropCaps{едва только ушёл как вернулся; не успел уйти, как вернулся‎}

\textstyleCaptioncharacters{مَا كِدْتُ أَجْلِسُ حَتَّى نَظَرَ اِلَىَّ }\textstyleDropCaps{едва только я сел, как он посмотрел на меня‎}

\textstyleCaptioncharacters{أَخَذَ يَقْرَأُ }\textstyleDropCaps{начал читать‎}

\textstyleCaptioncharacters{سَقَطَ الْمَطَرُ }\textstyleDropCaps{пошёл дождь‎}

\textstyleCaptioncharacters{بَلَّ }\textstyleDropCaps{намочил‎}

\textstyleCaptioncharacters{سَوْفَ يَذْهَبُ }\textstyleDropCaps{он пой­дет, будет идти‎}

\subsection[Урок 83‎]{\textstyleDropCaps{Урок 83‎}}
\textstyleCaptioncharacters{بِطِّيخٌ }\textstyleDropCaps{арбуз‎}

\textstyleCaptioncharacters{شَمَّامٌ }\textstyleDropCaps{дыня‎}

\textstyleCaptioncharacters{لِفْتٌ }\textstyleDropCaps{репа‎}

\textstyleCaptioncharacters{بَصَلٌ }\textstyleDropCaps{лук‎}

\textstyleCaptioncharacters{كُرُنْبٌ }\textstyleDropCaps{капуста‎}

\textstyleCaptioncharacters{خِيَارٌ }\textstyleDropCaps{огурец‎}

\textstyleCaptioncharacters{فِجْلٌ }\textstyleDropCaps{редиска‎}

\textstyleCaptioncharacters{جَزَرٌ }\textstyleDropCaps{морковь‎}

\textstyleCaptioncharacters{ثُومٌ }\textstyleDropCaps{чеснок‎}

\textstyleCaptioncharacters{رَفِيقٌ }\textstyleDropCaps{товарищ‎}

\textstyleCaptioncharacters{اَسْرَعَ اِلَى...ِ }\textstyleDropCaps{поспешил‎}

\textstyleCaptioncharacters{دُونَ اَنْ يَجْلِسَ }\textstyleDropCaps{не сев‎}

\textstyleCaptioncharacters{وَ لَوْ...ِ }\textstyleDropCaps{хоть, хотя бы‎}

\textstyleCaptioncharacters{اَعْطِنِى وَ لَوْ دِرْهَمًا }\textstyleDropCaps{дай мне хоть один дирхем‎}

\textstyleCaptioncharacters{مَبْقَلَةٌ }\textstyleDropCaps{огород‎}

\textstyleCaptioncharacters{مِنَارَةٌ }\textstyleDropCaps{минарет‎}

\textstyleCaptioncharacters{مِنْ عَلَى الطَّاوِلَةِ }\textstyleDropCaps{со стола‎}

\textstyleCaptioncharacters{مَنْ }\textstyleDropCaps{тот, кто‎}

\textstyleCaptioncharacters{رَأَيْتُ مَنْ جَاءَكَ }\textstyleDropCaps{я ви­дел того, кто приходил к тебе‎}

\textstyleCaptioncharacters{نَحْوَ الْبَيْتِ }\textstyleDropCaps{по направле­нию к дому, в сторону дома‎}

\textstyleCaptioncharacters{مُتَوَجِّهًا نَحْوَ الْبَيْتِ }\textstyleDropCaps{направившись к дому‎}

\textstyleCaptioncharacters{اِنْطَلَقَ }\textstyleDropCaps{быстро пошёл‎}

\textstyleCaptioncharacters{لاَ يَلْوِى عَلَى شَىْءٍ }\textstyleDropCaps{не­взирая ни на что, не обращая внимания ни на что‎}

\textstyleCaptioncharacters{دَاخِلَ...ِ }\textstyleDropCaps{внутри, во­внутрь‎}

\textstyleCaptioncharacters{دَاخِلَ الْمَدِينَةِ }\textstyleDropCaps{внутри города‎}

\textstyleCaptioncharacters{خَالِدٌ }\textstyleDropCaps{вечный, бессмертый‎}

\textstyleCaptioncharacters{نَبَتَ }\textstyleDropCaps{рос‎}

\textstyleCaptioncharacters{هَمٌّ }\textstyleDropCaps{забота‎}

\textstyleCaptioncharacters{غَالٍ }\textstyleDropCaps{дорогой (по цене)‎}

\textstyleCaptioncharacters{أَغْلَى }\textstyleDropCaps{дороже‎}

\textstyleCaptioncharacters{كَمْ سَنَةً عُمْرُكَ؟ }\textstyleDropCaps{сколь­ко тебе лет?‎}

\textstyleCaptioncharacters{نَاهَزْتُ خَمْسِينَ سَنَةً }\textstyleDropCaps{мне под пятьдесят, мне скоро будет пятьдесят‎}

\textstyleCaptioncharacters{بُلُوغٌ }\textstyleDropCaps{совершеннолетие‎}

\textstyleCaptioncharacters{نَاهَزْتُ الْبُلُوغَ }\textstyleDropCaps{я скоро стану совершеннолетним‎}

\textstyleCaptioncharacters{اِذْذَّاكَ }\textstyleDropCaps{тогда‎}

\textstyleCaptioncharacters{عَلَى الأَكْثَرِ }\textstyleDropCaps{самое большое, по большей части‎}

\textstyleCaptioncharacters{عَلَى الأَقَلِّ }\textstyleDropCaps{по меньшей мере, по крайней мере‎}

\textstyleCaptioncharacters{رَاتِبٌ }\textstyleDropCaps{оклад, жалованье‎}

\textstyleCaptioncharacters{تَقَاضَى }\textstyleDropCaps{получил‎}

\textstyleCaptioncharacters{قُوَّةٌ }\textstyleDropCaps{сила‎}

\textstyleCaptioncharacters{مِقْدَارٌ }\textstyleDropCaps{размер, количество‎}

\textstyleCaptioncharacters{يَوْمُ الْقِيَامَةِ }\textstyleDropCaps{Судный день, день страшного суда‎}

\textstyleCaptioncharacters{اِلْتَحَقَ بِ...ِ }\textstyleDropCaps{поступил‎}

\textstyleCaptioncharacters{اِلْتَحَقَ بِالْجَامِعَةِ }\textstyleDropCaps{посту­пил в университет‎}

\subsection[Урок 84‎]{\textstyleDropCaps{Урок 84‎}}
\textstyleCaptioncharacters{تَزَوَّجَ }\textstyleDropCaps{женился‎}

\textstyleCaptioncharacters{تَزَوَّجَتْ }\textstyleDropCaps{вышла замуж‎}

\textstyleCaptioncharacters{اَنْجَبَ }\textstyleDropCaps{породил‎}

\textstyleCaptioncharacters{مَمَرٌّ }\textstyleDropCaps{проход‎}

\textstyleCaptioncharacters{لَعَلَّ }\textstyleDropCaps{наверное, возможно, может быть‎}

\textstyleCaptioncharacters{اَىْ }\textstyleDropCaps{то есть ‎}

\textstyleCaptioncharacters{رَأْسٌ }\textstyleDropCaps{голова‎}

\textstyleCaptioncharacters{اِصْطَدَمَ بِ... }\textstyleDropCaps{столкнул­ся, ударился о что-л.‎}

\textstyleCaptioncharacters{لاَ يَسَعُنِى اِلاَّ... }\textstyleDropCaps{не могу не...‎}

\textstyleCaptioncharacters{رِجْلٌ }\textstyleDropCaps{нога‎}

\textstyleCaptioncharacters{سَاكِنٌ }\textstyleDropCaps{житель‎}

\textstyleCaptioncharacters{سُكَّانُ الأَرْيَاف }\textstyleDropCaps{жители деревень‎}

\textstyleCaptioncharacters{مِنَ الْمَعْلُومِ أَنَّ }\textstyleDropCaps{из­вестно, что...‎}

\textstyleCaptioncharacters{بِوَاسِطَةِ... }\textstyleDropCaps{посредством, при помощи‎}

\textstyleCaptioncharacters{اَنْقَذَ }\textstyleDropCaps{спас, избавил‎}

\textstyleCaptioncharacters{رَدَّ }\textstyleDropCaps{ответил‎}

\textstyleCaptioncharacters{هَزَّ }\textstyleDropCaps{покачал‎}

\textstyleCaptioncharacters{هَزَّ رَأْسَهُ }\textstyleDropCaps{покачал голо­вой‎}

\textstyleCaptioncharacters{قَائِلاً }\textstyleDropCaps{сказав, со словами‎}

\textstyleCaptioncharacters{قَبِلَ }\textstyleDropCaps{принял‎}

\textstyleCaptioncharacters{وَحِيدٌ }\textstyleDropCaps{единственный‎}

\textstyleCaptioncharacters{سَائِرُ...ِ }\textstyleDropCaps{остальное, все остальное‎}

\textstyleCaptioncharacters{جَاءَ بِ }\textstyleDropCaps{принес, привел‎}

\textstyleCaptioncharacters{رَضِىَ }\textstyleDropCaps{был довольным, удовлетворился, согласился‎}

\textstyleCaptioncharacters{جَعَلَ يَضْحَكُ }\textstyleDropCaps{начал, стал смеяться‎}

\textstyleCaptioncharacters{يَدُ الْعَوْنِ }\textstyleDropCaps{рука помощи‎}

\textstyleCaptioncharacters{مَدَّ يَدَ الْعَوْنِ }\textstyleDropCaps{протянул руку помощи‎}

\textstyleCaptioncharacters{وَقَعَ عَلَى لأَرْضِ }\textstyleDropCaps{упал наземь‎}

\textstyleCaptioncharacters{اَللَّهُ تَعَالَى }\textstyleDropCaps{Всевышний Бог‎}

\textstyleCaptioncharacters{أَقَامَ }\textstyleDropCaps{поднял‎}

\textstyleCaptioncharacters{أَقَامَ الْقَاعِدَ }\textstyleDropCaps{поднял сидя­щего‎}

\subsection[Урок 85‎]{\textstyleDropCaps{Урок 85‎}}
\textstyleCaptioncharacters{لَمْ تَعُدْ طِفْلاً }\textstyleDropCaps{ты уже не ребенок‎}

\textstyleCaptioncharacters{لاَ يَلِيقُ بِ... }\textstyleDropCaps{не подоба­ет, не прилично, не к лицу‎}

\textstyleCaptioncharacters{سَفَاسِفُ الأُمُورِ }\textstyleDropCaps{пустя­ки, глупости‎}

\textstyleCaptioncharacters{اِشْتَغَلَ سَفَاسِفِ الأُمُورِ }\textstyleDropCaps{занимался пустяками‎}

\textstyleCaptioncharacters{طَالِبُ عِلْمٍ }\textstyleDropCaps{искатель зна­ния, учения, студент‎}

\textstyleCaptioncharacters{يَنْبَغِى اَنْ لاَّ... }\textstyleDropCaps{не следуе­т, не должен‎}

\textstyleCaptioncharacters{خَيْرٌ }\textstyleDropCaps{добро, благо, хоро­шее‎}

\textstyleCaptioncharacters{قِيَامُ اللَّيْلِ }\textstyleDropCaps{вставать но­чью‎}

\textstyleCaptioncharacters{صَلاَةُ اللَّيْلِ }\textstyleDropCaps{ночной на­маз‎}

\textstyleCaptioncharacters{صَلاَةُ التَّهَجُّدِ }\textstyleDropCaps{ночной намаз, совершаемый проснувшись‎}

\textstyleCaptioncharacters{خَافَ }\textstyleDropCaps{боялся‎}

\textstyleCaptioncharacters{سِوَى... }\textstyleDropCaps{кроме, исключая‎}

\textstyleCaptioncharacters{وَحْدَهُ }\textstyleDropCaps{он один, только он‎}

\textstyleCaptioncharacters{وَحْدَكَ }\textstyleDropCaps{ты один‎}

\textstyleCaptioncharacters{وَحْدِى }\textstyleDropCaps{я один‎}

\textstyleCaptioncharacters{كَفَاهُ اللَّهُ }\textstyleDropCaps{Бог, хватит ему‎}

\textstyleCaptioncharacters{مُحَاضَرَةٌ }\textstyleDropCaps{лекция‎}

\textstyleCaptioncharacters{اَلْقَى مُحَاضَرَةً }\textstyleDropCaps{читал лекцию‎}

\textstyleCaptioncharacters{حَاجَةٌ }\textstyleDropCaps{нужда, потреб­ность‎}

\textstyleCaptioncharacters{هَوَ بِحَاجَةٍ اِلَى... }\textstyleDropCaps{ему нужен‎}

\textstyleCaptioncharacters{تَعِبٌ }\textstyleDropCaps{усталый‎}

\textstyleCaptioncharacters{عَادِلٌ }\textstyleDropCaps{справедливый‎}

\textstyleCaptioncharacters{ظَالِمٌ }\textstyleDropCaps{несправедливый, же­стокий‎}

\textstyleCaptioncharacters{عَدْلٌ }\textstyleDropCaps{справедливость‎}

\textstyleCaptioncharacters{ظُلْمٌ }\textstyleDropCaps{несправедливость, гнет, угнетение‎}

\textstyleCaptioncharacters{جَهْلٌ }\textstyleDropCaps{невежество, темно­та‎}

\textstyleCaptioncharacters{اَصَابَ }\textstyleDropCaps{постиг, настиг, по­разил‎}

\textstyleCaptioncharacters{اُصِيبَ بِمَرَضٍ }\textstyleDropCaps{заболел‎}

\textstyleCaptioncharacters{ضُعْفٌ }\textstyleDropCaps{слабость‎}

\textstyleCaptioncharacters{غَلَبَ }\textstyleDropCaps{победил‎}

\textstyleCaptioncharacters{زَادَ عَنْ... }\textstyleDropCaps{превысил, пре­взошел‎}

\textstyleCaptioncharacters{كَمَا هُوَ الْحَالُ }\textstyleDropCaps{как об­стоит дело‎}

\textstyleCaptioncharacters{اَلاَ تُرِيدُ؟ }\textstyleDropCaps{не хотел бы ты?‎}

\subsection[Урок 86‎]{\textstyleDropCaps{Урок 86‎}}
\textstyleCaptioncharacters{سَمَاوِىٌّ }\textstyleDropCaps{небесный‎}

\textstyleCaptioncharacters{دِينٌ سَمَاوِىٌّ }\textstyleDropCaps{небесная религия‎}

\textstyleCaptioncharacters{سَعَادَةٌ }\textstyleDropCaps{счастье‎}

\textstyleCaptioncharacters{نَالَ }\textstyleDropCaps{получил‎}

\textstyleCaptioncharacters{اَلدُّنْيَا }\textstyleDropCaps{мирская жизнь, этот свет‎}

\textstyleCaptioncharacters{اَلآخِرَةُ }\textstyleDropCaps{загробная жизнь, тот свет‎}

\textstyleCaptioncharacters{فِى الدُّنْيَا وَ الآخِرَةِ }\textstyleDropCaps{на этом и на том свете‎}

\textstyleCaptioncharacters{رَفَضَ }\textstyleDropCaps{отверг, отказал‎}

\textstyleCaptioncharacters{شَقِىَ }\textstyleDropCaps{был, стал несчаст­ным‎}

\textstyleCaptioncharacters{آخِرٌ }\textstyleDropCaps{последний‎}

\textstyleCaptioncharacters{اِحْتَاجَ اِلَى... }\textstyleDropCaps{нуждался‎}

\textstyleCaptioncharacters{مُحْتَاجٌ اِلَى... }\textstyleDropCaps{нуждаю­щийся‎}

\textstyleCaptioncharacters{اَلصِّينُ }\textstyleDropCaps{Китай‎}

\textstyleCaptioncharacters{عِبَادَةٌ }\textstyleDropCaps{поклонение, служе­ние‎}

\textstyleCaptioncharacters{مَعِيشَةٌ }\textstyleDropCaps{жизнь, житие‎}

\textstyleCaptioncharacters{اَيْنَمَا... }\textstyleDropCaps{где бы ни‎}

\textstyleCaptioncharacters{حَيْثُمَا... }\textstyleDropCaps{куда бы ни‎}

\textstyleCaptioncharacters{وُجِدَ }\textstyleDropCaps{был, находился‎}

\textstyleCaptioncharacters{اَقْرَبُ }\textstyleDropCaps{ближе, ближайший‎}

\textstyleCaptioncharacters{أَحَبُّ }\textstyleDropCaps{самый любимый, бо­лее любимый, милее‎}

\textstyleCaptioncharacters{حَبِيبٌ }\textstyleDropCaps{дорогой, люби­мый, милый‎}

\textstyleCaptioncharacters{طَلَبَ الْعِلْمَ }\textstyleDropCaps{искал зна­ние, учился‎}

\textstyleCaptioncharacters{مَالٌ }\textstyleDropCaps{имущество, деньги‎}

\textstyleCaptioncharacters{صَادِقٌ }\textstyleDropCaps{истинный, настоя­щий‎}

\textstyleCaptioncharacters{كَامِلٌ }\textstyleDropCaps{совершенный, пол­ноценный,полный‎}

\textstyleCaptioncharacters{أَنْفُسُهُمْ }\textstyleDropCaps{они сами‎}

\textstyleCaptioncharacters{فِى اَىِّ بَلَدٍ آخَرَ }\textstyleDropCaps{в лю­бой другой стране‎}

\textstyleCaptioncharacters{قَدَرَ عَلَى... }\textstyleDropCaps{мог, был в си­лах, был способным‎}

\textstyleCaptioncharacters{خَيْرٌ }\textstyleDropCaps{хорошее, добро‎}

\textstyleCaptioncharacters{شَرٌّ }\textstyleDropCaps{плохое, зло‎}

\textstyleCaptioncharacters{شَقِىٌّ }\textstyleDropCaps{несчастный, жал­кий‎}

\subsection[Урок 87‎]{\textstyleDropCaps{Урок 87‎}}
\textstyleCaptioncharacters{ذُو... }\textstyleDropCaps{имеющий, облада­тель‎}

\textstyleCaptioncharacters{ذَاتُ... }\textstyleDropCaps{имеющая, облада­тельница‎}

\textstyleCaptioncharacters{ذُو مَالٍ }\textstyleDropCaps{богатый, денеж­ный‎}

\textstyleCaptioncharacters{عِبَارَةٌ عَنْ... }\textstyleDropCaps{представ­ляет собой‎}

\textstyleCaptioncharacters{مَبْنًى }\textstyleDropCaps{здание‎}

\textstyleCaptioncharacters{طَابِقٌ }\textstyleDropCaps{этаж‎}

\textstyleCaptioncharacters{سِتَارَةٌ }\textstyleDropCaps{штора, занавеска‎}

\textstyleCaptioncharacters{حَرِيرٌ }\textstyleDropCaps{шёлк‎}

\textstyleCaptioncharacters{سَخِرَ }\textstyleDropCaps{насмехался, изде­вался‎}

\textstyleCaptioncharacters{أَيُّهُمْ؟ }\textstyleDropCaps{кто из них, какой, который из них?‎}

\textstyleCaptioncharacters{عِيَالٌ }\textstyleDropCaps{семья‎}

\textstyleCaptioncharacters{مُحْتَرَمٌ }\textstyleDropCaps{уважаемый‎}

\textstyleCaptioncharacters{لاَ...اِلاَّ... }\textstyleDropCaps{нет… кроме, только‎}

\textstyleCaptioncharacters{اَنْفَقَ }\textstyleDropCaps{тратил, израсходовал‎}

\textstyleCaptioncharacters{اِزْدِهَارٌ }\textstyleDropCaps{процветание‎}

\textstyleCaptioncharacters{اِزْدِهَارُ الْعِلْمِ }\textstyleDropCaps{процвета­ние науки‎}

\textstyleCaptioncharacters{سِلاَحٌ }\textstyleDropCaps{оружие‎}

\textstyleCaptioncharacters{حَارَبَ }\textstyleDropCaps{воевал, сражался‎}

\textstyleCaptioncharacters{عَبَدَ }\textstyleDropCaps{поклонялся‎}

\textstyleCaptioncharacters{نِظَامٌ }\textstyleDropCaps{порядок‎}

\textstyleCaptioncharacters{بِنِظَامٍ }\textstyleDropCaps{строем, в порядке‎}

\textstyleCaptioncharacters{خُطْبَةٌ }\textstyleDropCaps{речь, выступление; проповедь‎}

\textstyleCaptioncharacters{مَوْعِظَةٌ }\textstyleDropCaps{проповедь‎}

\textstyleCaptioncharacters{تَلَقَّى }\textstyleDropCaps{получил‎}

\textstyleCaptioncharacters{بَنَى }\textstyleDropCaps{строил‎}

\textstyleCaptioncharacters{شَتَّى }\textstyleDropCaps{разные, различные‎}

\textstyleCaptioncharacters{بَعْدَ اَنْ... }\textstyleDropCaps{после того, как…‎}

\textstyleCaptioncharacters{اِسْتَقَرَّ }\textstyleDropCaps{устроился, обосно­вался, посетил‎}

\textstyleCaptioncharacters{سَعَى اِلَى... }\textstyleDropCaps{стремился‎}

\textstyleCaptioncharacters{صَمْتٌ }\textstyleDropCaps{молчание‎}

\textstyleCaptioncharacters{بِصَمْتٍ }\textstyleDropCaps{молча, с молчани­ем‎}

\textstyleCaptioncharacters{بَطْنٌ }\textstyleDropCaps{живот‎}

\textstyleCaptioncharacters{أَرْضَعَ }\textstyleDropCaps{кормить грудью‎}

\textstyleCaptioncharacters{هَيَّأَ }\textstyleDropCaps{готовил, подготовил‎}

\textstyleCaptioncharacters{قَبَّلَ }\textstyleDropCaps{целовал‎}

\textstyleCaptioncharacters{وَجُودٌ }\textstyleDropCaps{существование, бы­тие‎}

\textstyleCaptioncharacters{عَظِيمٌ }\textstyleDropCaps{великий‎}

\textstyleCaptioncharacters{مَا أَعْظَمَهُ }\textstyleDropCaps{как он велик! какой он великий!‎}

\textstyleCaptioncharacters{مَا أَجْمَلَهُ }\textstyleDropCaps{какой он кра­сивый!‎}

\textstyleCaptioncharacters{أَمَرَ }\textstyleDropCaps{приказал‎}

\textstyleCaptioncharacters{كَمَا أَمَرَ اللَّهُ }\textstyleDropCaps{как Бог приказал‎}

\textstyleCaptioncharacters{حَفِظَكَ اللَّهُ }\textstyleDropCaps{да сохранит тебя Бог‎}

\textstyleCaptioncharacters{حَىٌّ }\textstyleDropCaps{живой‎}

\textstyleCaptioncharacters{صَبَاحَ مَسَاءَ }\textstyleDropCaps{утром и ве­чером‎}

\textstyleCaptioncharacters{خَيْرٌ }\textstyleDropCaps{лучше‎}

\subsection[Урок 88‎]{\textstyleDropCaps{Урок 88‎}}
\textstyleCaptioncharacters{دَرَّاجَةٌ }\textstyleDropCaps{велосипед‎}

\textstyleCaptioncharacters{دَرَّاجَةٌ نَارِيَّةٌ }\textstyleDropCaps{мотоцикл‎}

\textstyleCaptioncharacters{حَلَّ }\textstyleDropCaps{настал, наступил‎}

\textstyleCaptioncharacters{قَطَفَ }\textstyleDropCaps{собрал, срывал (пло­ды, урожай)‎}

\textstyleCaptioncharacters{ثَمَرٌ }\textstyleDropCaps{плод‎}

\textstyleCaptioncharacters{حِرَاثَةٌ }\textstyleDropCaps{пахота, вспашка‎}

\textstyleCaptioncharacters{حَقًّا مَّا تَقُولُ }\textstyleDropCaps{правду, правильно ты говоришь‎}

\textstyleCaptioncharacters{آلِىٌّ }\textstyleDropCaps{автоматический, ме­ханизированный‎}

\textstyleCaptioncharacters{مِحْرَاثٌ آلِىٌّ }\textstyleDropCaps{автоплуг‎}

\textstyleCaptioncharacters{لاَ تُزْعِجْ نَفْسَكَ }\textstyleDropCaps{не бес­покой себя, не беспокойся‎}

\textstyleCaptioncharacters{جَاهِزٌ }\textstyleDropCaps{готовый‎}

\textstyleCaptioncharacters{كَفَى كُفْرَانًا }\textstyleDropCaps{хватит быть неблагодарным!‎}

\textstyleCaptioncharacters{كُفْرَانٌ بِالنِّعَمِ }\textstyleDropCaps{неблаго­дарность‎}

\textstyleCaptioncharacters{نِعْمَةٌ }\textstyleDropCaps{благодеяние, ми­лость‎}

\textstyleCaptioncharacters{نِعَمُ اللَّهِ }\textstyleDropCaps{милости Бога‎}

\textstyleCaptioncharacters{شَرَابٌ }\textstyleDropCaps{напиток‎}

\textstyleCaptioncharacters{لاَ يُحْصَى }\textstyleDropCaps{бесчислен­ный, неисчислимый‎}

\textstyleCaptioncharacters{حَمِدَ }\textstyleDropCaps{хвалил, восхвалял‎}

\textstyleCaptioncharacters{نَبَأٌ }\textstyleDropCaps{новость, известие, сообщ­ение‎}

\textstyleCaptioncharacters{سَلْ }\textstyleDropCaps{спроси‎}

\textstyleCaptioncharacters{لاَ تَسَلْ }\textstyleDropCaps{не спрашивай‎}

\textstyleCaptioncharacters{اَنَا فِى انْتِظَارِهِ }\textstyleDropCaps{я жду его, я в ожидании его‎}

\textstyleCaptioncharacters{قَدِمَ }\textstyleDropCaps{прибыл‎}

\textstyleCaptioncharacters{اَحْرَزَ }\textstyleDropCaps{достиг, добился, по­лучил‎}

\textstyleCaptioncharacters{اَحْرَزَ النَّجَاحَ }\textstyleDropCaps{добился успеха‎}

\textstyleCaptioncharacters{بِفَارِغِ الصَّبْرِ }\textstyleDropCaps{с нетерпе­нием‎}

\textstyleCaptioncharacters{إِلَى أَنْ... }\textstyleDropCaps{до тех пор, пока не…‎}

\textstyleCaptioncharacters{قَرِيبًا }\textstyleDropCaps{скоро, в ближайшее время‎}

\textstyleCaptioncharacters{جَادٌّ }\textstyleDropCaps{серьезный‎}

\textstyleCaptioncharacters{هَازِلٌ }\textstyleDropCaps{подшучивающий‎}

\textstyleCaptioncharacters{أَتَرْضَى؟ }\textstyleDropCaps{ты согласен? ты доволен?‎}

\textstyleCaptioncharacters{مَنَحَ }\textstyleDropCaps{дал, даровал‎}

\textstyleCaptioncharacters{مَبْلَغٌ }\textstyleDropCaps{сумма‎}

\textstyleCaptioncharacters{عَلَى الرَّأْسِ وَ الْعَيْنِ }\textstyleDropCaps{с удовольствием!‎}

\textstyleCaptioncharacters{يَا بُنَىَّ }\textstyleDropCaps{сынок! о мой сы­нок!‎}

\textstyleCaptioncharacters{مَلَكَ }\textstyleDropCaps{овладел, имел, обла­дал‎}

\textstyleCaptioncharacters{يَمْلِكُ }\textstyleDropCaps{имеет‎}

\subsection[Урок 89‎]{\textstyleDropCaps{Урок 89‎}}
\textstyleCaptioncharacters{شَطْرَنْجٌ }\textstyleDropCaps{шахматы‎}

\textstyleCaptioncharacters{دَامَا }\textstyleDropCaps{шашки‎}

\textstyleCaptioncharacters{بَطَلٌ }\textstyleDropCaps{чемпион‎}

\textstyleCaptioncharacters{بِسُهُولَةٍ }\textstyleDropCaps{легко‎}

\textstyleCaptioncharacters{شَبْعَانُ }\textstyleDropCaps{сытый‎}

\textstyleCaptioncharacters{جَوْعَانُ }\textstyleDropCaps{голодный‎}

\textstyleCaptioncharacters{طَعِمَ }\textstyleDropCaps{ел, кушал‎}

\textstyleCaptioncharacters{أَرْضَى }\textstyleDropCaps{удовлетворил‎}

\textstyleCaptioncharacters{بَيَّنَ }\textstyleDropCaps{разъяснил, объяснил, показал‎}

\textstyleCaptioncharacters{خَلَقَ }\textstyleDropCaps{создал, творил‎}

\textstyleCaptioncharacters{عَدَمٌ }\textstyleDropCaps{небытие‎}

\textstyleCaptioncharacters{أَحْيَا }\textstyleDropCaps{оживил, возродил‎}

\textstyleCaptioncharacters{أَمَاتَ }\textstyleDropCaps{умертвил‎}

\textstyleCaptioncharacters{رَزَقَ }\textstyleDropCaps{дал, даровал, наде­лил‎}

\textstyleCaptioncharacters{اَدْخَلَ }\textstyleDropCaps{ввёл, завёл‎}

\textstyleCaptioncharacters{جَنَّةٌ }\textstyleDropCaps{рай‎}

\textstyleCaptioncharacters{نَارٌ }\textstyleDropCaps{огонь; ад‎}

\textstyleCaptioncharacters{عَصَى }\textstyleDropCaps{ослушался, не под­чинился‎}

\textstyleCaptioncharacters{أَشَدُّ }\textstyleDropCaps{сильнее; более‎}

\textstyleCaptioncharacters{حُمْرَةٌ }\textstyleDropCaps{краснота‎}

\textstyleCaptioncharacters{أَشَدُّ حُمْرَةً }\textstyleDropCaps{более крас­ный‎}

\textstyleCaptioncharacters{أَقَلُّ اجْتِهَادًا }\textstyleDropCaps{менее ста­рательный‎}

\textstyleCaptioncharacters{بِفَضْلِ اللَّهِ }\textstyleDropCaps{по милости Бога‎}

\textstyleCaptioncharacters{بِرَحْمَةِ اللَّهِ }\textstyleDropCaps{по милосер­дию Бога‎}

\textstyleCaptioncharacters{عَدَدٌ }\textstyleDropCaps{число, количество‎}

\textstyleCaptioncharacters{يُصْبِحُ أَقْوَى فَأَقْوَى }\textstyleDropCaps{становится все сильнее и сильнее‎}

\textstyleCaptioncharacters{يُصْبِحُ اَكْثَرَ فَاَكْثَرَ }\textstyleDropCaps{ста­новится все больше и больше‎}

\textstyleCaptioncharacters{قَبْلَ كُلِّ شَىْءٍ }\textstyleDropCaps{прежде всего‎}

\subsection[Урок 90‎]{\textstyleDropCaps{Урок 90‎}}
\textstyleCaptioncharacters{أَصْبَاحًا أَمْ مَسَاءً؟ \ }\textstyleDropCaps{утром или вечером‎}

\textstyleCaptioncharacters{أَدَبٌ }\textstyleDropCaps{литература‎}

\textstyleCaptioncharacters{مِنْ أَقْدَمِ الْعُصُورِ إِلَى يَوْمِنَا هَذَا }\textstyleDropCaps{с древнейших времён до наших дней ‎}

\textstyleCaptioncharacters{أَتَى }\textstyleDropCaps{пришёл‎}

\textstyleCaptioncharacters{وَاللَّهِ }\textstyleDropCaps{клянусь Богом! ей-богу!‎}

\textstyleCaptioncharacters{بِالدِّقَّةِ }\textstyleDropCaps{точно‎}

\textstyleCaptioncharacters{اِقْتَرَبَ }\textstyleDropCaps{приблизился, подо­шёл‎}

\textstyleCaptioncharacters{اِقْتَرَبَ أَجَلُهُ }\textstyleDropCaps{приблизил­ся его час‎}

\textstyleCaptioncharacters{جُمْهُورٌ }\textstyleDropCaps{толпа, масса, пуб­лика‎}

\textstyleCaptioncharacters{مَقْهًى }\textstyleDropCaps{кафе‎}

\textstyleCaptioncharacters{يُمْكِنُكَ }\textstyleDropCaps{ты можешь‎}

\textstyleCaptioncharacters{يُمْكِنُنِى }\textstyleDropCaps{я могу‎}

\textstyleCaptioncharacters{كَالْعَادَةِ }\textstyleDropCaps{как обычно‎}

\textstyleCaptioncharacters{بَاقٍ }\textstyleDropCaps{оставшийся‎}

\textstyleCaptioncharacters{صَفْحَةٌ }\textstyleDropCaps{страница‎}

\textstyleCaptioncharacters{اِحْتَشَدَ }\textstyleDropCaps{скопился, со­брался‎}

\textstyleCaptioncharacters{أَظُنُّ }\textstyleDropCaps{(я) думаю‎}

\textstyleCaptioncharacters{مِينَاءٌ }\textstyleDropCaps{порт, гавань‎}

\textstyleCaptioncharacters{رَصِيفٌ }\textstyleDropCaps{перрон‎}

\textstyleCaptioncharacters{بَاخِرَةٌ }\textstyleDropCaps{пароход‎}

\textstyleCaptioncharacters{شِفَاهِيًّا }\textstyleDropCaps{устно‎}

\textstyleCaptioncharacters{كِتَابِيًّا }\textstyleDropCaps{письменно‎}

\textstyleCaptioncharacters{أَنَا عَلَى يَقِينٍ }\textstyleDropCaps{я уверен‎}

\textstyleCaptioncharacters{خَذَلَ }\textstyleDropCaps{оставил без помо­щи, поддержки‎}

\textstyleCaptioncharacters{أَمَلٌ }\textstyleDropCaps{надежда‎}

\textstyleCaptioncharacters{وَثِقَ بِ… }\textstyleDropCaps{верил, дове­рил‎}

\textstyleCaptioncharacters{صَرَخَ }\textstyleDropCaps{кричал, орал‎}

\textstyleCaptioncharacters{بِأَعْلَى صَوْتِهِ }\textstyleDropCaps{во весь го­лос‎}

\textstyleCaptioncharacters{أَوَّاهْ! }\textstyleDropCaps{ох! ах!‎}

\textstyleCaptioncharacters{هَيَّا بِنَا }\textstyleDropCaps{пошли! скорее!‎}

\textstyleCaptioncharacters{مَطْعَمٌ }\textstyleDropCaps{ресторан, столовая‎}

\subsection[Урок 91‎]{\textstyleDropCaps{Урок 91‎}}
\textstyleCaptioncharacters{مَا اَحْسَنَ بِكَ! }\textstyleDropCaps{как заме­чательно, когда ты…‎}

\textstyleCaptioncharacters{مَا اَقْبَحَ بِكَ! }\textstyleDropCaps{как некрасив­о, когда ты…‎}

\textstyleCaptioncharacters{صَدَقَ }\textstyleDropCaps{сказал правду‎}

\textstyleCaptioncharacters{بِالتَّاْكِيدِ }\textstyleDropCaps{конечно, несо­мненно‎}

\textstyleCaptioncharacters{دُونَ شَكٍّ }\textstyleDropCaps{несомненно‎}

\textstyleCaptioncharacters{أَجَادَ }\textstyleDropCaps{хорошо знал, умел‎}

\textstyleCaptioncharacters{فِى السَّنَوَاتِ مَا بَعْدَ الثَّوْرَةِ }\textstyleDropCaps{в послереволюционные годы ‎}

\textstyleCaptioncharacters{اَلشَّرْقُ }\textstyleDropCaps{восток‎}

\textstyleCaptioncharacters{اَلْغَرْبُ }\textstyleDropCaps{запад‎}

\textstyleCaptioncharacters{سَوَاءٌ اَقُلْتَ اَمْ لَمْ تَقُلْ }\textstyleDropCaps{все равно, сказал ли ты или нет ‎}

\textstyleCaptioncharacters{سَوَاءٌ فِى الْغَرْبِ أَمْ فِى الشَّرْقِ }\textstyleDropCaps{будь то на западе или на востоке ‎}

\textstyleCaptioncharacters{فِى طَرِيقِ النَّهْضَةِ }\textstyleDropCaps{на пути к возрождению‎}

\textstyleCaptioncharacters{صَحَا }\textstyleDropCaps{очнулся, пришел в себя, пробудился‎}

\textstyleCaptioncharacters{مُؤَامَرَةٌ }\textstyleDropCaps{заговор‎}

\textstyleCaptioncharacters{ضِدَّ... }\textstyleDropCaps{против…‎}

\textstyleCaptioncharacters{بَاءَ بِالْفَشَلِ }\textstyleDropCaps{окончился провалом, терпел неудачу‎}

\textstyleCaptioncharacters{لاَبُدَّ وَ أَنْ يَعُودَ }\textstyleDropCaps{он обя­зательно, непременно вернется‎}

\textstyleCaptioncharacters{وَعَدَ }\textstyleDropCaps{обещал‎}

\textstyleCaptioncharacters{ذَا }\textstyleDropCaps{это, то‎}

\textstyleCaptioncharacters{مُتَّهَمٌ }\textstyleDropCaps{обвиняемый, подсу­димый‎}

\textstyleCaptioncharacters{حَكَمَ عَلَى... }\textstyleDropCaps{пригово­рил, вынесприговор, судил‎}

\textstyleCaptioncharacters{سَجَنَ }\textstyleDropCaps{посадил, заключил в тюрьму‎}

\textstyleCaptioncharacters{لِمُدَّةِ أُسْبُوعٍ }\textstyleDropCaps{на неде­лю, сроком на неделю‎}

\textstyleCaptioncharacters{كَاَنَّمَا... }\textstyleDropCaps{как будто бы‎}

\textstyleCaptioncharacters{هَذَا لاَ يَعْنِينِى }\textstyleDropCaps{это меня не касается, не трогает‎}

\textstyleCaptioncharacters{عَاقَبَ }\textstyleDropCaps{наказал‎}

\textstyleCaptioncharacters{بِلاَدُ الإِنْجِلِيزِ }\textstyleDropCaps{страна ан­гличан, Англия‎}

\textstyleCaptioncharacters{بِلاَدُ الأَمْرِيكَانِ }\textstyleDropCaps{страна американцев, Америка‎}

\textstyleCaptioncharacters{جَرِيمَةٌ }\textstyleDropCaps{преступление‎}

\textstyleCaptioncharacters{مَدَى الْحَيَاةِ }\textstyleDropCaps{пожизнен­но‎}

\textstyleCaptioncharacters{اَلسَّجْنُ مَدَى الْحَيَاةِ }\textstyleDropCaps{пожизненное заключение‎}

\textstyleCaptioncharacters{بَيْنَمَا... }\textstyleDropCaps{в то время как…; между тем, как…‎}

\textstyleCaptioncharacters{اِعْدَامٌ }\textstyleDropCaps{расстрел, казнь‎}

\textstyleCaptioncharacters{شَنْقٌ }\textstyleDropCaps{повешение‎}

\textstyleCaptioncharacters{بَشَرِيَّةٌ }\textstyleDropCaps{человечество, че­ловеческий род‎}

\textstyleCaptioncharacters{حُكْمٌ }\textstyleDropCaps{приговор, решение‎}

\textstyleCaptioncharacters{صَلُحَ }\textstyleDropCaps{улучшился, испра­вился, пришёл вх орошее состояние‎}

\textstyleCaptioncharacters{بِمِثْلِ الإِسْلاَمِ }\textstyleDropCaps{как Ислам‎}

\textstyleCaptioncharacters{سُنَّةٌ }\textstyleDropCaps{закон‎}

\textstyleCaptioncharacters{غَيَّرَ }\textstyleDropCaps{изменил‎}

\textstyleCaptioncharacters{اُمَّةٌ }\textstyleDropCaps{нация, народ, умма‎}

\textstyleCaptioncharacters{ذُلٌّ }\textstyleDropCaps{унижение, низость‎}

\textstyleCaptioncharacters{عِزٌّ }\textstyleDropCaps{сила, могущество; ве­личие‎}

\textstyleCaptioncharacters{سُنَّةُ اللَّهِ }\textstyleDropCaps{закон Бога, объективный закон‎}

\textstyleCaptioncharacters{جَرَتْ سُنَّةُ اللَّهِ }\textstyleDropCaps{у Бога есть закон, по закону Бога принято‎}

\textstyleCaptioncharacters{اَلإِعْدَامُ شَنْقًا }\textstyleDropCaps{казнь че­рез повешение‎}

\textstyleCaptioncharacters{بَذَلَ وُسْعَهُ }\textstyleDropCaps{приложил все усилия, делал все возможное‎}

\textstyleCaptioncharacters{سَعَى فِى نُصْرَةِ الدِّينِ }\textstyleDropCaps{старался помочь религии, добивался помощи религии‎}

\subsection[Урок 92‎]{\textstyleDropCaps{Урок 92‎}}
\textstyleCaptioncharacters{نَعْلٌ }\textstyleDropCaps{сандалии, тапочки, бо­соножки ‎}

\textstyleCaptioncharacters{مِخْرَزٌ }\textstyleDropCaps{шило‎}

\textstyleCaptioncharacters{شَالٌ }\textstyleDropCaps{шарф‎}

\textstyleCaptioncharacters{دُكَّانٌ }\textstyleDropCaps{лавка, ларёк‎}

\textstyleCaptioncharacters{صَادِقٌ }\textstyleDropCaps{правдивый‎}

\textstyleCaptioncharacters{غَدِيرٌ }\textstyleDropCaps{лужа, пруд‎}

\textstyleCaptioncharacters{نَقَّ }\textstyleDropCaps{квакал‎}

\textstyleCaptioncharacters{بَرَّمَائِىٌّ }\textstyleDropCaps{земноводный‎}

\textstyleCaptioncharacters{بَرَى }\textstyleDropCaps{чинил, заострил‎}

\textstyleCaptioncharacters{خَاطَبَ }\textstyleDropCaps{обратился (к кому-л.)‎}

\textstyleCaptioncharacters{مُخَاطِبًا }\textstyleDropCaps{обращаясь‎}

\textstyleCaptioncharacters{اِسْكَافٌ }\textstyleDropCaps{сапожник‎}

\textstyleCaptioncharacters{ضِفْدَعٌ }\textstyleDropCaps{лягушка‎}

\textstyleCaptioncharacters{ثَقَبَ }\textstyleDropCaps{сверлил, проколол‎}

\textstyleCaptioncharacters{حَضَرَتِ الْقَهْوَةُ }\textstyleDropCaps{кофе го­тов, кофе подали‎}

\textstyleCaptioncharacters{حَضَرَتِ الصَّلاَةُ }\textstyleDropCaps{наста­ло время намаза‎}

\textstyleCaptioncharacters{يَسُرُّنِى }\textstyleDropCaps{я рад, что…, меня радует‎}

\textstyleCaptioncharacters{جَلَبَ }\textstyleDropCaps{привёз, принёс‎}

\textstyleCaptioncharacters{بِكُلِّ سُرُورٍ }\textstyleDropCaps{с большой ра­достью‎}

\textstyleCaptioncharacters{تَقِىٌّ }\textstyleDropCaps{богобоязненный, ‎}

\textstyleCaptioncharacters{مُحِبٌّ لِلْعِلْمِ }\textstyleDropCaps{благоче­стивый, любознательный, любящий знание, ‎}

\textstyleCaptioncharacters{اَللَّهُ يَرْضَى عَنْكَ }\textstyleDropCaps{да бу­дет доволен тобой Бог‎}

\textstyleCaptioncharacters{حَرَصَ }\textstyleDropCaps{старался‎}

\textstyleCaptioncharacters{صَاحَبَ }\textstyleDropCaps{дружил‎}

\subsection[Урок 93‎]{\textstyleDropCaps{Урок 93‎}}
\textstyleCaptioncharacters{هُدْبٌ }\textstyleDropCaps{ресницы‎}

\textstyleCaptioncharacters{جَفْنٌ }\textstyleDropCaps{веко‎}

\textstyleCaptioncharacters{حَاجِبٌ }\textstyleDropCaps{бровь‎}

\textstyleCaptioncharacters{أَنْفٌ }\textstyleDropCaps{нос‎}

\textstyleCaptioncharacters{شَارِبٌ }\textstyleDropCaps{ус‎}

\textstyleCaptioncharacters{لِحْيَةٌ }\textstyleDropCaps{борода‎}

\textstyleCaptioncharacters{عُضْو }\textstyleDropCaps{орган (тела)‎}

\textstyleCaptioncharacters{وَظِيفَةٌ }\textstyleDropCaps{функция, назначе­ние‎}

\textstyleCaptioncharacters{أُحَادُ }\textstyleDropCaps{по одному‎}

\textstyleCaptioncharacters{مَثْنَى }\textstyleDropCaps{по два‎}

\textstyleCaptioncharacters{مِنْهَا مَا هُوَ }\textstyleDropCaps{из них есть то, что…‎}

\textstyleCaptioncharacters{شَمَّ }\textstyleDropCaps{понюхал, обонял‎}

\textstyleCaptioncharacters{رَائِحَةٌ }\textstyleDropCaps{запах‎}

\textstyleCaptioncharacters{لِمَ لاَ تَقْرَأُ؟ }\textstyleDropCaps{почему ты не читаешь?‎}

\textstyleCaptioncharacters{عَمَّا قَرِيبٍ }\textstyleDropCaps{скоро, вскоре‎}

\textstyleCaptioncharacters{جَزَّ }\textstyleDropCaps{стриг‎}

\textstyleCaptioncharacters{جَزَّ الصُّوفَ }\textstyleDropCaps{стриг шерсть‎}

\textstyleCaptioncharacters{اَعْفَى اللِّحْيَةَ }\textstyleDropCaps{отпустил бороду‎}

\textstyleCaptioncharacters{شَعْرَةٌ }\textstyleDropCaps{волос, волосок‎}

\textstyleCaptioncharacters{شَعْرٌ }\textstyleDropCaps{волосы‎}

\textstyleCaptioncharacters{وَ لَوْ مَرَّةً }\textstyleDropCaps{хоть раз‎}

\textstyleCaptioncharacters{طَرَفٌ }\textstyleDropCaps{край, конец‎}

\textstyleCaptioncharacters{قَلَمَ }\textstyleDropCaps{обрезал‎}

\textstyleCaptioncharacters{قَلَمَ الظُّفْرَ }\textstyleDropCaps{обрезал ногти‎}

\textstyleCaptioncharacters{دَهَنَ }\textstyleDropCaps{смазал маслом‎}

\textstyleCaptioncharacters{سَرَّحَ }\textstyleDropCaps{причесал‎}

\textstyleCaptioncharacters{سَرَّحَ شَعْرَهُ }\textstyleDropCaps{причёсывалс­я‎}

\textstyleCaptioncharacters{أَحْفَى الشَّارِبَ }\textstyleDropCaps{коротко стриг усы‎}

\textstyleCaptioncharacters{مَجُوسِىٌّ }\textstyleDropCaps{маг, огне­поклонник‎}

\subsection[Урок 94‎]{\textstyleDropCaps{Урок 94‎}}
\textstyleCaptioncharacters{بُنْدُقِيَّةٌ }\textstyleDropCaps{ружьё‎}

\textstyleCaptioncharacters{خَرْطُوشَةٌ }\textstyleDropCaps{патрон‎}

\textstyleCaptioncharacters{كَرِيمٌ }\textstyleDropCaps{благородный, вели­кодушный‎}

\textstyleCaptioncharacters{سَخِىٌّ }\textstyleDropCaps{щедрый‎}

\textstyleCaptioncharacters{بَخِيلٌ }\textstyleDropCaps{скупой‎}

\textstyleCaptioncharacters{فِيمَا مَضَى }\textstyleDropCaps{в прошлом‎}

\textstyleCaptioncharacters{غَيْرُهُ }\textstyleDropCaps{другой, кроме него‎}

\textstyleCaptioncharacters{اِسْتَغْنَى عَنْ... }\textstyleDropCaps{обхо­дился, не нуждался‎}

\textstyleCaptioncharacters{سَائِلٌ }\textstyleDropCaps{нищий, попро­шайка‎}

\textstyleCaptioncharacters{رَدَّ }\textstyleDropCaps{отогнал, отбил, отразил‎}

\textstyleCaptioncharacters{قِطْعَةٌ }\textstyleDropCaps{кусок‎}

\textstyleCaptioncharacters{شَرْبَةٌ }\textstyleDropCaps{глоток‎}

\textstyleCaptioncharacters{بِخِلاَفِ... }\textstyleDropCaps{в отличии, в противоположность ‎}

\textstyleCaptioncharacters{تَصَدَّقَ عَلَى... }\textstyleDropCaps{дал ми­лостыню‎}

\textstyleCaptioncharacters{مَعْرُوفٌ بِ... }\textstyleDropCaps{известный‎}

\textstyleCaptioncharacters{كَرَمٌ }\textstyleDropCaps{благородство, вели­кодушие‎}

\textstyleCaptioncharacters{شُحٌّ }\textstyleDropCaps{жадность, скупость‎}

\textstyleCaptioncharacters{سَخَاءٌ }\textstyleDropCaps{щедрость‎}

\textstyleCaptioncharacters{بُخْلٌ }\textstyleDropCaps{скупость‎}

\textstyleCaptioncharacters{بَقِىَ }\textstyleDropCaps{остался‎}

\textstyleCaptioncharacters{لَهُ اَنْ... }\textstyleDropCaps{он может, имеет право‎}

\textstyleCaptioncharacters{لَيْسَ لَهُ اَنْ... }\textstyleDropCaps{он не мо­жет, он не имеет права‎}

\textstyleCaptioncharacters{اِفْتَخَرَ }\textstyleDropCaps{гордился‎}

\textstyleCaptioncharacters{اِنْطَلَقَ }\textstyleDropCaps{отправился, пу­стился в путь‎}

\textstyleCaptioncharacters{اِحْتَطَبَ }\textstyleDropCaps{собрал дрова‎}

\textstyleCaptioncharacters{اِسْتَصْحَبَ }\textstyleDropCaps{взял с собой‎}

\textstyleCaptioncharacters{وَحْشٌ }\textstyleDropCaps{зверь, дикое жи­вотное‎}

\textstyleCaptioncharacters{صَادَ الْوَحْشَ }\textstyleDropCaps{поймал, убил зверя, охотился на зверя‎}

\textstyleCaptioncharacters{فَارِغٌ }\textstyleDropCaps{пустой‎}

\textstyleCaptioncharacters{عَبَّأَ }\textstyleDropCaps{наполнил, насыпал‎}

\subsection[Урок 95‎]{\textstyleDropCaps{Урок 95‎}}
\textstyleCaptioncharacters{ظَهَرَ الإِسْلاَمُ عَلَى سَائِرِ الأَدْيَانِ }\textstyleDropCaps{Ислам одержал верх над другими религиями ‎}

\textstyleCaptioncharacters{فِى شَأْنِ... }\textstyleDropCaps{по поводу…, на счет…‎}

\textstyleCaptioncharacters{مَاضٍ }\textstyleDropCaps{продолжающийся, не прекращающийся‎}

\textstyleCaptioncharacters{بِمَا يَقْدِرُ }\textstyleDropCaps{чем может‎}

\textstyleCaptioncharacters{وَجَبَ }\textstyleDropCaps{стал обязательным, должным‎}

\textstyleCaptioncharacters{وَاجِبٌ }\textstyleDropCaps{обязательный‎}

\textstyleCaptioncharacters{وَ هَكَذَا دَوَالَيْكَ }\textstyleDropCaps{и так да­лее‎}

\textstyleCaptioncharacters{تَفَكَّرَ }\textstyleDropCaps{думал, размышлял‎}

\textstyleCaptioncharacters{مِرْآةٌ }\textstyleDropCaps{зеркало‎}

\textstyleCaptioncharacters{حَسَّنَ }\textstyleDropCaps{украсил‎}

\textstyleCaptioncharacters{كَىْ يَكُونَ }\textstyleDropCaps{чтобы был‎}

\textstyleCaptioncharacters{مَيِّتٌ }\textstyleDropCaps{мёртвый‎}

\textstyleCaptioncharacters{رَسْمٌ }\textstyleDropCaps{рисование‎}

\textstyleCaptioncharacters{تَبَعْثَرَ }\textstyleDropCaps{рассеялся, рассы­пался‎}

\textstyleCaptioncharacters{ضَاعَ }\textstyleDropCaps{пропал‎}

\textstyleCaptioncharacters{حَافَظَ عَلَى... }\textstyleDropCaps{соблюдал‎}

\textstyleCaptioncharacters{حَافَظَ عَلَى النَّظَافَةِ }\textstyleDropCaps{соблюдал чистоту‎}

\textstyleCaptioncharacters{حَافَظَ عَلَى الصَّلاَةِ }\textstyleDropCaps{соблюдал намаз‎}

\textstyleCaptioncharacters{صَامَ }\textstyleDropCaps{постился, соблюдал пост, уразу‎}

\textstyleCaptioncharacters{وَسَّخَ }\textstyleDropCaps{пачкал‎}

\textstyleCaptioncharacters{عَلَى وَجْهِ لأَرْرْضِ }\textstyleDropCaps{на поверхности Земли‎}

\textstyleCaptioncharacters{حُكْمٌ }\textstyleDropCaps{власть‎}

\subsection[Урок 96‎]{\textstyleDropCaps{Урок 96‎}}
\textstyleCaptioncharacters{آلَةُ التَّصْوِيرِ }\textstyleDropCaps{фотоаппа­рат‎}

\textstyleCaptioncharacters{لِجَامٌ }\textstyleDropCaps{узда‎}

\textstyleCaptioncharacters{سَرْجٌ }\textstyleDropCaps{седло‎}

\textstyleCaptioncharacters{حَاضِنَةٌ }\textstyleDropCaps{нянька, мамка‎}

\textstyleCaptioncharacters{وَلِيدٌ }\textstyleDropCaps{дитя, новорождён­ный‎}

\textstyleCaptioncharacters{سَقَاهُ مَاءً }\textstyleDropCaps{напоил водой‎}

\textstyleCaptioncharacters{قَمَطَ }\textstyleDropCaps{пеленал, свил (ребён­ка)‎}

\textstyleCaptioncharacters{نَوَّمَ }\textstyleDropCaps{уложил спать, усыпил‎}

\textstyleCaptioncharacters{نَوَّمَ الْوَلِيدَ }\textstyleDropCaps{уложил ребён­ка спать‎}

\textstyleCaptioncharacters{مُرْ }\textstyleDropCaps{прикажи‎}

\textstyleCaptioncharacters{خَادِمٌ }\textstyleDropCaps{слуга‎}

\textstyleCaptioncharacters{خَادِمَةٌ }\textstyleDropCaps{служанка‎}

\textstyleCaptioncharacters{مَهْدٌ }\textstyleDropCaps{люлька, колыбель‎}

\textstyleCaptioncharacters{لَحْدٌ }\textstyleDropCaps{могила‎}

\textstyleCaptioncharacters{أَلْجَمَ }\textstyleDropCaps{надел узду‎}

\textstyleCaptioncharacters{أَسْرَجَ }\textstyleDropCaps{оседлал‎}

\textstyleCaptioncharacters{مَصْنُوعٌ }\textstyleDropCaps{сделанный, изго­товленный‎}

\textstyleCaptioncharacters{عَزَمَ عَلَى... }\textstyleDropCaps{решил, воз­намерился‎}

\textstyleCaptioncharacters{جِلْدٌ }\textstyleDropCaps{кожа‎}

\textstyleCaptioncharacters{عَلَى مَهْلٍ }\textstyleDropCaps{медленно‎}

\textstyleCaptioncharacters{مَشَى عَلَى مَهْلٍ }\textstyleDropCaps{хо­дил, пошёл медленно‎}

\textstyleCaptioncharacters{حُزْمَةٌ }\textstyleDropCaps{кипа, пачка‎}

\textstyleCaptioncharacters{سَمِيكٌ }\textstyleDropCaps{толстый‎}

\textstyleCaptioncharacters{لَوَازِمُ الْكِتَابَةِ }\textstyleDropCaps{письмен­ные принадлежности‎}

\textstyleCaptioncharacters{تَعَوَّدَ }\textstyleDropCaps{привык‎}

\textstyleCaptioncharacters{صَوَّرَ }\textstyleDropCaps{фотографировал, снял‎}

\textstyleCaptioncharacters{مُصَوِّرٌ }\textstyleDropCaps{фотограф‎}

\textstyleCaptioncharacters{غَرِيبٌ }\textstyleDropCaps{странный‎}

\textstyleCaptioncharacters{غَرِيبُ الشَّكْلِ }\textstyleDropCaps{странной формы‎}

\textstyleCaptioncharacters{اَشْبَهَ }\textstyleDropCaps{был похожим‎}

\textstyleCaptioncharacters{لاَ يُشْبِهُ }\textstyleDropCaps{непохожий‎}

\textstyleCaptioncharacters{حَجْمٌ }\textstyleDropCaps{размер, габарит‎}

\textstyleCaptioncharacters{صَغِيرُ الْحَجْمِ }\textstyleDropCaps{малогаба­ритный, небольшого размера‎}

\textstyleCaptioncharacters{كَبِيرُ الْحَجْمِ }\textstyleDropCaps{громозд­кий, крупногабаритный‎}

\textstyleCaptioncharacters{عَادِىٌّ }\textstyleDropCaps{обыкновенный, обычный‎}

\textstyleCaptioncharacters{اِتَّخَذَ صَدِيقًا }\textstyleDropCaps{сделал, из­брал себе друга‎}

\textstyleCaptioncharacters{دَارٌ }\textstyleDropCaps{дом, жилище; страна‎}

\textstyleCaptioncharacters{كَفَرَ }\textstyleDropCaps{не веровал, был гяу­ром‎}

\textstyleCaptioncharacters{كَفَرَ بِاللَّهِ }\textstyleDropCaps{не верил в Бога‎}

\textstyleCaptioncharacters{رَضِيتُ بِاللَّهِ رَبًّا }\textstyleDropCaps{я удовлетворён Богом как Господом‎}

\textstyleCaptioncharacters{رَضِيتُ بِالإِسْلاَمِ دِينًا }\textstyleDropCaps{я удовлетворён Исламом как религией‎}

\subsection[Урок 97‎]{\textstyleDropCaps{Урок 97‎}}
\textstyleCaptioncharacters{وَصِيَّةٌ }\textstyleDropCaps{завещание, завет‎}

\textstyleCaptioncharacters{أَوْصَى }\textstyleDropCaps{завещал, наказал; советовал‎}

\textstyleCaptioncharacters{خَالَفَ الْوَصِيَّةَ }\textstyleDropCaps{нарушил завещание‎}

\textstyleCaptioncharacters{اِغْتَسَلَ }\textstyleDropCaps{купался, умылся‎}

\textstyleCaptioncharacters{اِسْتَدْعَى }\textstyleDropCaps{вызвал‎}

\textstyleCaptioncharacters{عَايَنَ }\textstyleDropCaps{осмотрел‎}

\textstyleCaptioncharacters{عَايَنَ الْمَرِيضَ }\textstyleDropCaps{осмот­рел больного‎}

\textstyleCaptioncharacters{كَشَفَ }\textstyleDropCaps{обнаружил, вы­явил‎}

\textstyleCaptioncharacters{دَاءٌ }\textstyleDropCaps{болезнь‎}

\textstyleCaptioncharacters{وَصَفَ الدَّوَاءَ }\textstyleDropCaps{выписал, прописал лекарство‎}

\textstyleCaptioncharacters{مُعَالَجَةٌ }\textstyleDropCaps{лечение‎}

\textstyleCaptioncharacters{دَامَ }\textstyleDropCaps{длился, продолжался‎}

\textstyleCaptioncharacters{اَمْرٌ }\textstyleDropCaps{приказ, распоряжение‎}

\textstyleCaptioncharacters{أَوَامِرُ اللَّهِ }\textstyleDropCaps{приказы, пред­писания Бога‎}

\textstyleCaptioncharacters{عِقَابٌ }\textstyleDropCaps{наказание‎}

\textstyleCaptioncharacters{تَابَ }\textstyleDropCaps{покаялся, пожалел‎}

\textstyleCaptioncharacters{عَادَ اِلَى... }\textstyleDropCaps{возобновил, де­лал вторично, повторил‎}

\textstyleCaptioncharacters{لَمْ يَعُدْ اِلَى... }\textstyleDropCaps{он больше не повторил‎}

\textstyleCaptioncharacters{فَرْضٌ }\textstyleDropCaps{обязанность, запо­ведь, предписание‎}

\textstyleCaptioncharacters{مُعَقَّدٌ }\textstyleDropCaps{сложный, запутан­ный‎}

\textstyleCaptioncharacters{مَسْئَلَةٌ مُعَقَّدَةٌ }\textstyleDropCaps{сложная задача‎}

\textstyleCaptioncharacters{وَرَدَ }\textstyleDropCaps{значился, встретился, приводился‎}

\textstyleCaptioncharacters{اَكْرَمَ }\textstyleDropCaps{почитал, уважал‎}

\textstyleCaptioncharacters{شَيْخُوخَةٌ }\textstyleDropCaps{старость‎}

\textstyleCaptioncharacters{رِفَاقٌ }\textstyleDropCaps{товарищи‎}

\textstyleCaptioncharacters{غِيَابٌ }\textstyleDropCaps{отсутствие‎}

\textstyleCaptioncharacters{هَذِهِ الْمُدَّةَ }\textstyleDropCaps{в этот пери­од, в это время‎}

\textstyleCaptioncharacters{صَلَّى الْفَرْضَ }\textstyleDropCaps{совершил обязательный намаз‎}

\subsection[Урок 98‎]{\textstyleDropCaps{Урок 98‎}}
\textstyleCaptioncharacters{رَاكِبٌ }\textstyleDropCaps{пассажир‎}

\textstyleCaptioncharacters{سِنٌّ }\textstyleDropCaps{возраст‎}

\textstyleCaptioncharacters{كَبِيرُ السِّنِّ }\textstyleDropCaps{старый‎}

\textstyleCaptioncharacters{تَعَجَّبَ مِنْ... }\textstyleDropCaps{удивился‎}

\textstyleCaptioncharacters{غَرَسَ }\textstyleDropCaps{посадил (дерево)‎}

\textstyleCaptioncharacters{غَرَسَ شَجَرَةً }\textstyleDropCaps{посадил дерево‎}

\textstyleCaptioncharacters{أَمَّلَ }\textstyleDropCaps{надеялся‎}

\textstyleCaptioncharacters{أَتُؤَمِّلُ؟ }\textstyleDropCaps{ты надеешься?‎}

\textstyleCaptioncharacters{مَا لَكَ؟ }\textstyleDropCaps{что с тобой?‎}

\textstyleCaptioncharacters{مَا لَكَ تَضْحَكُ؟ }\textstyleDropCaps{что ты смеёшься?п очему ты смеёшься?‎}

\textstyleCaptioncharacters{مَا لَكَ لاَ تَقْرَأُ؟ }\textstyleDropCaps{что ты не читаешь?‎}

\textstyleCaptioncharacters{أَعِدُكَ }\textstyleDropCaps{обещаю тебе, даю тебе слово‎}

\textstyleCaptioncharacters{طَبَّلَ }\textstyleDropCaps{бил в барабан, бара­банил‎}

\textstyleCaptioncharacters{نَائِمٌ }\textstyleDropCaps{спящий‎}

\textstyleCaptioncharacters{قَالَ قَائِلٌ }\textstyleDropCaps{кто-то сказал‎}

\textstyleCaptioncharacters{أُمْنِيَّةٌ }\textstyleDropCaps{желание, мечта‎}

\textstyleCaptioncharacters{أُمْنِيَّتِىَ الْوَحِيدَةُ }\textstyleDropCaps{моё единственное желание‎}

\textstyleCaptioncharacters{كَانَتْ أُمْنِيَّتِىَ الْوَحِيدَةُ }\textstyleDropCaps{у меня единственное желание было…‎}

\textstyleCaptioncharacters{اِسْتَنْفَدَ }\textstyleDropCaps{исчерпал, погло­тил‎}

\textstyleCaptioncharacters{اِسْتَنْفَدَ الْمَالَ }\textstyleDropCaps{исчер­пал деньги‎}

\subsection[Урок 99‎]{\textstyleDropCaps{Урок 99‎}}
\textstyleCaptioncharacters{مَرْفَأٌ }\textstyleDropCaps{порт‎}

\textstyleCaptioncharacters{قَلَعَ }\textstyleDropCaps{вырвал, выдёргивал‎}

\textstyleCaptioncharacters{ضَارٌّ }\textstyleDropCaps{вредный‎}

\textstyleCaptioncharacters{أَصْلَحَ }\textstyleDropCaps{ремонтировал, по­чинил‎}

\textstyleCaptioncharacters{تَعَاوَنَ عَلَى... }\textstyleDropCaps{сотрудни­чали, совместно работали‎}

\textstyleCaptioncharacters{جَمَّلَ }\textstyleDropCaps{украсил‎}

\textstyleCaptioncharacters{نَظَّمَ }\textstyleDropCaps{приводил в порядок, регулировал‎}

\textstyleCaptioncharacters{بَارٌّ }\textstyleDropCaps{добрый, хороший‎}

\textstyleCaptioncharacters{رَسَا }\textstyleDropCaps{причалил, становился на якорь‎}

\textstyleCaptioncharacters{رَسَتِ السَّفِينَةُ }\textstyleDropCaps{корабль становился на якорь‎}

\textstyleCaptioncharacters{عَرَبَةُ الأَطْفَالِ }\textstyleDropCaps{детская коляска‎}

\textstyleCaptioncharacters{عَرَبَةٌ }\textstyleDropCaps{коляска, тачка‎}

\textstyleCaptioncharacters{مَرَّةً }\textstyleDropCaps{однажды‎}

\textstyleCaptioncharacters{قَسَمَ }\textstyleDropCaps{делил‎}

\textstyleCaptioncharacters{عَلَى سَوَاءٍ }\textstyleDropCaps{одинаково, равным образом‎}

\textstyleCaptioncharacters{نَادَى }\textstyleDropCaps{звал, окликал‎}

\textstyleCaptioncharacters{لاَعَبَ }\textstyleDropCaps{играл (с кем-л.)‎}

\textstyleCaptioncharacters{لاَعَبَ الْهِرَّةَ }\textstyleDropCaps{играл с кош­кой‎}

\textstyleCaptioncharacters{اَكْمَلَ }\textstyleDropCaps{закончил, завершил‎}

\textstyleCaptioncharacters{حَالاً }\textstyleDropCaps{тотчас, немедленно‎}

\textstyleCaptioncharacters{ضَاحَكَ }\textstyleDropCaps{посмеялся (с кем-л.)‎}

\textstyleCaptioncharacters{ضَاحَكَ الطِّفْلَ }\textstyleDropCaps{посмеял­ся с ребёнком‎}

\textstyleCaptioncharacters{غَنَّى }\textstyleDropCaps{пел‎}

\subsection[Урок 100‎]{\textstyleDropCaps{Урок 100‎}}
\textstyleCaptioncharacters{خَطٌّ }\textstyleDropCaps{линия, черта‎}

\textstyleCaptioncharacters{زَائِِرٌ }\textstyleDropCaps{посетитель‎}

\textstyleCaptioncharacters{سَافَرَ }\textstyleDropCaps{уехал, поехал‎}

\textstyleCaptioncharacters{حِسَابٌ }\textstyleDropCaps{арафметика‎}

\textstyleCaptioncharacters{حَارِسٌ }\textstyleDropCaps{сторож‎}

\textstyleCaptioncharacters{بَاضَ }\textstyleDropCaps{нести яйца‎}

\textstyleCaptioncharacters{فِضَّةٌ }\textstyleDropCaps{серебро‎}

\textstyleCaptioncharacters{كَثَّرَ }\textstyleDropCaps{умножил, приумано­жил‎}

\textstyleCaptioncharacters{عَلَفٌ }\textstyleDropCaps{корм‎}

\textstyleCaptioncharacters{اِنْشَقَّ }\textstyleDropCaps{разорвался‎}

\textstyleCaptioncharacters{حَوْصَلَةٌ }\textstyleDropCaps{зоб (птицы)‎}

\subsection[Урок 101‎]{\textstyleDropCaps{Урок 101‎}}
\textstyleCaptioncharacters{صُوصٌ }\textstyleDropCaps{ципленок‎}

\textstyleCaptioncharacters{حُزْمَةٌ }\textstyleDropCaps{сноп‎}

\textstyleCaptioncharacters{خُمٌّ }\textstyleDropCaps{курятник‎}

\textstyleCaptioncharacters{خُمُّ الدَّجَاجِ }\textstyleDropCaps{курятник; корзина для кур‎}

\textstyleCaptioncharacters{سُنْبُلٌ }\textstyleDropCaps{колос‎}

\textstyleCaptioncharacters{اِصْفَرَّ }\textstyleDropCaps{пожелтел‎}

\textstyleCaptioncharacters{مُصَفَّرٌ }\textstyleDropCaps{пожелтевший‎}

\textstyleCaptioncharacters{عَزِيمَةٌ }\textstyleDropCaps{твердая реши­мость‎}

\textstyleCaptioncharacters{بِعَزِيمَةٍ }\textstyleDropCaps{решительно‎}

\textstyleCaptioncharacters{نَشَاطٌ }\textstyleDropCaps{активность‎}

\textstyleCaptioncharacters{بِنَشَاطٍ }\textstyleDropCaps{активно‎}

\textstyleCaptioncharacters{سَاقٌ }\textstyleDropCaps{стебель; ствол‎}

\textstyleCaptioncharacters{سَاقُ نَبَاتٍ }\textstyleDropCaps{стебель рас­тения‎}

\textstyleCaptioncharacters{دَابَّةٌ }\textstyleDropCaps{животное‎}

\textstyleCaptioncharacters{بَيْدَرٌ }\textstyleDropCaps{кумно, ток‎}

\textstyleCaptioncharacters{تَعِبَ }\textstyleDropCaps{устал‎}

\textstyleCaptioncharacters{أَقْعَى }\textstyleDropCaps{сел на задние лапы‎}

\textstyleCaptioncharacters{عَتَبَةٌ }\textstyleDropCaps{порог‎}

\textstyleCaptioncharacters{غَلَبَهُ النُّعَاسُ }\textstyleDropCaps{им овла­дел сон‎}

\textstyleCaptioncharacters{حَرَسَ }\textstyleDropCaps{охранял‎}

\textstyleCaptioncharacters{يَا حَبِيبِي }\textstyleDropCaps{о мой люби­мый‎}

\textstyleCaptioncharacters{كِسْرَةٌ }\textstyleDropCaps{кусок‎}

\textstyleCaptioncharacters{كَسْرَةُ خُبْزٍ }\textstyleDropCaps{кусок хлеба‎}

\textstyleCaptioncharacters{اِبْتَعَدَ عَنْ... }\textstyleDropCaps{удалился, отошел‎}

\textstyleCaptioncharacters{خَبِيثٌ }\textstyleDropCaps{скверный, мерз­кий, коварный‎}

\textstyleCaptioncharacters{فِي الخَارِجِ }\textstyleDropCaps{на улице, снаружи‎}

\textstyleCaptioncharacters{تَرَصَّدَ }\textstyleDropCaps{подстерегал, подкар­аулил, выслеживал‎}

\textstyleCaptioncharacters{عَمِلَ بِـ... }\textstyleDropCaps{действовал, по­ступил согласно чему-л., выполнил‎}

\textstyleCaptioncharacters{عَمِلَ بِالنَّصِيحَةِ }\textstyleDropCaps{посту­пил согласно совету, последовал совету‎}

\textstyleCaptioncharacters{عَمِلَ بِكِتَابِ اللَّهِ }\textstyleDropCaps{дей­ствавал согласно писанию Бога‎}

\textstyleCaptioncharacters{تَسَاءَلَ }\textstyleDropCaps{задался вопросом‎}

\textstyleCaptioncharacters{فَتَّشَ عَنْ... }\textstyleDropCaps{искал‎}

\textstyleCaptioncharacters{فِي كُلِّ مَكَانٍ }\textstyleDropCaps{везде, всюду‎}

\textstyleCaptioncharacters{أَخِيرًا }\textstyleDropCaps{наконец‎}

\textstyleCaptioncharacters{جَزَاءٌ }\textstyleDropCaps{возмездие, наказа­ние‎}

\subsection[Урок 102‎]{\textstyleDropCaps{Урок 102‎}}
\textstyleCaptioncharacters{حَفَرَ }\textstyleDropCaps{копал‎}

\textstyleCaptioncharacters{دَفَنَ }\textstyleDropCaps{зарыл, закопал‎}

\textstyleCaptioncharacters{أَظَلَّ }\textstyleDropCaps{покрыл тенью, дал тень‎}

\textstyleCaptioncharacters{أَقْشَعَ }\textstyleDropCaps{рассеялся‎}

\textstyleCaptioncharacters{أَقْشَعَتِ السَّحَابَةُ }\textstyleDropCaps{рас­сеялось облако‎}

\textstyleCaptioncharacters{سَحَابَةٌ }\textstyleDropCaps{облако, туча‎}

\textstyleCaptioncharacters{عَلاَمَةٌ }\textstyleDropCaps{знак, метка‎}

\textstyleCaptioncharacters{أَضَاعَ }\textstyleDropCaps{губил, терял, тратил‎}

\textstyleCaptioncharacters{مِسْكِينٌ }\textstyleDropCaps{бедный, бедняга‎}

\textstyleCaptioncharacters{جَسَّ }\textstyleDropCaps{щупал‎}

\textstyleCaptioncharacters{نَبْضٌ }\textstyleDropCaps{пульс‎}

\textstyleCaptioncharacters{جَسَّ النَّبْضَ }\textstyleDropCaps{щупал пульс‎}

\textstyleCaptioncharacters{قَاسَ }\textstyleDropCaps{мерил, измерил‎}

\textstyleCaptioncharacters{دَرَجَةُ الحَرَارَةِ }\textstyleDropCaps{темпера­тура‎}

\textstyleCaptioncharacters{قَاسَ دَرَجَةَ الحَرَارَةِ }\textstyleDropCaps{мерил температуру‎}

\textstyleCaptioncharacters{اِسْتَطَاعَ }\textstyleDropCaps{смог, был в со­стоянии‎}

\textstyleCaptioncharacters{مَرْبُوطٌ }\textstyleDropCaps{связанный, при­вязанный‎}

\textstyleCaptioncharacters{مِنَ الدَّاخِلِ }\textstyleDropCaps{внутри, из­нутри‎}

\textstyleCaptioncharacters{مَشْغُولُ البَالِ }\textstyleDropCaps{озабо­ченный‎}

\textstyleCaptioncharacters{اِنْتَبَهَ لِـ... }\textstyleDropCaps{заметил, обра­тил внимание‎}

\textstyleCaptioncharacters{غُلاَمٌ }\textstyleDropCaps{мальчик, парень‎}

\textstyleCaptioncharacters{لَقَطَ }\textstyleDropCaps{подобрал, поднял с земли‎}

\textstyleCaptioncharacters{لَحِقَ }\textstyleDropCaps{догнал‎}

\textstyleCaptioncharacters{كَافَأَ }\textstyleDropCaps{вознаградил‎}

\textstyleCaptioncharacters{أَبَى }\textstyleDropCaps{отверг, отказал‎}

\subsection[Урок 103‎]{\textstyleDropCaps{Урок 103‎}}
\textstyleCaptioncharacters{جَوْزَةٌ }\textstyleDropCaps{орех (один)‎}

\textstyleCaptioncharacters{ظَنَّ }\textstyleDropCaps{думал, полагал‎}

\textstyleCaptioncharacters{عَضَّ }\textstyleDropCaps{укусил‎}

\textstyleCaptioncharacters{مُرٌّ }\textstyleDropCaps{горький‎}

\textstyleCaptioncharacters{مَا أَمَرَّ هَذِهِ التُّفَّاحَةَ }\textstyleDropCaps{какое горькое это яблоко!‎}

\textstyleCaptioncharacters{أَكْبَرُ مِنْهُ سِنًّا }\textstyleDropCaps{старше его‎}

\textstyleCaptioncharacters{قَشَرَ }\textstyleDropCaps{снял кожицу, корку, очистил‎}

\textstyleCaptioncharacters{قِشْرٌ }\textstyleDropCaps{корка, кожица, кожу­ра‎}

\textstyleCaptioncharacters{لُبٌّ }\textstyleDropCaps{сердцевина, ядро‎}

\textstyleCaptioncharacters{أَبْقَى }\textstyleDropCaps{оставил‎}

\textstyleCaptioncharacters{ضَجِرَ مِنْ ... }\textstyleDropCaps{испытывал досаду, скучал‎}

\textstyleCaptioncharacters{وَحْدَةٌ }\textstyleDropCaps{одиночество‎}

\textstyleCaptioncharacters{ضَجِرَ مِنْ وَحْدَتِهِ }\textstyleDropCaps{ску­чал от одиночества‎}

\textstyleCaptioncharacters{أَلْبَسَ }\textstyleDropCaps{одел‎}

\textstyleCaptioncharacters{أَلْبَسَ الطِّفْلَ لِبَاسًا }\textstyleDropCaps{одел ребенка в одежду‎}

\textstyleCaptioncharacters{صَارَ }\textstyleDropCaps{стал, становился‎}

\textstyleCaptioncharacters{صَارَ كَبِيرًا }\textstyleDropCaps{стал большой‎}

\textstyleCaptioncharacters{لَوْزٌ }\textstyleDropCaps{миндаль‎}

\textstyleCaptioncharacters{شَجَرَةُ اللَّوْزِ }\textstyleDropCaps{миндальное дерево‎}

\textstyleCaptioncharacters{تَأَلَّمَ }\textstyleDropCaps{мучился, страдал, чув­ствовал боль‎}

\textstyleCaptioncharacters{صَدَقَ }\textstyleDropCaps{сказал правду‎}

\textstyleCaptioncharacters{آسِفٌ }\textstyleDropCaps{сожалеющий‎}

\textstyleCaptioncharacters{إِنَّنِي آسِفٌ جِدًّا }\textstyleDropCaps{поис­тине, я очень сожалею‎}

\subsection[Урок 104‎]{\textstyleDropCaps{Урок 104‎}}
\textstyleCaptioncharacters{أَصْفَرُ }\textstyleDropCaps{желтый‎}

\textstyleCaptioncharacters{وَ إِلاَّ فَلاَ }\textstyleDropCaps{если нет, то нет‎}

\textstyleCaptioncharacters{هُنَاكَ }\textstyleDropCaps{имеется, существу­ет‎}

\textstyleCaptioncharacters{طَرِيقَةٌ }\textstyleDropCaps{метод, способ‎}

\textstyleCaptioncharacters{هُنَاكَ طَرِيقَةٌ }\textstyleDropCaps{есть способ‎}

\textstyleCaptioncharacters{رَسَبَ }\textstyleDropCaps{осел, опустился вниз‎}

\textstyleCaptioncharacters{طَفَا }\textstyleDropCaps{держался на воде, не тонул, плавал‎}

\textstyleCaptioncharacters{فَحَصَ }\textstyleDropCaps{осмотрел, обсле­довал‎}

\textstyleCaptioncharacters{قَلْبٌ }\textstyleDropCaps{сердце‎}

\textstyleCaptioncharacters{حَلْقٌ }\textstyleDropCaps{горло‎}

\textstyleCaptioncharacters{أَقْبَلَ }\textstyleDropCaps{пришел, подошел, приблизился‎}

\textstyleCaptioncharacters{فَاسِدٌ }\textstyleDropCaps{испорченный, не­годный, гнилой‎}

\textstyleCaptioncharacters{جِسْمٌ }\textstyleDropCaps{тело‎}

\textstyleCaptioncharacters{صَحِيحُ الجِسْمِ }\textstyleDropCaps{здоро­вый, со здоровым телом‎}

\textstyleCaptioncharacters{شُجَاعٌ }\textstyleDropCaps{храбрый, смелый‎}

\textstyleCaptioncharacters{هَا هُوَ ذَا }\textstyleDropCaps{вот он‎}

\textstyleCaptioncharacters{خَجِلَ }\textstyleDropCaps{стыдился, сконфу­зился‎}

\textstyleCaptioncharacters{أَعْلَى... }\textstyleDropCaps{вершина, мокуш­ка‎}

\textstyleCaptioncharacters{أَعْلَى الشَّجَرَةِ }\textstyleDropCaps{макушка дерева‎}

\textstyleCaptioncharacters{تَنَاوَلَ }\textstyleDropCaps{взял‎}

\textstyleCaptioncharacters{رَمَى بِحَجَرٍ }\textstyleDropCaps{бросил кам­нем, швырял, ударил камнем‎}

\textstyleCaptioncharacters{يَسُرُّنِي }\textstyleDropCaps{я рад, меня радует‎}

\textstyleCaptioncharacters{يَسُرُّنِي أَنِّي مَا رَمَيْتُهُ بِحَجَرٍ }\textstyleDropCaps{я рад, что не бросил в него камнем ‎}

\textstyleCaptioncharacters{تَسَمَّعَ إِلَى... }\textstyleDropCaps{прислуши­вался‎}

\subsection[Урок 105‎]{\textstyleDropCaps{Урок 105‎}}
\textstyleCaptioncharacters{قُفَّةٌ }\textstyleDropCaps{корзина‎}

\textstyleCaptioncharacters{شَاطِئٌ }\textstyleDropCaps{берег‎}

\textstyleCaptioncharacters{شَاطِئُ البَحْرِ }\textstyleDropCaps{берег моря‎}

\textstyleCaptioncharacters{رَمْلٌ }\textstyleDropCaps{песок‎}

\textstyleCaptioncharacters{نَزَلَ إِلَى النَّهْرِ }\textstyleDropCaps{спустил­ся к реке‎}

\textstyleCaptioncharacters{خَلَعَ ثِيَابَهُ }\textstyleDropCaps{разделся, снял одежду‎}

\textstyleCaptioncharacters{اِسْتَحَمَّ }\textstyleDropCaps{купался, принял ванну‎}

\textstyleCaptioncharacters{هَلْ أَحْبَبْتَ؟ }\textstyleDropCaps{тебе по­нравилось?‎}

\textstyleCaptioncharacters{يَظْهَرُ }\textstyleDropCaps{видно, по-видимо­му‎}

\textstyleCaptioncharacters{أَفْهَمَ }\textstyleDropCaps{объяснил, дал по­нять‎}

\textstyleCaptioncharacters{مِلْحٌ }\textstyleDropCaps{соленый‎}

\textstyleCaptioncharacters{مَاءٌ مِلْحٌ }\textstyleDropCaps{соленая вода‎}

\textstyleCaptioncharacters{مُفَتِّشٌ }\textstyleDropCaps{контролер, кон­дуктор‎}

\textstyleCaptioncharacters{مُفَتِّشُ سِكَّةِ الحَدِيدِ }\textstyleDropCaps{железнодорожный контролер‎}

\textstyleCaptioncharacters{خَطَرٌ }\textstyleDropCaps{опасность‎}

\textstyleCaptioncharacters{مُعَيَّنٌ }\textstyleDropCaps{назначенный, опре­деленный‎}

\textstyleCaptioncharacters{فِي الوَقْتِ المُعَيَّنِ }\textstyleDropCaps{в назначенное время‎}

\textstyleCaptioncharacters{سَاعَةٌ }\textstyleDropCaps{час, момент‎}

\textstyleCaptioncharacters{قَبَضَ }\textstyleDropCaps{взял, получил‎}

\textstyleCaptioncharacters{قَبَضَ الأَجْرَ }\textstyleDropCaps{получил плату‎}

\textstyleCaptioncharacters{أَجْرٌ }\textstyleDropCaps{плата‎}

\textstyleCaptioncharacters{مَحَطَّةٌ }\textstyleDropCaps{станция, останов­ка‎}

\textstyleCaptioncharacters{تَقَدَّمَ }\textstyleDropCaps{опередил‎}

\textstyleCaptioncharacters{تَأَخَّرَ }\textstyleDropCaps{опоздал‎}

\textstyleCaptioncharacters{بَائِعٌ }\textstyleDropCaps{продавец, торговец‎}

\textstyleCaptioncharacters{بَاعَةُُ الصُّحُفِ }\textstyleDropCaps{продавец журналов и газет‎}

\textstyleCaptioncharacters{رَافَقَ }\textstyleDropCaps{сопроводил, пошел, поехал вместе‎}

\textstyleCaptioncharacters{رَافَقَ وَالِدَهُ إِلَى السُّوقِ }\textstyleDropCaps{пошел с отцом на базар ‎}

\textstyleCaptioncharacters{رَطْلٌ }\textstyleDropCaps{ратль (мера веса, разная в разных странах)‎}

\textstyleCaptioncharacters{حُزْمَةٌ }\textstyleDropCaps{пучок, связка‎}

\textstyleCaptioncharacters{حُزْمَةٌ مِنَ البَصَلِ }\textstyleDropCaps{пу­чок лука‎}

\textstyleCaptioncharacters{سِلْقٌ }\textstyleDropCaps{турнепс‎}

\textstyleCaptioncharacters{اِمْتَلَأَ }\textstyleDropCaps{наполнился, запол­нился‎}

\textstyleCaptioncharacters{أَفْرَغَ }\textstyleDropCaps{опорожнил, вылил, высыпал‎}

\subsection[Урок 106‎]{\textstyleDropCaps{Урок 106‎}}
\textstyleCaptioncharacters{عِصَابَةٌ }\textstyleDropCaps{повязка‎}

\textstyleCaptioncharacters{دَوَّرَ }\textstyleDropCaps{крутил‎}

\textstyleCaptioncharacters{دَفَعَ }\textstyleDropCaps{толкнул‎}

\textstyleCaptioncharacters{رَقَصَ }\textstyleDropCaps{танцевал, плясал‎}

\textstyleCaptioncharacters{صَفَّقَ }\textstyleDropCaps{хлопал‎}

\textstyleCaptioncharacters{زِدْ }\textstyleDropCaps{еще, давай еще‎}

\textstyleCaptioncharacters{زِدِ اقْتَرِبْ }\textstyleDropCaps{еще прибли­жайся‎}

\textstyleCaptioncharacters{لَمْ تُصِبْ }\textstyleDropCaps{не попал (ты), ошибся, промахнулся (ты)‎}

\textstyleCaptioncharacters{هَيَّا أَمْسِكْنَا }\textstyleDropCaps{давай лови нас‎}

\textstyleCaptioncharacters{اِلْتَحَقَ بِـ... }\textstyleDropCaps{догнал‎}

\textstyleCaptioncharacters{مَدَّ }\textstyleDropCaps{протянул‎}

\textstyleCaptioncharacters{مَدَّ يَدَهُ }\textstyleDropCaps{протянул руку‎}

\textstyleCaptioncharacters{تَحَسَّسَ }\textstyleDropCaps{ощупывал, на­щупывал‎}

\textstyleCaptioncharacters{زَلْزَلَةٌ }\textstyleDropCaps{землетрясение‎}

\textstyleCaptioncharacters{زُلْزِلَتِ الأَرْضُ }\textstyleDropCaps{сотряса­лась земля, произошло землетрясение‎}

\textstyleCaptioncharacters{فَائِدَةٌ }\textstyleDropCaps{польза‎}

\textstyleCaptioncharacters{بِدُونِ فَائِدَةٍ }\textstyleDropCaps{бесполезно‎}

\textstyleCaptioncharacters{بَيْنَمَا... }\textstyleDropCaps{в то время, как…; в то время, когда…‎}

\textstyleCaptioncharacters{كَانَ الفَصْلُ صَيْفًا }\textstyleDropCaps{было лето, время было летом‎}

\textstyleCaptioncharacters{تَزَاحَمَ }\textstyleDropCaps{теснился, толкался‎}

\textstyleCaptioncharacters{تَصَدَّعَ }\textstyleDropCaps{трескался, дал тре­щины‎}

\textstyleCaptioncharacters{جِدَارٌ }\textstyleDropCaps{стена‎}

\textstyleCaptioncharacters{تَصَدَّعَتِ الجُدْرَانُ }\textstyleDropCaps{по­трескались стены‎}

\textstyleCaptioncharacters{تَسَاقَطَ }\textstyleDropCaps{выпал, осыпался, обвалился‎}

\textstyleCaptioncharacters{سَقْفٌ }\textstyleDropCaps{потолок, крыша‎}

\textstyleCaptioncharacters{أَذًى }\textstyleDropCaps{неприятность, вред, ущерб‎}

\textstyleCaptioncharacters{مَا أُصِيبَ أَحَدٌ بِأَذًى }\textstyleDropCaps{никто не пострадал ‎}

\textstyleCaptioncharacters{مَدْرَسِيٌّ }\textstyleDropCaps{школьный, учеб­ный‎}

\textstyleCaptioncharacters{الكِتَابُ المَدْرَسِيُّ }\textstyleDropCaps{учеб­ник‎}

\textstyleCaptioncharacters{خِلاَلَ الدَّرْسِ }\textstyleDropCaps{на уроке, во время урока‎}

\textstyleCaptioncharacters{حِجَارَةٌ }\textstyleDropCaps{камни‎}

\subsection[Урок 107‎]{\textstyleDropCaps{Урок 107‎}}
\textstyleCaptioncharacters{مَرَّ بِالجِبَلِ }\textstyleDropCaps{прошел мимо горы‎}

\textstyleCaptioncharacters{أَتُرِيدُ مِنِّي؟ }\textstyleDropCaps{хочешь, чтобы я?‎}

\textstyleCaptioncharacters{عَارٍ }\textstyleDropCaps{голый, нагой‎}

\textstyleCaptioncharacters{مَشَى عَارِيًا }\textstyleDropCaps{ходил го­лый‎}

\textstyleCaptioncharacters{تَبَارَى }\textstyleDropCaps{соревновался, со­стязался‎}

\textstyleCaptioncharacters{فِرْقَةٌ }\textstyleDropCaps{команда‎}

\textstyleCaptioncharacters{فِرَقُ الأَلْعَابِ الرِّيَاضِيَّةِ }\textstyleDropCaps{спортивные команды ‎}

\textstyleCaptioncharacters{جُمْهُورٌ }\textstyleDropCaps{толпа, масса‎}

\textstyleCaptioncharacters{إِبْرَةٌ }\textstyleDropCaps{иголка‎}

\textstyleCaptioncharacters{عَيْنُ الإِبْرَةِ }\textstyleDropCaps{ушко иголки‎}

\textstyleCaptioncharacters{قَمِيصٌ }\textstyleDropCaps{рубашка, соочка‎}

\textstyleCaptioncharacters{جَانِبٌ }\textstyleDropCaps{сторона‎}

\textstyleCaptioncharacters{حَكَمٌ }\textstyleDropCaps{судья, арбитр‎}

\textstyleCaptioncharacters{صَفَرَ }\textstyleDropCaps{свистел‎}

\textstyleCaptioncharacters{ابْتَدَأَ }\textstyleDropCaps{начал‎}

\textstyleCaptioncharacters{اِبْتَدَأَ اللَّعِبَ }\textstyleDropCaps{начал игру‎}

\textstyleCaptioncharacters{حَاوَلَ جُهْدَهُ }\textstyleDropCaps{пытался что есть мочи‎}

\textstyleCaptioncharacters{تَسَاوَى }\textstyleDropCaps{сравнялся‎}

\textstyleCaptioncharacters{تَسَاوَتِ الفِرْقَتَانِ }\textstyleDropCaps{обе команды сравнялись‎}

\textstyleCaptioncharacters{إِصَابَةُ الأَهْدَافِ }\textstyleDropCaps{попада­ние в цель, гол‎}

\textstyleCaptioncharacters{اِدَّعَى }\textstyleDropCaps{утверждал, претен­довал ‎}

\textstyleCaptioncharacters{هَتَفَ }\textstyleDropCaps{кричал, сопрово­ждал криками, овациями‎}

\textstyleCaptioncharacters{عَجَزَ عَنْ... }\textstyleDropCaps{не мог, не умел, был не в состоянии‎}

\textstyleCaptioncharacters{خَلِيقَةٌ }\textstyleDropCaps{народ, люди‎}

\textstyleCaptioncharacters{هَاتِ }\textstyleDropCaps{давай, давай сюда‎}

\textstyleCaptioncharacters{أُنْبُوبَةٌ }\textstyleDropCaps{трубка‎}

\textstyleCaptioncharacters{قَامَ عَلَى قَدَمَيْهِ }\textstyleDropCaps{стал на ноги‎}

\textstyleCaptioncharacters{جَعَلَ يَصِيحُ }\textstyleDropCaps{начал кри­чать‎}

\textstyleCaptioncharacters{مَهَارَةٌ }\textstyleDropCaps{мастерство, искус­ство, навык‎}

\textstyleCaptioncharacters{قَامَةٌ }\textstyleDropCaps{рост‎}

\textstyleCaptioncharacters{مِنْ قَامَتِهِ }\textstyleDropCaps{со стойки‎}

\textstyleCaptioncharacters{أَضَاعَ }\textstyleDropCaps{губил, тратил, терял‎}

\textstyleCaptioncharacters{عَبَثًا }\textstyleDropCaps{зря, напрасно‎}

\textstyleCaptioncharacters{أَفَادَ }\textstyleDropCaps{дал, принес пользу‎}

\textstyleCaptioncharacters{عَمَلٌ لاَ يُفِيدُ }\textstyleDropCaps{бесполез­ное дело‎}

\subsection[Урок 108‎]{\textstyleDropCaps{Урок 108‎}}
\textstyleCaptioncharacters{طَرِيقُ السَّيَّارَاتِ }\textstyleDropCaps{авто­трасса‎}

\textstyleCaptioncharacters{خَطِرٌ }\textstyleDropCaps{опасный‎}

\textstyleCaptioncharacters{صَارَ يَلْعَبُ }\textstyleDropCaps{начал, стал иг­рать‎}

\textstyleCaptioncharacters{أَحَدُهُمْ }\textstyleDropCaps{один из них, кто-то из них‎}

\textstyleCaptioncharacters{أَحَدُنَا }\textstyleDropCaps{один из нас‎}

\textstyleCaptioncharacters{لَزِمَ }\textstyleDropCaps{держался, придержал­ся‎}

\textstyleCaptioncharacters{لَزِمَ جَانِبَ الطَّرِيقِ }\textstyleDropCaps{дер­жался краешка дороги ‎}

\textstyleCaptioncharacters{آتٍ }\textstyleDropCaps{идущий‎}

\textstyleCaptioncharacters{شِمَالٌ }\textstyleDropCaps{левая сторона‎}

\textstyleCaptioncharacters{اِلْتَفَتَ شِمَالاً }\textstyleDropCaps{повернулс­я налево‎}

\textstyleCaptioncharacters{وَقَفَ }\textstyleDropCaps{стал, остановился‎}

\textstyleCaptioncharacters{رَيْثَمَا... }\textstyleDropCaps{пока, пока не…‎}

\textstyleCaptioncharacters{قَدَمٌ }\textstyleDropCaps{ступня, нога‎}

\textstyleCaptioncharacters{عَبَرَ }\textstyleDropCaps{пересек, перешел‎}

\textstyleCaptioncharacters{فَرَكَ }\textstyleDropCaps{потер, протер‎}

\textstyleCaptioncharacters{مَشَّطَ }\textstyleDropCaps{причесал, расчесал‎}

\textstyleCaptioncharacters{مَشَّطَ شَعَرَهُ }\textstyleDropCaps{причесался‎}

\textstyleCaptioncharacters{مِنْدِيلٌ }\textstyleDropCaps{носовой платок‎}

\textstyleCaptioncharacters{حَافَظَ عَلَى النَّظَافَةِ }\textstyleDropCaps{соблюдал чистоту‎}

\textstyleCaptioncharacters{مَرَّنَ }\textstyleDropCaps{тренировал, приучал‎}

\textstyleCaptioncharacters{يَرُوحُ وَ يَجِيءُ }\textstyleDropCaps{ходит туда-сюда‎}

\textstyleCaptioncharacters{كَانَ يَرُوحُ وَ يَجِيءُ }\textstyleDropCaps{хо­дил туда-сюда ‎}

\textstyleCaptioncharacters{أَعْمَالُ البَيْتِ }\textstyleDropCaps{домаш­ние дела, работы‎}

\textstyleCaptioncharacters{كَوَى }\textstyleDropCaps{гладил, утюжил‎}

\textstyleCaptioncharacters{كَوَى الثِّيَابَ }\textstyleDropCaps{гладил оде­жду‎}

\textstyleCaptioncharacters{عَلَفَ }\textstyleDropCaps{дал корм, кормил‎}

\textstyleCaptioncharacters{عَلَفَ الدَّجَاجَ }\textstyleDropCaps{кормил кур‎}

\textstyleCaptioncharacters{صَاحِبَةٌ }\textstyleDropCaps{подруга‎}

\textstyleCaptioncharacters{فِي سُرْعَةٍ }\textstyleDropCaps{быстро‎}

\subsection[Урок 109‎]{\textstyleDropCaps{Урок 109‎}}
\textstyleCaptioncharacters{أَبْلَهُ }\textstyleDropCaps{глупый, тупой‎}

\textstyleCaptioncharacters{القَاهِرَةُ }\textstyleDropCaps{Каир‎}

\textstyleCaptioncharacters{اليَابَانُ }\textstyleDropCaps{Япония‎}

\textstyleCaptioncharacters{تَفَكَّرَ }\textstyleDropCaps{подумал‎}

\textstyleCaptioncharacters{سَافَرَ }\textstyleDropCaps{уехал, поехал‎}

\textstyleCaptioncharacters{سَافَرَ بِالقِطَارِ }\textstyleDropCaps{уехал поез­дом‎}

\textstyleCaptioncharacters{اِتَّبَعَ }\textstyleDropCaps{следовал‎}

\textstyleCaptioncharacters{اِتَّبَعَ طَرِيقًا }\textstyleDropCaps{следовал, по­шел дорогой‎}

\textstyleCaptioncharacters{سَفِينَةٌ }\textstyleDropCaps{корабль, судно‎}

\textstyleCaptioncharacters{السُّوَيْسُ }\textstyleDropCaps{Суэц‎}

\textstyleCaptioncharacters{رُبَّانٌ }\textstyleDropCaps{капитан (корабля)‎}

\textstyleCaptioncharacters{رُبَّانُ السَّفِينَةِ }\textstyleDropCaps{капитан корабля‎}

\textstyleCaptioncharacters{رَصِيفٌ }\textstyleDropCaps{тротуар, мостовая‎}

\textstyleCaptioncharacters{مِرَارًا }\textstyleDropCaps{несколько раз, неод­нократно‎}

\textstyleCaptioncharacters{كُلَّمَا }\textstyleDropCaps{всякий раз, как; вся­кий раз, когда‎}

\textstyleCaptioncharacters{كُلَّمَا أَرَادَ أَنْ يَخْرُجَ }\textstyleDropCaps{всякий раз как захочет выйти ‎}

\textstyleCaptioncharacters{أَخَذَ بِيَدِهِ }\textstyleDropCaps{взял за руку‎}

\textstyleCaptioncharacters{عَابِرٌ }\textstyleDropCaps{проходящий, проез­жий, проезжающий‎}

\textstyleCaptioncharacters{سَيَّارَةٌ عَابِرَةٌ }\textstyleDropCaps{проходя­щая машина‎}

\textstyleCaptioncharacters{عَلَى يَمِينِهِ }\textstyleDropCaps{справа от него‎}

\textstyleCaptioncharacters{عَلَى يَسَارِهِ }\textstyleDropCaps{слева от него‎}

\textstyleCaptioncharacters{حَرَكَةٌ }\textstyleDropCaps{движение‎}

\textstyleCaptioncharacters{حَرَكَةُ السَّيَّارَاتِ }\textstyleDropCaps{авто­мобильное движение‎}

\textstyleCaptioncharacters{أَوْصَلَ }\textstyleDropCaps{отвел, отвез, доста­вил‎}

\textstyleCaptioncharacters{دَفْتَرُ الإِمْلاَءِ }\textstyleDropCaps{тетрадь диктанта‎}

\textstyleCaptioncharacters{خَطٌّ }\textstyleDropCaps{почерк‎}

\textstyleCaptioncharacters{وَاضِحٌ }\textstyleDropCaps{ясный‎}

\textstyleCaptioncharacters{خَطٌّ \ }\textstyleDropCaps{ясный почерк‎}

\textstyleCaptioncharacters{أَثْنَى عَلَى... }\textstyleDropCaps{похвалил‎}

\textstyleCaptioncharacters{كَرِيمُ الأَخْلاَقِ }\textstyleDropCaps{благо­нравный, с благородным характером‎}

\textstyleCaptioncharacters{بَعْدَ أَنْ... }\textstyleDropCaps{после того, как‎}

\subsection[Урок 110‎]{\textstyleDropCaps{Урок 110‎}}
\textstyleCaptioncharacters{قَمْحَةٌ }\textstyleDropCaps{пшеничное зерно‎}

\textstyleCaptioncharacters{نَمْلَةٌ }\textstyleDropCaps{муравей‎}

\textstyleCaptioncharacters{قَرْيَةُ النَّمْلِ }\textstyleDropCaps{муравейник‎}

\textstyleCaptioncharacters{صِنْفٌ }\textstyleDropCaps{сорт‎}

\textstyleCaptioncharacters{رِزْقٌ }\textstyleDropCaps{пропитание, средства к существованию‎}

\textstyleCaptioncharacters{طَلَبَ الرِّزْقَ }\textstyleDropCaps{искал про­питание‎}

\textstyleCaptioncharacters{سَعَى }\textstyleDropCaps{старался, стремил­ся, добивался‎}

\textstyleCaptioncharacters{سَعَى فِي طَلَبِ الرِّزْقِ }\textstyleDropCaps{стремился искать пропитание ‎}

\textstyleCaptioncharacters{سَعَى فِي طَلَبِ العِلْمِ }\textstyleDropCaps{стремился искать знания ‎}

\textstyleCaptioncharacters{رَفِيقَةٌ }\textstyleDropCaps{подруга‎}

\textstyleCaptioncharacters{عَائِلَةٌ }\textstyleDropCaps{семья‎}

\textstyleCaptioncharacters{قَرْنٌ }\textstyleDropCaps{рог‎}

\textstyleCaptioncharacters{حَكَّ }\textstyleDropCaps{тёр‎}

\textstyleCaptioncharacters{حَكَّ يَدَهُ بِشَيْءٍ }\textstyleDropCaps{тёр свою руку о что-л. ‎}

\textstyleCaptioncharacters{كَأَنَّ... }\textstyleDropCaps{как будто, как если бы‎}

\textstyleCaptioncharacters{كَأَنَّهُ يَقُولُ }\textstyleDropCaps{как будто он говорит‎}

\textstyleCaptioncharacters{بَرِّيَّةٌ }\textstyleDropCaps{пустыня‎}

\textstyleCaptioncharacters{فَرَشَ }\textstyleDropCaps{постелил‎}

\textstyleCaptioncharacters{زُرْبِيَّةٌ }\textstyleDropCaps{ковёр‎}

\textstyleCaptioncharacters{تَعَهَّدَ بِـ... }\textstyleDropCaps{обязался, взял на себя‎}

\textstyleCaptioncharacters{طَرَفٌ }\textstyleDropCaps{край, конец‎}

\textstyleCaptioncharacters{حَطَبٌ }\textstyleDropCaps{дрова‎}

\textstyleCaptioncharacters{أَوْقَدَ }\textstyleDropCaps{зажёг, разжёг‎}

\textstyleCaptioncharacters{نَضِجَ }\textstyleDropCaps{сварился, прожарил­ся‎}

\textstyleCaptioncharacters{نَضِجَ الغَدَاءُ }\textstyleDropCaps{обед сва­рился, обед готов‎}

\textstyleCaptioncharacters{شَهِيَّةٌ }\textstyleDropCaps{аппетит‎}

\textstyleCaptioncharacters{أَكَلَ بِشَهِيَّةٍ }\textstyleDropCaps{покушал с аппетитом‎}

\textstyleCaptioncharacters{تَمَدَّدَ }\textstyleDropCaps{растянулся, зава­лился‎}

\textstyleCaptioncharacters{تَمَدَّدَ لِلرَّاحَةِ }\textstyleDropCaps{завалился отдыхать‎}

\textstyleCaptioncharacters{دَبَشٌ }\textstyleDropCaps{хлам, тряпье, рух­лядь‎}

\textstyleCaptioncharacters{جَمَعَ أَدْبَاشَهُ }\textstyleDropCaps{собрал свой хлам‎}

\textstyleCaptioncharacters{تَمَتَّعَ بِـ... }\textstyleDropCaps{наслаждался, пользовался‎}

\textstyleCaptioncharacters{نُزْهَةٌ }\textstyleDropCaps{прогулка, экскурсия‎}

\textstyleCaptioncharacters{تَمَتَّعَ بِالنُّزْهَةِ }\textstyleDropCaps{насла­ждался прогулкой ‎}

\textstyleCaptioncharacters{قِرْشٌ }\textstyleDropCaps{пиастр‎}

\textstyleCaptioncharacters{ثَمَنٌ }\textstyleDropCaps{цена, стоимость‎}

\textstyleCaptioncharacters{كَمْ ثَمَنُهُ؟ }\textstyleDropCaps{какая у него цена? сколько он стоит?‎}

\subsection[Урок 111‎]{\textstyleDropCaps{Урок 111‎}}
\textstyleCaptioncharacters{حَدَّادٌ }\textstyleDropCaps{кузнец‎}

\textstyleCaptioncharacters{لاَعِبٌ }\textstyleDropCaps{игрок‎}

\textstyleCaptioncharacters{مَدْرَجٌ }\textstyleDropCaps{трибуны‎}

\textstyleCaptioncharacters{قَالَ لِنَفْسِهِ }\textstyleDropCaps{сказал про себя‎}

\textstyleCaptioncharacters{عَجَبًا }\textstyleDropCaps{удивительно‎}

\textstyleCaptioncharacters{صَوْتٌ عَالٍ }\textstyleDropCaps{громкий го­лос, громкий звук‎}

\textstyleCaptioncharacters{خَفِيٌّ }\textstyleDropCaps{скрытый, незамет­ный‎}

\textstyleCaptioncharacters{صَوْتٌ خَفِيٌّ }\textstyleDropCaps{скрытый звук‎}

\textstyleCaptioncharacters{عَاوَنَ }\textstyleDropCaps{помог, способства­вал‎}

\textstyleCaptioncharacters{مَضَغَ }\textstyleDropCaps{жевал‎}

\textstyleCaptioncharacters{مَضَغَ الطَّعَامَ }\textstyleDropCaps{жевал пищу‎}

\textstyleCaptioncharacters{بَعْدَ بُرْهَةٍ مِنَ الزَّمَانِ }\textstyleDropCaps{через некоторе время ‎}

\textstyleCaptioncharacters{قَدِمَ }\textstyleDropCaps{прибыл‎}

\textstyleCaptioncharacters{قَدِمَ المُسَافِرُ }\textstyleDropCaps{путеше­ственник прибыл‎}

\textstyleCaptioncharacters{وَاحِدًا إِثْرَ وَاحِدٍ }\textstyleDropCaps{один за другим ‎}

\textstyleCaptioncharacters{مُمَرِّنٌ }\textstyleDropCaps{тренер‎}

\textstyleCaptioncharacters{مُقَابَلَةٌ }\textstyleDropCaps{встреча‎}

\textstyleCaptioncharacters{اِبْتَدَأَتِ المُقَابَلَةُ }\textstyleDropCaps{встре­ча началась‎}

\textstyleCaptioncharacters{مُتَفَرِّجٌ }\textstyleDropCaps{зритель‎}

\textstyleCaptioncharacters{فِنَاءٌ }\textstyleDropCaps{двор‎}

\textstyleCaptioncharacters{مُبَارَاةٌ }\textstyleDropCaps{соревнование‎}

\textstyleCaptioncharacters{شَيِّقٌ }\textstyleDropCaps{интересный, занимат­ельный, увлекательный‎}

\textstyleCaptioncharacters{مُبَارَاةٌ شَيِّقَةٌ }\textstyleDropCaps{интерес­ное соревнование‎}

\textstyleCaptioncharacters{تَقَاتَلَ }\textstyleDropCaps{бился, сражался‎}

\textstyleCaptioncharacters{غَالِبٌ }\textstyleDropCaps{победивший, побе­дитель‎}

\textstyleCaptioncharacters{مَغْلُوبٌ }\textstyleDropCaps{побежденный‎}

\textstyleCaptioncharacters{فِنَاءُ الدَّارِ }\textstyleDropCaps{домашний двор‎}

\textstyleCaptioncharacters{مَضَى }\textstyleDropCaps{ушел‎}

\textstyleCaptioncharacters{مِنْ سَاعَتِهِ }\textstyleDropCaps{в тот же час‎}

\textstyleCaptioncharacters{مِنْ لَيْلَتِهِ }\textstyleDropCaps{в ту же ночь‎}

\textstyleCaptioncharacters{مَأْوًى }\textstyleDropCaps{приют, кров‎}

\textstyleCaptioncharacters{صَعِدَ }\textstyleDropCaps{поднялся, взобрал­ся, влез‎}

\textstyleCaptioncharacters{صَعِدَ الجَبَلَ }\textstyleDropCaps{поднялся на гору‎}

\textstyleCaptioncharacters{جَارِحَةٌ }\textstyleDropCaps{хищная птица‎}

\textstyleCaptioncharacters{بَصُرَ بِـ... }\textstyleDropCaps{увидел‎}

\textstyleCaptioncharacters{صَفَّقَ }\textstyleDropCaps{хлопал крыльями‎}

\textstyleCaptioncharacters{اِنْقَضَّ عَلَى... }\textstyleDropCaps{накинул­ся, набросился, внезапно напал‎}

\textstyleCaptioncharacters{جِسْرٌ }\textstyleDropCaps{мост‎}

\textstyleCaptioncharacters{ظِلٌّ }\textstyleDropCaps{тень‎}

\textstyleCaptioncharacters{اِخْتَطَفَ }\textstyleDropCaps{схватил, унес‎}

\subsection[Урок 112‎]{\textstyleDropCaps{Урок 112‎}}
\textstyleCaptioncharacters{مُلاَكِمٌ }\textstyleDropCaps{боксер‎}

\textstyleCaptioncharacters{مُصَارِعٌ }\textstyleDropCaps{борец‎}

\textstyleCaptioncharacters{كَرْمٌ }\textstyleDropCaps{виноград‎}

\textstyleCaptioncharacters{شَجَرَةُ الكَرْمِ }\textstyleDropCaps{виноград­ная лоза‎}

\textstyleCaptioncharacters{بَحَثَ فِي... }\textstyleDropCaps{обсуждал, рассмативал, разбирал‎}

\textstyleCaptioncharacters{قَضِيَّةٌ }\textstyleDropCaps{вопрос, проблема, дело‎}

\textstyleCaptioncharacters{بَحَثَ فِي قَضِيَّةٍ }\textstyleDropCaps{обсу­ждал вопрос‎}

\textstyleCaptioncharacters{فَرَّحَ }\textstyleDropCaps{обрадовал‎}

\textstyleCaptioncharacters{اَلْمُصْحَفُ }\textstyleDropCaps{Коран‎}

\textstyleCaptioncharacters{أَهْدَى }\textstyleDropCaps{подарил‎}

\textstyleCaptioncharacters{جُزْءٌ }\textstyleDropCaps{джуз (тридцатая часть Корана)‎}

\textstyleCaptioncharacters{فَرَحًا }\textstyleDropCaps{от радости‎}

\textstyleCaptioncharacters{فَرَحًا بِمَا فَعَلَ بِطَلَبٍ مِنِّي }\textstyleDropCaps{обрадовавшись тому, что он делал по моей просьбе‎}

\textstyleCaptioncharacters{تَمَّ لَهُ خَمْسُونَ سَنَةً }\textstyleDropCaps{ему исполнилось 50 лет ‎}

\textstyleCaptioncharacters{طَعَنَ فِي الخَمْسِينَ }\textstyleDropCaps{ему пошел пятидесятый (год) ‎}

\textstyleCaptioncharacters{كَلاَّ! }\textstyleDropCaps{нет! совсем нет!‎}

\textstyleCaptioncharacters{فِي رَبِيعِ هَذَا العَامِ }\textstyleDropCaps{весной этого года ‎}

\textstyleCaptioncharacters{هَكَذَا؟ }\textstyleDropCaps{так ли? разве так?‎}

\textstyleCaptioncharacters{بِكَثِيرٍ }\textstyleDropCaps{намного‎}

\textstyleCaptioncharacters{أَكْبَرُ بِكَثِيرٍ }\textstyleDropCaps{намного больше, намного старше‎}

\textstyleCaptioncharacters{أَصْغَرُ بِكَثِيرٍ }\textstyleDropCaps{намного младше, намного моложе‎}

\textstyleCaptioncharacters{يَبْدُو }\textstyleDropCaps{выглядит‎}

\textstyleCaptioncharacters{كَيْفَ تَرَى؟ }\textstyleDropCaps{как тебе нра­вится‎}

\textstyleCaptioncharacters{لاَ بَأْسَ بِهِ }\textstyleDropCaps{неплохой, хо­роший‎}

\textstyleCaptioncharacters{عِلْمٌ لاَ بَأْسَ بِهِ }\textstyleDropCaps{непло­хие знания‎}

\textstyleCaptioncharacters{رَغْمَ ذَلِكَ }\textstyleDropCaps{несмотря на то, вопреки этому‎}

\textstyleCaptioncharacters{كِبَرُ السِّنِّ }\textstyleDropCaps{старость‎}

\textstyleCaptioncharacters{رَغْمَ كِبَرِ سِنِّهِ }\textstyleDropCaps{несмот­ря на его старость‎}

\textstyleCaptioncharacters{مَحْبُوبٌ }\textstyleDropCaps{любимый‎}

\textstyleCaptioncharacters{أَقْسَمَ بِـ... }\textstyleDropCaps{поклялся‎}

\textstyleCaptioncharacters{أُقْسِمُ بِاللَّهِ }\textstyleDropCaps{клянусь Бо­гом‎}

\textstyleCaptioncharacters{قَضَى حَاجَتَهُ }\textstyleDropCaps{удовле­творил его желание‎}

\textstyleCaptioncharacters{نَفِدَ }\textstyleDropCaps{кончился, иссяк, ис­черпался‎}

\textstyleCaptioncharacters{نَفِدَ مَالُهُ }\textstyleDropCaps{его деньги кон­чились‎}

\textstyleCaptioncharacters{مَا بَالُكَ؟ }\textstyleDropCaps{что с тобой? как ты? почему ты?‎}

\textstyleCaptioncharacters{حَزِينٌ }\textstyleDropCaps{печальный‎}

\textstyleCaptioncharacters{مَا بَالُكَ حَزِينًا؟ }\textstyleDropCaps{почему ты печальный? ‎}

\textstyleCaptioncharacters{قَرِيبٌ }\textstyleDropCaps{родной, близкий‎}

\textstyleCaptioncharacters{سَبِيلٌ }\textstyleDropCaps{дорога, путь‎}

\textstyleCaptioncharacters{بَدَنِيٌّ }\textstyleDropCaps{телесный, физиче­ский‎}

\subsection[УРОК 113‎]{\textstyleDropCaps{УРОК 113‎}}
\textstyleCaptioncharacters{فُسْتَانٌ }\textstyleDropCaps{платье‎}

\textstyleCaptioncharacters{مِعْطَفٌ }\textstyleDropCaps{пальто, шинель ‎}

\textstyleCaptioncharacters{مَنَامَةٌ }\textstyleDropCaps{пижама ‎}

\textstyleCaptioncharacters{مِقْلاَةٌ }\textstyleDropCaps{сковородка‎}

\textstyleCaptioncharacters{رُبَّمَا }\textstyleDropCaps{возможно, может быть‎}

\textstyleCaptioncharacters{رُبَّمَا يُرِيدُ }\textstyleDropCaps{может быть, он хочет‎}

\textstyleCaptioncharacters{رُبَّمَا جَاءَ }\textstyleDropCaps{возможно, он пришёл‎}

\textstyleCaptioncharacters{هَاكَ }\textstyleDropCaps{на, на тебе, вот тебе‎}

\textstyleCaptioncharacters{هَاكَ القَلَمَ }\textstyleDropCaps{на карандаш‎}

\textstyleCaptioncharacters{لاَ أَحَدَ }\textstyleDropCaps{никто, никого ‎}

\textstyleCaptioncharacters{حَافِلَةٌ }\textstyleDropCaps{автобус‎}

\textstyleCaptioncharacters{قَادِمٌ }\textstyleDropCaps{идущий‎}

\textstyleCaptioncharacters{رَاحَ }\textstyleDropCaps{ушёл‎}

\textstyleCaptioncharacters{حَمَّامٌ }\textstyleDropCaps{баня‎}

\textstyleCaptioncharacters{اِخْتَفَى }\textstyleDropCaps{скрылся, спрятал­ся‎}

\textstyleCaptioncharacters{فَتَّشَ عَنْ... }\textstyleDropCaps{поискал‎}

\textstyleCaptioncharacters{بِالبَابِ }\textstyleDropCaps{у двери‎}

\textstyleCaptioncharacters{قُدَّامَ... }\textstyleDropCaps{перед...‎}

\textstyleCaptioncharacters{دُولاَبٌ }\textstyleDropCaps{шкаф‎}

\textstyleCaptioncharacters{دُولاَبُ المَلاَبِسِ \ }\textstyleDropCaps{ши­фоньер, гардероб‎}

\textstyleCaptioncharacters{شَوَى }\textstyleDropCaps{жарил ‎}

\textstyleCaptioncharacters{شُوَاءٌ }\textstyleDropCaps{жаркое‎}

\textstyleCaptioncharacters{صَبَرَ }\textstyleDropCaps{терпел‎}

\textstyleCaptioncharacters{حَضَرَ الطَّعَامُ }\textstyleDropCaps{кушать го­тово, кушать приготовилось‎}

\textstyleCaptioncharacters{جَفْنَةٌ }\textstyleDropCaps{миска, блюдо‎}

\textstyleCaptioncharacters{كُسْكُسٌ }\textstyleDropCaps{кускус (мучное блюдо)‎}

\textstyleCaptioncharacters{عِيدُ مِيلاَدٍ }\textstyleDropCaps{день рожде­ния‎}

\textstyleCaptioncharacters{مُسْتَعِدٌّ }\textstyleDropCaps{подготовивший­ся, готовый‎}

\textstyleCaptioncharacters{عِيدٌ سَعِيدْ! }\textstyleDropCaps{с праздни­ком!‎}

\subsection[урок 114‎]{\textstyleDropCaps{урок 114‎}}
\textstyleCaptioncharacters{حَذَارِ }\textstyleDropCaps{берегись! осторож­но!‎}

\textstyleCaptioncharacters{جَرَحَ }\textstyleDropCaps{ранил‎}

\textstyleCaptioncharacters{جَرَحَ يَدَهُ }\textstyleDropCaps{ранил свою руку‎}

\textstyleCaptioncharacters{سَمَّرَ }\textstyleDropCaps{прибил, заколотил гвоздями‎}

\textstyleCaptioncharacters{سَمَّرَ اللَّوْحَ }\textstyleDropCaps{прибил дос­ку гвоздями‎}

\textstyleCaptioncharacters{فُجْأَةً }\textstyleDropCaps{внезапно, вдруг, неожиданно‎}

\textstyleCaptioncharacters{أَطْلَقَ }\textstyleDropCaps{испустил‎}

\textstyleCaptioncharacters{أَطْلَقَ صَرْحَةً }\textstyleDropCaps{испустил крик‎}

\textstyleCaptioncharacters{عَنِيدٌ }\textstyleDropCaps{упрямый, упорный ‎}

\textstyleCaptioncharacters{مُضْحِكٌ }\textstyleDropCaps{смешной, коми­ческий ‎}

\textstyleCaptioncharacters{نَعْسَانُ }\textstyleDropCaps{сонливый, сон­ный ‎}

\textstyleCaptioncharacters{سَهِرَ }\textstyleDropCaps{бодрствовал, не спал‎}

\textstyleCaptioncharacters{يَوْمُ عُطْلَةٍ }\textstyleDropCaps{выходной день, нерабочий, неучебный день‎}

\textstyleCaptioncharacters{كَمَّلَ }\textstyleDropCaps{завершил, закончил‎}

\textstyleCaptioncharacters{وَ فَوْقَ ذَلِكَ }\textstyleDropCaps{кроме того, сверх того, помимо того‎}

\textstyleCaptioncharacters{حِجْرٌ }\textstyleDropCaps{колени‎}

\textstyleCaptioncharacters{فِي حِجْرِهَا }\textstyleDropCaps{на её коле­нях‎}

\textstyleCaptioncharacters{حَالاً }\textstyleDropCaps{немедленно, сейчас же, тотчас‎}

\textstyleCaptioncharacters{نَعَسَ }\textstyleDropCaps{задремал, клонило ко сну‎}

\textstyleCaptioncharacters{رَقَدَ }\textstyleDropCaps{поспал; лёг, лежал ‎}

\textstyleCaptioncharacters{قُومُوا إِلَى النَّوْمِ \ }\textstyleDropCaps{идите спать‎}

\textstyleCaptioncharacters{سَهْرَةٌ }\textstyleDropCaps{вечеринка, вечер ‎}

\textstyleCaptioncharacters{اِنْتَهَتِ السَّهْرَةُ \ }\textstyleDropCaps{вечер закончился ‎}

\textstyleCaptioncharacters{دَعْ }\textstyleDropCaps{брось, оставь ‎}

\textstyleCaptioncharacters{رَوَّحَ }\textstyleDropCaps{вернулся домой‎}

\textstyleCaptioncharacters{وَ إِلاَّ }\textstyleDropCaps{иначе, а то, в против­ном случае‎}

\textstyleCaptioncharacters{مُجْتَمِعٌ }\textstyleDropCaps{собравшийся ‎}

\textstyleCaptioncharacters{مُنْتَظِرٌ }\textstyleDropCaps{ожидающий ‎}

\textstyleCaptioncharacters{بَقِيَّةٌ }\textstyleDropCaps{остальное, остаток ‎}

\textstyleCaptioncharacters{قِصَّةٌ }\textstyleDropCaps{рассказ, история‎}

\subsection[урок 115‎]{\textstyleDropCaps{урок 115‎}}
\textstyleCaptioncharacters{مُمَرِّضَةٌ }\textstyleDropCaps{медсестра ‎}

\textstyleCaptioncharacters{سَمَّاعَةٌ }\textstyleDropCaps{наушники ‎}

\textstyleCaptioncharacters{مَطَرِيَّةٌ \ }\textstyleDropCaps{зонтик‎}

\textstyleCaptioncharacters{حَبَا }\textstyleDropCaps{ползал ‎}

\textstyleCaptioncharacters{مَا زَالَ }\textstyleDropCaps{ещё, всё ещё‎}

\textstyleCaptioncharacters{مَا زَالَ الوَلِيدُ يَحْبُو \ }\textstyleDropCaps{ребенок всё ещё ползает ‎}

\textstyleCaptioncharacters{تَنَاوَلَ }\textstyleDropCaps{взял‎}

\textstyleCaptioncharacters{قَذِرٌ }\textstyleDropCaps{грязный‎}

\textstyleCaptioncharacters{جَاهِزٌ }\textstyleDropCaps{готовый‎}

\textstyleCaptioncharacters{الشَّايُ جَاهِزٌ \ }\textstyleDropCaps{чай готов‎}

\textstyleCaptioncharacters{أَفْطَرَ }\textstyleDropCaps{завтракал‎}

\textstyleCaptioncharacters{تَأَلَّمَ }\textstyleDropCaps{страдал, мучился, по­чувствовал боль‎}

\textstyleCaptioncharacters{أَحَسَّ بـ... \ }\textstyleDropCaps{чувствовал, ощущал‎}

\textstyleCaptioncharacters{حُمَّى }\textstyleDropCaps{лихорадка‎}

\textstyleCaptioncharacters{أَحَسَّ بِالبَرْدِ }\textstyleDropCaps{ощущал холод, ему было холодно‎}

\textstyleCaptioncharacters{يُحِسُّ بِالحُمَّى \ }\textstyleDropCaps{его ли­хорадит ‎}

\textstyleCaptioncharacters{حَبَّةٌ }\textstyleDropCaps{таблетка, пилюля ‎}

\textstyleCaptioncharacters{نَزَعَ }\textstyleDropCaps{снял‎}

\textstyleCaptioncharacters{نَزَعَ ثِيَابَهُ }\textstyleDropCaps{снял одежду, разделся‎}

\textstyleCaptioncharacters{طَفِيفٌ }\textstyleDropCaps{незначительный, пустяковый, малый‎}

\textstyleCaptioncharacters{زَالَ }\textstyleDropCaps{прошёл, исчез, скрыл­ся‎}

\textstyleCaptioncharacters{لَزِمَ }\textstyleDropCaps{держался, не покинул‎}

\textstyleCaptioncharacters{لَزِمَ الفِرَاشَ }\textstyleDropCaps{был прико­ван к постели‎}

\textstyleCaptioncharacters{لَزِمَ البَيْتَ }\textstyleDropCaps{сидел дома‎}

\textstyleCaptioncharacters{تَذَكَّرَ }\textstyleDropCaps{вспомнил‎}

\textstyleCaptioncharacters{تَذَكَّرْتُ! }\textstyleDropCaps{вспомнил я!‎}

\textstyleCaptioncharacters{حُقْنَةٌ }\textstyleDropCaps{укол, инъекция‎}

\textstyleCaptioncharacters{بِنَايَةٌ }\textstyleDropCaps{здание‎}

\textstyleCaptioncharacters{لاَ بَأْسَ عَلَيْكَ }\textstyleDropCaps{не бойся! ничего!‎}

\textstyleCaptioncharacters{فِي أَثْنَاءِ ذَلِكَ }\textstyleDropCaps{в то вре­мя, тем временем‎}

\subsection[Урок 116‎]{\textstyleDropCaps{Урок 116‎}}
\textstyleCaptioncharacters{نَظَّارَةٌ }\textstyleDropCaps{очки‎}

\textstyleCaptioncharacters{مِذْيَاعٌ }\textstyleDropCaps{радио‎}

\textstyleCaptioncharacters{قُبَّعَةٌ }\textstyleDropCaps{шляпа‎}

\textstyleCaptioncharacters{أَخْفَى }\textstyleDropCaps{скрыл, спрятал‎}

\textstyleCaptioncharacters{ظَهْرٌ }\textstyleDropCaps{спина‎}

\textstyleCaptioncharacters{حَرْفٌ }\textstyleDropCaps{буква‎}

\textstyleCaptioncharacters{ضَعُفَ }\textstyleDropCaps{ослабел‎}

\textstyleCaptioncharacters{بَصَرٌ }\textstyleDropCaps{зрение‎}

\textstyleCaptioncharacters{ضَعُفَ بَصَرُهُ }\textstyleDropCaps{у него ухуд­шилось зрение‎}

\textstyleCaptioncharacters{فِي السِّنِّ المُبَكِّرَةِ }\textstyleDropCaps{в раннем возрасте‎}

\textstyleCaptioncharacters{أَنْتَ تَظْلِمُنِي! }\textstyleDropCaps{ты меня обижаешь‎}

\textstyleCaptioncharacters{خَبَّأَ }\textstyleDropCaps{прятал, припрятал‎}

\textstyleCaptioncharacters{أَظْلَمَ }\textstyleDropCaps{стемнел‎}

\textstyleCaptioncharacters{أَظْلَمَ اللَّيْلُ \ }\textstyleDropCaps{ночь стем­нела ‎}

\textstyleCaptioncharacters{لَحْظَةً }\textstyleDropCaps{одну минуту‎}

\textstyleCaptioncharacters{فَتَحَ المِذْيَاعَ }\textstyleDropCaps{включил радиоприёмник‎}

\textstyleCaptioncharacters{عَشِيَّةٌ }\textstyleDropCaps{вечер‎}

\textstyleCaptioncharacters{كُلَّ عَشِيَّةٍ }\textstyleDropCaps{каждый вечер‎}

\textstyleCaptioncharacters{الرُّسُومُ المُتَحَرِّكَةُ \ }\textstyleDropCaps{мультфильм ‎}

\textstyleCaptioncharacters{شَخَرَ }\textstyleDropCaps{храпел‎}

\textstyleCaptioncharacters{فَكَّكَ }\textstyleDropCaps{разобрал‎}

\textstyleCaptioncharacters{رَكَّبَ }\textstyleDropCaps{собрал‎}

\textstyleCaptioncharacters{اِسْتَغَاثَ بِـ... }\textstyleDropCaps{просил по­мощь, призывал на помощь ‎}

\textstyleCaptioncharacters{قَضَى }\textstyleDropCaps{провёл‎}

\textstyleCaptioncharacters{قَضَى الوَقْتَ }\textstyleDropCaps{провёл вре­мя‎}

\textstyleCaptioncharacters{سَاعَاتُ الفَرَاغِ \ }\textstyleDropCaps{свобод­ные часы ‎}

\textstyleCaptioncharacters{مَتْحَفٌ }\textstyleDropCaps{музей‎}

\textstyleCaptioncharacters{الآثَارُ التَّارِيخِيَّةُ \ }\textstyleDropCaps{исто­рические памятники‎}

\textstyleCaptioncharacters{ضَوَاحِى المَدِينَةِ }\textstyleDropCaps{окрестности города ‎}

\textstyleCaptioncharacters{أَحْيَانًا }\textstyleDropCaps{иногда, временами‎}

\textstyleCaptioncharacters{ضَاحِيَةٌ }\textstyleDropCaps{пригород, окрестность‎}

\textstyleCaptioncharacters{ذِكْرٌ }\textstyleDropCaps{зикр, воспоминание Бога, молитва‎}

\textstyleCaptioncharacters{دُعَاءٌ }\textstyleDropCaps{дуа, мольба, зов, мо­литва‎}

\textstyleCaptioncharacters{كَثِيرًامَّا }\textstyleDropCaps{часто, частенько ‎}

\textstyleCaptioncharacters{اِسْتَمْتَعَ بِـ... }\textstyleDropCaps{насла­ждался‎}

\textstyleCaptioncharacters{الهَوَاءُ الطَّلْقُ }\textstyleDropCaps{открытый воздух, лоно природы‎}

\textstyleCaptioncharacters{فِي الهَوَاءِ الطَّلْقِ }\textstyleDropCaps{на открытом воздухе, на природе‎}

\textstyleCaptioncharacters{تَلْفَزَةٌ }\textstyleDropCaps{телевизор‎}

\subsection[Урок 117‎]{\textstyleDropCaps{Урок 117‎}}
\textstyleCaptioncharacters{حُفْرَةٌ }\textstyleDropCaps{яма‎}

\textstyleCaptioncharacters{دُكَّانٌ }\textstyleDropCaps{лавка, ларёк‎}

\textstyleCaptioncharacters{دُكَّانُ البَقَّالِ }\textstyleDropCaps{бакалей­ная лавка‎}

\textstyleCaptioncharacters{بَقَّالٌ }\textstyleDropCaps{бакалейщик‎}

\textstyleCaptioncharacters{يَا تُرَى؟ }\textstyleDropCaps{спрашивается, интересно бы знать‎}

\textstyleCaptioncharacters{طَالَ }\textstyleDropCaps{был долгим, длитель­ным, длился‎}

\textstyleCaptioncharacters{طَالَ انْتِظَارِي }\textstyleDropCaps{долго я ждал‎}

\textstyleCaptioncharacters{غَنَّى }\textstyleDropCaps{пел‎}

\textstyleCaptioncharacters{يَا عِفْرِيتُ! }\textstyleDropCaps{ой хитер! ой бесёнок!‎}

\textstyleCaptioncharacters{غَرَبَ }\textstyleDropCaps{закатился, зашёл ‎}

\textstyleCaptioncharacters{غَرَبَتِ الشَّمْسُ }\textstyleDropCaps{зашло солнце‎}

\textstyleCaptioncharacters{حَتَّى أَمْلَأَ }\textstyleDropCaps{пока заполню ‎}

\textstyleCaptioncharacters{مَا أَلَذَّهُ! }\textstyleDropCaps{какой он вкус­ный! ‎}

\textstyleCaptioncharacters{وَاحِدٌ آخَرُ }\textstyleDropCaps{ещё один ‎}

\textstyleCaptioncharacters{رَاسَلَ }\textstyleDropCaps{переписывался ‎}

\textstyleCaptioncharacters{اِنْقَطَعَ \ }\textstyleDropCaps{прекратился‎}

\textstyleCaptioncharacters{تَمَامًا }\textstyleDropCaps{полностью‎}

\textstyleCaptioncharacters{يَبْدُو }\textstyleDropCaps{кажется, видно, по-видимому‎}

\textstyleCaptioncharacters{طِوَالَ سِنِي الدِّرَاسَةِ }\textstyleDropCaps{все годы учёбы‎}

\textstyleCaptioncharacters{عَلَى الدَّوَامِ }\textstyleDropCaps{постоянно, беспрерывно‎}

\textstyleCaptioncharacters{كُلُّ عَامٍ وَ أَنْتُمْ بِخَيْرٍ! }\textstyleDropCaps{С Новым годом! ‎}

\textstyleCaptioncharacters{وَ أَنْتُمْ بِالصِّحَّةِ وَ السَّلاَمَةِ }\textstyleDropCaps{и вам желаем здравия и благополучия‎}

\textstyleCaptioncharacters{كَيْفَ رَأَيْتَ؟ }\textstyleDropCaps{как тебе понравился?‎}

\textstyleCaptioncharacters{عَظِيمٌ! }\textstyleDropCaps{великолепно! Пре­красно!‎}

\textstyleCaptioncharacters{أَعْجَبَ }\textstyleDropCaps{понравился‎}

\textstyleCaptioncharacters{هَلْ أَعْجَبَكَ؟ }\textstyleDropCaps{тебе по­нравилось?‎}

\textstyleCaptioncharacters{يُعْجِبَنِي }\textstyleDropCaps{мне нравится‎}

\textstyleCaptioncharacters{اِحْذَرْ! }\textstyleDropCaps{берегись! осторож­но! смотри!‎}

\textstyleCaptioncharacters{أَجْرَى }\textstyleDropCaps{заставил бежать, погнал‎}

\textstyleCaptioncharacters{لَمْ يَلْتَفِتْ إِلَى... }\textstyleDropCaps{не обратил внимания‎}

\textstyleCaptioncharacters{حَذَّرَ }\textstyleDropCaps{предупредил‎}

\textstyleCaptioncharacters{تَحْذِيرٌ }\textstyleDropCaps{предупреждение‎}

\textstyleCaptioncharacters{اِنْقَلَبَ }\textstyleDropCaps{перевернулся‎}

\textstyleCaptioncharacters{اِنْقَلَبَتْ بِهِ دَرَّاجَتُهُ }\textstyleDropCaps{вме­сте с ним перевернулся его велосипед ‎}

\subsection[Урок 118‎]{\textstyleDropCaps{Урок 118‎}}
\textstyleCaptioncharacters{أَصْلٌ }\textstyleDropCaps{основа‎}

\textstyleCaptioncharacters{رَبَّى }\textstyleDropCaps{вырастил, вскормил ‎}

\textstyleCaptioncharacters{عَالَمٌ }\textstyleDropCaps{мир, вселенная, свет ‎}

\textstyleCaptioncharacters{بِنِعْمَةِ اللَّهِ }\textstyleDropCaps{по милости Бога‎}

\textstyleCaptioncharacters{مِنْ عَدَمٍ إِلَى وُجُودٍ \ }\textstyleDropCaps{из небытия в бытие ‎}

\textstyleCaptioncharacters{مَعْبُودٌ }\textstyleDropCaps{предмет поклоне­ния, бог, божество‎}

\textstyleCaptioncharacters{دَلِيلٌ }\textstyleDropCaps{довод, доказатель­ство‎}

\textstyleCaptioncharacters{صِرَاطٌ }\textstyleDropCaps{путь, дорога ‎}

\textstyleCaptioncharacters{الصِّرَاطُ المُسْتَقِيمُ }\textstyleDropCaps{пра­вильный путь ‎}

\textstyleCaptioncharacters{كُلُّ مَا سِوَى اللَّهِ }\textstyleDropCaps{все кроме Бога, все помимо Бога‎}

\textstyleCaptioncharacters{ِلأَيِّ شَيْءٍ؟ }\textstyleDropCaps{для чего? зачем? почему?‎}

\textstyleCaptioncharacters{اِتَّبَعَ }\textstyleDropCaps{следовал, соблюдал‎}

\textstyleCaptioncharacters{اِتَّبَعَ أَمْرَهُ }\textstyleDropCaps{следовал пове­лению его, соблюдал приказ его‎}

\textstyleCaptioncharacters{نَهْيٌ }\textstyleDropCaps{запрет, запрещение ‎}

\textstyleCaptioncharacters{اِجْتَنَبَ }\textstyleDropCaps{избегал, сторо­нился‎}

\textstyleCaptioncharacters{اِجْتَنَبَ نَهْيَهُ }\textstyleDropCaps{соблюдал запрет его, воздержался от запрещённого им‎}

\textstyleCaptioncharacters{تَوْحِيدٌ }\textstyleDropCaps{единобожие‎}

\textstyleCaptioncharacters{شِرْكٌ }\textstyleDropCaps{многобожие, языче­ство‎}

\textstyleCaptioncharacters{طَاعَةٌ }\textstyleDropCaps{повиновение, покло­нение‎}

\textstyleCaptioncharacters{طَاعَةُ اللَّهِ }\textstyleDropCaps{ритуальное поклонение Богу‎}

\textstyleCaptioncharacters{إِنْسِيٌّ }\textstyleDropCaps{человек (в отличие от джиннов)‎}

\textstyleCaptioncharacters{اِنْقَادَ لِـ... }\textstyleDropCaps{подчинился, по­виновался‎}

\textstyleCaptioncharacters{مَنْهَجٌ }\textstyleDropCaps{программа, путь, до­рога‎}

\textstyleCaptioncharacters{شَرْعٌ }\textstyleDropCaps{шариат, божий закон‎}

\textstyleCaptioncharacters{شَرْعُ اللَّهِ \ }\textstyleDropCaps{закон Бога ‎}

\textstyleCaptioncharacters{أَقَامَ }\textstyleDropCaps{установил, устроил‎}

\textstyleCaptioncharacters{أَقَامَ دَوْلَةَ الإِسْلاَمِ }\textstyleDropCaps{установил Исламское государство ‎}

\textstyleCaptioncharacters{إِقَامَةُ حُكْمِ اللَّهِ فِي الأَرْضِ \ }\textstyleDropCaps{установление власти Бога на Земле ‎}

\textstyleCaptioncharacters{بَرِئَ }\textstyleDropCaps{отрёкся, был непри­частным, к кому-л., чему-л., заявил о своей не причастности ‎}

\textstyleCaptioncharacters{بَرَاءَةٌ مِنَ الشِّرْكِ }\textstyleDropCaps{отре­чение, отказ от многобожия ‎}

\textstyleCaptioncharacters{وَضْعِيٌّ }\textstyleDropCaps{составленный, установленный человеком‎}

\textstyleCaptioncharacters{قَانُونٌ وَضْعِيٌّ }\textstyleDropCaps{закон, со­ставленный самим человеком (в отличие от }\textstyleDropCaps{божественного)‎}

\textstyleCaptioncharacters{رَفْضٌ بَاتٌّ }\textstyleDropCaps{решительное отклонение, решительный отказ‎}

\textstyleCaptioncharacters{مُرْسَلٌ }\textstyleDropCaps{посланник, мис­сионер‎}

\textstyleCaptioncharacters{مِنْ لَدُنْ آدَمَ }\textstyleDropCaps{от Адама, начиная от Адама‎}

\textstyleCaptioncharacters{طَلَوَاتُ اللَّهِ وَ سَلاَمُهُ }\textstyleDropCaps{благословения Бога и Его мир ‎}

\textstyleCaptioncharacters{ذُرِّيَّةٌ \ }\textstyleDropCaps{потомство, дети‎}

\textstyleCaptioncharacters{تُرَابٌ }\textstyleDropCaps{земля, прах ‎}

\textstyleCaptioncharacters{جِنِيٌّ }\textstyleDropCaps{джинн, дух, демон‎}

\subsection[Урок 119‎]{\textstyleDropCaps{Урок 119‎}}
\textstyleCaptioncharacters{أَعْظَمُ }\textstyleDropCaps{величайший, са­мый большой‎}

\textstyleCaptioncharacters{أَشْرَكَ بِاللَّهِ }\textstyleDropCaps{сделал со­участников для Бога совершил ширк, придавал Богу сотоварищей, был многобожником‎}

\textstyleCaptioncharacters{ذَنْبٌ }\textstyleDropCaps{грех‎}

\textstyleCaptioncharacters{عَلَى الإِطْلاَقِ }\textstyleDropCaps{абсолют­но, вообще‎}

\textstyleCaptioncharacters{خَمْرٌ }\textstyleDropCaps{прелюбодеяние, про­ституция вино, спиртное‎}

\textstyleCaptioncharacters{قَتَلَ }\textstyleDropCaps{убил‎}

\textstyleCaptioncharacters{قَطَعَ الطَّرِيقَ }\textstyleDropCaps{разбойни­чал, занимался разбоем‎}

\textstyleCaptioncharacters{قَطْعُ الطَّرِيقِ }\textstyleDropCaps{разбой‎}

\textstyleCaptioncharacters{دُونَ... }\textstyleDropCaps{меньше, ниже‎}

\textstyleCaptioncharacters{عقَّ }\textstyleDropCaps{был непослушным, не­почтительным (к родителям)‎}

\textstyleCaptioncharacters{غَفَرَ }\textstyleDropCaps{простил грех ‎}

\textstyleCaptioncharacters{شَاءَ }\textstyleDropCaps{хотел, желал‎}

\textstyleCaptioncharacters{لِمَنْ يَشَاءُ }\textstyleDropCaps{кому пожела­ет‎}

\textstyleCaptioncharacters{خَفِيَ }\textstyleDropCaps{был неизвестным, тайным для кого-л.‎}

\textstyleCaptioncharacters{يَخْفَى عَلَى كَثِيرٍ مِنَ النَّاسِ \ }\textstyleDropCaps{многим из людей не известно ‎}

\textstyleCaptioncharacters{بَلْ }\textstyleDropCaps{даже, более того‎}

\textstyleCaptioncharacters{أَخْفَى مِنْ }\textstyleDropCaps{более скры­тый, более тайный, незаметный‎}

\textstyleCaptioncharacters{دَبَّ }\textstyleDropCaps{ползал‎}

\textstyleCaptioncharacters{أَخْفَى مِنْ دَبِيبِ النَّمْلِ }\textstyleDropCaps{более скрытый, чем ползание муравья ‎}

\textstyleCaptioncharacters{صَخْرَةٌ \ }\textstyleDropCaps{скала ‎}

\textstyleCaptioncharacters{صَخْرَةٌ صَمَّاءُ }\textstyleDropCaps{твердая, плотная скала‎}

\textstyleCaptioncharacters{لَيْلَةٌ ظَلْمَاءُ }\textstyleDropCaps{тёмная ночь‎}

\textstyleCaptioncharacters{أَبْطَلَ }\textstyleDropCaps{погубил, уничто­жил, сделал пустым, недействительным‎}

\textstyleCaptioncharacters{الصَّالِحَاتُ }\textstyleDropCaps{добрые дела, благодеяния‎}

\textstyleCaptioncharacters{حَبِطَ }\textstyleDropCaps{пропал, стал тщет­ным‎}

\textstyleCaptioncharacters{لَئِنْ... }\textstyleDropCaps{если вдруг, если же‎}

\textstyleCaptioncharacters{لَئِنْ أَشْرَكْتَ لَيَحْبَطَنَّ عَمَلُكَ }\textstyleDropCaps{если вдруг ты совершишь ширк, непременно пропадут твои деяния‎}

\textstyleCaptioncharacters{َلاَقْتُلَنَّكَ }\textstyleDropCaps{непременно, клянусь, я убью тебя‎}

\textstyleCaptioncharacters{خَاسِرٌ }\textstyleDropCaps{потерпевший убы­ток, проигравший, пропащий‎}

\textstyleCaptioncharacters{أَحْبَطَ }\textstyleDropCaps{погубил, прова­лил, сделал тщетным‎}

\textstyleCaptioncharacters{تَابَ مِنْ ذَنْبِهِ }\textstyleDropCaps{раскаял­ся о своем грехе ‎}

\textstyleCaptioncharacters{صَرَفَ }\textstyleDropCaps{тратил, расходо­вал, направил, обратил‎}

\textstyleCaptioncharacters{صَرَفَ العِبَادَةَ لِغَيْرِ اللَّهِ }\textstyleDropCaps{стал поклоняться другому, помимо Бога ‎}

\textstyleCaptioncharacters{صَرَفَ المَالَ فِي وُجُوهِ الخَيْرِ }\textstyleDropCaps{тратил деньги на благие стороны ‎}

\textstyleCaptioncharacters{عَلَيْكَ أَنْ تَحْيَا حَيَاتَكَ عَلَى وَجْهٍ يُرْضِي اللَّهَ }\textstyleDropCaps{угодным Богу образом (удовлетворяющим Бога)т ы должен прожить свою жизнь ‎}

\textstyleCaptioncharacters{مَا حُكْمُهُ فِي الإِسْلاَمِ؟ }\textstyleDropCaps{каково его положение в Исламе? какое решение ему выносит Ислам?‎}

\textstyleCaptioncharacters{إِنَّ اللَّهَ لاَ يَغْفِرُ أَنْ يُشْرَكَ بِهِ }\textstyleDropCaps{поистине, Бог не прощает, чтобы Ему придавали сотоварищей‎}

\subsection[Урок 120‎]{\textstyleDropCaps{Урок 120‎}}
\textstyleCaptioncharacters{دُعَاءٌ }\textstyleDropCaps{мольба, зов, просьба, дуа‎}

\textstyleCaptioncharacters{دَعَا اللَّهَ }\textstyleDropCaps{взывал к Богу, просил Бога‎}

\textstyleCaptioncharacters{أَثْقَلَ }\textstyleDropCaps{тяготил, был тяже­лым‎}

\textstyleCaptioncharacters{إِذَا أَثْقَلَكَ }\textstyleDropCaps{если тебе будет тяжело‎}

\textstyleCaptioncharacters{تَحْمِيلٌ \ }\textstyleDropCaps{погрузка‎}

\textstyleCaptioncharacters{بِضَاعَةٌ \ }\textstyleDropCaps{товар ‎}

\textstyleCaptioncharacters{رَاحِلَةٌ }\textstyleDropCaps{вьючная или верхо­вая верблюдица‎}

\textstyleCaptioncharacters{يَجُوزُ }\textstyleDropCaps{можно, разрешает­ся, допускается‎}

\textstyleCaptioncharacters{لاَ يَجُوزُ \ }\textstyleDropCaps{нельзя ‎}

\textstyleCaptioncharacters{جَائِزٌ }\textstyleDropCaps{разрешённый, допус­каемый, такой, что можно‎}

\textstyleCaptioncharacters{غَيْرُ جَائِزٍ }\textstyleDropCaps{неразрешён­ный, непозволительный‎}

\textstyleCaptioncharacters{لاَ بَأْسَ بِهِ }\textstyleDropCaps{неплохо, не мешает, не беда‎}

\textstyleCaptioncharacters{شَبَّ }\textstyleDropCaps{вспыхнул, разгорелся‎}

\textstyleCaptioncharacters{حَرِيقٌ \ }\textstyleDropCaps{пожар ‎}

\textstyleCaptioncharacters{إِصْطَبْلٌ }\textstyleDropCaps{хлев‎}

\textstyleCaptioncharacters{اِسْتَغَاثَ بِـ... }\textstyleDropCaps{призвал на помощь‎}

\textstyleCaptioncharacters{أَطْْفَأَ الحَرِيقَ }\textstyleDropCaps{потушил пожар‎}

\textstyleCaptioncharacters{فِرْقَةُ مَطَافِئَ }\textstyleDropCaps{пожарная команда‎}

\textstyleCaptioncharacters{مِضَخَّةٌ }\textstyleDropCaps{насос‎}

\textstyleCaptioncharacters{خُرْطُومٌ }\textstyleDropCaps{шланг‎}

\textstyleCaptioncharacters{خَرَاطِيمُ المَاءِ }\textstyleDropCaps{водяные шланги‎}

\textstyleCaptioncharacters{دَوَّارَةٌ }\textstyleDropCaps{водоворот‎}

\textstyleCaptioncharacters{لَمْ يَكُنْ يُحْسِنُ السِّبَاحَةَ \ }\textstyleDropCaps{он не умел плавать‎}

\textstyleCaptioncharacters{أَلْقَى }\textstyleDropCaps{бросил ‎}

\textstyleCaptioncharacters{تَعَلَّقَ بِـ... }\textstyleDropCaps{зацепился за что‎}

\textstyleCaptioncharacters{أَشْرَفَ عَلَى... \ }\textstyleDropCaps{был бли­зок к чему, был накануне чего-л.‎}

\textstyleCaptioncharacters{أَشْرَفَ عَلَى الغَرَقِ }\textstyleDropCaps{вот-вот тонул ‎}

\textstyleCaptioncharacters{غَرِقَ }\textstyleDropCaps{потонул‎}

\textstyleCaptioncharacters{نَجَا }\textstyleDropCaps{спасся, избавился ‎}

\textstyleCaptioncharacters{دَيْنٌ }\textstyleDropCaps{долг, задолженность ‎}

\textstyleCaptioncharacters{حَلَّ أَجَلُهُ \ }\textstyleDropCaps{наступил его срок‎}

\textstyleCaptioncharacters{قَضَى دَيْنَهُ }\textstyleDropCaps{уплатил свой долг‎}

\textstyleCaptioncharacters{اِسْتَعَانَ بِـ... }\textstyleDropCaps{просил по­мочь, обратился за помощью‎}

\textstyleCaptioncharacters{لاَ مَانِعَ مِنْهُ عَادَةً }\textstyleDropCaps{это можно, возражений нет ‎}

\textstyleCaptioncharacters{عَادَةً }\textstyleDropCaps{обычно, обыкновен­но‎}

\textstyleCaptioncharacters{نَقَضَ }\textstyleDropCaps{нарушил, разрушил, сделал недействительным‎}

\textstyleCaptioncharacters{تَوْحِيدَهُ }\textstyleDropCaps{разрушил своё единобожие‎}

\textstyleCaptioncharacters{وَ العِيَاذُ بِاللَّهِ }\textstyleDropCaps{Боже упа­си! Боже сохрани! не дай Бог!‎}

\textstyleCaptioncharacters{سَيَأْتِي \ }\textstyleDropCaps{ниже следует, бу­дет приведен‎}

\textstyleCaptioncharacters{مِثَالٌ }\textstyleDropCaps{пример‎}

\subsection[Урок 121‎]{\textstyleDropCaps{Урок 121‎}}
\textstyleCaptioncharacters{أَغَاثَ }\textstyleDropCaps{помог, выручил, спас‎}

\textstyleCaptioncharacters{أَغِثْنِي }\textstyleDropCaps{выручай меня‎}

\textstyleCaptioncharacters{كَشَفَ البَلاَءَ }\textstyleDropCaps{удалил беду, отстранил беду‎}

\textstyleCaptioncharacters{كَشَفَ الغَمَّ }\textstyleDropCaps{удалил, раз­веял печаль‎}

\textstyleCaptioncharacters{بَلاَءٌ }\textstyleDropCaps{беда, несчастье‎}

\textstyleCaptioncharacters{نَزَلَ بِهِ البَلاَءُ }\textstyleDropCaps{его постигло несчастье, с ним случалась беда‎}

\textstyleCaptioncharacters{شَفَى \ }\textstyleDropCaps{вылечил ‎}

\textstyleCaptioncharacters{كَثُرَ }\textstyleDropCaps{умножился, размно­жился, стало много‎}

\textstyleCaptioncharacters{إِيمَانٌ }\textstyleDropCaps{вера, уверование‎}

\textstyleCaptioncharacters{عِنْدَ... }\textstyleDropCaps{при, во время‎}

\textstyleCaptioncharacters{عِنْدَ المَوْتِ }\textstyleDropCaps{при смерти‎}

\textstyleCaptioncharacters{قَرَّبَ }\textstyleDropCaps{приблизил, сблизил‎}

\textstyleCaptioncharacters{مُقَرَّبٌ }\textstyleDropCaps{приближённый‎}

\textstyleCaptioncharacters{مُذْنِبٌ }\textstyleDropCaps{грешный‎}

\textstyleCaptioncharacters{نَقِيٌّ \ }\textstyleDropCaps{чистый‎}

\textstyleCaptioncharacters{شَفَعَ }\textstyleDropCaps{ходатайствовал за кого-л.‎}

\textstyleCaptioncharacters{شَفَعَ عِنْدَ اللَّهِ }\textstyleDropCaps{засту­пился перед Богом‎}

\textstyleCaptioncharacters{نَحْوُ ذَلِكَ }\textstyleDropCaps{тому подобное‎}

\textstyleCaptioncharacters{مَلَكٌ \ }\textstyleDropCaps{ангел ‎}

\textstyleCaptioncharacters{شَيْخٌ }\textstyleDropCaps{шейх, глава суфий­ского ордена‎}

\textstyleCaptioncharacters{أُسْتَاذٌ }\textstyleDropCaps{учитель, настав­ник; шейх, суфийский глава‎}

\textstyleCaptioncharacters{قَبْرٌ }\textstyleDropCaps{могила‎}

\textstyleCaptioncharacters{مِنْ دُونِ اللَّهِ }\textstyleDropCaps{помимо Бога, кроме Бога‎}

\textstyleCaptioncharacters{إِيَّاكَ نَعْبُدُ وَ إِيَّاكَ نَسْتَعِينُ }\textstyleDropCaps{Тебе только мы поклоняемся, и у Тебя только просим помощи‎}

\textstyleCaptioncharacters{اِسْتَجَابَ }\textstyleDropCaps{внял (мольбе), ответил (на зов)‎}

\textstyleCaptioncharacters{اِسْتَجَابَ اللَّهُ دُعَاءَهُ }\textstyleDropCaps{Бог ответил на его зов ‎}

\textstyleCaptioncharacters{إِذًا }\textstyleDropCaps{тогда, если так‎}

\textstyleCaptioncharacters{ظَالِمٌ }\textstyleDropCaps{неправедный, нече­стивый, несправедливый‎}

\textstyleCaptioncharacters{إِنَّمَا }\textstyleDropCaps{только‎}

\textstyleCaptioncharacters{إِنَّمَا أَدْعُو رَبِّي }\textstyleDropCaps{я взы­ваю только к своему Господу‎}

\textstyleCaptioncharacters{إِنَّمَا أَنَا بَشَرٌ }\textstyleDropCaps{я только че­ловек, я никто иной как человек‎}

\textstyleCaptioncharacters{لاَ إِلَهَ إِلاَّ اللَّهِ }\textstyleDropCaps{нет боже­ства, кроме Бога‎}

\subsection[Урок 122‎]{\textstyleDropCaps{Урок 122‎}}
\textstyleCaptioncharacters{اِتَّخَذَ }\textstyleDropCaps{сделал себе, взял себе, имел, заимел‎}

\textstyleCaptioncharacters{اِتَّخَذَ شَيْخًا }\textstyleDropCaps{заимел шей­ха‎}

\textstyleCaptioncharacters{صُوفِيٌّ }\textstyleDropCaps{суфий, суфийский‎}

\textstyleCaptioncharacters{وَاسِطَةٌ \ }\textstyleDropCaps{посредник ‎}

\textstyleCaptioncharacters{تَرَكَهُ وَرَاءَ ظَهْرِهِ }\textstyleDropCaps{оста­вил за спиной, без внимания‎}

\textstyleCaptioncharacters{اِحْتَكَمَ إِلَى... }\textstyleDropCaps{обратился (как к судье) для решеия вопроса ‎}

\textstyleCaptioncharacters{عَمِلَ بِقَوْلِهِ }\textstyleDropCaps{действовал по его словам, согласно его словам‎}

\textstyleCaptioncharacters{وَافَقَ }\textstyleDropCaps{соответствовал, сов­пал‎}

\textstyleCaptioncharacters{الكِتَابُ وَ السُّنَّةُ }\textstyleDropCaps{Китаб и Сунна, Коран и Хадисы‎}

\textstyleCaptioncharacters{نُورٌ }\textstyleDropCaps{свет‎}

\textstyleCaptioncharacters{عِلْمٌ لَدُنِيٌّ \ }\textstyleDropCaps{мистическое знание ‎}

\textstyleCaptioncharacters{عِلْمٌ بَاطِنِيٌّ }\textstyleDropCaps{таинствен­ное, сокровенное знание‎}

\textstyleCaptioncharacters{يَزْعُمُ }\textstyleDropCaps{он утверждает, гово­рит (что-то недостоверное, сомнительное)‎}

\textstyleCaptioncharacters{كَمَا يَزْعُمُ \ }\textstyleDropCaps{он также утверждает ‎}

\textstyleCaptioncharacters{كَمَا يَقُولُ \ }\textstyleDropCaps{он также го­ворит ‎}

\textstyleCaptioncharacters{أُمُورُ الدِّينِ }\textstyleDropCaps{вопросы ре­лигии‎}

\textstyleCaptioncharacters{بَاطِلٌ }\textstyleDropCaps{ложный, невер- ный; ложь, неправда‎}

\textstyleCaptioncharacters{عَبَأَ }\textstyleDropCaps{придал значение, обра­тил внимание‎}

\textstyleCaptioncharacters{لاَ نَعْبَأُ بِهِ }\textstyleDropCaps{мы не обраща­ем на него внимания‎}

\textstyleCaptioncharacters{رَدَّ عَلَيْهِ قَوْلَهُ }\textstyleDropCaps{отверг, не принял его слова‎}

\textstyleCaptioncharacters{صَاحِبٌ }\textstyleDropCaps{хозяин, владелец‎}

\textstyleCaptioncharacters{وَلِيٌّ }\textstyleDropCaps{вали, угодник, при­ближённый к Богу человек‎}

\textstyleCaptioncharacters{دُونَ نِزَاعٍ }\textstyleDropCaps{бесспорно‎}

\textstyleCaptioncharacters{مُطَهَّرٌ }\textstyleDropCaps{очищенный‎}

\textstyleCaptioncharacters{صِلَةٌ }\textstyleDropCaps{связь‎}

\textstyleCaptioncharacters{لَهُ صِلَةٌ }\textstyleDropCaps{он имеет связь‎}

\textstyleCaptioncharacters{اَلْمَلَأُ الأَعْلَى }\textstyleDropCaps{общество ангелов, верховное общество‎}

\textstyleCaptioncharacters{اِطَّلَعَ عَلَى... }\textstyleDropCaps{ознакомил­ся, осведомился с чем-л., взглянул на что-л.‎}

\textstyleCaptioncharacters{اللَّوْحُ المَحْفُوظُ }\textstyleDropCaps{храни­мая доска (где записано всё Божье предопределение)‎}

\textstyleCaptioncharacters{أَخَذَ بِيَدِهِ }\textstyleDropCaps{взял кого-л. за руку‎}

\textstyleCaptioncharacters{غَيْبِيٌّ }\textstyleDropCaps{сокровенный, та­инственный, скрытый‎}

\textstyleCaptioncharacters{نَجِسٌ }\textstyleDropCaps{грязный, нечистый‎}

\textstyleCaptioncharacters{مَلِكٌ }\textstyleDropCaps{король, царь‎}

\textstyleCaptioncharacters{عَامَّةٌ }\textstyleDropCaps{простой народ, мас­са‎}

\textstyleCaptioncharacters{تَقَرَّبَ إِلَى... }\textstyleDropCaps{сблизился, искал близости с кем‎}

\textstyleCaptioncharacters{تَبْلِيغٌ }\textstyleDropCaps{доведение, сообще­ние, передача‎}

\textstyleCaptioncharacters{حَاشِيَةٌ }\textstyleDropCaps{свита, челядь, при­ближённые‎}

\textstyleCaptioncharacters{شَرِيعَةٌ }\textstyleDropCaps{шариат, божий за­кон ‎}

\textstyleCaptioncharacters{اللَّهُ عَزَّ وَ جَلَّ }\textstyleDropCaps{Бог, Мо­гущественный, Величественный‎}

\textstyleCaptioncharacters{إِنَّا لِلَّهِ وَ إِنَّا إِلَيْهِ رَاجِعُونَ! }\textstyleDropCaps{поистине, мы принадлежим Богу и мы к Нему возвращаемся‎}

\textstyleCaptioncharacters{أَعَاذَنَا اللَّهُ }\textstyleDropCaps{Да сохранит Бог нас! Боже упаси!‎}

\textstyleCaptioncharacters{خَالَفَ }\textstyleDropCaps{противоречил, не соответствовал‎}

\subsection[УРОК 123‎]{\textstyleDropCaps{УРОК 123‎}}
\textstyleCaptioncharacters{لَيْسَ كَمِثْلِهِ شَيْءٌ \ }\textstyleDropCaps{нет ничего подобного Ему ‎}

\textstyleCaptioncharacters{رَجَا }\textstyleDropCaps{надеялся‎}

\textstyleCaptioncharacters{تَوَكَّلَ عَلَى... }\textstyleDropCaps{уповал, рассчитывал‎}

\textstyleCaptioncharacters{إِنَّهُ لَيَقُولُ }\textstyleDropCaps{поистине, он говорит, он же говорит‎}

\textstyleCaptioncharacters{عَابِدٌ }\textstyleDropCaps{поклоняющийся, поклонник‎}

\textstyleCaptioncharacters{فِي القَدِيمِ }\textstyleDropCaps{раньше, в ста­рину‎}

\textstyleCaptioncharacters{وَثَنٌ }\textstyleDropCaps{идол‎}

\textstyleCaptioncharacters{عُبَّادُ الأَوْثَانِ }\textstyleDropCaps{идоло­поклонники‎}

\textstyleCaptioncharacters{سَوَاءً }\textstyleDropCaps{совершенно одина­ково‎}

\textstyleCaptioncharacters{شَفِيعٌ }\textstyleDropCaps{заступник, ходатай‎}

\textstyleCaptioncharacters{زُلْفَى }\textstyleDropCaps{приближение‎}

\textstyleCaptioncharacters{شَبَّهَ بِـ... }\textstyleDropCaps{уподобил чему-л.‎}

\textstyleCaptioncharacters{ضَرَبَ مَثَلاً }\textstyleDropCaps{привел при­мер, притчу‎}

\textstyleCaptioncharacters{مَا أَبْعَدَهُ عَنِ الحَقِّ }\textstyleDropCaps{как он далёк от истины!‎}

\textstyleCaptioncharacters{أَبْلَغَ }\textstyleDropCaps{сообщил, доложил, передал‎}

\textstyleCaptioncharacters{رَعِيَّةٌ }\textstyleDropCaps{подданные, гражда­не‎}

\textstyleCaptioncharacters{مَحْجُوبٌ عَنْ... }\textstyleDropCaps{закры­тый, отделенный от кого-либо‎}

\textstyleCaptioncharacters{قَصْرٌ }\textstyleDropCaps{дворец‎}

\textstyleCaptioncharacters{يُجِيبُ دَعْوَةَ الدَّاعِ }\textstyleDropCaps{от­вечает на зов зовущего ‎}

\textstyleCaptioncharacters{حَبْلُ الوَرِيد }\textstyleDropCaps{шейная ар­терия‎}

\textstyleCaptioncharacters{تَأْيِيدٌ }\textstyleDropCaps{поддержка, под­крепление‎}

\textstyleCaptioncharacters{غَنِيٌّ عن... }\textstyleDropCaps{не нуждаю­щийся в чём-л., в ком- л.‎}

\textstyleCaptioncharacters{تَعَالَى اللَّهُ عَنْ ذَلِكَ عُلُوًّا كَبِيرًا }\textstyleDropCaps{Бог превыше того, далёк от того на большую высоту‎}

\textstyleCaptioncharacters{كُنْ عَلَى حَذَرٍ }\textstyleDropCaps{будь осторожен‎}

\textstyleCaptioncharacters{اِنْخَدَعَ }\textstyleDropCaps{обманулся‎}

\textstyleCaptioncharacters{اِهْتَدَى }\textstyleDropCaps{стал, попал на вер­ный путь‎}

\subsection[Урок 124‎]{\textstyleDropCaps{Урок 124‎}}
\textstyleCaptioncharacters{قُبُورِيٌّ }\textstyleDropCaps{могилопоклон­ник, почитающий могилу клоняющийся могилам‎}

\textstyleCaptioncharacters{مُعْظَمُ... }\textstyleDropCaps{большая часть, основная часть чего-либо‎}

\textstyleCaptioncharacters{صَرَفَ مُعْظَمَ هَمِّهِ }\textstyleDropCaps{основную свою заботу направил‎}

\textstyleCaptioncharacters{مَيِّتٌ }\textstyleDropCaps{мёртвый, покойник ‎}

\textstyleCaptioncharacters{ظَنًّا مِنْهُ }\textstyleDropCaps{думая, что... ‎}

\textstyleCaptioncharacters{أَنْفَقَ }\textstyleDropCaps{расходовал, тратил ‎}

\textstyleCaptioncharacters{أَمْوَالٌ طَائِلَةٌ }\textstyleDropCaps{большие деньги‎}

\textstyleCaptioncharacters{اِعْتَقَدَ }\textstyleDropCaps{был убеждённым, твёрдо верил‎}

\textstyleCaptioncharacters{كُلَّمَا كَانَ أَكْثَرَ كَانَ أَفْضَلَ }\textstyleDropCaps{чем больше, тем лучше ‎}

\textstyleCaptioncharacters{تَبَرَّكَ بِـ... }\textstyleDropCaps{искал благо­дать, баракат через что-л.‎}

\textstyleCaptioncharacters{صَاحِبُ القَبْرِ }\textstyleDropCaps{обитатель могилы, покойник данной, могилы‎}

\textstyleCaptioncharacters{أَلْصَقَ بِـ... }\textstyleDropCaps{прижал, при­льнул, приник к чему-л.‎}

\textstyleCaptioncharacters{تَمَسَّحَ بِـ... \ }\textstyleDropCaps{тёрся обо что-л.‎}

\textstyleCaptioncharacters{خِرْقَةٌ \ }\textstyleDropCaps{тряпка, лоскут‎}

\textstyleCaptioncharacters{تُرْبَةٌ }\textstyleDropCaps{земля, грунт ‎}

\textstyleCaptioncharacters{نَفِيسٌ }\textstyleDropCaps{ценный, драгоцен­ный‎}

\textstyleCaptioncharacters{يَا لَهُمْ مِنْ جُهَّالٍ! }\textstyleDropCaps{ну и невежды! какие невежды! ‎}

\textstyleCaptioncharacters{قَلَّ }\textstyleDropCaps{уменьшился, стал ма­лым‎}

\textstyleCaptioncharacters{فَشَا \ }\textstyleDropCaps{распространился‎}

\textstyleCaptioncharacters{بَعُدَ عَن... }\textstyleDropCaps{удалился от‎}

\textstyleCaptioncharacters{عَلَيْكَ بِـ... }\textstyleDropCaps{ты держись за..., берись за...‎}

\textstyleCaptioncharacters{تِبْيَانٌ }\textstyleDropCaps{разъяснение‎}

\textstyleCaptioncharacters{(يَا) حَبَّذَا لَوْ... }\textstyleDropCaps{как хо­рошо было бы, если ‎}

\textstyleCaptioncharacters{نَشَرَ }\textstyleDropCaps{распространил ‎}

\textstyleCaptioncharacters{أَنْشَأَ }\textstyleDropCaps{создал ‎}

\textstyleCaptioncharacters{كَسَا }\textstyleDropCaps{одел кого-л. ‎}

\textstyleCaptioncharacters{عَارٍ }\textstyleDropCaps{голый, нагой ‎}

\textstyleCaptioncharacters{بَعَثَ }\textstyleDropCaps{послал, отправил‎}

\textstyleCaptioncharacters{بَيْتُ اللَّهِ الحَرَامُ }\textstyleDropCaps{святой Дом Бога (Кааба) ‎}

\textstyleCaptioncharacters{عَلَى سَبِيلِ العِبَادَةِ }\textstyleDropCaps{в виде поклонения, в качестве поклонения ‎}

\textstyleCaptioncharacters{ضَرِيحٌ }\textstyleDropCaps{мавзолей, гробни­ца‎}

\textstyleCaptioncharacters{نَعُوذُ بِاللَّهِ مِنْهُ }\textstyleDropCaps{прибе­гаем к Богу от него; упаси Боже‎}

\textstyleCaptioncharacters{فِينَا مَنْ إِذَا دَخَلَ لَمْ يُسَلِّمْ }\textstyleDropCaps{есть среди нас такие, что не приветствует, когда входит‎}

\subsection[Урок 125‎]{\textstyleDropCaps{Урок 125‎}}
\textstyleCaptioncharacters{مَشْرُوعٌ }\textstyleDropCaps{законный‎}

\textstyleCaptioncharacters{بَشَرِيَّةٌ }\textstyleDropCaps{человечество, че­ловеческий род‎}

\textstyleCaptioncharacters{أَبُو الْبَشَرِيَّةِ }\textstyleDropCaps{прароди­тель человечества‎}

\textstyleCaptioncharacters{أَخْفَى }\textstyleDropCaps{скрыл, утаил‎}

\textstyleCaptioncharacters{مَوْقِعٌ }\textstyleDropCaps{место, местополо­жение‎}

\textstyleCaptioncharacters{...كَذَا }\textstyleDropCaps{такой-то‎}

\textstyleCaptioncharacters{فِي يَوْمِ كَذَا }\textstyleDropCaps{в такой-то день‎}

\textstyleCaptioncharacters{أَفْضَلُ مِنْ... }\textstyleDropCaps{лучше, предпоч тительнее‎}

\textstyleCaptioncharacters{قُبَّةٌ }\textstyleDropCaps{купол,свод‎}

\textstyleCaptioncharacters{قُبَّةُ الضَّرِيحِ }\textstyleDropCaps{купол гроб­ницы‎}

\textstyleCaptioncharacters{بَخِلَ }\textstyleDropCaps{скупился для кого-л.‎}

\textstyleCaptioncharacters{بُخْلًا عَلَيْكَ }\textstyleDropCaps{скупясь для тебя‎}

\textstyleCaptioncharacters{حَاشَ لِلَّهِ! }\textstyleDropCaps{совсем нет! это исключается! это невероятно!‎}

\textstyleCaptioncharacters{لَعَنَ }\textstyleDropCaps{проклинал‎}

\textstyleCaptioncharacters{يَهُودِيٌّ }\textstyleDropCaps{иудеи, евреи‎}

\textstyleCaptioncharacters{نَصْرَانِيٌّ }\textstyleDropCaps{христианин‎}

\textstyleCaptioncharacters{صَلُّوا عَلَيَّ }\textstyleDropCaps{читайте салят за меня, благословите меня‎}

\textstyleCaptioncharacters{فِي بَعْضِ الأَحْيَانِ }\textstyleDropCaps{ино­гда‎}

\textstyleCaptioncharacters{دَعَا لَهُ \ }\textstyleDropCaps{молился за него‎}

\textstyleCaptioncharacters{أُدْعُ لِى }\textstyleDropCaps{молись за меня‎}

\textstyleCaptioncharacters{شِرْكِيَّاتٌ }\textstyleDropCaps{языческие по­ступки‎}

\textstyleCaptioncharacters{وَرَدَ فِي الْكِتَابِ }\textstyleDropCaps{в кни­ге говорится, приводится ‎}

\textstyleCaptioncharacters{لَمْ يَرِدْ فِي الْحَدِيثِ }\textstyleDropCaps{в хадисе не говорится, не приводится ‎}

\textstyleCaptioncharacters{فِي... }\textstyleDropCaps{по поводу, насчёт‎}

\textstyleCaptioncharacters{مَا تَقُولُ فِي هَذَا؟ }\textstyleDropCaps{что ты по этому поводу скажешь? ‎}

\textstyleCaptioncharacters{أَذِنَ }\textstyleDropCaps{разрешил‎}

\textstyleCaptioncharacters{أَذِنَ لِي }\textstyleDropCaps{мне разрешено‎}

\textstyleCaptioncharacters{ذَكَّرَ }\textstyleDropCaps{напомнил‎}

\subsection[Урок 126‎]{\textstyleDropCaps{Урок 126‎}}
\textstyleCaptioncharacters{مَصْدَرٌ }\textstyleDropCaps{источник‎}

\textstyleCaptioncharacters{إِنْ قِيلَ لَكَ }\textstyleDropCaps{если тебе скажут‎}

\textstyleCaptioncharacters{تَمَسَّكَ بِ... }\textstyleDropCaps{держался, крепко держался за что-л.‎}

\textstyleCaptioncharacters{مَا... }\textstyleDropCaps{пока‎}

\textstyleCaptioncharacters{مَا تَمَسَّكَ بِالْقُرْآنِ }\textstyleDropCaps{пока ты держишься за Коран‎}

\textstyleCaptioncharacters{ضَلَّ }\textstyleDropCaps{заблудился, сбился‎}

\textstyleCaptioncharacters{كُلُّ كِتَابٍ }\textstyleDropCaps{каждая книга‎}

\textstyleCaptioncharacters{كُلُّ الْكِتَابِ }\textstyleDropCaps{вся книга‎}

\textstyleCaptioncharacters{تَدَبَّرِ }\textstyleDropCaps{размышлял, рассмат­ривал, вдумался во что‎}

\textstyleCaptioncharacters{لا تَلْتَفِتْ إِلَى قَوْلِهِ }\textstyleDropCaps{не обращай внимания на его слова ‎}

\textstyleCaptioncharacters{سَيِّدُ الْمُرْسَلِينَ }\textstyleDropCaps{Сайй­ид посланников, Старший из посланников‎}

\textstyleCaptioncharacters{أَنَّى }\textstyleDropCaps{где там! куда там!‎}

\textstyleCaptioncharacters{أَنَّى لَنَا أَنْ نَفْهَمَهُ؟ }\textstyleDropCaps{откуда нам понять его! ‎}

\textstyleCaptioncharacters{كَيْدٌ }\textstyleDropCaps{козни, махинации‎}

\textstyleCaptioncharacters{مَكْرٌ \ }\textstyleDropCaps{хитрость, обман‎}

\textstyleCaptioncharacters{مِنْ قِبَلِ... }\textstyleDropCaps{со стороны‎}

\textstyleCaptioncharacters{قَائِدٌ }\textstyleDropCaps{вождь, руководитель‎}

\textstyleCaptioncharacters{قَادَةُ الْكُفْرِ }\textstyleDropCaps{руководите­ли безбожия, главари неверия‎}

\textstyleCaptioncharacters{صِهْيُونِيٌّ }\textstyleDropCaps{сионист‎}

\textstyleCaptioncharacters{صَرَفَ عَنْ... }\textstyleDropCaps{отвёл, от­влёк от чего-л.; отвернул‎}

\textstyleCaptioncharacters{سَهُلَ عَلى... }\textstyleDropCaps{был лёг­ким, нетрудным для кого-л.‎}

\textstyleCaptioncharacters{اِسْتَوْلَى عَلَى... }\textstyleDropCaps{завла­дел, овладел‎}

\textstyleCaptioncharacters{عَقْلٌ }\textstyleDropCaps{ум, разум‎}

\textstyleCaptioncharacters{مَرِيضُ الْقَلْبِ }\textstyleDropCaps{с боль­ным сердцем‎}

\textstyleCaptioncharacters{جَاهِلٌ بِالدِّينِ }\textstyleDropCaps{не знаю­щий религии, не сведущий в религии‎}

\textstyleCaptioncharacters{مَكِيدَةٌ }\textstyleDropCaps{интрига, козни‎}

\textstyleCaptioncharacters{مَكَايِدُ الشَّيْطَانِ }\textstyleDropCaps{козни сатаны‎}

\textstyleCaptioncharacters{صَعُبَ عَلَى... }\textstyleDropCaps{был труд­ным‎}

\textstyleCaptioncharacters{اِحْتَرَزَ عَنْ... }\textstyleDropCaps{оберегался, остерегался, предохранил себя‎}

\textstyleCaptioncharacters{الدَّعْوَةُ إِلَى اللَّهِ }\textstyleDropCaps{призыв к Богу‎}

\textstyleCaptioncharacters{طَاغُوتٌ }\textstyleDropCaps{идол, ложный бог‎}

\textstyleCaptioncharacters{وَ هُوَ لاَ يَدْرِي }\textstyleDropCaps{сам того не зная, незаметно для самого себя‎}

\section{}

\bigskip

\section{}
\section{}
\section[]{}
{\centering
بِسْـــمِ اللهِ الرَّحْمَنِ الرَّحِيمِ
\par}

{\centering\bfseries
الدُّرُوسُ الاَوَّلِيَّةُ
\par}


\bigskip
\end{document}
